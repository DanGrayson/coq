%\RequirePackage{ifpdf}
%\ifpdf
%  \documentclass[11pt,a4paper,pdftex]{book}
%\else
  \documentclass[11pt,a4paper]{book}
%\fi

\usepackage[latin1]{inputenc}
\usepackage[T1]{fontenc}
\usepackage{times}
\usepackage{url}
\usepackage{verbatim}
\usepackage{amsmath}
\usepackage{amssymb}
\usepackage{alltt}
\usepackage{hevea}
\usepackage{ifpdf}
\usepackage[headings]{fullpage}
\usepackage{headers} % in this directory
\usepackage{multicol}
\usepackage{xspace}

% for coqide
\ifpdf   % si on est pas en pdflatex
  \usepackage[pdftex]{graphicx}
\else
  \usepackage[dvips]{graphicx}
\fi


%\includeonly{Setoid}

\input{../common/version.tex}
\input{../common/macros.tex}% extension .tex pour htmlgen
%%%%%%%%%%%%%%%%%%%%%%%%%%%%%%%%
% File title.tex
% Page formatting commands
% Macro \coverpage
%%%%%%%%%%%%%%%%%%%%%%%%%%%%%%%%

%\setlength{\marginparwidth}{0pt}
%\setlength{\oddsidemargin}{0pt}
%\setlength{\evensidemargin}{0pt}
%\setlength{\marginparsep}{0pt}
%\setlength{\topmargin}{0pt}
%\setlength{\textwidth}{16.9cm}
%\setlength{\textheight}{22cm}
%\usepackage{fullpage}

%\newcommand{\printingdate}{\today}
%\newcommand{\isdraft}{\Large\bf\today\\[20pt]}
%\newcommand{\isdraft}{\vspace{20pt}}

\newcommand{\coverpage}[3]{
\thispagestyle{empty}
\begin{center}
\bfseries % for the rest of this page, until \end{center}
\Huge
The Coq Proof Assistant\\[12pt]
#1\\[20pt]
\Large\today\\[20pt]
Version \coqversion%\footnote[1]{This research was partly supported by IST working group ``Types''}

\vspace{0pt plus .5fill}
#2
\par\vfill
The Coq Development Team

\vspace*{15pt}
\end{center}
\newpage

\thispagestyle{empty}
\hbox{}\vfill % without \hbox \vfill does not work at the top of the page
\begin{flushleft}
%BEGIN LATEX
V\coqversion, \today
\par\vspace{20pt}
%END LATEX
\copyright INRIA 1999-2004 ({\Coq} versions 7.x)

\copyright INRIA 2004-2011 ({\Coq} versions 8.x)

#3
\end{flushleft}
} % end of \coverpage definition


% \newcommand{\shorttitle}[1]{
% \begin{center}
% \begin{huge}
% \begin{bf}
% The Coq Proof Assistant\\
% \vspace{10pt}
%     #1\\
% \end{bf}
% \end{huge}
% \end{center}
% \vspace{5pt}
% }

% Local Variables: 
% mode: LaTeX
% TeX-master: ""
% End: 

% $Id$ 
% extension .tex pour htmlgen
%\input{headers}

\usepackage[linktocpage,colorlinks]{hyperref}
% The manual advises to load hyperref package last to be able to redefine
% necessary commands.
% The above should work for both latex and pdflatex. Even if PDF is produced
% through DVI and PS using dvips and ps2pdf, hyperlinks should still work.
% linktocpage option makes page numbers, not section names, to be links in
% the table of contents.
% colorlinks option colors the links instead of using boxes.

% The command \tocnumber was added to HEVEA in version 1.06-6.
% It instructs HEVEA to put chapter numbers into the table of
% content entries. The table of content is produced by HACHA using
% the options -tocbis -o toc.html. HEVEA produces a warning when
% a command is not recognized, so versions earlier than 1.06-6 can
% still be used.
%HEVEA\tocnumber

\begin{document}
%BEGIN LATEX
\sloppy\hbadness=5000
%END LATEX

%BEGIN LATEX
\coverpage{Reference Manual}
{The Coq Development Team}
{This material may be distributed only subject to the terms and
conditions set forth in the Open Publication License, v1.0 or later
(the latest version is presently available at
\url{http://www.opencontent.org/openpub}).
Options A and B of the licence are {\em not} elected.}
%END LATEX

%\defaultheaders
\include{RefMan-int}% Introduction
\include{RefMan-pre}% Credits

%BEGIN LATEX
\tableofcontents
%END LATEX

\part{The language}
%BEGIN LATEX
\defaultheaders
%END LATEX
\include{RefMan-gal.v}% Gallina
\chapter[Extensions of \Gallina{}]{Extensions of \Gallina{}\label{Gallina-extension}\index{Gallina}}

{\gallina} is the kernel language of {\Coq}. We describe here extensions of
the Gallina's syntax.

\section{Record types
\comindex{Record}
\comindex{Inductive}
\comindex{CoInductive}
\label{Record}}

The \verb+Record+ construction is a macro allowing the definition of
records as is done in many programming languages.  Its syntax is
described on Figure~\ref{record-syntax}.  In fact, the \verb+Record+
macro is more general than the usual record types, since it allows
also for ``manifest'' expressions. In this sense, the \verb+Record+
construction allows to define ``signatures''.

\begin{figure}[h]
\begin{centerframe}
\begin{tabular}{lcl}
{\sentence} & ++= & {\record}\\
  & & \\
{\record} & ::= &
   {\recordkw} {\ident} \zeroone{\binders} \zeroone{{\tt :} {\sort}} \verb.:=. \\
&& ~~~~\zeroone{\ident}
       \verb!{! \zeroone{\nelist{\field}{;}} \verb!}! \verb:.:\\
  & & \\
{\recordkw} & ::= &
   {\tt Record} $|$ {\tt Inductive} $|$ {\tt CoInductive}\\
  & & \\
{\field} & ::= & {\name} \zeroone{\binders} : {\type} [ {\tt where} {\it notation} ] \\
 & $|$ & {\name} \zeroone{\binders} {\typecstr} := {\term}
\end{tabular}
\end{centerframe}
\caption{Syntax for the definition of {\tt Record}}
\label{record-syntax}
\end{figure}

\noindent In the expression

\smallskip
{\tt Record} {\ident} {\params} \texttt{:} 
   {\sort} := {\ident$_0$} \verb+{+
 {\ident$_1$} \binders$_1$ \texttt{:} {\term$_1$}; 
              \dots
  {\ident$_n$} \binders$_n$ \texttt{:} {\term$_n$} \verb+}+.
\smallskip
 
\noindent the identifier {\ident} is the name of the defined record
and {\sort} is its type. The identifier {\ident$_0$} is the name of
its constructor. If {\ident$_0$} is omitted, the default name {\tt
Build\_{\ident}} is used. If {\sort} is omitted, the default sort is ``{\Type}''.
The identifiers {\ident$_1$}, ..,
{\ident$_n$} are the names of fields and {\tt forall} \binders$_1${\tt ,} {\term$_1$}, ..., {\tt forall} \binders$_n${\tt ,} {\term$_n$}
their respective types. Remark that the type of {\ident$_i$} may
depend on the previous {\ident$_j$} (for $j<i$). Thus the order of the
fields is important. Finally, {\params} are the parameters of the
record.

More generally, a record may have explicitly defined (a.k.a.
manifest) fields. For instance, {\tt Record} {\ident} {\tt [}
{\params} {\tt ]} \texttt{:} {\sort} := \verb+{+ {\ident$_1$}
\texttt{:} {\type$_1$} \verb+;+ {\ident$_2$} \texttt{:=} {\term$_2$}
\verb+;+ {\ident$_3$} \texttt{:} {\type$_3$} \verb+}+ in which case
the correctness of {\type$_3$} may rely on the instance {\term$_2$} of
{\ident$_2$} and {\term$_2$} in turn may depend on {\ident$_1$}.


\Example
The set of rational numbers may be defined as:
\begin{coq_eval}
Reset Initial.
\end{coq_eval}
\begin{coq_example}
Record Rat : Set := mkRat
  {sign : bool;
   top : nat;
   bottom : nat;
   Rat_bottom_cond : 0 <> bottom;
   Rat_irred_cond :
    forall x y z:nat, (x * y) = top /\ (x * z) = bottom -> x = 1}.
\end{coq_example}

Remark here that the field
\verb+Rat_cond+ depends on the field \verb+bottom+. 

%Let us now see the work done by the {\tt Record} macro.
%First the macro generates an inductive definition
%with just one constructor:
%
%\medskip
%\noindent
%{\tt Inductive {\ident} \zeroone{\binders} : {\sort} := \\
%\mbox{}\hspace{0.4cm} {\ident$_0$} : forall ({\ident$_1$}:{\term$_1$}) .. 
%({\ident$_n$}:{\term$_n$}), {\ident} {\rm\sl params}.}
%\medskip

Let us now see the work done by the {\tt Record} macro.  First the
macro generates an inductive definition with just one constructor:
\begin{quote}
{\tt Inductive {\ident} {\params} :{\sort} :=} \\
\qquad {\tt
  {\ident$_0$} ({\ident$_1$}:{\term$_1$}) .. ({\ident$_n$}:{\term$_n$}).}
\end{quote}
To build an object of type {\ident}, one should provide the
constructor {\ident$_0$} with $n$ terms filling the fields of
the record.

As an example, let us define the rational $1/2$:
\begin{coq_example*}
Require Import Arith.
Theorem one_two_irred :
 forall x y z:nat, x * y = 1 /\ x * z = 2 -> x = 1.
\end{coq_example*}
\begin{coq_eval}
Lemma mult_m_n_eq_m_1 : forall m n:nat, m * n = 1 -> m = 1.
destruct m; trivial.
intros; apply f_equal with (f := S).
destruct m; trivial.
destruct n; simpl in H.
 rewrite <- mult_n_O in H.
   discriminate.
 rewrite <- plus_n_Sm in H.
   discriminate.
Qed.

intros x y z [H1 H2].
 apply mult_m_n_eq_m_1 with (n := y); trivial.
\end{coq_eval}
\ldots
\begin{coq_example*}
Qed.
\end{coq_example*}
\begin{coq_example}
Definition half := mkRat true 1 2 (O_S 1) one_two_irred.
\end{coq_example}
\begin{coq_example}
Check half.
\end{coq_example}

The macro generates also, when it is possible, the projection
functions for destructuring an object of type {\ident}.  These
projection functions have the same name that the corresponding
fields. If a field is named ``\verb=_='' then no projection is built
for it.  In our example:

\begin{coq_example}
Eval compute in half.(top).
Eval compute in half.(bottom).
Eval compute in half.(Rat_bottom_cond).
\end{coq_example}
\begin{coq_eval}
Reset Initial.
\end{coq_eval}

Records defined  with the {\tt Record}  keyword are not  allowed to be
recursive (references  to the record's name  in the type  of its field
raises an  error). To define recursive  records, one can  use the {\tt
  Inductive} and {\tt CoInductive} keywords, resulting in an inductive
or  co-inductive record.  A  \emph{caveat}, however,  is that  records
cannot appear in mutually inductive (or co-inductive) definitions.

\begin{Warnings}
\item {\tt Warning: {\ident$_i$} cannot be defined.}

  It can happen that the definition of a projection is impossible.
  This message is followed by an explanation of this impossibility.
  There may be three reasons:
   \begin{enumerate}
   \item The name {\ident$_i$} already exists in the environment (see
     Section~\ref{Axiom}).
   \item The body of {\ident$_i$} uses an incorrect elimination for
     {\ident} (see Sections~\ref{Fixpoint} and~\ref{Caseexpr}).
   \item The type of the projections {\ident$_i$} depends on previous
   projections which themselves could not be defined.
   \end{enumerate}  
\end{Warnings}     

\begin{ErrMsgs}

\item \errindex{Records declared with the keyword Record or Structure cannot be recursive.}

  The record name {\ident} appears in the type of its fields, but uses
  the keyword  {\tt Record}. Use  the keyword {\tt Inductive}  or {\tt
    CoInductive} instead.
\item \errindex{Cannot handle mutually (co)inductive records.}

  Records  cannot  be  defined  as  part  of  mutually  inductive  (or
  co-inductive) definitions,  whether with records only  or mixed with
  standard definitions.
\item During the definition of the one-constructor inductive
  definition, all the errors of inductive definitions, as described in
  Section~\ref{gal-Inductive-Definitions}, may also occur.

\end{ErrMsgs}

\SeeAlso Coercions and records in Section~\ref{Coercions-and-records}
of the chapter devoted to coercions.

\Rem {\tt Structure} is a synonym of the keyword {\tt Record}.

\Rem Creation of an object of record type can be done by calling {\ident$_0$}
and passing arguments in the correct order.

\begin{coq_example}
Record point := { x : nat; y : nat }.
Definition a := Build_point 5 3.
\end{coq_example}

The following syntax allows to create objects by using named fields. The
fields do not have to be in any particular order, nor do they have to be all
present if the missing ones can be inferred or prompted for (see
Section~\ref{Program}).

\begin{coq_example}
Definition b := {| x := 5; y := 3 |}.
Definition c := {| y := 3; x := 5 |}.
\end{coq_example}

This syntax can be disabled globally for printing by
\begin{quote}
{\tt Unset Printing Records.}
\end{quote}
For a given type, one can override this using either
\begin{quote}
{\tt Add Printing Record {\ident}.}
\end{quote}
to get record syntax or
\begin{quote}
{\tt Add Printing Constructor {\ident}.}
\end{quote}
to get constructor syntax.

This syntax can also be used for pattern matching.

\begin{coq_example}
Eval compute in (
  match b with
  | {| y := S n |} => n
  | _ => 0
  end).
\end{coq_example}

\begin{coq_eval}
Reset Initial.
\end{coq_eval}

\Rem An experimental syntax for projections based on a dot notation is
available. The command to activate it is
\begin{quote}
{\tt Set Printing Projections.}
\end{quote}

\begin{figure}[t]
\begin{centerframe}
\begin{tabular}{lcl}
{\term} & ++= & {\term} {\tt .(} {\qualid} {\tt )}\\
 & $|$ & {\term} {\tt .(} {\qualid} \nelist{\termarg}{} {\tt )}\\
 & $|$ & {\term} {\tt .(} {@}{\qualid} \nelist{\term}{} {\tt )}
\end{tabular}
\end{centerframe}
\caption{Syntax of \texttt{Record} projections}
\label{fig:projsyntax}
\end{figure}

The corresponding grammar rules are given Figure~\ref{fig:projsyntax}.
When {\qualid} denotes a projection, the syntax {\tt
  {\term}.({\qualid})} is equivalent to {\qualid~\term}, the syntax
{\term}{\tt .(}{\qualid}~{\termarg}$_1$ {\ldots} {\termarg}$_n${\tt )} to
{\qualid~{\termarg}$_1$ {\ldots} {\termarg}$_n$~\term}, and the syntax
{\term}{\tt .(@}{\qualid}~{\term}$_1$~\ldots~{\term}$_n${\tt )} to
{@\qualid~{\term}$_1$ {\ldots} {\term}$_n$~\term}. In each case, {\term}
is the object projected and the other arguments are the parameters of
the inductive type.

To deactivate the printing of projections, use 
{\tt Unset Printing Projections}.


\section{Variants and extensions of {\mbox{\tt match}}
\label{Extensions-of-match}
\index{match@{\tt match\ldots with\ldots end}}}

\subsection{Multiple and nested pattern-matching
\index{ML-like patterns}
\label{Mult-match}}

The basic version of \verb+match+ allows pattern-matching on simple
patterns. As an extension, multiple nested patterns or disjunction of
patterns are allowed, as in ML-like languages.

The extension just acts as a macro that is expanded during parsing
into a sequence of {\tt match} on simple patterns. Especially, a
construction defined using the extended {\tt match} is generally
printed under its expanded form (see~\texttt{Set Printing Matching} in
section~\ref{SetPrintingMatching}).

\SeeAlso Chapter~\ref{Mult-match-full}.

\subsection{Pattern-matching on boolean values: the {\tt if} expression
\label{if-then-else}
\index{if@{\tt if ... then ... else}}}

For inductive types with exactly two constructors and for
pattern-matchings expressions which do not depend on the arguments of
the constructors, it is possible to use a {\tt if ... then ... else}
notation. For instance, the definition

\begin{coq_example}
Definition not (b:bool) :=
  match b with
  | true => false
  | false => true
  end.
\end{coq_example}

\noindent can be alternatively written

\begin{coq_eval}
Reset not.
\end{coq_eval}
\begin{coq_example}
Definition not (b:bool) := if b then false else true.
\end{coq_example}

More generally, for an inductive type with constructors {\tt C$_1$}
and {\tt C$_2$}, we have the following equivalence

\smallskip

{\tt if {\term} \zeroone{\ifitem} then {\term}$_1$ else {\term}$_2$} $\equiv$
\begin{tabular}[c]{l}
{\tt match {\term} \zeroone{\ifitem} with}\\
{\tt \verb!|! C$_1$ \_ {\ldots} \_ \verb!=>! {\term}$_1$} \\
{\tt \verb!|! C$_2$ \_ {\ldots} \_ \verb!=>! {\term}$_2$} \\
{\tt end}
\end{tabular}

Here is an example.

\begin{coq_example}
Check (fun x (H:{x=0}+{x<>0}) =>
  match H with
  | left _ => true
  | right _ => false
  end).
\end{coq_example}

Notice that the printing uses the {\tt if} syntax because {\tt sumbool} is
declared as such (see Section~\ref{printing-options}).

\subsection{Irrefutable patterns: the destructuring {\tt let} variants 
\index{let in@{\tt let ... in}}
\label{Letin}}

Pattern-matching on terms inhabiting inductive type having only one
constructor can be alternatively written using {\tt let ... in ...}
constructions. There are two variants of them.

\subsubsection{First destructuring {\tt let} syntax}
The expression {\tt let
(}~{\ident$_1$},\ldots,{\ident$_n$}~{\tt ) :=}~{\term$_0$}~{\tt
in}~{\term$_1$} performs case analysis on a {\term$_0$} which must be in
an inductive type with one constructor having itself $n$ arguments. Variables
{\ident$_1$}\ldots{\ident$_n$} are bound to the $n$ arguments of the
constructor in expression {\term$_1$}. For instance, the definition

\begin{coq_example}
Definition fst (A B:Set) (H:A * B) := match H with
                                      | pair x y => x
                                      end.
\end{coq_example}

can be alternatively written 

\begin{coq_eval}
Reset fst.
\end{coq_eval}
\begin{coq_example}
Definition fst (A B:Set) (p:A * B) := let (x, _) := p in x.
\end{coq_example}
Notice that reduction is different from regular {\tt let ... in ...}
construction since it happens only if {\term$_0$} is in constructor
form. Otherwise, the reduction is blocked.

The pretty-printing of a definition by matching on a
irrefutable pattern can either be done using {\tt match} or the {\tt
let} construction (see Section~\ref{printing-options}).

If {\term} inhabits an inductive type with one constructor {\tt C},
we have an equivalence between

{\tt let ({\ident}$_1$,\ldots,{\ident}$_n$) \zeroone{\ifitem} := {\term} in {\term}'}

\noindent and

{\tt match {\term} \zeroone{\ifitem} with C {\ident}$_1$ {\ldots} {\ident}$_n$ \verb!=>! {\term}' end}


\subsubsection{Second destructuring {\tt let} syntax\index{let '... in}}

Another destructuring {\tt let} syntax is available for inductive types with
one constructor by giving an arbitrary pattern instead of just a tuple
for all the arguments. For example, the preceding example can be written:
\begin{coq_eval}
Reset fst.
\end{coq_eval}
\begin{coq_example}
Definition fst (A B:Set) (p:A*B) := let 'pair x _ := p in x.
\end{coq_example}

This is useful to match deeper inside tuples and also to use notations
for the pattern, as the syntax {\tt let 'p := t in b} allows arbitrary
patterns to do the deconstruction. For example:

\begin{coq_example}
Definition deep_tuple (A:Set) (x:(A*A)*(A*A)) : A*A*A*A :=
  let '((a,b), (c, d)) := x in (a,b,c,d).
Notation " x 'With' p " := (exist _ x p) (at level 20).
Definition proj1_sig' (A:Set) (P:A->Prop) (t:{ x:A | P x }) : A :=
  let 'x With p := t in x.
\end{coq_example}

When printing definitions which are written using this construct it
takes precedence over {\tt let} printing directives for the datatype
under consideration (see Section~\ref{printing-options}).

\subsection{Controlling pretty-printing of {\tt match} expressions
\label{printing-options}}

The following commands give some control over the pretty-printing of
{\tt match} expressions.

\subsubsection{Printing nested patterns
\label{SetPrintingMatching}
\comindex{Set Printing Matching}
\comindex{Unset Printing Matching}
\comindex{Test Printing Matching}}

The Calculus of Inductive Constructions knows pattern-matching only
over simple patterns. It is however convenient to re-factorize nested
pattern-matching into a single pattern-matching over a nested pattern.
{\Coq}'s printer try to do such limited re-factorization.

\begin{quote}
{\tt Set Printing Matching.}
\end{quote}
This tells {\Coq} to try to use nested patterns. This is the default
behavior.

\begin{quote}
{\tt Unset Printing Matching.}
\end{quote}
This tells {\Coq} to print only simple pattern-matching problems in
the same way as the {\Coq} kernel handles them.

\begin{quote}
{\tt Test Printing Matching.}
\end{quote}
This tells if the printing matching mode is on or off. The default is
on.

\subsubsection{Printing of wildcard pattern
\comindex{Set Printing Wildcard}
\comindex{Unset Printing Wildcard}
\comindex{Test Printing Wildcard}}

Some variables in a pattern may not occur in the right-hand side of
the pattern-matching clause.  There are options to control the
display of these variables.

\begin{quote}
{\tt Set Printing Wildcard.}
\end{quote}
The variables having no occurrences in the right-hand side of the
pattern-matching clause are just printed using the wildcard symbol
``{\tt \_}''.

\begin{quote}
{\tt Unset Printing Wildcard.}
\end{quote}
The variables, even useless, are printed using their usual name. But some
non dependent variables have no name. These ones are still printed
using a ``{\tt \_}''.

\begin{quote}
{\tt Test Printing Wildcard.}
\end{quote}
This tells if the wildcard printing mode is on or off. The default is
to print wildcard for useless variables.

\subsubsection{Printing of the elimination predicate
\comindex{Set Printing Synth}
\comindex{Unset Printing Synth}
\comindex{Test Printing Synth}}

In most of the cases, the type of the result of a matched term is
mechanically synthesizable. Especially, if the result type does not
depend of the matched term.

\begin{quote}
{\tt Set Printing Synth.}
\end{quote}
The result type is not printed when {\Coq} knows that it can
re-synthesize it.

\begin{quote}
{\tt Unset Printing Synth.}
\end{quote}
This forces the result type to be always printed.

\begin{quote}
{\tt Test Printing Synth.}
\end{quote}
This tells if the non-printing of synthesizable types is on or off.
The default is to not print synthesizable types.

\subsubsection{Printing matching on irrefutable pattern
\label{AddPrintingLet}
\comindex{Add Printing Let {\ident}}
\comindex{Remove Printing Let {\ident}}
\comindex{Test Printing Let for {\ident}}
\comindex{Print Table Printing Let}}

If an inductive type has just one constructor,
pattern-matching can be written using {\tt let} ... {\tt :=}
... {\tt in}~...

\begin{quote}
{\tt Add Printing Let {\ident}.}
\end{quote}
This adds {\ident} to the list of inductive types for which
pattern-matching is written using a {\tt let} expression.

\begin{quote}
{\tt Remove Printing Let {\ident}.}
\end{quote}
This removes {\ident} from this list.

\begin{quote}
{\tt Test Printing Let for {\ident}.}
\end{quote}
This tells if {\ident} belongs to the list.

\begin{quote}
{\tt Print Table Printing Let.}
\end{quote}
This prints the list of inductive types for which pattern-matching is
written using a {\tt let} expression.

The list of inductive types for which pattern-matching is written
using a {\tt let} expression is managed synchronously. This means that
it is sensible to the command {\tt Reset}.

\subsubsection{Printing matching on booleans
\comindex{Add Printing If {\ident}}
\comindex{Remove Printing If {\ident}}
\comindex{Test Printing If for {\ident}}
\comindex{Print Table Printing If}}

If an inductive type is isomorphic to the boolean type,
pattern-matching can be written using {\tt if} ... {\tt then} ... {\tt
  else} ...

\begin{quote}
{\tt Add Printing If {\ident}.}
\end{quote}
This adds {\ident} to the list of inductive types for which
pattern-matching is written using an {\tt if} expression.

\begin{quote}
{\tt Remove Printing If {\ident}.}
\end{quote}
This removes {\ident} from this list.

\begin{quote}
{\tt Test Printing If for {\ident}.}
\end{quote}
This tells if {\ident} belongs to the list.

\begin{quote}
{\tt Print Table Printing If.}
\end{quote}
This prints the list of inductive types for which pattern-matching is
written using an {\tt if} expression.

The list of inductive types for which pattern-matching is written
using an {\tt if} expression is managed synchronously. This means that
it is sensible to the command {\tt Reset}.

\subsubsection{Example}

This example emphasizes what the printing options offer.

\begin{coq_example}
Test Printing Let for prod.
Print fst.
Remove Printing Let prod.
Unset Printing Synth.
Unset Printing Wildcard.
Print fst.
\end{coq_example}

% \subsection{Still not dead old notations}

% The following variant of {\tt match} is inherited from older version
% of {\Coq}. 

% \medskip
% \begin{tabular}{lcl}
% {\term} & ::= & {\annotation} {\tt Match} {\term} {\tt with} {\terms} {\tt end}\\
% \end{tabular}
% \medskip

% This syntax is a macro generating a combination of {\tt match} with {\tt
% Fix} implementing a combinator for primitive recursion equivalent to
% the {\tt Match} construction of \Coq\ V5.8. It is provided only for
% sake of compatibility with \Coq\ V5.8. It is recommended to avoid it.
% (see Section~\ref{Matchexpr}).

% There is also a notation \texttt{Case} that is the
% ancestor of \texttt{match}. Again, it is still in the code for
% compatibility with old versions but the user should not use it.

% Explained in RefMan-gal.tex
%% \section{Forced type}

%% In some cases, one may wish to assign a particular type to a term. The
%% syntax to force the type of a term is the following:

%% \medskip
%% \begin{tabular}{lcl}
%% {\term} & ++= & {\term} {\tt :} {\term}\\
%% \end{tabular}
%% \medskip

%% It forces the first term to be of type the second term. The
%% type must be compatible with
%% the term. More precisely it must be either a type convertible to
%% the automatically inferred type (see Chapter~\ref{Cic}) or a type
%% coercible to it, (see \ref{Coercions}). When the type of a
%% whole expression is forced, it is usually not necessary to give the types of
%% the variables involved in the term.

%% Example:

%% \begin{coq_example}
%% Definition ID := forall X:Set, X -> X.
%% Definition id := (fun X x => x):ID.
%% Check id.
%% \end{coq_example}

\section{Advanced recursive functions}

The \emph{experimental} command 
\begin{center}
   \texttt{Function {\ident} {\binder$_1$}\ldots{\binder$_n$}
     \{decrease\_annot\} : type$_0$ := \term$_0$}
   \comindex{Function}
   \label{Function}
\end{center}
can be seen as a generalization of {\tt Fixpoint}.  It is actually a
wrapper for several ways of defining a function \emph{and other useful
  related objects}, namely: an induction principle that reflects the
recursive structure of the function (see \ref{FunInduction}), and its
fixpoint equality.  The meaning of this
declaration is to define a function {\it ident}, similarly to {\tt
  Fixpoint}. Like in {\tt Fixpoint}, the decreasing argument must be
given (unless the function is not recursive), but it must not
necessary be \emph{structurally} decreasing. The point of the {\tt
  \{\}} annotation is to name the decreasing argument \emph{and} to
describe which kind of decreasing criteria must be used to ensure
termination of recursive calls.

The {\tt Function} construction enjoys also the {\tt with} extension
to define mutually recursive definitions. However, this feature does
not work for non structural recursive functions. % VRAI??

See the documentation of {\tt functional induction}
(see Section~\ref{FunInduction}) and {\tt Functional Scheme}
(see Section~\ref{FunScheme} and \ref{FunScheme-examples}) for how to use the
induction principle to easily reason about the function.

\noindent {\bf Remark: } To obtain the right principle, it is better
to put rigid parameters of the function as first arguments. For
example it is better to define plus like this:

\begin{coq_example*}
Function plus (m n : nat) {struct n} : nat :=
  match n with
  | 0 => m
  | S p => S (plus m p)
  end.
\end{coq_example*}
\noindent than like this:
\begin{coq_eval}
Reset plus.
\end{coq_eval}
\begin{coq_example*}
Function plus (n m : nat) {struct n} : nat :=
  match n with
  | 0 => m
  | S p => S (plus p m)
  end.
\end{coq_example*}

\paragraph[Limitations]{Limitations\label{sec:Function-limitations}}
\term$_0$ must be build as a \emph{pure pattern-matching tree}
(\texttt{match...with}) with applications only \emph{at the end} of
each branch.  

Function does not support partial application of the function being defined. Thus, the following example cannot be accepted due to the presence of partial application of \ident{wrong} into the body of \ident{wrong}~:
\begin{coq_example*}
  Function wrong (C:nat) {\ldots} : nat := 
    List.hd(List.map wrong (C::nil)).
\end{coq_example*}

For now dependent cases are not treated for non structurally terminating functions.



\begin{ErrMsgs}
\item \errindex{The recursive argument must be specified}
\item \errindex{No argument name \ident}
\item \errindex{Cannot use mutual definition with well-founded
    recursion or measure}

\item \errindex{Cannot define graph for \ident\dots} (warning)

  The generation of the graph relation \texttt{(R\_\ident)} used to
  compute the induction scheme of \ident\ raised a typing error. Only
  the ident is defined, the induction scheme will not be generated.

  This error happens generally when:

  \begin{itemize}
  \item the definition uses pattern matching on dependent types, which
    \texttt{Function} cannot deal with yet.
  \item the definition is not a \emph{pattern-matching tree} as
    explained above.
  \end{itemize}

\item \errindex{Cannot define principle(s) for \ident\dots} (warning)

  The generation of the graph relation \texttt{(R\_\ident)} succeeded
  but the induction principle could not be built. Only the ident is
  defined. Please report.

\item \errindex{Cannot build functional inversion principle} (warning)

  \texttt{functional inversion} will not be available for the
  function.
\end{ErrMsgs}


\SeeAlso{\ref{FunScheme}, \ref{FunScheme-examples}, \ref{FunInduction}}

Depending on the {\tt \{$\ldots$\}} annotation, different definition
mechanisms are used by {\tt Function}. More precise description
given below.

\begin{Variants}
\item \texttt{ Function {\ident} {\binder$_1$}\ldots{\binder$_n$}
    : type$_0$ := \term$_0$}

  Defines the not recursive function \ident\ as if declared with
  \texttt{Definition}.  Moreover the following are defined:

  \begin{itemize}
  \item {\tt\ident\_rect}, {\tt\ident\_rec} and {\tt\ident\_ind},
    which reflect the pattern matching structure of \term$_0$ (see the
    documentation of {\tt Inductive} \ref{Inductive});
  \item The inductive \texttt{R\_\ident} corresponding to the graph of
    \ident\ (silently);
  \item \texttt{\ident\_complete} and \texttt{\ident\_correct} which are
    inversion information linking the function and its graph.
  \end{itemize}
\item \texttt{Function {\ident} {\binder$_1$}\ldots{\binder$_n$}
    {\tt \{}{\tt struct} \ident$_0${\tt\}} : type$_0$ := \term$_0$}
  
  Defines the structural recursive function \ident\ as if declared
  with \texttt{Fixpoint}.  Moreover the following are defined:

  \begin{itemize}
  \item The same objects as above;
  \item The fixpoint equation of \ident: \texttt{\ident\_equation}.
  \end{itemize}
  
\item \texttt{Function {\ident} {\binder$_1$}\ldots{\binder$_n$} {\tt
      \{}{\tt measure \term$_1$} \ident$_0${\tt\}} : type$_0$ :=
    \term$_0$}
\item \texttt{Function {\ident} {\binder$_1$}\ldots{\binder$_n$}
 {\tt \{}{\tt wf \term$_1$} \ident$_0${\tt\}} : type$_0$ := \term$_0$}

Defines a recursive function by well founded recursion. \textbf{The
module \texttt{Recdef} of the standard library must be loaded for this
feature}. The {\tt \{\}} annotation is mandatory and must be one of
the following:
\begin{itemize}
\item {\tt \{measure} \term$_1$ \ident$_0${\tt\}} with \ident$_0$
      being the decreasing argument and \term$_1$ being a function
      from type of \ident$_0$ to \texttt{nat} for which value on the
      decreasing argument decreases (for the {\tt lt} order on {\tt
      nat}) at each recursive call of \term$_0$, parameters of the
      function are bound in  \term$_0$;
\item {\tt \{wf} \term$_1$ \ident$_0${\tt\}} with \ident$_0$ being
      the decreasing argument and \term$_1$ an ordering relation on
      the type of \ident$_0$ (i.e. of type T$_{\ident_0}$
      $\to$ T$_{\ident_0}$ $\to$ {\tt Prop}) for which
      the decreasing argument decreases at each recursive call of
      \term$_0$. The order must be well founded. parameters of the
      function are bound in  \term$_0$.
\end{itemize} 

Depending on the annotation, the user is left with some proof
obligations that will be used to define the function. These proofs
are: proofs that each recursive call is actually decreasing with
respect to the given criteria, and (if the criteria is \texttt{wf}) a
proof that the ordering relation is well founded.

%Completer sur measure et wf

Once proof obligations are discharged, the following objects are
defined:

\begin{itemize}
\item The same objects as with the \texttt{struct};
\item The lemma \texttt{\ident\_tcc} which collects all proof
  obligations in one property;
\item The lemmas \texttt{\ident\_terminate} and \texttt{\ident\_F}
  which is needed to be inlined during extraction of \ident.
\end{itemize}



%Complete!!
The way this recursive function is defined is the subject of several
papers by Yves Bertot and Antonia Balaa on the one hand, and Gilles Barthe,
Julien Forest, David Pichardie, and Vlad Rusu on the other hand.

%Exemples ok ici

\bigskip

\noindent {\bf Remark: } Proof obligations are presented as several
subgoals belonging to a Lemma {\ident}{\tt\_tcc}. % These subgoals are independent which means that in order to
% abort them you will have to abort each separately.



%The decreasing argument cannot be dependent of another??

%Exemples faux ici
\end{Variants}


\section{Section mechanism
\index{Sections}
\label{Section}}

The sectioning mechanism allows to organize a proof in structured
sections. Then local declarations become available (see
Section~\ref{Basic-definitions}).

\subsection{\tt Section {\ident}\comindex{Section}}

This command is used to open a section named {\ident}.

%% Discontinued ?
%% \begin{Variants}
%% \comindex{Chapter}
%% \item{\tt Chapter {\ident}}\\
%%         Same as {\tt Section {\ident}}
%% \end{Variants}

\subsection{\tt End {\ident}
\comindex{End}}

This command closes the section named {\ident}. After closing of the
section, the local declarations (variables and local definitions) get
{\em discharged}, meaning that they stop being visible and that all
global objects defined in the section are generalized with respect to
the variables and local definitions they each depended on in the
section.


Here is an example :
\begin{coq_example}
Section s1.
Variables x y : nat.
Let y' := y.
Definition x' := S x.
Definition x'' := x' + y'.
Print x'.
End s1.
Print x'.
Print x''.
\end{coq_example}
Notice the difference between the value of {\tt x'} and {\tt x''}
inside section {\tt s1} and outside.

\begin{ErrMsgs}
\item \errindex{This is not the last opened section}
\end{ErrMsgs}

\begin{Remarks}
\item Most commands, like {\tt Hint}, {\tt Notation}, option management, ...
which appear inside a section are canceled when the
section is closed.
% see Section~\ref{LongNames}
%\item Usually all identifiers must be distinct. 
%However, a name already used in a closed section (see \ref{Section})
%can be reused. In this case, the old name is no longer accessible.

% Obsol�te
%\item A module implicitly open a section. Be careful not to name a
%module with an identifier already used in the module (see \ref{compiled}).
\end{Remarks}

\input{RefMan-mod.v}

\section{Libraries and qualified names}

\subsection{Names of libraries and files
\label{Libraries}
\index{Libraries}
\index{Physical paths}
\index{Logical paths}}

\paragraph{Libraries}

The theories developed in {\Coq} are stored in {\em library files}
which are hierarchically classified into {\em libraries} and {\em
sublibraries}. To express this hierarchy, library names are
represented by qualified identifiers {\qualid}, i.e. as list of
identifiers separated by dots (see Section~\ref{qualid}). For
instance, the library file {\tt Mult} of the standard {\Coq} library
{\tt Arith} has name {\tt Coq.Arith.Mult}. The identifier
that starts the name of a library is called a {\em library root}.
All library files of the standard library of {\Coq} have reserved root
{\tt Coq} but library file names based on other roots can be obtained
by using {\tt coqc} options {\tt -I} or {\tt -R} (see
Section~\ref{coqoptions}). Also, when an interactive {\Coq} session
starts, a library of root {\tt Top} is started, unless option {\tt
-top} or {\tt -notop} is set (see Section~\ref{coqoptions}).

As library files are stored on the file system of the underlying
operating system, a translation from file-system names to {\Coq} names
is needed. In this translation, names in the file system are called
{\em physical} paths while {\Coq} names are contrastingly called {\em
logical} names. Logical names are mapped to physical paths using the
commands {\tt Add LoadPath} or {\tt Add Rec LoadPath} (see
Sections~\ref{AddLoadPath} and~\ref{AddRecLoadPath}).

\subsection{Qualified names
\label{LongNames}
\index{Qualified identifiers}
\index{Absolute names}}

Library files are modules which possibly contain submodules which
eventually contain constructions (axioms, parameters, definitions,
lemmas, theorems, remarks or facts). The {\em absolute name}, or {\em
full name}, of a construction in some library file is a qualified
identifier starting with the logical name of the library file,
followed by the sequence of submodules names encapsulating the
construction and ended by the proper name of the construction.
Typically, the absolute name {\tt Coq.Init.Logic.eq} denotes Leibniz'
equality defined in the module {\tt Logic} in the sublibrary {\tt
Init} of the standard library of \Coq.

The proper name that ends the name of a construction is the {\it short
name} (or sometimes {\it base name}) of the construction (for
instance, the short name of {\tt Coq.Init.Logic.eq} is {\tt eq}). Any
partial suffix of the absolute name is a {\em partially qualified name}
(e.g. {\tt Logic.eq} is a partially qualified name for {\tt
Coq.Init.Logic.eq}).  Especially, the short name of a construction is
its shortest partially qualified name.

{\Coq} does not accept two constructions (definition, theorem, ...)
with the same absolute name but different constructions can have the
same short name (or even same partially qualified names as soon as the
full names are different).

Notice that the notion of absolute, partially qualified and
short names also applies to library file names.

\paragraph{Visibility}

{\Coq} maintains a table called {\it name table} which maps partially
qualified names of constructions to absolute names. This table is
updated by the commands {\tt Require} (see \ref{Require}), {\tt
Import} and {\tt Export} (see \ref{Import}) and also each time a new
declaration is added to the context. An absolute name is called {\it
visible} from a given short or partially qualified name when this
latter name is enough to denote it. This means that the short or
partially qualified name is mapped to the absolute name in {\Coq} name
table. Definitions flagged as {\tt Local} are only accessible with their
fully qualified name (see \ref{Definition}).

A similar table exists for library file names. It is updated by the
vernacular commands {\tt Add LoadPath} and {\tt Add Rec LoadPath} (or
their equivalent as options of the {\Coq} executables, {\tt -I} and
{\tt -R}).

It may happen that a visible name is hidden by the short name or a
qualified name of another construction. In this case, the name that
has been hidden must be referred to using one more level of
qualification. To ensure that a construction always remains
accessible, absolute names can never be hidden.

Examples:
\begin{coq_eval}
Reset Initial.
\end{coq_eval}
\begin{coq_example}
Check 0.
Definition nat := bool.
Check 0.
Check Datatypes.nat.
Locate nat.
\end{coq_example}

\SeeAlso Command {\tt Locate} in Section~\ref{Locate} and {\tt Locate
Library} in Section~\ref{Locate Library}.

%% \paragraph{The special case of remarks and facts}
%% 
%% In contrast with definitions, lemmas, theorems, axioms and parameters,
%% the absolute name of remarks includes the segment of sections in which
%% it is defined. Concretely, if a remark {\tt R} is defined in
%% subsection {\tt S2} of section {\tt S1} in module {\tt M}, then its
%% absolute name is {\tt M.S1.S2.R}. The same for facts, except that the
%% name of the innermost section is dropped from the full name. Then, if
%% a fact {\tt F} is defined in subsection {\tt S2} of section {\tt S1}
%% in module {\tt M}, then its absolute name is {\tt M.S1.F}.

\section{Implicit arguments
\index{Implicit arguments}
\label{Implicit Arguments}}

An implicit argument of a function is an argument which can be
inferred from contextual knowledge. There are different kinds of
implicit arguments that can be considered implicit in different
ways. There are also various commands to control the setting or the
inference of implicit arguments.

\subsection{The different kinds of implicit arguments}

\subsubsection{Implicit arguments inferable from the knowledge of other 
arguments of a function}

The first kind of implicit arguments covers the arguments that are
inferable from the knowledge of the type of other arguments of the
function, or of the type of the surrounding context of the
application.  Especially, such implicit arguments correspond to 
parameters dependent in the type of the function. Typical implicit
arguments are the type arguments in polymorphic functions.  
There are several kinds of such implicit arguments.

\paragraph{Strict Implicit Arguments.} 
An implicit argument can be either strict or non strict. An implicit
argument is said {\em strict} if, whatever the other arguments of the
function are, it is still inferable from the type of some other
argument. Technically, an implicit argument is strict if it
corresponds to a parameter which is not applied to a variable which
itself is another parameter of the function (since this parameter
may erase its arguments), not in the body of a {\tt match}, and not
itself applied or matched against patterns (since the original
form of the argument can be lost by reduction).

For instance, the first argument of
\begin{quote}
\verb|cons: forall A:Set, A -> list A -> list A|
\end{quote}
in module {\tt List.v} is strict because {\tt list} is an inductive
type and {\tt A} will always be inferable from the type {\tt
list A} of the third argument of {\tt cons}.
On the contrary, the second argument of a term of type 
\begin{quote}
\verb|forall P:nat->Prop, forall n:nat, P n -> ex nat P|
\end{quote}
is implicit but not strict, since it can only be inferred from the
type {\tt P n} of the third argument and if {\tt P} is, e.g., {\tt
fun \_ => True}, it reduces to an expression where {\tt n} does not
occur any longer. The first argument {\tt P} is implicit but not
strict either because it can only be inferred from {\tt P n} and {\tt
P} is not canonically inferable from an arbitrary {\tt n} and the
normal form of {\tt P n} (consider e.g. that {\tt n} is {\tt 0} and
the third argument has type {\tt True}, then any {\tt P} of the form
{\tt fun n => match n with 0 => True | \_ => \mbox{\em anything} end} would
be a solution of the inference problem).

\paragraph{Contextual Implicit Arguments.} 
An implicit argument can be {\em contextual} or not. An implicit
argument is said {\em contextual} if it can be inferred only from the
knowledge of the type of the context of the current expression. For
instance, the only argument of
\begin{quote}
\verb|nil : forall A:Set, list A|
\end{quote}
is contextual. Similarly, both arguments of a term of type
\begin{quote}
\verb|forall P:nat->Prop, forall n:nat, P n \/ n = 0|
\end{quote}
are contextual (moreover, {\tt n} is strict and {\tt P} is not).

\paragraph{Reversible-Pattern Implicit Arguments.}
There is another class of implicit arguments that can be reinferred
unambiguously if all the types of the remaining arguments are
known. This is the class of implicit arguments occurring in the type
of another argument in position of reversible pattern, which means it
is at the head of an application but applied only to uninstantiated
distinct variables. Such an implicit argument is called {\em
reversible-pattern implicit argument}. A typical example is the
argument {\tt P} of {\tt nat\_rec} in
\begin{quote}
{\tt nat\_rec : forall P : nat -> Set,
       P 0 -> (forall n : nat, P n -> P (S n)) -> forall x : nat, P x}.
\end{quote}
({\tt P} is reinferable by abstracting over {\tt n} in the type {\tt P n}).

See Section~\ref{SetReversiblePatternImplicit} for the automatic declaration
of reversible-pattern implicit arguments.

\subsubsection{Implicit arguments inferable by resolution}

This corresponds to a class of non dependent implicit arguments that
are solved based on the structure of their type only.

\subsection{Maximal or non maximal insertion of implicit arguments}

In case a function is partially applied, and the next argument to be
applied is an implicit argument, two disciplines are applicable. In the
first case, the function is considered to have no arguments furtherly:
one says that the implicit argument is not maximally inserted. In
the second case, the function is considered to be implicitly applied
to the implicit arguments it is waiting for: one says that the
implicit argument is maximally inserted.

Each implicit argument can be declared to have to be inserted
maximally or non maximally. This can be governed argument per argument
by the command {\tt Implicit Arguments} (see~\ref{ImplicitArguments})
or globally by the command {\tt Set Maximal Implicit Insertion}
(see~\ref{SetMaximalImplicitInsertion}). See also
Section~\ref{PrintImplicit}.

\subsection{Casual use of implicit arguments}

In a given expression, if it is clear that some argument of a function
can be inferred from the type of the other arguments, the user can
force the given argument to be guessed by replacing it by ``{\tt \_}''. If
possible, the correct argument will be automatically generated.

\begin{ErrMsgs}

\item \errindex{Cannot infer a term for this placeholder}

  {\Coq} was not able to deduce an instantiation of a ``{\tt \_}''.

\end{ErrMsgs}

\subsection{Declaration of implicit arguments for a constant
\comindex{Arguments}}
\label{ImplicitArguments}

In case one wants that some arguments of a given object (constant,
inductive types, constructors, assumptions, local or not) are always
inferred by Coq, one may declare once and for all which are the expected
implicit arguments of this object. There are two ways to do this,
a priori and a posteriori.

\subsubsection{Implicit Argument Binders}

In the first setting, one wants to explicitly give the implicit
arguments of a constant as part of its definition. To do this, one has
to surround the bindings of implicit arguments by curly braces:
\begin{coq_eval}
Reset Initial.
\end{coq_eval}
\begin{coq_example}
Definition id {A : Type} (x : A) : A := x.
\end{coq_example}

This automatically declares the argument {\tt A} of {\tt id} as a
maximally inserted implicit argument. One can then do as-if the argument
was absent in every situation but still be able to specify it if needed:
\begin{coq_example}
Definition compose {A B C} (g : B -> C) (f : A -> B) := 
  fun x => g (f x).
Goal forall A, compose id id = id (A:=A).
\end{coq_example}

The syntax is supported in all top-level definitions: {\tt Definition},
{\tt Fixpoint}, {\tt Lemma} and so on. For (co-)inductive datatype
declarations, the semantics are the following: an inductive parameter
declared as an implicit argument need not be repeated in the inductive
definition but will become implicit for the constructors of the
inductive only, not the inductive type itself. For example:

\begin{coq_example}
Inductive list {A : Type} : Type :=
| nil : list
| cons : A -> list -> list.
Print list.
\end{coq_example}

One can always specify the parameter if it is not uniform using the
usual implicit arguments disambiguation syntax.

\subsubsection{Declaring Implicit Arguments}

To set implicit arguments for a constant a posteriori, one can use the
command:
\begin{quote}
\tt Arguments {\qualid} \nelist{\possiblybracketedident}{}
\end{quote}
where the list of {\possiblybracketedident} is the list of all arguments of
{\qualid} where the ones to be declared implicit are surrounded by
square brackets and the ones to be declared as maximally inserted implicits
are surrounded by curly braces.

After the above declaration is issued, implicit arguments can just (and
have to) be skipped in any expression involving an application of
{\qualid}.

\begin{Variants}
\item {\tt Global Arguments {\qualid} \nelist{\possiblybracketedident}{}
\comindex{Global Arguments}}

Tell to recompute the implicit arguments of {\qualid} after ending of
the current section if any, enforcing the implicit arguments known
from inside the section to be the ones declared by the command.

\item {\tt Local Arguments {\qualid} \nelist{\possiblybracketedident}{}
\comindex{Local Arguments}}

When in a module, tell not to activate the implicit arguments of
{\qualid} declared by this command to contexts that require the
module.

\item {\tt \zeroone{Global {\sl |} Local} Arguments {\qualid} \sequence{\nelist{\possiblybracketedident}{}}{,}}

For names of constants, inductive types, constructors, lemmas which
can only be applied to a fixed number of arguments (this excludes for
instance constants whose type is polymorphic), multiple 
implicit arguments decflarations can be given. 
Depending on the number of arguments {\qualid} is applied
to in practice, the longest applicable list of implicit arguments is
used to select which implicit arguments are inserted.

For printing, the omitted arguments are the ones of the longest list
of implicit arguments of the sequence.

\end{Variants}

\Example
\begin{coq_eval}
Reset Initial.
\end{coq_eval}
\begin{coq_example*}
Inductive list (A:Type) : Type :=
 | nil : list A 
 | cons : A -> list A -> list A.
\end{coq_example*}
\begin{coq_example}
Check (cons nat 3 (nil nat)).
Arguments cons [A] _ _.
Arguments nil [A].
Check (cons 3 nil).
Fixpoint map (A B:Type) (f:A->B) (l:list A) : list B :=
  match l with nil => nil | cons a t => cons (f a) (map A B f t) end.
Fixpoint length (A:Type) (l:list A) : nat :=
  match l with nil => 0 | cons _ m => S (length A m) end.
Arguments map [A B] f l.
Arguments length {A} l. (* A has to be maximally inserted *)
Check (fun l:list (list nat) => map length l).
Arguments map [A B] f l, [A] B f l, A B f l.
Check (fun l => map length l = map (list nat) nat length l).
\end{coq_example}

\Rem To know which are the implicit arguments of an object, use the command
{\tt Print Implicit} (see \ref{PrintImplicit}).

\Rem If the list of arguments is empty, the command removes the
implicit arguments of {\qualid}.

\subsection{Automatic declaration of implicit arguments for a constant}

{\Coq} can also automatically detect what are the implicit arguments
of a defined object. The command is just
\begin{quote}
{\tt Arguments {\qualid} : default implicits
\comindex{Arguments}}
\end{quote}
The auto-detection is governed by options telling if strict,
contextual, or reversible-pattern implicit arguments must be
considered or not (see
Sections~\ref{SetStrictImplicit},~\ref{SetContextualImplicit},~\ref{SetReversiblePatternImplicit}
and also~\ref{SetMaximalImplicitInsertion}).

\begin{Variants}
\item {\tt Global Arguments {\qualid} : default implicits
\comindex{Global Arguments}}

Tell to recompute the implicit arguments of {\qualid} after ending of
the current section if any.

\item {\tt Local Arguments {\qualid} : default implicits
\comindex{Local Arguments}}

When in a module, tell not to activate the implicit arguments of
{\qualid} computed by this declaration to contexts that requires the
module.

\end{Variants}

\Example
\begin{coq_eval}
Reset Initial.
\end{coq_eval}
\begin{coq_example*}
Inductive list (A:Set) : Set := 
  | nil : list A 
  | cons : A -> list A -> list A.
\end{coq_example*}
\begin{coq_example}
Arguments cons : default implicits.
Print Implicit cons.
Arguments nil : default implicits.
Print Implicit nil.
Set Contextual Implicit.
Arguments nil : default implicits.
Print Implicit nil.
\end{coq_example}

The computation of implicit arguments takes account of the
unfolding of constants.  For instance, the variable {\tt p} below has
type {\tt (Transitivity R)} which is reducible to {\tt forall x,y:U, R x
y -> forall z:U, R y z -> R x z}. As the variables {\tt x}, {\tt y} and
{\tt z} appear strictly in body of the type, they are implicit.

\begin{coq_example*}
Variable X : Type.
Definition Relation := X -> X -> Prop.
Definition Transitivity (R:Relation) :=
  forall x y:X, R x y -> forall z:X, R y z -> R x z.
Variables (R : Relation) (p : Transitivity R).
Arguments p : default implicits.
\end{coq_example*}
\begin{coq_example}
Print p.
Print Implicit p.
\end{coq_example}
\begin{coq_example*}
Variables (a b c : X) (r1 : R a b) (r2 : R b c).
\end{coq_example*}
\begin{coq_example}
Check (p r1 r2).
\end{coq_example}

Implicit arguments can be cleared with the following syntax:

\begin{quote}
{\tt Arguments {\qualid} : clear implicits
\comindex{Arguments}}
\end{quote}

In the following example implict arguments declarations for the {\tt nil}
constant are cleared:
\begin{coq_example}
Arguments cons : clear implicits.
Print Implicit cons.
\end{coq_example}


\subsection{Mode for automatic declaration of implicit arguments
\label{Auto-implicit}
\comindex{Set Implicit Arguments}
\comindex{Unset Implicit Arguments}}

In case one wants to systematically declare implicit the arguments
detectable as such, one may switch to the automatic declaration of
implicit arguments mode by using the command
\begin{quote}
\tt Set Implicit Arguments.
\end{quote}
Conversely, one may unset the mode by using {\tt Unset Implicit
Arguments}.  The mode is off by default. Auto-detection of implicit
arguments is governed by options controlling whether strict and
contextual implicit arguments have to be considered or not.

\subsection{Controlling strict implicit arguments
\comindex{Set Strict Implicit}
\comindex{Unset Strict Implicit}
\label{SetStrictImplicit}}

When the mode for automatic declaration of implicit arguments is on,
the default is to automatically set implicit only the strict implicit
arguments plus, for historical reasons, a small subset of the non
strict implicit arguments. To relax this constraint and to
set implicit all non strict implicit arguments by default, use the command
\begin{quote}
\tt Unset Strict Implicit.
\end{quote}
Conversely, use the command {\tt Set Strict Implicit} to
restore the original mode that declares implicit only the strict implicit arguments plus a small subset of the non strict implicit arguments.

In the other way round, to capture exactly the strict implicit arguments and no more than the strict implicit arguments, use the command:
\comindex{Set Strongly Strict Implicit}
\comindex{Unset Strongly Strict Implicit}
\begin{quote}
\tt Set Strongly Strict Implicit.
\end{quote}
Conversely, use the command {\tt Unset Strongly Strict Implicit} to
let the option ``{\tt Strict Implicit}'' decide what to do.

\Rem In versions of {\Coq} prior to version 8.0, the default was to
declare the strict implicit arguments as implicit.

\subsection{Controlling contextual implicit arguments
\comindex{Set Contextual Implicit}
\comindex{Unset Contextual Implicit}
\label{SetContextualImplicit}}

By default, {\Coq} does not automatically set implicit the contextual
implicit arguments. To tell {\Coq} to infer also contextual implicit
argument, use command  
\begin{quote}
\tt Set Contextual Implicit. 
\end{quote}
Conversely, use command {\tt Unset Contextual Implicit} to
unset the contextual implicit mode.

\subsection{Controlling reversible-pattern implicit arguments
\comindex{Set Reversible Pattern Implicit}
\comindex{Unset Reversible Pattern Implicit}
\label{SetReversiblePatternImplicit}}

By default, {\Coq} does not automatically set implicit the reversible-pattern
implicit arguments. To tell {\Coq} to infer also reversible-pattern implicit
argument, use command  
\begin{quote}
\tt Set Reversible Pattern Implicit. 
\end{quote}
Conversely, use command {\tt Unset Reversible Pattern Implicit} to
unset the reversible-pattern implicit mode.

\subsection{Controlling the insertion of implicit arguments not followed by explicit arguments
\comindex{Set Maximal Implicit Insertion}
\comindex{Unset Maximal Implicit Insertion}
\label{SetMaximalImplicitInsertion}}

Implicit arguments can be declared to be automatically inserted when a
function is partially applied and the next argument of the function is
an implicit one. In case the implicit arguments are automatically
declared (with the command {\tt Set Implicit Arguments}), the command
\begin{quote}
\tt Set Maximal Implicit Insertion. 
\end{quote}
is used to tell to declare the implicit arguments with a maximal
insertion status. By default, automatically declared implicit
arguments are not declared to be insertable maximally.  To restore the
default mode for maximal insertion, use command {\tt Unset Maximal
Implicit Insertion}.

\subsection{Explicit applications
\index{Explicitly given implicit arguments}
\label{Implicits-explicitation}
\index{qualid@{\qualid}} \index{\symbol{64}}}

In presence of non strict or contextual argument, or in presence of
partial applications, the synthesis of implicit arguments may fail, so
one may have to give explicitly certain implicit arguments of an
application. The syntax for this is {\tt (\ident:=\term)} where {\ident}
is the name of the implicit argument and {\term} is its corresponding
explicit term. Alternatively, one can locally deactivate the hiding of
implicit arguments of a function by using the notation
{\tt @{\qualid}~{\term}$_1$..{\term}$_n$}. This syntax extension is
given Figure~\ref{fig:explicitations}.
\begin{figure}
\begin{centerframe}
\begin{tabular}{lcl}
{\term} & ++= & @ {\qualid} \nelist{\term}{}\\
& $|$ & @ {\qualid}\\
& $|$ & {\qualid} \nelist{\textrm{\textsl{argument}}}{}\\
\\
{\textrm{\textsl{argument}}} & ::= & {\term} \\
& $|$ & {\tt ({\ident}:={\term})}\\
\end{tabular}
\end{centerframe}
\caption{Syntax for explicitly giving implicit arguments}
\label{fig:explicitations}
\end{figure}

\noindent {\bf Example (continued): }
\begin{coq_example}
Check (p r1 (z:=c)).
Check (p (x:=a) (y:=b) r1 (z:=c) r2).
\end{coq_example}

\subsection{Renaming implicit arguments
\comindex{Arguments}
}

Implicit arguments names can be redefined using the following syntax:
\begin{quote}
{\tt Arguments {\qualid} \nelist{\name}{}  : rename}
\end{quote}

Without the {\tt rename} flag, {\tt Arguments} can be used to assert
that a given constant has the expected number of arguments and that
these arguments are named as expected.

\noindent {\bf Example (continued): }
\begin{coq_example}
Arguments p [s t] _ [u] _: rename.
Check (p r1 (u:=c)).
Check (p (s:=a) (t:=b) r1 (u:=c) r2).
Fail Arguments p [s t] _ [w] _.
\end{coq_example}


\subsection{Displaying what the implicit arguments are
\comindex{Print Implicit}
\label{PrintImplicit}}

To display the implicit arguments associated to an object, and to know
if each of them is to be used maximally or not, use the command
\begin{quote}
\tt Print Implicit {\qualid}.
\end{quote}

\subsection{Explicit displaying of implicit arguments for pretty-printing
\comindex{Set Printing Implicit}
\comindex{Unset Printing Implicit}
\comindex{Set Printing Implicit Defensive}
\comindex{Unset Printing Implicit Defensive}}

By default the basic pretty-printing rules hide the inferable implicit
arguments of an application. To force printing all implicit arguments,
use command
\begin{quote}
{\tt Set Printing Implicit.}
\end{quote}
Conversely, to restore the hiding of implicit arguments, use command
\begin{quote}
{\tt Unset Printing Implicit.}
\end{quote}

By default the basic pretty-printing rules display the implicit arguments that are not detected as strict implicit arguments. This ``defensive'' mode can quickly make the display cumbersome so this can be deactivated by using the command
\begin{quote}
{\tt Unset Printing Implicit Defensive.}
\end{quote}
Conversely, to force the display of non strict arguments, use command
\begin{quote}
{\tt Set Printing Implicit Defensive.}
\end{quote}

\SeeAlso {\tt Set Printing All} in Section~\ref{SetPrintingAll}.

\subsection{Interaction with subtyping}

When an implicit argument can be inferred from the type of more than
one of the other arguments, then only the type of the first of these
arguments is taken into account, and not an upper type of all of
them.  As a consequence, the inference of the implicit argument of
``='' fails in

\begin{coq_example*}
Check nat = Prop.
\end{coq_example*}

but succeeds in 

\begin{coq_example*}
Check Prop = nat.
\end{coq_example*}

\subsection{Deactivation of implicit arguments for parsing}
\comindex{Set Parsing Explicit}
\comindex{Unset Parsing Explicit}

Use of implicit arguments can be deactivated by issuing the command:
\begin{quote}
{\tt Set Parsing Explicit.}
\end{quote}

In this case, all arguments of constants, inductive types,
constructors, etc, including the arguments declared as implicit, have
to be given as if none arguments were implicit. By symmetry, this also
affects printing. To restore parsing and normal printing of implicit
arguments, use:
\begin{quote}
{\tt Set Parsing Explicit.}
\end{quote}

\subsection{Canonical structures
\comindex{Canonical Structure}}

A canonical structure is an instance of a record/structure type that
can be used to solve unification problems involving a projection
applied to an unknown structure instance (an implicit argument) and
a value.  The complete documentation of canonical structures can be found
in Chapter~\ref{CS-full}, here only a simple example is given.

Assume that {\qualid} denotes an object $(Build\_struc~ c_1~ \ldots~ c_n)$ in
the
structure {\em struct} of which the fields are $x_1$, ...,
$x_n$. Assume that {\qualid} is declared as a canonical structure
using the command
\begin{quote}
{\tt Canonical Structure {\qualid}.}
\end{quote}
Then, each time an equation of the form $(x_i~
\_)=_{\beta\delta\iota\zeta}c_i$ has to be solved during the
type-checking process, {\qualid} is used as a solution. Otherwise
said, {\qualid} is canonically used to extend the field $c_i$ into a
complete structure built on $c_i$.

Canonical structures are particularly useful when mixed with
coercions and strict implicit arguments. Here is an example.
\begin{coq_example*}
Require Import Relations.
Require Import EqNat.
Set Implicit Arguments.
Unset Strict Implicit.
Structure Setoid : Type := 
  {Carrier :> Set;
   Equal : relation Carrier;
   Prf_equiv : equivalence Carrier Equal}.
Definition is_law (A B:Setoid) (f:A -> B) :=
  forall x y:A, Equal x y -> Equal (f x) (f y).
Axiom eq_nat_equiv : equivalence nat eq_nat.
Definition nat_setoid : Setoid := Build_Setoid eq_nat_equiv.
Canonical Structure nat_setoid.
\end{coq_example*}

Thanks to \texttt{nat\_setoid} declared as canonical, the implicit
arguments {\tt A} and {\tt B} can be synthesized in the next statement.
\begin{coq_example}
Lemma is_law_S : is_law S.
\end{coq_example}

\Rem If a same field occurs in several canonical structure, then
only the structure declared first as canonical is considered.

\begin{Variants}
\item {\tt Canonical Structure {\ident} := {\term} : {\type}.}\\
 {\tt Canonical Structure {\ident} := {\term}.}\\
 {\tt Canonical Structure {\ident} : {\type} := {\term}.}

These are equivalent to a regular definition of {\ident} followed by
the declaration 

{\tt Canonical Structure {\ident}}.
\end{Variants}

\SeeAlso more examples in user contribution \texttt{category}
(\texttt{Rocq/ALGEBRA}).

\subsubsection{Print Canonical Projections.
\comindex{Print Canonical Projections}}

This displays the list of global names that are components of some
canonical structure. For each of them, the canonical structure of
which it is a projection is indicated. For instance, the above example 
gives the following output:

\begin{coq_example}
Print Canonical Projections.
\end{coq_example}

\subsection{Implicit types of variables}
\comindex{Implicit Types}

It is possible to bind variable names to a given type (e.g. in a
development using arithmetic, it may be convenient to bind the names
{\tt n} or {\tt m} to the type {\tt nat} of natural numbers). The
command for that is
\begin{quote}
\tt Implicit Types \nelist{\ident}{} : {\type}
\end{quote}
The effect of the command is to automatically set the type of bound
variables starting with {\ident} (either {\ident} itself or
{\ident} followed by one or more single quotes, underscore or digits)
to be {\type} (unless the bound variable is already declared with an
explicit type in which case, this latter type is considered).

\Example
\begin{coq_example}
Require Import List.
Implicit Types m n : nat.
Lemma cons_inj_nat : forall m n l, n :: l = m :: l -> n = m.
intros m n.
Lemma cons_inj_bool : forall (m n:bool) l, n :: l = m :: l -> n = m.
\end{coq_example}

\begin{Variants}
\item {\tt Implicit Type {\ident} : {\type}}\\
This is useful for declaring the implicit type of a single variable.
\item
 {\tt Implicit Types\,%
(\,{\ident$_{1,1}$}\ldots{\ident$_{1,k_1}$}\,{\tt :}\,{\term$_1$} {\tt )}\,%
\ldots\,{\tt (}\,{\ident$_{n,1}$}\ldots{\ident$_{n,k_n}$}\,{\tt :}\,%
{\term$_n$} {\tt )}.}\\ 
  Adds $n$ blocks of implicit types with different specifications.
\end{Variants}


\subsection{Implicit generalization
\label{implicit-generalization}
\comindex{Generalizable Variables}}

Implicit generalization is an automatic elaboration of a statement with
free variables into a closed statement where these variables are
quantified explicitly. Implicit generalization is done inside binders
starting with a \verb|`| and terms delimited by \verb|`{ }| and
\verb|`( )|, always introducing maximally inserted implicit arguments for
the generalized variables. Inside implicit generalization
delimiters, free variables in the current context are automatically
quantified using a product or a lambda abstraction to generate a closed
term. In the following statement for example, the variables \texttt{n}
and \texttt{m} are autamatically generalized and become explicit
arguments of the lemma as we are using \verb|`( )|:

\begin{coq_example}
Generalizable All Variables.
Lemma nat_comm : `(n = n + 0).
\end{coq_example}
\begin{coq_eval}
Abort.
\end{coq_eval}
One can control the set of generalizable identifiers with the
\texttt{Generalizable} vernacular command to avoid unexpected
generalizations when mistyping identifiers. There are three variants of
the command:

\begin{quote}
{\tt Generalizable (All|No) Variable(s)? ({\ident$_1$ \ident$_n$})?.}
\end{quote}

\begin{Variants}
\item {\tt Generalizable All Variables.} All variables are candidate for 
  generalization if they appear free in the context under a
  generalization delimiter. This may result in confusing errors in
  case of typos. In such cases, the context will probably contain some
  unexpected generalized variable.

\item {\tt Generalizable No Variables.} Disable implicit generalization 
  entirely. This is the default behavior.

\item {\tt Generalizable Variable(s)? {\ident$_1$ \ident$_n$}.} 
  Allow generalization of the given identifiers only. Calling this
  command multiple times adds to the allowed identifiers.

\item {\tt Global Generalizable} Allows to export the choice of
  generalizable variables.
\end{Variants}

One can also use implicit generalization for binders, in which case the
generalized variables are added as binders and set maximally implicit.
\begin{coq_example*}
Definition id `(x : A) : A := x.
\end{coq_example*}
\begin{coq_example}
Print id.
\end{coq_example}

The generalizing binders \verb|`{ }| and \verb|`( )| work similarly to
their explicit counterparts, only binding the generalized variables
implicitly, as maximally-inserted arguments. In these binders, the
binding name for the bound object is optional, whereas the type is
mandatory, dually to regular binders.

\section{Coercions
\label{Coercions}
\index{Coercions}}

Coercions can be used to implicitly inject terms from one {\em class} in
which they reside into another one. A {\em class} is either a sort
(denoted by the keyword {\tt Sortclass}), a product type (denoted by the
keyword {\tt Funclass}), or a type constructor (denoted by its name),
e.g. an inductive type or any constant with a type of the form
\texttt{forall} $(x_1:A_1) .. (x_n:A_n),~s$ where $s$ is a sort.

Then the user is able to apply an
object that is not a function, but can be coerced to a function, and
more generally to consider that a term of type A is of type B provided
that there is a declared coercion between A and B. The main command is
\comindex{Coercion}
\begin{quote}
\tt Coercion {\qualid} : {\class$_1$} >-> {\class$_2$}.
\end{quote}
which declares the construction denoted by {\qualid} as a
coercion between {\class$_1$} and {\class$_2$}.

More details and examples, and a description of the commands related
to coercions are provided in Chapter~\ref{Coercions-full}.

\section[Printing constructions in full]{Printing constructions in full\label{SetPrintingAll}
\comindex{Set Printing All}
\comindex{Unset Printing All}}

Coercions, implicit arguments, the type of pattern-matching, but also
notations (see Chapter~\ref{Addoc-syntax}) can obfuscate the behavior
of some tactics (typically the tactics applying to occurrences of
subterms are sensitive to the implicit arguments). The command
\begin{quote}
{\tt Set Printing All.}
\end{quote}
deactivates all high-level printing features such as coercions,
implicit arguments, returned type of pattern-matching, notations and
various syntactic sugar for pattern-matching or record projections.
Otherwise said, {\tt Set Printing All} includes the effects
of the commands {\tt Set Printing Implicit}, {\tt Set Printing
Coercions}, {\tt Set Printing Synth}, {\tt Unset Printing Projections}
and {\tt Unset Printing Notations}.  To reactivate the high-level
printing features, use the command
\begin{quote}
{\tt Unset Printing All.}
\end{quote}

\section[Printing universes]{Printing universes\label{PrintingUniverses}
\comindex{Set Printing Universes}
\comindex{Unset Printing Universes}}

The following command:
\begin{quote}
{\tt Set Printing Universes}
\end{quote}
activates the display of the actual level of each occurrence of
{\Type}. See Section~\ref{Sorts} for details.  This wizard option, in
combination with \texttt{Set Printing All} (see
section~\ref{SetPrintingAll}) can help to diagnose failures to unify
terms apparently identical but internally different in the Calculus of
Inductive Constructions. To reactivate the display of the actual level
of the occurrences of {\Type}, use
\begin{quote}
{\tt Unset Printing Universes.}
\end{quote}

\comindex{Print Universes}
\comindex{Print Sorted Universes}

The constraints on the internal level of the occurrences of {\Type}
(see Section~\ref{Sorts}) can be printed using the command
\begin{quote}
{\tt Print \zeroone{Sorted} Universes.}
\end{quote}
If the optional {\tt Sorted} option is given, each universe will be
made equivalent to a numbered label reflecting its level (with a
linear ordering) in the universe hierarchy.

This command also accepts an optional output filename:
\begin{quote}
\tt Print \zeroone{Sorted} Universes {\str}.
\end{quote}
If {\str} ends in \texttt{.dot} or \texttt{.gv}, the constraints are
printed in the DOT language, and can be processed by Graphviz
tools. The format is unspecified if {\str} doesn't end in
\texttt{.dot} or \texttt{.gv}.

\section[Existential variables]{Existential variables\label{ExistentialVariables}}

Coq terms can include existential variables. An existential variable
is a placeholder intended to eventually be replaced by an actual
subterm though which subterm it will be replaced by is still unknown.

Existential variables are generated in place of unsolvable implicit
arguments when using commands such as \texttt{Check} (see
Section~\ref{Check}) or in place of unsolvable instances when using
tactics such as \texttt{eapply} (see Section~\ref{eapply}). They can
only appear as the result of a command displaying a term and they are
represented by ``?'' followed by a number. They cannot be entered by
the user (though they can be generated from ``\_'' when the
corresponding implicit argument is unsolvable).

A given existential variable name can occur several times in a term
meaning the corresponding expected instance is shared. Each
existential variable is relative to a context, as shown by {\tt Show
  Existential} when in the process of proving a goal (see
Section~\ref{ShowExistentials}). Henceforth, each occurrence of an
existential variable in a term is subject to an instance of the
variables of its context of definition which is specific to this
occurrence.

\subsection{Explicit displaying of existential instances for pretty-printing
\comindex{Set Printing Existential Instances}
\comindex{Unset Printing Existential Instances}}

The command:
\begin{quote}
{\tt Set Printing Existential Instances}
\end{quote}
activates the display of how the context of an existential variable is
instantiated on each of its occurrences.

To deactivate the display of the instances of existential variables, use
\begin{quote}
{\tt Unset Printing Existential Instances.}
\end{quote}

\subsection{Solving existential variables using tactics}
\index{\tt \textdollar( \ldots )\textdollar}

\def\expr{\textrm{\textsl{tacexpr}}}

Instead of letting the unification engine try to solve an existential variable
by itself, one can also provide an explicit hole together with a tactic to solve
it. Using the syntax {\tt \textdollar(\expr)\textdollar}, the user can put a
tactic anywhere a term is expected. The order of resolution is not specified and
is implementation-dependent. The inner tactic may use any variable defined in
its scope, including repeated alternations between variables introduced by term
binding as well as those introduced by tactic binding. The expression {\expr}
can be any tactic expression as described at section~\ref{TacticLanguage}.

\begin{coq_example*}
Definition foo (x : A) : A := $( exact x )$.
\end{coq_example*}

This construction is useful when one wants to define complicated terms using
highly automated tactics without resorting to writing the proof-term by means of
the interactive proof engine.

This mechanism is comparable to the {\tt Declare Implicit Tactic} command
defined at~\ref{DeclareImplicit}, except that the used tactic is local to each
hole instead of being declared globally.

%%% Local Variables: 
%%% mode: latex
%%% TeX-master: "Reference-Manual"
%%% End: 
% Gallina extensions
\include{RefMan-lib.v}% The coq library
\include{RefMan-cic.v}% The Calculus of Constructions
\include{RefMan-modr}% The module system


\part{The proof engine}
\chapter[Vernacular commands]{Vernacular commands\label{Vernacular-commands}
\label{Other-commands}}

\section{Displaying}

\subsection[\tt Print {\qualid}.]{\tt Print {\qualid}.\comindex{Print}}
This command displays on the screen informations about the declared or
defined object referred by {\qualid}.

\begin{ErrMsgs}
\item {\qualid} \errindex{not a defined object}
\end{ErrMsgs}

\begin{Variants}
\item {\tt Print Term {\qualid}.}
\comindex{Print Term}\\ 
This is a synonym to {\tt Print {\qualid}} when {\qualid} denotes a
global constant. 

\item {\tt About {\qualid}.}
\label{About}
\comindex{About}\\ 
This displays various informations about the object denoted by {\qualid}:
its kind (module, constant, assumption, inductive,
constructor, abbreviation, \ldots), long name, type, implicit
arguments and argument scopes. It does not print the body of
definitions or proofs.

%\item {\tt Print Proof {\qualid}.}\comindex{Print Proof}\\
%In case \qualid\ denotes an opaque theorem defined in a section,
%it is stored on a special unprintable form and displayed as 
%{\tt <recipe>}. {\tt Print Proof} forces the printable form of \qualid\
%to be computed and displays it.
\end{Variants}

\subsection[\tt Print All.]{\tt Print All.\comindex{Print All}}
This command displays informations about the current state of the
environment, including sections and modules.

\begin{Variants}
\item {\tt Inspect \num.}\comindex{Inspect}\\
This command displays the {\num} last objects of the current
environment, including sections and modules.
\item {\tt Print Section {\ident}.}\comindex{Print Section}\\
should correspond to a currently open section, this command
displays the objects defined since the beginning of this section.
% Discontinued
%% \item {\tt Print.}\comindex{Print}\\
%% This command displays the axioms and variables declarations in the
%% environment as well as the constants defined since the last variable
%% was introduced.
\end{Variants}

\section{Flags, Options and Tables}

{\Coq} configurability is based on flags (e.g. {\tt Set Printing All} in
Section~\ref{SetPrintingAll}), options (e.g. {\tt Set Printing Width
  {\integer}} in Section~\ref{SetPrintingWidth}), or tables (e.g. {\tt
  Add Printing Record {\ident}}, in Section~\ref{AddPrintingLet}). The
names of flags, options and tables are made of non-empty sequences of
identifiers (conventionally with capital initial letter). The general
commands handling flags, options and tables are given below.

\subsection[\tt Set {\rm\sl flag}.]{\tt Set {\rm\sl flag}.\comindex{Set}}
This command switches {\rm\sl flag} on. The original state of
{\rm\sl flag} is restored when the current module ends.

\begin{Variants}
\item {\tt Local Set {\rm\sl flag}.}\\
This command switches {\rm\sl flag} on. The original state of
{\rm\sl flag} is restored when the current \emph{section} ends.
\item {\tt Global Set {\rm\sl flag}.}\\
This command switches {\rm\sl flag} on. The original state of
{\rm\sl flag} is \emph{not} restored at the end of the module. Additionally,
if set in a file, {\rm\sl flag} is switched on when the file is
{\tt Require}-d.
\end{Variants}

\subsection[\tt Unset {\rm\sl flag}.]{\tt Unset {\rm\sl flag}.\comindex{Unset}}
This command switches {\rm\sl flag} off. The original state of {\rm\sl flag}
is restored when the current module ends.

\begin{Variants}
\item {\tt Local Unset {\rm\sl flag}.\comindex{Local Unset}}\\
This command switches {\rm\sl flag} off. The original state of {\rm\sl flag}
is restored when the current \emph{section} ends.
\item {\tt Global Unset {\rm\sl flag}.\comindex{Global Unset}}\\
This command switches {\rm\sl flag} off.  The original state of
{\rm\sl flag} is \emph{not} restored at the end of the module. Additionally,
if set in a file, {\rm\sl flag} is switched on when the file is
{\tt Require}-d.
\end{Variants}

\subsection[\tt Test {\rm\sl flag}.]{\tt Test {\rm\sl flag}.\comindex{Test}}
This command prints whether {\rm\sl flag} is on or off.

\subsection[\tt Set {\rm\sl option} {\rm\sl value}.]{\tt Set {\rm\sl option} {\rm\sl value}.\comindex{Set}}
This command sets {\rm\sl option} to {\rm\sl value}. The original value of
{\rm\sl option} is restored when the current module ends.

\begin{Variants}
\item {\tt Local Set {\rm\sl option} {\rm\sl value}.\comindex{Local Set}}
This command sets {\rm\sl option} to {\rm\sl value}. The original value of
\item {\tt Global Set {\rm\sl option} {\rm\sl value}.\comindex{Global Set}}
This command sets {\rm\sl option} to {\rm\sl value}. The original value of
{\rm\sl option} is \emph{not} restored at the end of the module. Additionally,
if set in a file, {\rm\sl option} is set to {\rm\sl value} when the file is
{\tt Require}-d.
\end{Variants}

\subsection[\tt Unset {\rm\sl option}.]{\tt Unset {\rm\sl option}.\comindex{Unset}}
This command resets {\rm\sl option} to its default value.

\begin{Variants}
\item {\tt Local Unset {\rm\sl option}.\comindex{Local Unset}}\\
This command resets {\rm\sl option} to its default value. The original state of {\rm\sl option}
is restored when the current \emph{section} ends.
\item {\tt Global Unset {\rm\sl option}.\comindex{Global Unset}}\\
This command resets {\rm\sl option} to its default value.  The original state of
{\rm\sl option} is \emph{not} restored at the end of the module. Additionally,
if unset in a file, {\rm\sl option} is reset to its default value when the file is
{\tt Require}-d.
\end{Variants}

\subsection[\tt Test {\rm\sl option}.]{\tt Test {\rm\sl option}.\comindex{Test}}
This command prints the current value of {\rm\sl option}.

\subsection{Tables}
The general commands for tables are {\tt Add {\rm\sf table} {\rm\sl
    value}}, {\tt Remove {\rm\sf table} {\rm\sl value}}, {\tt Test
    {\rm\sf table}}, {\tt Test {\rm\sf table} for {\rm\sl value}} and
  {\tt Print Table {\rm\sf table}}.

\subsection[\tt Print Options.]{\tt Print Options.\comindex{Print Options}}
This command lists all available flags, options and tables.

\begin{Variants}
\item {\tt Print Tables}.\comindex{Print Tables}\\
This is a synonymous of {\tt Print Options.}
\end{Variants}

\section{Requests to the environment}

\subsection[\tt Check {\term}.]{\tt Check {\term}.\label{Check}
\comindex{Check}}
This command displays the type of {\term}. When called in proof mode, 
the term is checked in the local context of the current subgoal.

\subsection[\tt Eval {\rm\sl convtactic} in {\term}.]{\tt Eval {\rm\sl convtactic} in {\term}.\comindex{Eval}}

This command performs the specified reduction on {\term}, and displays
the resulting term with its type. The term to be reduced may depend on
hypothesis introduced in the first subgoal (if a proof is in
progress).

\SeeAlso Section~\ref{Conversion-tactics}.

\subsection[\tt Compute {\term}.]{\tt Compute {\term}.\comindex{Compute}}

This command performs a call-by-value evaluation of {\term} by using
the bytecode-based virtual machine. It is a shortcut for
{\tt Eval vm\_compute in {\term}}.

\SeeAlso Section~\ref{Conversion-tactics}.

\subsection[\tt Extraction \term.]{\tt Extraction \term.\label{ExtractionTerm}
\comindex{Extraction}} 
This command displays the extracted term from
{\term}. The extraction is processed according to the distinction
between {\Set} and {\Prop}; that is to say, between logical and
computational content (see Section~\ref{Sorts}). The extracted term is
displayed in Objective Caml syntax, where global identifiers are still
displayed as in \Coq\ terms.

\begin{Variants}
\item \texttt{Recursive Extraction} {\qualid$_1$} \ldots{} {\qualid$_n$}{\tt .}\\
  Recursively extracts all the material needed for the extraction of 
  globals {\qualid$_1$}, \ldots, {\qualid$_n$}.
\end{Variants}

\SeeAlso Chapter~\ref{Extraction}.

\subsection[\tt Print Assumptions {\qualid}.]{\tt Print Assumptions {\qualid}.\comindex{Print Assumptions}}
\label{PrintAssumptions}

This commands display all the assumptions (axioms, parameters and
variables) a theorem or definition depends on.  Especially, it informs
on the assumptions with respect to which the validity of a theorem
relies.

\begin{Variants}
\item \texttt{\tt Print Opaque Dependencies {\qualid}.
  \comindex{Print Opaque Dependencies}}\\
  Displays the set of opaque constants {\qualid} relies on in addition
  to the assumptions.
\item \texttt{\tt Print Transparent Dependencies {\qualid}.
  \comindex{Print Transparent Dependencies}}\\
  Displays the set of transparent constants {\qualid} relies on in addition
  to the assumptions.
\item \texttt{\tt Print All Dependencies {\qualid}.
  \comindex{Print All Dependencies}}\\
  Displays all assumptions and constants {\qualid} relies on.
\end{Variants}

\subsection[\tt Search {\qualid}.]{\tt Search {\qualid}.\comindex{Search}}
This command displays the name and type of all objects (theorems,
axioms, etc) of the current context whose statement contains \qualid.
This command is useful to remind the user of the name of library
lemmas.

\begin{ErrMsgs}
\item \errindex{The reference \qualid\ was not found in the current
environment}\\
    There is no constant in the environment named \qualid.
\end{ErrMsgs}

\newcommand{\termpatternorstr}{{\termpattern}\textrm{\textsl{-}}{\str}}

\begin{Variants}
\item {\tt Search {\str}.}

If {\str} is a valid identifier, this command displays the name and type
of all objects (theorems, axioms, etc) of the current context whose
name contains {\str}. If {\str} is a notation's string denoting some
reference {\qualid} (referred to by its main symbol as in \verb="+"=
or by its notation's string as in \verb="_ + _"= or \verb="_ 'U' _"=, see
Section~\ref{Notation}), the command works like {\tt Search
{\qualid}}.

\item {\tt Search {\str}\%{\delimkey}.}

The string {\str} must be a notation or the main symbol of a notation
which is then interpreted in the scope bound to the delimiting key
{\delimkey} (see Section~\ref{scopechange}).

\item {\tt Search {\termpattern}.}

This searches for all statements or types of definition that contains
a subterm that matches the pattern {\termpattern} (holes of the
pattern are either denoted by ``{\texttt \_}'' or
by ``{\texttt ?{\ident}}'' when non linear patterns are expected).

\item {\tt Search \nelist{\zeroone{-}{\termpatternorstr}}{}.}\\

\noindent where {\termpatternorstr} is a
{\termpattern} or a {\str}, or a {\str} followed by a scope
delimiting key {\tt \%{\delimkey}}.

This generalization of {\tt Search} searches for all objects
whose statement or type contains a subterm matching {\termpattern} (or
{\qualid} if {\str} is the notation for a reference {\qualid}) and
whose name contains all {\str} of the request that correspond to valid
identifiers. If a {\termpattern} or a {\str} is prefixed by ``-'', the
search excludes the objects that mention that {\termpattern} or that
{\str}.

\item
  {\tt Search} \nelist{{\termpatternorstr}}{}
    {\tt inside} {\module$_1$} \ldots{} {\module$_n$}{\tt .}

This restricts the search to constructions defined in modules
{\module$_1$} \ldots{} {\module$_n$}.

\item
  {\tt Search \nelist{{\termpatternorstr}}{}
     outside {\module$_1$}...{\module$_n$}.}

This restricts the search to constructions not defined in modules
{\module$_1$} \ldots{} {\module$_n$}.

\end{Variants}

\examples

\begin{coq_example*}
Require Import ZArith.
\end{coq_example*}
\begin{coq_example}
Search Z.mul Z.add "distr".
Search "+"%Z "*"%Z "distr" -positive -Prop.
Search (?x * _ + ?x * _)%Z outside OmegaLemmas.
\end{coq_example}

\Warning \comindex{SearchAbout} Up to Coq version 8.4, {\tt Search}
had the behavior of current {\tt SearchHead} and the behavior of
current {\tt Search} was obtained with command {\tt SearchAbout}. For
compatibility, the deprecated name {\tt SearchAbout} can still be used
as a synonym of {\tt Search}. For compatibility, the list of objects to
search when using {\tt SearchAbout} may also be enclosed by optional
{\tt [ ]} delimiters.

\subsection[\tt SearchHead {\term}.]{\tt SearchHead {\term}.\comindex{SearchHead}}
This command displays the name and type of all theorems of the current
context whose statement's conclusion has the form {\tt ({\term} t1 ..
  tn)}.  This command is useful to remind the user of the name of
library lemmas.

\begin{coq_example*}
Reset Initial.
\end{coq_example*}

\begin{coq_example}
SearchHead le.
SearchHead (@eq bool).
\end{coq_example}

\begin{Variants}
\item
{\tt SearchHead} {\term} {\tt inside} {\module$_1$} \ldots{} {\module$_n$}{\tt .}

This restricts the search to constructions defined in modules
{\module$_1$} \ldots{} {\module$_n$}.

\item {\tt SearchHead} {\term} {\tt outside} {\module$_1$} \ldots{} {\module$_n$}{\tt .}

This restricts the search to constructions not defined in modules
{\module$_1$} \ldots{} {\module$_n$}.

\begin{ErrMsgs}
\item \errindex{Module/section \module{} not found}
No module \module{} has been required (see Section~\ref{Require}).
\end{ErrMsgs}

\end{Variants}

\Warning Up to Coq version 8.4, {\tt SearchHead} was named {\tt Search}.

\subsection[\tt SearchPattern {\termpattern}.]{\tt SearchPattern {\term}.\comindex{SearchPattern}}

This command displays the name and type of all theorems of the current
context whose statement's conclusion or last hypothesis and conclusion
matches the expression {\term} where holes in the latter are denoted
by ``{\texttt \_}''. It is a variant of {\tt Search
  {\termpattern}} that does not look for subterms but searches for
statements whose conclusion has exactly the expected form, or whose
statement finishes by the given series of hypothesis/conclusion.

\begin{coq_example}
Require Import Arith.
SearchPattern (_ + _ = _ + _).
SearchPattern (nat -> bool).
SearchPattern (forall l : list _, _ l l).
\end{coq_example}

Patterns need not be linear: you can express that the same expression
must occur in two places by using pattern variables `{\texttt
?{\ident}}''.

\begin{coq_example}
Require Import Arith.
SearchPattern (?X1 + _ = _ + ?X1).
\end{coq_example}

\begin{Variants}
\item {\tt SearchPattern {\term} inside
{\module$_1$} \ldots{} {\module$_n$}.}

This restricts the search to constructions defined in modules
{\module$_1$} \ldots{} {\module$_n$}.

\item {\tt SearchPattern {\term} outside {\module$_1$} \ldots{} {\module$_n$}.}

This restricts the search to constructions not defined in modules
{\module$_1$} \ldots{} {\module$_n$}.

\end{Variants}

\subsection[\tt SearchRewrite {\term}.]{\tt SearchRewrite {\term}.\comindex{SearchRewrite}}

This command displays the name and type of all theorems of the current
context whose statement's conclusion is an equality of which one side matches
the expression {\term}. Holes in {\term} are denoted by ``{\texttt \_}''.

\begin{coq_example}
Require Import Arith.
SearchRewrite (_ + _ + _).
\end{coq_example}

\begin{Variants}
\item {\tt SearchRewrite {\term} inside
{\module$_1$} \ldots{} {\module$_n$}.}

This restricts the search to constructions defined in modules
{\module$_1$} \ldots{} {\module$_n$}.

\item {\tt SearchRewrite {\term} outside {\module$_1$} \ldots{} {\module$_n$}.}

This restricts the search to constructions not defined in modules
{\module$_1$} \ldots{} {\module$_n$}.

\end{Variants}

\subsubsection{Nota Bene:}
For the {\tt Search}, {\tt SearchHead}, {\tt SearchPattern} and
{\tt SearchRewrite} queries, it is possible to globally filter
the search results via the command
{\tt Add Search Blacklist "substring1"}.
A lemma whose fully-qualified name contains any of the declared substrings
will be removed from the search results.
The default blacklisted substrings are {\tt "\_admitted"
  "\_subproof" "Private\_"}. The command {\tt Remove Search Blacklist
  ...} allows to expunge this blacklist.

% \begin{tabbing}
% \ \ \ \ \=11.\ \=\kill
% \>1.\>$A=B\mx{ if }A\stackrel{\bt{}\io{}}{\lra{}}B$\\
% \>2.\>$\sa{}x:A.B=\sa{}y:A.B[x\la{}y]\mx{ if }y\not\in{}FV(\sa{}x:A.B)$\\
% \>3.\>$\Pi{}x:A.B=\Pi{}y:A.B[x\la{}y]\mx{ if }y\not\in{}FV(\Pi{}x:A.B)$\\
% \>4.\>$\sa{}x:A.B=\sa{}x:B.A\mx{ if }x\not\in{}FV(A,B)$\\
% \>5.\>$\sa{}x:(\sa{}y:A.B).C=\sa{}x:A.\sa{}y:B[y\la{}x].C[x\la{}(x,y)]$\\
% \>6.\>$\Pi{}x:(\sa{}y:A.B).C=\Pi{}x:A.\Pi{}y:B[y\la{}x].C[x\la{}(x,y)]$\\
% \>7.\>$\Pi{}x:A.\sa{}y:B.C=\sa{}y:(\Pi{}x:A.B).(\Pi{}x:A.C[y\la{}(y\sm{}x)]$\\
% \>8.\>$\sa{}x:A.unit=A$\\
% \>9.\>$\sa{}x:unit.A=A[x\la{}tt]$\\
% \>10.\>$\Pi{}x:A.unit=unit$\\
% \>11.\>$\Pi{}x:unit.A=A[x\la{}tt]$
% \end{tabbing}

% For more informations about the exact working of this command, see
% \cite{Del97}.

\subsection[\tt Locate {\qualid}.]{\tt Locate {\qualid}.\comindex{Locate}
\label{Locate}}
This command displays the full name of the qualified identifier {\qualid}
and consequently the \Coq\ module in which it is defined.

\begin{coq_eval}
(*************** The last line should produce **************************)
(*********** Error: I.Dont.Exist not a defined object ******************)
\end{coq_eval}
\begin{coq_eval}
Set Printing Depth 50.
\end{coq_eval}
\begin{coq_example}
Locate nat.
Locate Datatypes.O.
Locate Init.Datatypes.O.
Locate Coq.Init.Datatypes.O.
Locate I.Dont.Exist.
\end{coq_example}

\SeeAlso Section \ref{LocateSymbol}

\subsection{The {\sc Whelp} searching tool
\label{Whelp}}

{\sc Whelp} is an experimental searching and browsing tool for the
whole {\Coq} library and the whole set of {\Coq} user contributions.
{\sc Whelp} requires a browser to work. {\sc Whelp} has been developed
at the University of Bologna as part of the HELM\footnote{Hypertextual
Electronic Library of Mathematics} and MoWGLI\footnote{Mathematics on
the Web, Get it by Logics and Interfaces} projects.  It can be invoked
directly from the {\Coq} toplevel or from {\CoqIDE}, assuming a
graphical environment is also running. The browser to use can be
selected by setting the environment variable {\tt
COQREMOTEBROWSER}. If not explicitly set, it defaults to
\verb!firefox -remote \"OpenURL(%s,new-tab)\" || firefox %s &"!  or
\verb!C:\\PROGRA~1\\INTERN~1\\IEXPLORE %s!, depending on the
underlying operating system (in the command, the string \verb!%s!
serves as metavariable for the url to open).
The Whelp tool relies on a dedicated Whelp server and on another server
called Getter that retrieves formal documents. The default Whelp server name
can be obtained using the command {\tt Test Whelp Server}
\comindex{Test Whelp Server} and the default Getter can be obtained
using the command: {\tt Test Whelp Getter} \comindex{Test Whelp
Getter}. The Whelp server name can be changed using the command:

\smallskip
\noindent {\tt Set Whelp Server {\str}}.\\
where {\str} is a URL (e.g. {\tt http://mowgli.cs.unibo.it:58080}).
\comindex{Set Whelp Server}
\smallskip

\noindent The Getter can be changed using the command:
\smallskip

\noindent {\tt Set Whelp Getter {\str}}.\\
where {\str} is a URL (e.g. {\tt http://mowgli.cs.unibo.it:58081}).  
\comindex{Set Whelp Getter}

\bigskip

The {\sc Whelp} commands are:

\subsubsection{\tt Whelp Locate "{\sl reg\_expr}".
\comindex{Whelp Locate}}

This command opens a browser window and displays the result of seeking
for all names that match the regular expression {\sl reg\_expr} in the
{\Coq} library and user contributions. The regular expression can
contain the special operators are * and ? that respectively stand for
an arbitrary substring and for exactly one character.

\variant {\tt Whelp Locate {\ident}.}\\
This is equivalent to {\tt Whelp Locate "{\ident}"}.

\subsubsection{\tt Whelp Match {\pattern}.
\comindex{Whelp Match}}

This command opens a browser window and displays the result of seeking
for all statements that match the pattern {\pattern}. Holes in the
pattern are represented by the wildcard character ``\_''.

\subsubsection[\tt Whelp Instance {\pattern}.]{\tt Whelp Instance {\pattern}.\comindex{Whelp Instance}}

This command opens a browser window and displays the result of seeking
for all statements that are instances of the pattern {\pattern}. The
pattern is here assumed to be an universally quantified expression.

\subsubsection[\tt Whelp Elim {\qualid}.]{\tt Whelp Elim {\qualid}.\comindex{Whelp Elim}}

This command opens a browser window and displays the result of seeking
for all statements that have the ``form'' of an elimination scheme
over the type denoted by {\qualid}.

\subsubsection[\tt Whelp Hint {\term}.]{\tt Whelp Hint {\term}.\comindex{Whelp Hint}}

This command opens a browser window and displays the result of seeking
for all statements that can be instantiated so that to prove the
statement {\term}.

\variant {\tt Whelp Hint.}\\ This is equivalent to {\tt Whelp Hint
{\sl goal}} where {\sl goal} is the current goal to prove. Notice that
{\Coq} does not send the local environment of definitions to the {\sc
Whelp} tool so that it only works on requests strictly based on, only,
definitions of the standard library and user contributions.

\section{Loading files}

\Coq\ offers the possibility of loading different
parts of a whole development stored in separate files. Their contents
will be loaded as if they were entered from the keyboard. This means
that the loaded files are ASCII files containing sequences of commands
for \Coq's toplevel. This kind of file is called a {\em script} for
\Coq\index{Script file}. The standard (and default) extension of
\Coq's script files is {\tt .v}.

\subsection[\tt Load {\ident}.]{\tt Load {\ident}.\comindex{Load}\label{Load}}
This command loads the file named {\ident}{\tt .v}, searching
successively in each of the directories specified in the {\em
  loadpath}. (see Section~\ref{loadpath})

\begin{Variants}
\item {\tt Load {\str}.}\label{Load-str}\\
  Loads the file denoted by the string {\str}, where {\str} is any
  complete filename. Then the \verb.~. and {\tt ..}
  abbreviations are allowed as well as shell variables. If no
  extension is specified, \Coq\ will use the default extension {\tt
    .v}
\item {\tt Load Verbose {\ident}.}, 
  {\tt Load Verbose {\str}}\\
  \comindex{Load Verbose}
  Display, while loading, the answers of \Coq\ to each command
  (including tactics) contained in the loaded file
  \SeeAlso Section~\ref{Begin-Silent}
\end{Variants}

\begin{ErrMsgs}
\item \errindex{Can't find file {\ident} on loadpath}
\end{ErrMsgs}

\section[Compiled files]{Compiled files\label{compiled}\index{Compiled files}}

This section describes the commands used to load compiled files (see
Chapter~\ref{Addoc-coqc} for documentation on how to compile a file).
A compiled file is a particular case of module called {\em library file}.

%%%%%%%%%%%%
% Import and Export described in RefMan-mod.tex
% the minor difference (to avoid multiple Exporting of libraries) in
% the treatment of normal modules and libraries by Export omitted

\subsection[\tt Require {\qualid}.]{\tt Require {\qualid}.\label{Require}
\comindex{Require}}

This command looks in the loadpath for a file containing
module {\qualid} and adds the corresponding module to the environment
of {\Coq}. As library files have dependencies in other library files,
the command {\tt Require {\qualid}} recursively requires all library
files the module {\qualid} depends on and adds the corresponding modules to the
environment of {\Coq} too. {\Coq} assumes that the compiled files have
been produced by a valid {\Coq} compiler and their contents are then not
replayed nor rechecked.

To locate the file in the file system, {\qualid} is decomposed under
the form {\dirpath}{\tt .}{\textsl{ident}} and the file {\ident}{\tt
.vo} is searched in the physical directory of the file system that is
mapped in {\Coq} loadpath to the logical path {\dirpath} (see
Section~\ref{loadpath}). The mapping between physical directories and
logical names at the time of requiring the file must be consistent
with the mapping used to compile the file.

\begin{Variants}
\item {\tt Require Import {\qualid}.} \comindex{Require} 

  This loads and declares the module {\qualid} and its dependencies
  then imports the contents of {\qualid} as described in
  Section~\ref{Import}.

  It does not import the modules on which {\qualid} depends unless
  these modules were itself required in module {\qualid} using {\tt
  Require Export}, as described below, or recursively required through
  a sequence of {\tt Require Export}.

  If the module required has already been loaded, {\tt Require Import
  {\qualid}} simply imports it, as {\tt Import {\qualid}} would.

\item {\tt Require Export {\qualid}.}
  \comindex{Require Export}

  This command acts as {\tt Require Import} {\qualid}, but if a
  further module, say {\it A}, contains a command {\tt Require
  Export} {\it B}, then the command {\tt Require Import} {\it A}
  also imports the module {\it B}.

\item {\tt Require \zeroone{Import {\sl |} Export}} {\qualid}$_1$ {\ldots} {\qualid}$_n${\tt .}

  This loads the modules {\qualid}$_1$, \ldots, {\qualid}$_n$ and
  their recursive dependencies. If {\tt Import} or {\tt Export} is
  given, it also imports {\qualid}$_1$, \ldots, {\qualid}$_n$ and all
  the recursive dependencies that were marked or transitively marked
  as {\tt Export}.

\item {\tt Require \zeroone{Import {\sl |} Export} {\str}.}

  This shortcuts the resolution of the qualified name into a library
  file name by directly requiring the module to be found in file
  {\str}.vo.
\end{Variants}

\begin{ErrMsgs}

\item \errindex{Cannot load {\qualid}: no physical path bound to {\dirpath}}

\item \errindex{Cannot find library foo in loadpath}

  The command did not find the file {\tt foo.vo}. Either {\tt
    foo.v} exists but is not compiled or {\tt foo.vo} is in a directory
  which is not in your {\tt LoadPath} (see Section~\ref{loadpath}).

\item \errindex{Compiled library {\ident}.vo makes inconsistent assumptions over library {\qualid}}

  The command tried to load library file {\ident}.vo that depends on
  some specific version of library {\qualid} which is not the one
  already loaded in the current {\Coq} session. Probably {\ident}.v
  was not properly recompiled with the last version of the file
  containing module {\qualid}.

\item \errindex{Bad magic number}

  \index{Bad-magic-number@{\tt Bad Magic Number}}
  The file {\tt{\ident}.vo} was found but either it is not a \Coq\
  compiled module, or it was compiled with an older and incompatible
  version of \Coq.

\item \errindex{The file {\ident}.vo contains library {\dirpath} and not
  library {\dirpath'}}

  The library file {\dirpath'} is indirectly required by the {\tt
  Require} command but it is bound in the current loadpath to the file
  {\ident}.vo which was bound to a different library name {\dirpath}
  at the time it was compiled.

\end{ErrMsgs}

\SeeAlso Chapter~\ref{Addoc-coqc}

\subsection[\tt Print Libraries.]{\tt Print Libraries.\comindex{Print Libraries}}

This command displays the list of library files loaded in the current
{\Coq} session. For each of these libraries, it also tells if it is
imported.

\subsection[\tt Declare ML Module {\str$_1$} .. {\str$_n$}.]{\tt Declare ML Module {\str$_1$} .. {\str$_n$}.\comindex{Declare ML Module}}
This commands loads the Objective Caml compiled files {\str$_1$} {\ldots}
{\str$_n$} (dynamic link). It is mainly used to load tactics
dynamically.
% (see Chapter~\ref{WritingTactics}).
 The files are
searched into the current Objective Caml loadpath (see the command {\tt
Add ML Path} in the Section~\ref{loadpath}).  Loading of Objective Caml
files is only possible under the bytecode version of {\tt coqtop}
(i.e. {\tt coqtop} called with options {\tt -byte}, see chapter 
\ref{Addoc-coqc}), or when Coq has been compiled with a version of
Objective Caml that supports native {\tt Dynlink} ($\ge$ 3.11).

\begin{Variants}
\item {\tt Local Declare ML Module {\str$_1$} .. {\str$_n$}.}\\
  This variant is not exported to the modules that import the module
  where they occur, even if outside a section.
\end{Variants}

\begin{ErrMsgs}
\item \errindex{File not found on loadpath : \str}
\item \errindex{Loading of ML object file forbidden in a native Coq}
\end{ErrMsgs}

\subsection[\tt Print ML Modules.]{\tt Print ML Modules.\comindex{Print ML Modules}}
This print the name of all \ocaml{} modules loaded with \texttt{Declare
  ML Module}. To know from where these module were loaded, the user
should use the command \texttt{Locate File} (see Section~\ref{Locate File})

\section[Loadpath]{Loadpath\label{loadpath}\index{Loadpath}}

There are currently two loadpaths in \Coq. A loadpath where seeking
{\Coq} files (extensions {\tt .v} or {\tt .vo} or {\tt .vi}) and one where
seeking Objective Caml files. The default loadpath contains the
directory ``\texttt{.}'' denoting the current directory and mapped to the empty logical path (see Section~\ref{LongNames}).

\subsection[\tt Pwd.]{\tt Pwd.\comindex{Pwd}\label{Pwd}}
This command displays the current working directory.

\subsection[\tt Cd {\str}.]{\tt Cd {\str}.\comindex{Cd}}
This command changes the current directory according to {\str} 
which can be any valid path.

\begin{Variants}
\item {\tt Cd.}\\
  Is equivalent to {\tt Pwd.}
\end{Variants}

\subsection[\tt Add LoadPath {\str} as {\dirpath}.]{\tt Add LoadPath {\str} as {\dirpath}.\comindex{Add LoadPath}\label{AddLoadPath}}

This command adds the physical directory {\str} to the current {\Coq}
loadpath and maps it to the logical directory {\dirpath}, which means
that every file \textrm{\textsl{dirname}}/\textrm{\textsl{basename.v}}
physically lying in subdirectory {\str}/\textrm{\textsl{dirname}}
becomes accessible in {\Coq} through absolute logical name
{\dirpath}{\tt .}\textrm{\textsl{dirname}}{\tt
.}\textrm{\textsl{basename}}.

\Rem {\tt Add LoadPath} also adds {\str} to the current ML loadpath.

\begin{Variants}
\item {\tt Add LoadPath {\str}.}\\
Performs as {\tt Add LoadPath {\str} as {\dirpath}} but for the empty directory path.
\end{Variants}

\subsection[\tt Add Rec LoadPath {\str} as {\dirpath}.]{\tt Add Rec LoadPath {\str} as {\dirpath}.\comindex{Add Rec LoadPath}\label{AddRecLoadPath}}
This command adds the physical directory {\str} and all its subdirectories to
the current \Coq\ loadpath. The top directory {\str} is mapped to the
logical directory {\dirpath} and any subdirectory {\textsl{pdir}} of it is
mapped to logical name {\dirpath}{\tt .}\textsl{pdir} and
recursively. Subdirectories corresponding to invalid {\Coq}
identifiers are skipped, and, by convention, subdirectories named {\tt
CVS} or {\tt \_darcs} are skipped too.

Otherwise, said, {\tt Add Rec LoadPath {\str} as {\dirpath}} behaves
as {\tt Add LoadPath {\str} as {\dirpath}} excepts that files lying in
validly named subdirectories of {\str} need not be qualified to be
found.

In case of files with identical base name, files lying in most recently
declared {\dirpath} are found first and explicit qualification is
required to refer to the other files of same base name.

If several files with identical base name are present in different
subdirectories of a recursive loadpath declared via a single instance of
{\tt Add Rec LoadPath}, which of these files is found first is
system-dependent and explicit qualification is recommended.

\Rem {\tt Add Rec LoadPath} also recursively adds {\str} to the current ML loadpath.

\begin{Variants}
\item {\tt Add Rec LoadPath {\str}.}\\
Works as {\tt Add Rec LoadPath {\str} as {\dirpath}} but for the empty logical directory path.
\end{Variants}

\subsection[\tt Remove LoadPath {\str}.]{\tt Remove LoadPath {\str}.\comindex{Remove LoadPath}}
This command removes the path {\str} from the current \Coq\ loadpath.

\subsection[\tt Print LoadPath.]{\tt Print LoadPath.\comindex{Print LoadPath}}
This command displays the current \Coq\ loadpath.

\begin{Variants}
\item {\tt Print LoadPath {\dirpath}.}\\
Works as {\tt Print LoadPath} but displays only the paths that extend the {\dirpath} prefix.
\end{Variants}

\subsection[\tt Add ML Path {\str}.]{\tt Add ML Path {\str}.\comindex{Add ML Path}}
This command adds the path {\str} to the current Objective Caml loadpath (see
the command {\tt Declare ML Module} in the Section~\ref{compiled}).

\Rem This command is implied by {\tt Add LoadPath {\str} as {\dirpath}}.

\subsection[\tt Add Rec ML Path {\str}.]{\tt Add Rec ML Path {\str}.\comindex{Add Rec ML Path}}
This command adds the directory {\str} and all its subdirectories 
to the current Objective Caml loadpath (see
the command {\tt Declare ML Module} in the Section~\ref{compiled}).

\Rem This command is implied by {\tt Add Rec LoadPath {\str} as {\dirpath}}.

\subsection[\tt Print ML Path {\str}.]{\tt Print ML Path {\str}.\comindex{Print ML Path}}
This command displays the current Objective Caml loadpath.
This command makes sense only under the bytecode version of {\tt
coqtop}, i.e. using option {\tt -byte} (see the
command {\tt Declare ML Module} in the section
\ref{compiled}).

\subsection[\tt Locate File {\str}.]{\tt Locate File {\str}.\comindex{Locate
  File}\label{Locate File}}
This command displays the location of file {\str} in the current loadpath.
Typically, {\str} is a \texttt{.cmo} or \texttt{.vo} or \texttt{.v} file.

\subsection[\tt Locate Library {\dirpath}.]{\tt Locate Library {\dirpath}.\comindex{Locate Library}\label{Locate Library}}
This command gives the status of the \Coq\ module {\dirpath}. It tells if the
module is loaded and if not searches in the load path for a module
of logical name {\dirpath}.

\section{Backtracking}

The backtracking commands described in this section can only be used
interactively, they cannot be part of a vernacular file loaded via
{\tt Load} or compiled by {\tt coqc}.

\subsection[\tt Reset \ident.]{\tt Reset \ident.\comindex{Reset}}
This command removes all the objects in the environment since \ident\ 
was introduced, including \ident. \ident\ may be the name of a defined
or declared object as well as the name of a section. One cannot reset
over the name of a module or of an object inside a module.

\begin{ErrMsgs}
\item \ident: \errindex{no such entry}
\end{ErrMsgs}

\begin{Variants}
 \item {\tt Reset Initial.}\comindex{Reset Initial}\\
   Goes back to the initial state, just after the start of the
   interactive session.
\end{Variants}

\subsection[\tt Back.]{\tt Back.\comindex{Back}}

This commands undoes all the effects of the last vernacular
command. Commands read from a vernacular file via a {\tt Load} are
considered as a single command. Proof managment commands
are also handled by this command (see Chapter~\ref{Proof-handling}).
For that, {\tt Back} may have to undo more than one command in order
to reach a state where the proof managment information is available.
For instance, when the last command is a {\tt Qed}, the managment
information about the closed proof has been discarded. In this case,
{\tt Back} will then undo all the proof steps up to the statement of
this proof.

\begin{Variants}
\item {\tt Back $n$} \\
  Undoes $n$ vernacular commands. As for {\tt Back}, some extra
  commands may be undone in order to reach an adequate state.
  For instance {\tt Back n} will not re-enter a closed proof,
  but rather go just before that proof.
\end{Variants}

\begin{ErrMsgs}
\item \errindex{Invalid backtrack} \\
  The user wants to undo more commands than available in the history.
\end{ErrMsgs}

\subsection[\tt BackTo $\num$.]{\tt BackTo $\num$.\comindex{BackTo}}
\label{sec:statenums}

This command brings back the system to the state labelled $\num$,
forgetting the effect of all commands executed after this state.
The state label is an integer which grows after each successful command.
It is displayed in the prompt when in \texttt{-emacs} mode.
Just as {\tt Back} (see above), the {\tt BackTo} command now handles
proof states. For that, it may have to undo some
extra commands and end on a state $\num' \leq \num$ if necessary.

\begin{Variants}
\item {\tt Backtrack $\num_1$ $\num_2$ $\num_3$}.\comindex{Backtrack}\\
  {\tt Backtrack} is a \emph{deprecated} form of {\tt BackTo} which
  allows to explicitely manipulate the proof environment. The three
  numbers $\num_1$, $\num_2$ and $\num_3$ represent the following:
\begin{itemize}
\item $\num_3$: Number of \texttt{Abort} to perform, i.e. the number
  of currently opened nested proofs that must be canceled (see
  Chapter~\ref{Proof-handling}).
\item $\num_2$: \emph{Proof state number} to unbury once aborts have
  been done. Coq will compute the number of \texttt{Undo} to perform
  (see Chapter~\ref{Proof-handling}).
\item $\num_1$: State label to reach, as for {\tt BackTo}.
\end{itemize}
\end{Variants}

\begin{ErrMsgs}
\item \errindex{Invalid backtrack} \\
  The destination state label is unknown.
\end{ErrMsgs}

\section{State files}

\subsection[\tt Write State \str.]{\tt Write State \str.\comindex{Write State}}
Writes the current state into a file \str{} for
use in a further session. This file can be given as the {\tt
  inputstate} argument of the commands {\tt coqtop} and {\tt coqc}.

\begin{Variants}
\item {\tt Write State \ident}\\
 Equivalent to {\tt Write State "}{\ident}{\tt .coq"}.
 The state is saved in the current directory (see Section~\ref{Pwd}).
\end{Variants}

\subsection[\tt Restore State \str.]{\tt Restore State \str.\comindex{Restore State}}
  Restores the state contained in the file \str.

\begin{Variants}
\item {\tt Restore State \ident}\\
 Equivalent to {\tt Restore State "}{\ident}{\tt .coq"}.
\end{Variants}

\section{Quitting and debugging}

\subsection[\tt Quit.]{\tt Quit.\comindex{Quit}}
This command permits to quit \Coq.

\subsection[\tt Drop.]{\tt Drop.\comindex{Drop}\label{Drop}}

This is used mostly as a debug facility by \Coq's implementors
and does not concern the casual user.
This command permits to leave {\Coq} temporarily and enter the
Objective Caml toplevel. The Objective Caml command:

\begin{flushleft}
\begin{verbatim}
#use "include";;
\end{verbatim}
\end{flushleft}

\noindent add the right loadpaths and loads some toplevel printers for
all abstract types of \Coq - section\_path, identifiers, terms, judgments,
\dots. You can also use the file \texttt{base\_include} instead,
that loads only the pretty-printers for section\_paths and
identifiers.
% See Section~\ref{test-and-debug} more information on the
% usage of the toplevel.
You can return back to \Coq{} with the command: 

\begin{flushleft}
\begin{verbatim}
go();;
\end{verbatim}
\end{flushleft}

\begin{Warnings}
\item It only works with the bytecode version of {\Coq} (i.e. {\tt coqtop} called with option {\tt -byte}, see the contents of Section~\ref{binary-images}).
\item You must have compiled {\Coq} from the source package and set the
  environment variable \texttt{COQTOP} to the root of your copy of the sources (see Section~\ref{EnvVariables}).
\end{Warnings}

\subsection[\tt Time \textrm{\textsl{command}}.]{\tt Time \textrm{\textsl{command}}.\comindex{Time}
\label{time}}
This command executes the vernacular command \textrm{\textsl{command}}
and display the time needed to execute it.


\subsection[\tt Timeout \textrm{\textsl{int}} \textrm{\textsl{command}}.]{\tt Timeout \textrm{\textsl{int}} \textrm{\textsl{command}}.\comindex{Timeout}
\label{timeout}}

This command executes the vernacular command \textrm{\textsl{command}}. If
the command has not terminated after the time specified by the integer
(time expressed in seconds), then it is interrupted and an error message
is displayed.

\subsection[\tt Set Default Timeout \textrm{\textsl{int}}.]{\tt Set
  Default Timeout \textrm{\textsl{int}}.\comindex{Set Default Timeout}}

After using this command, all subsequent commands behave as if they
were passed to a {\tt Timeout} command. Commands already starting by
a {\tt Timeout} are unaffected.

\subsection[\tt Unset Default Timeout.]{\tt Unset Default Timeout.\comindex{Unset Default Timeout}}

This command turns off the use of a default timeout.

\subsection[\tt Test Default Timeout.]{\tt Test Default Timeout.\comindex{Test Default Timeout}}

This command displays whether some default timeout has be set or not.

\section{Controlling display}

\subsection[\tt Set Silent.]{\tt Set Silent.\comindex{Set Silent}
\label{Begin-Silent}
\index{Silent mode}}
This command turns off the normal displaying.

\subsection[\tt Unset Silent.]{\tt Unset Silent.\comindex{Unset Silent}}
This command turns the normal display on.

\subsection[\tt Set Printing Width {\integer}.]{\tt Set Printing Width {\integer}.\comindex{Set Printing Width}}
\label{SetPrintingWidth}
This command sets which left-aligned part of the width of the screen
is used for display. 

\subsection[\tt Unset Printing Width.]{\tt Unset Printing Width.\comindex{Unset Printing Width}}
This command resets the width of the screen used for display to its
default value (which is 78 at the time of writing this documentation).

\subsection[\tt Test Printing Width.]{\tt Test Printing Width.\comindex{Test Printing Width}}
This command displays the current screen width used for display.

\subsection[\tt Set Printing Depth {\integer}.]{\tt Set Printing Depth {\integer}.\comindex{Set Printing Depth}}
This command sets the nesting depth of the formatter used for
pretty-printing. Beyond this depth, display of subterms is replaced by
dots. 

\subsection[\tt Unset Printing Depth.]{\tt Unset Printing Depth.\comindex{Unset Printing Depth}}
This command resets the nesting depth of the formatter used for
pretty-printing to its default value (at the
time of writing this documentation, the default value is 50).

\subsection[\tt Test Printing Depth.]{\tt Test Printing Depth.\comindex{Test Printing Depth}}
This command displays the current nesting depth used for display.

%\subsection{\tt Abstraction ...}
%Not yet documented.

\section{Controlling the reduction strategies and the conversion algorithm}
\label{Controlling reduction strategy}

{\Coq} provides reduction strategies that the tactics can invoke and
two different algorithms to check the convertibility of types.
The first conversion algorithm lazily
compares applicative terms while the other is a brute-force but efficient
algorithm that first normalizes the terms before comparing them.  The
second algorithm is based on a bytecode representation of terms
similar to the bytecode representation used in the ZINC virtual
machine~\cite{Leroy90}. It is especially useful for intensive
computation of algebraic values, such as numbers, and for reflexion-based
tactics. The commands to fine-tune the reduction strategies and the
lazy conversion algorithm are described first.

\subsection[{\tt Opaque} \qualid$_1$ {\ldots} \qualid$_n${\tt .}]{{\tt Opaque} \qualid$_1$ {\ldots} \qualid$_n${\tt .}\comindex{Opaque}\label{Opaque}}
This command has an effect on unfoldable constants, i.e. 
on constants defined by {\tt Definition} or {\tt Let} (with an explicit
body), or by a command assimilated to a definition such as {\tt
Fixpoint}, {\tt Program Definition}, etc, or by a proof ended by {\tt
Defined}. The command tells not to unfold
the constants {\qualid$_1$} {\ldots} {\qualid$_n$} in tactics using
$\delta$-conversion (unfolding a constant is replacing it by its
definition).

{\tt Opaque} has also on effect on the conversion algorithm of {\Coq},
telling to delay the unfolding of a constant as later as possible in
case {\Coq} has to check the conversion (see Section~\ref{conv-rules})
of two distinct applied constants.

The scope of {\tt Opaque} is limited to the current section, or
current file, unless the variant {\tt Global Opaque \qualid$_1$ {\ldots}
\qualid$_n$} is used.

\SeeAlso sections \ref{Conversion-tactics}, \ref{Automatizing},
\ref{Theorem}

\begin{ErrMsgs}
\item \errindex{The reference \qualid\ was not found in the current
environment}\\
    There is no constant referred by {\qualid} in the environment.
    Nevertheless, if you asked \texttt{Opaque foo bar}
    and if \texttt{bar} does not exist, \texttt{foo} is set opaque.
\end{ErrMsgs}

\subsection[{\tt Transparent} \qualid$_1$ {\ldots} \qualid$_n${\tt .}]{{\tt Transparent} \qualid$_1$ {\ldots} \qualid$_n${\tt .}\comindex{Transparent}\label{Transparent}}
This command is the converse of {\tt Opaque} and it applies on
unfoldable constants to restore their unfoldability after an {\tt
Opaque} command.

Note in particular that constants defined by a proof ended by {\tt
Qed} are not unfoldable and {\tt Transparent} has no effect on
them. This is to keep with the usual mathematical practice of {\em
proof irrelevance}: what matters in a mathematical development is the
sequence of lemma statements, not their actual proofs. This
distinguishes lemmas from the usual defined constants, whose actual
values are of course relevant in general.

The scope of {\tt Transparent} is limited to the current section, or
current file, unless the variant {\tt Global Transparent} \qualid$_1$
{\ldots} \qualid$_n$ is used.

\begin{ErrMsgs}
% \item \errindex{Can not set transparent.}\\
%     It is a constant from a required module or a parameter.
\item \errindex{The reference \qualid\ was not found in the current
environment}\\
    There is no constant referred by {\qualid} in the environment.
\end{ErrMsgs}

\SeeAlso sections \ref{Conversion-tactics}, \ref{Automatizing},
\ref{Theorem}

\subsection{{\tt Strategy} {\it level} {\tt [} \qualid$_1$ {\ldots} \qualid$_n$
  {\tt ].}\comindex{Strategy}\comindex{Local Strategy}\label{Strategy}}
This command generalizes the behavior of {\tt Opaque} and {\tt
  Transparent} commands. It is used to fine-tune the strategy for
unfolding constants, both at the tactic level and at the kernel
level. This command associates a level to \qualid$_1$ {\ldots}
\qualid$_n$. Whenever two expressions with two distinct head
constants are compared (for instance, this comparison can be triggered
by a type cast), the one with lower level is expanded first. In case
of a tie, the second one (appearing in the cast type) is expanded.

Levels can be one of the following (higher to lower):
\begin{description}
\item[opaque]: level of opaque constants. They cannot be expanded by
  tactics (behaves like $+\infty$, see next item).
\item[\num]: levels indexed by an integer. Level $0$ corresponds
  to the default behavior, which corresponds to transparent
  constants. This level can also be referred to as {\bf transparent}.
  Negative levels correspond to constants to be expanded before normal
  transparent constants, while positive levels correspond to constants
  to be expanded after normal transparent constants.
\item[expand]: level of constants that should be expanded first
  (behaves like $-\infty$)
\end{description}

These directives survive section and module closure, unless the
command is prefixed by {\tt Local}. In the latter case, the behavior
regarding sections and modules is the same as for the {\tt
  Transparent} and {\tt Opaque} commands.

\subsection{\tt Declare Reduction \ident\ := {\rm\sl convtactic}.}

This command allows to give a short name to a reduction expression,
for instance {\tt lazy beta delta [foo bar]}. This short name can
then be used in {\tt Eval \ident\ in ...} or {\tt eval} directives.
This command accepts the {\tt Local} modifier, for discarding
this reduction name at the end of the file or module. For the moment
the name cannot be qualified. In particular declaring the same name
in several modules or in several functor applications will be refused
if these declarations are not local. The name \ident\ cannot be used
directly as an Ltac tactic, but nothing prevent the user to also
perform a {\tt Ltac \ident\ := {\rm\sl convtactic}}.

\SeeAlso sections \ref{Conversion-tactics}

\subsection{\tt Set Virtual Machine
\label{SetVirtualMachine}
\comindex{Set Virtual Machine}}

This activates the bytecode-based conversion algorithm.

\subsection{\tt Unset Virtual Machine
\comindex{Unset Virtual Machine}}

This deactivates the bytecode-based conversion algorithm.

\subsection{\tt Test Virtual Machine
\comindex{Test Virtual Machine}}

This tells if the bytecode-based conversion algorithm is
activated. The default behavior is to have the bytecode-based
conversion algorithm deactivated.

\SeeAlso sections~\ref{vmcompute} and~\ref{vmoption}.

\section{Controlling the locality of commands}

\subsection{{\tt Local}, {\tt Global}
\comindex{Local}
\comindex{Global}
}

Some commands support a {\tt Local} or {\tt Global} prefix modifier to
control the scope of their effect. There are four kinds of commands:

\begin{itemize}
\item Commands whose default is to extend their effect both outside the
  section and the module or library file they occur in.

  For these commands, the {\tt Local} modifier limits the effect of
  the command to the current section or module it occurs in.

  As an example, the {\tt Coercion} (see Section~\ref{Coercions})
  and {\tt Strategy} (see Section~\ref{Strategy})
  commands belong to this category.

\item Commands whose default behavior is to stop their effect at the
  end of the section they occur in but to extent their effect outside
  the module or library file they occur in.

  For these commands, the {\tt Local} modifier limits the effect of
  the command to the current module if the command does not occur in a
  section and the {\tt Global} modifier extends the effect outside the
  current sections and current module if the command occurs in a
  section.

  As an example, the {\tt Implicit Arguments} (see
  Section~\ref{Implicit Arguments}), {\tt Ltac} (see
  Chapter~\ref{TacticLanguage}) or {\tt Notation} (see
  Section~\ref{Notation}) commands belong to this category.

  Notice that a subclass of these commands do not support extension of
  their scope outside sections at all and the {\tt Global} is not
  applicable to them.

\item Commands whose default behavior is to stop their effect at the
  end of the section or module they occur in.

  For these commands, the {\tt Global} modifier extends their effect
  outside the sections and modules they occurs in.

  The {\tt Transparent} and {\tt Opaque} (see
  Section~\ref{Controlling reduction strategy}) commands belong to
  this category.

\item Commands whose default behavior is to extend their effect
  outside sections but not outside modules when they occur in a
  section and to extend their effect outside the module or library
  file they occur in when no section contains them.

  For these commands, the {\tt Local} modifier limits the effect to
  the current section or module while the {\tt Global} modifier extends
  the effect outside the module even when the command occurs in a section.

  The {\tt Set} and {\tt Unset} commands belong to this category.
\end{itemize}


%%% Local Variables: 
%%% mode: latex
%%% TeX-master: "Reference-Manual"
%%% End: 
% Vernacular commands
\include{RefMan-pro.v}% Proof handling
% TODO: unify the use of \form and \type to mean a type
% or use \form specifically for a type of type Prop
\chapter{Tactics
\index{Tactics}
\label{Tactics}}

A deduction rule is a link between some (unique) formula, that we call
the {\em conclusion} and (several) formulas that we call the {\em
premises}. Indeed, a deduction rule can be read in two ways. The first
one has the shape: {\it ``if I know this and this then I can deduce
this''}. For instance, if I have a proof of $A$ and a proof of $B$
then I have a proof of $A \land B$. This is forward reasoning from
premises to conclusion. The other way says: {\it ``to prove this I
have to prove this and this''}. For instance, to prove $A \land B$, I
have to prove $A$ and I have to prove $B$. This is backward reasoning
which proceeds from conclusion to premises. We say that the conclusion
is {\em the goal}\index{goal} to prove and premises are {\em the
subgoals}\index{subgoal}.  The tactics implement {\em backward
reasoning}. When applied to a goal, a tactic replaces this goal with
the subgoals it generates. We say that a tactic reduces a goal to its
subgoal(s).

Each (sub)goal is denoted with a number. The current goal is numbered
1. By default, a tactic is applied to the current goal, but one can
address a particular goal in the list by writing {\sl n:\tac} which
means {\it ``apply tactic {\tac} to goal number {\sl n}''}.
We can show the list of subgoals by typing {\tt Show} (see
Section~\ref{Show}). 

Since not every rule applies to a given statement, every tactic cannot be
used to reduce any goal. In other words, before applying a tactic to a
given goal, the system checks that some {\em preconditions} are
satisfied. If it is not the case, the tactic raises an error message.

Tactics are build from atomic tactics and tactic expressions (which
extends the folklore notion of tactical) to combine those atomic
tactics. This chapter is devoted to atomic tactics. The tactic
language will be described in Chapter~\ref{TacticLanguage}.

There are, at least, three levels of atomic tactics. The simplest one
implements basic rules of the logical framework. The second level is
the one of {\em derived rules} which are built by combination of other
tactics. The third one implements heuristics or decision procedures to
build a complete proof of a goal.

\section{Invocation of tactics
\label{tactic-syntax}
\index{tactic@{\tac}}}

A tactic is applied as an ordinary command. If the tactic does not
address the first subgoal, the command may be preceded by the wished
subgoal number as shown below:

\begin{tabular}{lcl}
{\commandtac} & ::= & {\num} {\tt :} {\tac} {\tt .}\\
 & $|$ & {\tac} {\tt .}
\end{tabular}

\section{Explicit proof as a term}

\subsection{\tt exact \term
\tacindex{exact}
\label{exact}}

This tactic applies to any goal. It gives directly the exact proof
term of the goal. Let {\T} be our goal, let {\tt p} be a term of type
{\tt U} then {\tt exact p} succeeds iff {\tt T} and {\tt U} are
convertible (see Section~\ref{conv-rules}).

\begin{ErrMsgs}
\item \errindex{Not an exact proof}
\end{ErrMsgs}

\begin{Variants}
  \item \texttt{eexact \term}\tacindex{eexact} 
    
    This tactic behaves like \texttt{exact} but is able to handle terms with meta-variables. 

\end{Variants}


\subsection{\tt refine \term
\tacindex{refine}
\label{refine}
\index{?@{\texttt{?}}}}

This tactic allows to give an exact proof but still with some
holes. The holes are noted ``\texttt{\_}''.

\begin{ErrMsgs}
\item \errindex{invalid argument}: 
  the tactic \texttt{refine} doesn't know what to do
  with the term you gave.
\item \texttt{Refine passed ill-formed term}: the term you gave is not
  a valid proof (not easy to debug in general).
  This message may also occur in higher-level tactics, which call 
  \texttt{refine} internally.
\item \errindex{Cannot infer a term for this placeholder}
  there is a hole in the term you gave
  which type cannot be inferred. Put a cast around it.
\end{ErrMsgs}

An example of use is given in Section~\ref{refine-example}.

\section{Basics
\index{Typing rules}}

Tactics presented in this section implement the basic typing rules of
{\CIC} given in Chapter~\ref{Cic}.

\subsection{{\tt assumption}
\tacindex{assumption}}

This tactic applies to any goal. It implements the
``Var''\index{Typing rules!Var} rule given in
Section~\ref{Typed-terms}. It looks in the local context for an
hypothesis which type is equal to the goal.  If it is the case, the
subgoal is proved. Otherwise, it fails.

\begin{ErrMsgs}
\item  \errindex{No such assumption}
\end{ErrMsgs}

\begin{Variants}
\tacindex{eassumption}
  \item \texttt{eassumption}

    This tactic behaves like \texttt{assumption} but is able to handle
    goals with meta-variables.

\end{Variants}


\subsection{\tt clear {\ident}
\tacindex{clear}
\label{clear}}

This tactic erases the hypothesis named {\ident} in the local context
of the current goal. Then {\ident} is no more displayed and no more
usable in the proof development.

\begin{Variants}

\item {\tt clear {\ident$_1$} {\ldots} {\ident$_n$}}
  
  This is equivalent to {\tt clear {\ident$_1$}. {\ldots} clear
    {\ident$_n$}.}
  
\item {\tt clearbody {\ident}}\tacindex{clearbody}

  This tactic expects {\ident} to be a local definition then clears
  its body. Otherwise said, this tactic turns a definition into an
  assumption.

\item \texttt{clear - {\ident$_1$} {\ldots} {\ident$_n$}}

  This tactic clears all hypotheses except the ones depending in 
  the hypotheses named {\ident$_1$} {\ldots} {\ident$_n$} and in the
  goal.

\item \texttt{clear}

  This tactic clears all hypotheses except the ones depending in 
  goal.

\item {\tt clear dependent \ident \tacindex{clear dependent}}

 This clears the hypothesis \ident\ and all hypotheses
 which depend on it.

\end{Variants}

\begin{ErrMsgs}
\item \errindex{{\ident} not found}
\item \errindexbis{{\ident} is used in the conclusion}{is used in the
    conclusion} 
\item \errindexbis{{\ident} is used in the hypothesis {\ident'}}{is
    used in the hypothesis} 
\end{ErrMsgs}

\subsection{\tt move {\ident$_1$} after {\ident$_2$}
\tacindex{move}
\label{move}}

This moves the hypothesis named {\ident$_1$} in the local context
after the hypothesis named {\ident$_2$}.

If {\ident$_1$} comes before {\ident$_2$} in the order of dependences,
then all hypotheses between {\ident$_1$} and {\ident$_2$} which
(possibly indirectly) depend on {\ident$_1$} are moved also.

If {\ident$_1$} comes after {\ident$_2$} in the order of dependences,
then all hypotheses between {\ident$_1$} and {\ident$_2$} which 
(possibly indirectly) occur in {\ident$_1$} are moved also.

\begin{Variants}

\item {\tt move {\ident$_1$} before {\ident$_2$}}

This moves {\ident$_1$} towards and just before the hypothesis named {\ident$_2$}.

\item {\tt move {\ident} at top}

This moves {\ident} at the top of the local context (at the beginning of the context).

\item {\tt move {\ident} at bottom}

This moves {\ident} at the bottom of the local context (at the end of the context).

\end{Variants}

\begin{ErrMsgs}

\item \errindex{{\ident$_i$} not found}

\item \errindex{Cannot move {\ident$_1$} after {\ident$_2$}:
                   it occurs in {\ident$_2$}}

\item \errindex{Cannot move {\ident$_1$} after {\ident$_2$}:
                   it depends on {\ident$_2$}}

\end{ErrMsgs}

\subsection{\tt rename {\ident$_1$} into {\ident$_2$}
\tacindex{rename}}

This renames hypothesis {\ident$_1$} into {\ident$_2$} in the current
context\footnote{but it does not rename the hypothesis in the
  proof-term...}

\begin{Variants}

\item {\tt rename {\ident$_1$} into {\ident$_2$}, \ldots,
    {\ident$_{2k-1}$} into {\ident$_{2k}$}}

 Is equivalent to the sequence of the corresponding atomic {\tt rename}. 

\end{Variants}

\begin{ErrMsgs}

\item \errindex{{\ident$_1$} not found}

\item \errindexbis{{\ident$_2$} is already used}{is already used}

\end{ErrMsgs}

\subsection{\tt intro
\tacindex{intro}
\label{intro}}

This tactic applies to a goal which is either a product or starts with
a let binder. If the goal is a product, the tactic implements the
``Lam''\index{Typing rules!Lam} rule given in
Section~\ref{Typed-terms}\footnote{Actually, only the second subgoal will be
generated since the other one can be automatically checked.}.  If the
goal starts with a let binder then the tactic implements a mix of the
``Let''\index{Typing rules!Let} and ``Conv''\index{Typing rules!Conv}.

If the current goal is a dependent product {\tt forall $x$:$T$, $U$} (resp {\tt
let $x$:=$t$ in $U$}) then {\tt intro} puts {\tt $x$:$T$} (resp {\tt $x$:=$t$})
 in the local context.
% Obsolete (quantified names already avoid hypotheses names):
% Otherwise, it puts
% {\tt x}{\it n}{\tt :T} where {\it n} is such that {\tt x}{\it n} is a
%fresh name.
The new subgoal is $U$.
% If the {\tt x} has been renamed {\tt x}{\it n} then it is replaced 
% by {\tt x}{\it n} in {\tt U}. 

If the goal is a non dependent product {\tt $T$ -> $U$}, then it puts
in the local context either {\tt H}{\it n}{\tt :$T$} (if $T$ is of
type {\tt Set} or {\tt Prop}) or {\tt X}{\it n}{\tt :$T$} (if the type
of $T$ is {\tt Type}). The optional index {\it n} is such that {\tt
H}{\it n} or {\tt X}{\it n} is a fresh identifier.
In both cases the new subgoal is $U$.

If the goal is neither a product nor starting with a let definition,
the tactic {\tt intro} applies the tactic {\tt red} until the tactic
{\tt intro} can be applied or the goal is not reducible.

\begin{ErrMsgs}
\item \errindex{No product even after head-reduction}
\item \errindexbis{{\ident} is already used}{is already used}
\end{ErrMsgs}

\begin{Variants}

\item {\tt intros}\tacindex{intros}

  Repeats {\tt intro} until it meets the head-constant. It never reduces
  head-constants and it never fails.

\item {\tt intro {\ident}}

  Applies {\tt intro} but forces {\ident} to be the name of the
  introduced hypothesis.

  \ErrMsg \errindex{name {\ident} is already used}

  \Rem If a name used by {\tt intro} hides the base name of a global
  constant then the latter can still be referred to by a qualified name
  (see \ref{LongNames}).

\item {\tt intros \ident$_1$ \dots\ \ident$_n$} 
  
  Is equivalent to the composed tactic {\tt intro \ident$_1$; \dots\ ;
    intro \ident$_n$}.

  More generally, the \texttt{intros} tactic takes a pattern as
  argument in order to introduce names for components of an inductive
  definition or to clear introduced hypotheses; This is explained
  in~\ref{intros-pattern}.

\item {\tt intros until {\ident}} \tacindex{intros until}
  
  Repeats {\tt intro} until it meets a premise of the goal having form
  {\tt (} {\ident}~{\tt :}~{\term} {\tt )} and discharges the variable
  named {\ident} of the current goal.

  \ErrMsg \errindex{No such hypothesis in current goal}
  
\item {\tt intros until {\num}}  \tacindex{intros until}
  
  Repeats {\tt intro} until the {\num}-th non-dependent product. For
  instance, on the subgoal % 
  \verb+forall x y:nat, x=y -> y=x+ the tactic \texttt{intros until 1}
  is equivalent to \texttt{intros x y H}, as \verb+x=y -> y=x+ is the
  first non-dependent product. And on the subgoal %
  \verb+forall x y z:nat, x=y -> y=x+ the tactic \texttt{intros until 1}
  is equivalent to \texttt{intros x y z} as the product on \texttt{z}
  can be rewritten as a non-dependent product: %
  \verb+forall x y:nat, nat -> x=y -> y=x+


  \ErrMsg \errindex{No such hypothesis in current goal}

  Happens when {\num} is 0 or is greater than the number of non-dependent
  products of the goal.

\item {\tt intro after \ident} \tacindex{intro after}\\
      {\tt intro before \ident} \tacindex{intro before}\\
      {\tt intro at top} \tacindex{intro at top}\\
      {\tt intro at bottom} \tacindex{intro at bottom}

  Applies {\tt intro} and moves the freshly introduced hypothesis
  respectively after the hypothesis \ident{}, before the hypothesis
  \ident{}, at the top of the local context, or at the bottom of the
  local context. All hypotheses on which the new hypothesis depends
  are moved too so as to respect the order of dependencies between
  hypotheses. Note that {\tt intro at bottom} is a synonym for {\tt
  intro} with no argument.

\begin{ErrMsgs}
\item \errindex{No product even after head-reduction}
\item \errindex{No such hypothesis} : {\ident}
\end{ErrMsgs}

\item {\tt intro \ident$_1$ after \ident$_2$}\\
      {\tt intro \ident$_1$ before \ident$_2$}\\
      {\tt intro \ident$_1$ at top}\\
      {\tt intro \ident$_1$ at bottom}
  
  Behaves as previously but naming the introduced hypothesis
  \ident$_1$.  It is equivalent to {\tt intro \ident$_1$} followed by
  the appropriate call to {\tt move}~(see Section~\ref{move}).

\begin{ErrMsgs}
\item \errindex{No product even after head-reduction}
\item \errindex{No such hypothesis} : {\ident}
\end{ErrMsgs}

\end{Variants}

\subsection{\tt apply \term
\tacindex{apply}
\label{apply}}

This tactic applies to any goal.  The argument {\term} is a term
well-formed in the local context.  The tactic {\tt apply} tries to
match the current goal against the conclusion of the type of {\term}.
If it succeeds, then the tactic returns as many subgoals as the number
of non dependent premises of the type of {\term}. If the conclusion of
the type of {\term} does not match the goal {\em and} the conclusion
is an inductive type isomorphic to a tuple type, then each component
of the tuple is recursively matched to the goal in the left-to-right
order.

The tactic {\tt apply} relies on first-order unification with
dependent types unless the conclusion of the type of {\term} is of the
form {\tt ($P$~ $t_1$~\ldots ~$t_n$)} with $P$ to be instantiated.  In
the latter case, the behavior depends on the form of the goal. If the
goal is of the form {\tt (fun $x$ => $Q$)~$u_1$~\ldots~$u_n$} and the
$t_i$ and $u_i$ unifies, then $P$ is taken to be (fun $x$ => $Q$).
Otherwise, {\tt apply} tries to define $P$ by abstracting over
$t_1$~\ldots ~$t_n$ in the goal. See {\tt pattern} in
Section~\ref{pattern} to transform the goal so that it gets the form
{\tt (fun $x$ => $Q$)~$u_1$~\ldots~$u_n$}.

\begin{ErrMsgs}
\item \errindex{Impossible to unify \dots\ with \dots} 

  The {\tt apply}
  tactic failed to match the conclusion of {\term} and the current goal.
  You can help the {\tt apply} tactic by transforming your
  goal with the {\tt change} or {\tt pattern} tactics (see 
  sections~\ref{pattern},~\ref{change}).

\item \errindex{Unable to find an instance for the variables
{\ident} \ldots {\ident}}

  This occurs when some instantiations of the premises of {\term} are not
  deducible from the unification. This is the case, for instance, when
  you want to apply a transitivity property. In this case, you have to
  use one of the variants below:

\end{ErrMsgs}

\begin{Variants}

\item{\tt apply {\term} with {\term$_1$} \dots\ {\term$_n$}} 
  \tacindex{apply \dots\ with}
  
  Provides {\tt apply} with explicit instantiations for all dependent
  premises of the type of {\term} which do not occur in the conclusion
  and consequently cannot be found by unification. Notice that
  {\term$_1$} \dots\ {\term$_n$} must be given according to the order
  of these dependent premises of the type of {\term}.

  \ErrMsg \errindex{Not the right number of missing arguments}

\item{\tt apply {\term} with ({\vref$_1$} := {\term$_1$}) \dots\ ({\vref$_n$}
    := {\term$_n$})} 
  
  This also provides {\tt apply} with values for instantiating
  premises. Here, variables are referred by names and non-dependent
  products by increasing numbers (see syntax in Section~\ref{Binding-list}).

\item {\tt apply} {\term$_1$} {\tt ,} \ldots {\tt ,} {\term$_n$} 

  This is a shortcut for {\tt apply} {\term$_1$} {\tt ; [ ..~|}
   \ldots~{\tt ; [ ..~| {\tt apply} {\term$_n$} ]} \ldots~{\tt ]}, i.e. for the
   successive applications of {\term$_{i+1}$} on the last subgoal
   generated by {\tt apply} {\term$_i$}, starting from the application
   of {\term$_1$}.

\item {\tt eapply \term}\tacindex{eapply}\label{eapply}
  
  The tactic {\tt eapply} behaves as {\tt apply} but does not fail
  when no instantiation are deducible for some variables in the
  premises.  Rather, it turns these variables into so-called
  existential variables which are variables still to instantiate. An
  existential variable is identified by a name of the form {\tt ?$n$}
  where $n$ is a number.  The instantiation is intended to be found
  later in the proof.

  An example of use of {\tt eapply} is given in
  Section~\ref{eapply-example}. 

\item {\tt simple apply {\term}} \tacindex{simple apply} 

  This behaves like {\tt apply} but it reasons modulo conversion only
  on subterms that contain no variables to instantiate. For instance,
  if {\tt id := fun x:nat => x} and {\tt H : forall y, id y = y} then
  {\tt simple apply H} on goal {\tt O = O} does not succeed because it
  would require the conversion of {\tt f ?y} and {\tt O} where {\tt
  ?y} is a variable to instantiate. Tactic {\tt simple apply} does not
  either traverse tuples as {\tt apply} does.

  Because it reasons modulo a limited amount of conversion, {\tt
  simple apply} fails quicker than {\tt apply} and it is then
  well-suited for uses in used-defined tactics that backtrack often.

\item \zeroone{{\tt simple}} {\tt apply} {\term$_1$} \zeroone{{\tt with}
  {\bindinglist$_1$}} {\tt ,} \ldots {\tt ,} {\term$_n$} \zeroone{{\tt with}
  {\bindinglist$_n$}}\\
  \zeroone{{\tt simple}} {\tt eapply} {\term$_1$} \zeroone{{\tt with}
  {\bindinglist$_1$}} {\tt ,} \ldots {\tt ,} {\term$_n$} \zeroone{{\tt with}
  {\bindinglist$_n$}}

  This summarizes the different syntaxes for {\tt apply} and {\tt eapply}.

\item {\tt lapply {\term}} \tacindex{lapply} 

  This tactic applies to any goal, say {\tt G}.  The argument {\term}
  has to be well-formed in the current context, its type being
  reducible to a non-dependent product {\tt A -> B} with {\tt B}
  possibly containing products. Then it generates two subgoals {\tt
  B->G} and {\tt A}. Applying {\tt lapply H} (where {\tt H} has type
  {\tt A->B} and {\tt B} does not start with a product) does the same
  as giving the sequence {\tt cut B. 2:apply H.} where {\tt cut} is
  described below.

  \Warning When {\term} contains more than one non
  dependent product the tactic {\tt lapply} only takes into account the
  first product.

\end{Variants}

\subsection{{\tt set ( {\ident} {\tt :=} {\term} \tt )}
\label{tactic:set}
\tacindex{set}
\tacindex{pose}
\tacindex{remember}}

This replaces {\term} by {\ident} in the conclusion or in the
hypotheses of the current goal and adds the new definition {\ident
{\tt :=} \term} to the local context. The default is to make this
replacement only in the conclusion.

\begin{Variants}

\item {\tt set (} {\ident} {\tt :=} {\term} {\tt ) in {\occgoalset}}

This notation allows to specify which occurrences of {\term} have to
be substituted in the context. The {\tt in {\occgoalset}} clause is an
occurrence clause whose syntax and behavior is described in
Section~\ref{Occurrences clauses}.

\item {\tt set (} {\ident} \nelist{\binder}{} {\tt :=} {\term} {\tt )}

  This is equivalent to {\tt set (} {\ident} {\tt :=} {\tt fun}
  \nelist{\binder}{} {\tt =>} {\term} {\tt )}.

\item {\tt set } {\term}

  This behaves as {\tt set (} {\ident} := {\term} {\tt )} but {\ident}
  is generated by {\Coq}. This variant also supports an occurrence clause.

\item {\tt set (} {\ident$_0$} \nelist{\binder}{} {\tt :=} {\term}
      {\tt ) in {\occgoalset}}\\
      {\tt set {\term} in {\occgoalset}}

  These are the general forms which combine the previous possibilities.

\item {\tt remember {\term} {\tt as} {\ident}}

  This behaves as {\tt set (} {\ident} := {\term} {\tt ) in *} and using a
  logical (Leibniz's) equality instead of a local definition.

\item {\tt remember {\term} {\tt as} {\ident} in {\occgoalset}}

  This is a more general form of {\tt remember} that remembers the
  occurrences of {\term} specified by an occurrences set.

\item {\tt pose (  {\ident} {\tt :=} {\term} {\tt )}}
  
  This adds the local definition {\ident} := {\term} to the current
  context without performing any replacement in the goal or in the
  hypotheses. It is equivalent to {\tt set ( {\ident} {\tt :=}
  {\term} {\tt ) in |-}}.

\item {\tt pose (} {\ident} \nelist{\binder}{} {\tt :=} {\term} {\tt )}

  This is equivalent to {\tt pose (} {\ident} {\tt :=} {\tt fun}
  \nelist{\binder}{} {\tt =>} {\term} {\tt )}.

\item{\tt pose {\term}}

  This behaves as {\tt pose (} {\ident} := {\term} {\tt )} but
  {\ident} is generated by {\Coq}.

\end{Variants}

\subsection{{\tt assert ( {\ident} : {\form} \tt )}
\tacindex{assert}}

This tactic applies to any goal. {\tt assert (H : U)} adds a new
hypothesis of name \texttt{H} asserting \texttt{U} to the current goal
and opens a new subgoal \texttt{U}\footnote{This corresponds to the
  cut rule of sequent calculus.}. The subgoal {\texttt U} comes first
in the list of subgoals remaining to prove.

\begin{ErrMsgs}
\item \errindex{Not a proposition or a type}
  
  Arises when the argument {\form} is neither of type {\tt Prop}, {\tt
    Set} nor {\tt Type}.

\end{ErrMsgs}

\begin{Variants}

\item{\tt assert {\form}}
  
  This behaves as {\tt assert (} {\ident} : {\form} {\tt )} but
  {\ident} is generated by {\Coq}.

\item{\tt assert (} {\ident} := {\term} {\tt )}
  
  This behaves as {\tt assert ({\ident} : {\type});[exact
    {\term}|idtac]} where {\type} is the type of {\term}.

\item {\tt cut {\form}}\tacindex{cut} 
  
  This tactic applies to any goal. It implements the non dependent
  case of the ``App''\index{Typing rules!App} rule given in
  Section~\ref{Typed-terms}. (This is Modus Ponens inference rule.)
  {\tt cut U} transforms the current goal \texttt{T} into the two
  following subgoals: {\tt U -> T} and \texttt{U}.  The subgoal {\tt U
    -> T} comes first in the list of remaining subgoal to prove.

\item \texttt{assert {\form} by {\tac}}\tacindex{assert by}
  
  This tactic behaves like \texttt{assert} but applies {\tac}
  to solve the subgoals generated by \texttt{assert}.

\item \texttt{assert {\form} as {\intropattern}\tacindex{assert as}}

  If {\intropattern} is a naming introduction pattern (see
  Section~\ref{intros-pattern}), the hypothesis is named after this
  introduction pattern (in particular, if {\intropattern} is {\ident},
  the tactic behaves like \texttt{assert ({\ident} : {\form})}).

  If {\intropattern} is a disjunctive/conjunctive introduction
  pattern, the tactic behaves like \texttt{assert {\form}} then destructing the
  resulting hypothesis using the given introduction pattern.

\item \texttt{assert {\form} as {\intropattern} by {\tac}}

  This combines the two previous variants of {\tt assert}.

\item \texttt{pose proof {\term} as {\intropattern}\tacindex{pose proof}}

  This tactic behaves like \texttt{assert T as {\intropattern} by
  exact {\term}} where \texttt{T} is the type of {\term}.

  In particular, \texttt{pose proof {\term} as {\ident}} behaves as
  \texttt{assert ({\ident}:T) by exact {\term}} (where \texttt{T} is
  the type of {\term}) and \texttt{pose proof {\term} as
  {\disjconjintropattern}\tacindex{pose proof}} behaves
  like \texttt{destruct {\term} as {\disjconjintropattern}}.

\item {\tt specialize ({\ident} \term$_1$ {\ldots} \term$_n$)\tacindex{specialize}} \\
      {\tt specialize {\ident} with \bindinglist}

      The tactic {\tt specialize} works on local hypothesis \ident.
      The premises of this hypothesis (either universal
      quantifications or non-dependent implications) are instantiated
      by concrete terms coming either from arguments \term$_1$
      $\ldots$ \term$_n$ or from a bindings list (see
      Section~\ref{Binding-list} for more about bindings lists). In the
      second form, all instantiation elements must be given, whereas
      in the first form the application to \term$_1$ {\ldots}
      \term$_n$ can be partial. The first form is equivalent to 
      {\tt assert (\ident':=\ident \term$_1$ {\ldots} \term$_n$);
           clear \ident; rename \ident' into \ident}. 

      The name {\ident} can also refer to a global lemma or
      hypothesis. In this case, for compatibility reasons, the
      behavior of {\tt specialize} is close to that of {\tt
        generalize}: the instantiated statement becomes an additional 
      premise of the goal. 

%% Moreover, the old syntax allows the use of a number after {\tt specialize} 
%% for controlling the number of premises to instantiate. Giving this 
%% number should not be mandatory anymore (automatic detection of how
%% many premises can be eaten without leaving meta-variables). Hence 
%% no documentation for this integer optional argument of specialize

\end{Variants}

\subsection{{\tt apply {\term} in {\ident}}
\tacindex{apply \ldots\ in}}

This tactic applies to any goal.  The argument {\term} is a term
well-formed in the local context and the argument {\ident} is an
hypothesis of the context.  The tactic {\tt apply {\term} in {\ident}}
tries to match the conclusion of the type of {\ident} against a non
dependent premise of the type of {\term}, trying them from right to
left.  If it succeeds, the statement of hypothesis {\ident} is
replaced by the conclusion of the type of {\term}. The tactic also
returns as many subgoals as the number of other non dependent premises
in the type of {\term} and of the non dependent premises of the type
of {\ident}.  If the conclusion of the type of {\term} does not match
the goal {\em and} the conclusion is an inductive type isomorphic to a
tuple type, then the tuple is (recursively) decomposed and the first
component of the tuple of which a non dependent premise matches the
conclusion of the type of {\ident}. Tuples are decomposed in a
width-first left-to-right order (for instance if the type of {\tt H1}
is a \verb=A <-> B= statement, and the type of {\tt H2} is \verb=A=
then {\tt apply H1 in H2} transforms the type of {\tt H2} into {\tt
  B}). The tactic {\tt apply} relies on first-order pattern-matching
with dependent types.

\begin{ErrMsgs}
\item \errindex{Statement without assumptions}

This happens if the type of {\term} has no non dependent premise.

\item \errindex{Unable to apply}

This happens if the conclusion of {\ident} does not match any of the
non dependent premises of the type of {\term}.
\end{ErrMsgs}

\begin{Variants}
\item {\tt apply \nelist{\term}{,} in {\ident}}

This applies each of {\term} in sequence in {\ident}.

\item {\tt apply \nelist{{\term} with {\bindinglist}}{,} in {\ident}}

This does the same but uses the bindings in each {\bindinglist} to 
instantiate the parameters of the corresponding type of {\term}
(see syntax of bindings in Section~\ref{Binding-list}).

\item {\tt eapply \nelist{{\term} with {\bindinglist}}{,} in {\ident}}
\tacindex{eapply {\ldots} in}

This works as {\tt apply \nelist{{\term} with {\bindinglist}}{,} in
{\ident}} but turns unresolved bindings into existential variables, if
any, instead of failing.

\item {\tt apply \nelist{{\term}{,} with {\bindinglist}}{,} in {\ident} as {\disjconjintropattern}}

This works as {\tt apply \nelist{{\term}{,} with {\bindinglist}}{,} in
{\ident}} then destructs the hypothesis {\ident} along
{\disjconjintropattern} as {\tt destruct {\ident} as
{\disjconjintropattern}} would.

\item {\tt eapply \nelist{{\term}{,} with {\bindinglist}}{,} in {\ident} as {\disjconjintropattern}}

This works as {\tt apply \nelist{{\term}{,} with {\bindinglist}}{,} in {\ident} as {\disjconjintropattern}} but using {\tt eapply}.

\item {\tt simple apply {\term} in {\ident}}
\tacindex{simple apply {\ldots} in} 
\tacindex{simple eapply {\ldots} in} 

This behaves like {\tt apply {\term} in {\ident}} but it reasons
modulo conversion only on subterms that contain no variables to
instantiate. For instance, if {\tt id := fun x:nat => x} and {\tt H :
  forall y, id y = y -> True} and {\tt H0 : O = O} then {\tt simple
  apply H in H0} does not succeed because it would require the
conversion of {\tt f ?y} and {\tt O} where {\tt ?y} is a variable to
instantiate.  Tactic {\tt simple apply {\term} in {\ident}} does not
either traverse tuples as {\tt apply {\term} in {\ident}} does.

\item {\tt \zeroone{simple} apply \nelist{{\term} \zeroone{with {\bindinglist}}}{,} in {\ident} \zeroone{as {\disjconjintropattern}}}\\
{\tt \zeroone{simple} eapply \nelist{{\term} \zeroone{with {\bindinglist}}}{,} in {\ident} \zeroone{as {\disjconjintropattern}}}

This summarizes the different syntactic variants of {\tt apply {\term}
  in {\ident}} and {\tt eapply {\term} in {\ident}}.
\end{Variants}

\subsection{\tt generalize \term
\tacindex{generalize}
\label{generalize}}

This tactic applies to any goal. It generalizes the conclusion w.r.t.
one subterm of it. For example:

\begin{coq_eval}
Goal forall x y:nat, (0 <= x + y + y).
intros.
\end{coq_eval}
\begin{coq_example}
Show.
generalize (x + y + y).
\end{coq_example}

\begin{coq_eval}
Abort.
\end{coq_eval}

If the goal is $G$ and $t$ is a subterm of type $T$ in the goal, then
{\tt generalize} \textit{t} replaces the goal by {\tt forall (x:$T$), $G'$}
where $G'$ is obtained from $G$ by replacing all occurrences of $t$ by
{\tt x}. The name of the variable (here {\tt n}) is chosen based on $T$.

\begin{Variants}
\item {\tt generalize {\term$_1$ , \dots\ , \term$_n$}}
  
  Is equivalent to {\tt generalize \term$_n$; \dots\ ; generalize
    \term$_1$}. Note that the sequence of \term$_i$'s are processed
  from $n$ to $1$.

\item {\tt generalize {\term} at {\num$_1$ \dots\ \num$_i$}}
  
  Is equivalent to {\tt generalize \term} but generalizing only over
  the specified occurrences of {\term} (counting from left to right on the
  expression printed using option {\tt Set Printing All}).

\item {\tt generalize {\term} as {\ident}}
  
  Is equivalent to {\tt generalize \term} but use {\ident} to name the
  generalized hypothesis.

\item {\tt generalize {\term$_1$} at {\num$_{11}$ \dots\ \num$_{1i_1}$} 
                      as {\ident$_1$}
                      , {\ldots} ,
                      {\term$_n$} at {\num$_{n1}$ \dots\ \num$_{ni_n}$}
                      as {\ident$_2$}}
  
  This is the most general form of {\tt generalize} that combines the
  previous behaviors.
  
\item {\tt generalize dependent \term} \tacindex{generalize dependent}
  
  This generalizes {\term} but also {\em all} hypotheses which depend
  on {\term}. It clears the generalized hypotheses.

\end{Variants}


\subsection{\tt revert  \ident$_1$ \dots\ \ident$_n$
\tacindex{revert}
\label{revert}}

This applies to any goal with variables \ident$_1$ \dots\ \ident$_n$.
It moves the hypotheses (possibly defined) to the goal, if this respects
dependencies. This tactic is the inverse of {\tt intro}. 

\begin{ErrMsgs}
\item \errindexbis{{\ident} is used in the hypothesis {\ident'}}{is
    used in the hypothesis} 
\end{ErrMsgs}

\begin{Variants}
\item {\tt revert dependent \ident \tacindex{revert dependent}}

 This moves to the goal the hypothesis \ident\ and all hypotheses
 which depend on it.

\end{Variants}

\subsection{\tt change \term
\tacindex{change}
\label{change}}

This tactic applies to any goal. It implements the rule
``Conv''\index{Typing rules!Conv} given in Section~\ref{Conv}.  {\tt
  change U} replaces the current goal \T\ with \U\ providing that
\U\ is well-formed and that \T\ and \U\ are convertible.

\begin{ErrMsgs}
\item \errindex{Not convertible}
\end{ErrMsgs}

\tacindex{change \dots\ in}
\begin{Variants}
\item {\tt change \term$_1$ with \term$_2$} 
  
  This replaces the occurrences of \term$_1$ by \term$_2$ in the
  current goal.  The terms \term$_1$ and \term$_2$ must be
  convertible.

\item {\tt change \term$_1$ at \num$_1$ \dots\ \num$_i$ with \term$_2$} 
  
  This replaces the occurrences numbered \num$_1$ \dots\ \num$_i$ of
  \term$_1$ by \term$_2$ in the current goal.
  The terms \term$_1$ and \term$_2$ must be convertible.

  \ErrMsg {\tt Too few occurrences}

\item {\tt change {\term} in {\ident}}

\item {\tt change \term$_1$ with \term$_2$ in {\ident}}
  
\item {\tt change \term$_1$ at  \num$_1$ \dots\ \num$_i$ with \term$_2$ in
    {\ident}}
  
  This applies the {\tt change} tactic not to the goal but to the
  hypothesis {\ident}.

\end{Variants}

\SeeAlso \ref{Conversion-tactics}

\subsection{\tt fix {\ident} {\num}
\tacindex{fix}
\label{tactic:fix}}

This tactic is a primitive tactic to start a proof by induction. In
general, it is easier to rely on higher-level induction tactics such
as the ones described in Section~\ref{Tac-induction}.

In the syntax of the tactic, the identifier {\ident} is the name given
to the induction hypothesis. The natural number {\num} tells on which
premise of the current goal the induction acts, starting
from 1 and counting both dependent and non dependent
products. Especially, the current lemma must be composed of at least
{\num} products.

Like in a {\tt fix} expression, the induction
hypotheses have to be used on structurally smaller arguments.
The verification that inductive proof arguments are correct is done
only at the time of registering the lemma in the environment. To know
if the use of induction hypotheses is correct at some
time of the interactive development of a proof, use the command {\tt
  Guarded} (see Section~\ref{Guarded}).

\begin{Variants}
  \item {\tt fix} {\ident}$_1$ {\num} {\tt with (} {\ident}$_2$
    \nelist{{\binder}$_{2}$}{} \zeroone{{\tt \{ struct {\ident$'_2$}
      \}}} {\tt :} {\type}$_2$ {\tt )} {\ldots} {\tt (} {\ident}$_1$
    \nelist{{\binder}$_n$}{} \zeroone{{\tt \{ struct {\ident$'_n$} \}}}
           {\tt :} {\type}$_n$ {\tt )}

This starts a proof by mutual induction. The statements to be
simultaneously proved are respectively {\tt forall}
  \nelist{{\binder}$_2$}{}{\tt ,} {\type}$_2$, {\ldots}, {\tt forall}
  \nelist{{\binder}$_n$}{}{\tt ,} {\type}$_n$.  The identifiers
{\ident}$_1$ {\ldots} {\ident}$_n$ are the names of the induction
hypotheses. The identifiers {\ident}$'_2$ {\ldots} {\ident}$'_n$ are the
respective names of the premises on which the induction is performed
in the statements to be simultaneously proved (if not given, the
system tries to guess itself what they are).

\end{Variants}

\subsection{\tt cofix {\ident}
\tacindex{cofix}
\label{tactic:cofix}}

This tactic starts a proof by coinduction. The identifier {\ident} is
the name given to the coinduction hypothesis.  Like in a {\tt cofix}
expression, the use of induction hypotheses have to guarded by a
constructor.  The verification that the use of coinductive hypotheses
is correct is done only at the time of registering the lemma in the
environment. To know if the use of coinduction hypotheses is correct
at some time of the interactive development of a proof, use the
command {\tt Guarded} (see Section~\ref{Guarded}).


\begin{Variants}
  \item {\tt cofix} {\ident}$_1$ {\tt with (} {\ident}$_2$
    \nelist{{\binder}$_2$}{} {\tt :} {\type}$_2$ {\tt )} {\ldots} {\tt
      (} {\ident}$_1$ \nelist{{\binder}$_1$}{} {\tt :} {\type}$_n$
           {\tt )}

This starts a proof by mutual coinduction. The statements to be
simultaneously proved are respectively {\tt forall}
\nelist{{\binder}$_2$}{}{\tt ,} {\type}$_2$, {\ldots}, {\tt forall}
  \nelist{{\binder}$_n$}{}{\tt ,} {\type}$_n$. The identifiers
    {\ident}$_1$ {\ldots} {\ident}$_n$ are the names of the
    coinduction hypotheses.

\end{Variants}

\subsection{\tt evar (\ident:\term)
\tacindex{evar}
\label{evar}}

The {\tt evar} tactic creates a new local definition named \ident\ with
type \term\ in the context. The body of this binding is a fresh
existential variable.

\subsection{\tt instantiate (\num:= \term)
\tacindex{instantiate}
\label{instantiate}}

The {\tt instantiate} tactic allows to solve an existential variable
with the term \term. The \num\  argument is the position of the
existential variable from right to left in the conclusion. This cannot be
the number of the existential variable since this number is different
in every session.

\begin{Variants}
  \item {\tt instantiate (\num:=\term) in \ident}
  
  \item {\tt instantiate (\num:=\term) in (Value of \ident)}
  
  \item {\tt instantiate (\num:=\term) in (Type of \ident)}

These allow to refer respectively to existential variables occurring in 
a hypothesis or in the body or the type of a local definition.  

  \item {\tt instantiate}

    Without argument, the {\tt instantiate} tactic tries to solve as
    many existential variables as possible, using information gathered
    from other tactics in the same tactical. This is automatically
    done after each complete tactic (i.e. after a dot in proof mode),
    but not, for example, between each tactic when they are sequenced
    by semicolons.

\end{Variants}

\subsection{\tt admit
\tacindex{admit}
\label{admit}}

The {\tt admit} tactic ``solves'' the current subgoal by an
axiom. This typically allows to temporarily skip a subgoal so as to
progress further in the rest of the proof. To know if some proof still
relies on unproved subgoals, one can use the command {\tt Print
Assumptions} (see Section~\ref{PrintAssumptions}). Admitted subgoals
have names of the form {\ident}\texttt{\_admitted} possibly followed
by a number.

\subsection{\tt constr\_eq \term$_1$ \term$_2$
\tacindex{constr\_eq}
\label{constreq}}

This tactic applies to any goal. It checks whether its arguments are
equal modulo alpha conversion and casts.

\ErrMsg \errindex{Not equal}

\subsection{\tt is\_evar \term
\tacindex{is\_evar}
\label{isevar}}

This tactic applies to any goal. It checks whether its argument is an
existential variable. Existential variables are uninstantiated
variables generated by e.g. {\tt eapply} (see Section~\ref{apply}).

\ErrMsg \errindex{Not an evar}

\subsection{\tt has\_evar \term
\tacindex{has\_evar}
\label{hasevar}}

This tactic applies to any goal. It checks whether its argument has an
existential variable as a subterm. Unlike {\tt context} patterns
combined with {\tt is\_evar}, this tactic scans all subterms,
including those under binders.

\ErrMsg \errindex{No evars}

\subsection{\tt is\_var \term
\tacindex{is\_var}
\label{isvar}}

This tactic applies to any goal. It checks whether its argument is a
variable or hypothesis in the current goal context or in the opened sections.

\ErrMsg \errindex{Not a variable or hypothesis}

\subsection{Bindings list
\index{Binding list}
\label{Binding-list}}

Tactics that take a term as argument may also support a bindings list, so
as to instantiate some parameters of the term by name or position.
The general form of a term equipped with a bindings list is {\tt
{\term} with {\bindinglist}} where {\bindinglist} may be of two
different forms:

\begin{itemize}
\item In a bindings list of the form {\tt (\vref$_1$ := \term$_1$)
  \dots\ (\vref$_n$ := \term$_n$)}, {\vref} is either an {\ident} or a
  {\num}. The references are determined according to the type of
  {\term}. If \vref$_i$ is an identifier, this identifier has to be
  bound in the type of {\term} and the binding provides the tactic
  with an instance for the parameter of this name.  If \vref$_i$ is
  some number $n$, this number denotes the $n$-th non dependent
  premise of the {\term}, as determined by the type of {\term}.

  \ErrMsg \errindex{No such binder}

\item A bindings list can also be a simple list of terms {\tt
  \term$_1$ \dots\term$_n$}. In that case the references to
  which these terms correspond are determined by the tactic. In case
  of {\tt induction}, {\tt destruct}, {\tt elim} and {\tt case} (see
  Section~\ref{elim}) the terms have to provide instances for all the
  dependent products in the type of \term\ while in the case of {\tt
  apply}, or of {\tt constructor} and its variants, only instances for
  the dependent products which are not bound in the conclusion of the
  type are required.

  \ErrMsg \errindex{Not the right number of missing arguments}
\end{itemize}

\subsection{Occurrences sets and occurrences clauses}
\label{Occurrences clauses}
\index{Occurrences clauses}

An occurrences clause is a modifier to some tactics that obeys the
following syntax:

$\!\!\!$\begin{tabular}{lcl}
{\occclause} & ::= & {\tt in} {\occgoalset} \\
{\occgoalset} & ::= &
    \zeroone{{\ident$_1$} \zeroone{\atoccurrences} {\tt ,} \\
&   & {\dots} {\tt ,}\\
&   & {\ident$_m$} \zeroone{\atoccurrences}}\\
&   & \zeroone{{\tt |-} \zeroone{{\tt *} \zeroone{\atoccurrences}}}\\
& | &
    {\tt *} {\tt |-} \zeroone{{\tt *} \zeroone{\atoccurrences}}\\
& | &
    {\tt *}\\
{\atoccurrences} & ::= & {\tt at} {\occlist}\\
{\occlist} & ::= & \zeroone{{\tt -}} {\num$_1$} \dots\ {\num$_n$}
\end{tabular}

The role of an occurrence clause is to select a set of occurrences of
a {\term} in a goal. In the first case, the {{\ident$_i$}
\zeroone{{\tt at} {\num$_1^i$} \dots\ {\num$_{n_i}^i$}}} parts
indicate that occurrences have to be selected in the hypotheses named
{\ident$_i$}.  If no numbers are given for hypothesis {\ident$_i$},
then all occurrences of {\term} in the hypothesis are selected. If
numbers are given, they refer to occurrences of {\term} when the term
is printed using option {\tt Set Printing All} (see
Section~\ref{SetPrintingAll}), counting from left to right. In
particular, occurrences of {\term} in implicit arguments (see
Section~\ref{Implicit Arguments}) or coercions (see
Section~\ref{Coercions}) are counted.

If a minus sign is given between {\tt at} and the list of occurrences,
it negates the condition so that the clause denotes all the occurrences except
the ones explicitly mentioned after the minus sign.

As an exception to the left-to-right order, the occurrences in the
{\tt return} subexpression of a {\tt match} are considered {\em
before} the occurrences in the matched term.

In the second case, the {\tt *} on the left of {\tt |-} means that
all occurrences of {\term} are selected in every hypothesis.

In the first and second case, if {\tt *} is mentioned on the right of
{\tt |-}, the occurrences of the conclusion of the goal have to be
selected. If some numbers are given, then only the occurrences denoted
by these numbers are selected. In no numbers are given, all
occurrences of {\term} in the goal are selected.

Finally, the last notation is an abbreviation for {\tt * |- *}. Note
also that {\tt |-} is optional in the first case when no {\tt *} is
given.

Here are some tactics that understand occurrences clauses:
{\tt set}, {\tt remember}, {\tt induction}, {\tt destruct}.

\SeeAlso~Sections~\ref{tactic:set}, \ref{Tac-induction}, \ref{SetPrintingAll}.


\section{Negation and contradiction}

\subsection{\tt absurd \term
\tacindex{absurd}
\label{absurd}}

This tactic applies to any goal. The argument {\term} is any
proposition {\tt P} of type {\tt Prop}. This tactic applies {\tt
  False} elimination, that is it deduces the current goal from {\tt
  False}, and generates as subgoals {\tt $\sim$P} and {\tt P}. It is
very useful in proofs by cases, where some cases are impossible. In
most cases, \texttt{P} or $\sim$\texttt{P} is one of the hypotheses of
the local context.

\subsection{\tt contradiction
\label{contradiction}
\tacindex{contradiction}}

This tactic applies to any goal. The {\tt contradiction} tactic
attempts to find in the current context (after all {\tt intros}) one
hypothesis which is equivalent to {\tt False}. It permits to prune 
irrelevant cases. This tactic is a macro for the tactics sequence 
{\tt intros; elimtype False; assumption}. 

\begin{ErrMsgs}
\item \errindex{No such assumption}
\end{ErrMsgs}

\begin{Variants}
\item {\tt contradiction \ident}

The proof of {\tt False} is searched in the hypothesis named \ident.
\end{Variants}

\subsection {\tt contradict \ident}
\label{contradict}
\tacindex{contradict}

This tactic allows to manipulate negated hypothesis and goals. The
name \ident\ should correspond to a hypothesis. With 
{\tt contradict H}, the current goal and context is transformed in
the following way: 
\begin{itemize}
\item  {\tt H:$\neg$A $\vd$  B} \ becomes \ {\tt $\vd$ A}
\item  {\tt H:$\neg$A $\vd$ $\neg$B} \  becomes \ {\tt H: B $\vd$  A }
\item  {\tt H: A $\vd$  B} \ becomes \ {\tt $\vd$ $\neg$A}
\item  {\tt H: A $\vd$ $\neg$B} \ becomes \ {\tt H: B $\vd$ $\neg$A}
\end{itemize}

\subsection{\tt exfalso}
\label{exfalso}
\tacindex{exfalso}

This tactic implements the ``ex falso quodlibet'' logical principle:
an elimination of {\tt False} is performed on the current goal, and the
user is then required to prove that {\tt False} is indeed provable in
the current context. This tactic is a macro for {\tt elimtype False}.

\section{Conversion tactics
\index{Conversion tactics}
\label{Conversion-tactics}}

This set of tactics implements different specialized usages of the
tactic \texttt{change}.

All conversion tactics (including \texttt{change}) can be
parameterized by the parts of the goal where the conversion can
occur. This is done using \emph{goal clauses} which consists in a list
of hypotheses and, optionally, of a reference to the conclusion of the
goal. For defined hypothesis it is possible to specify if the
conversion should occur on the type part, the body part or both
(default).

\index{Clauses}
\index{Goal clauses}
Goal clauses are written after a conversion tactic (tactics
\texttt{set}~\ref{tactic:set},          \texttt{rewrite}~\ref{rewrite},
\texttt{replace}~\ref{tactic:replace}                               and
\texttt{autorewrite}~\ref{tactic:autorewrite} also use goal clauses)  and
are introduced by  the keyword \texttt{in}. If no goal clause is provided,
the default is to perform the conversion only in the conclusion.

The syntax and description of the various goal clauses is the following:
\begin{description}
\item[]\texttt{in {\ident}$_1$ $\ldots$ {\ident}$_n$ |- } only in hypotheses {\ident}$_1$
  \ldots {\ident}$_n$
\item[]\texttt{in {\ident}$_1$ $\ldots$ {\ident}$_n$ |- *} in hypotheses {\ident}$_1$ \ldots
  {\ident}$_n$ and in the conclusion
\item[]\texttt{in * |-} in every hypothesis
\item[]\texttt{in *} (equivalent to \texttt{in * |- *}) everywhere
\item[]\texttt{in (type of {\ident}$_1$) (value of {\ident}$_2$) $\ldots$ |-} in
  type part of {\ident}$_1$, in the value part of {\ident}$_2$, etc. 
\end{description}

For backward compatibility, the notation \texttt{in}~{\ident}$_1$\ldots {\ident}$_n$
performs the conversion in hypotheses {\ident}$_1$\ldots {\ident}$_n$.

%%%%%%%%%%%%%%%%%%%%%%%%%%%%%%%%%%
%voir reduction__conv_x : histoires d'univers.
%%%%%%%%%%%%%%%%%%%%%%%%%%%%%%%%%%

\subsection[{\tt cbv \flag$_1$ \dots\ \flag$_n$}, {\tt lazy \flag$_1$
\dots\ \flag$_n$} and {\tt compute}]
{{\tt cbv \flag$_1$ \dots\ \flag$_n$}, {\tt lazy \flag$_1$
\dots\ \flag$_n$} and {\tt compute}
\tacindex{cbv}
\tacindex{lazy}
\tacindex{compute}
\tacindex{vm\_compute}\label{vmcompute}}

These parameterized reduction tactics apply to any goal and perform
the normalization of the goal according to the specified flags. In
correspondence with the kinds of reduction considered in \Coq\, namely
$\beta$ (reduction of functional application), $\delta$ (unfolding of
transparent constants, see \ref{Transparent}), $\iota$ (reduction of
pattern-matching over a constructed term, and unfolding of {\tt fix}
and {\tt cofix} expressions) and $\zeta$ (contraction of local
definitions), the flag are either {\tt beta}, {\tt delta}, {\tt iota}
or {\tt zeta}. The {\tt delta} flag itself can be refined into {\tt
delta [\qualid$_1$\ldots\qualid$_k$]} or {\tt delta
-[\qualid$_1$\ldots\qualid$_k$]}, restricting in the first case the
constants to unfold to the constants listed, and restricting in the
second case the constant to unfold to all but the ones explicitly
mentioned. Notice that the {\tt delta} flag does not apply to
variables bound by a let-in construction inside the term itself (use
here the {\tt zeta} flag). In any cases, opaque constants are not
unfolded (see Section~\ref{Opaque}).

The goal may be normalized with two strategies: {\em lazy} ({\tt lazy}
tactic), or {\em call-by-value} ({\tt cbv} tactic). The lazy strategy
is a call-by-need strategy, with sharing of reductions: the arguments of a
function call are partially evaluated only when necessary, and if an
argument is used several times then it is computed only once. This
reduction is efficient for reducing expressions with dead code. For
instance, the proofs of a proposition {\tt exists~$x$. $P(x)$} reduce to a
pair of a witness $t$, and a proof that $t$ satisfies the predicate
$P$. Most of the time, $t$ may be computed without computing the proof
of $P(t)$, thanks to the lazy strategy.

The call-by-value strategy is the one used in ML languages: the
arguments of a function call are evaluated first, using a weak
reduction (no reduction under the $\lambda$-abstractions). Despite the
lazy strategy always performs fewer reductions than the call-by-value
strategy, the latter is generally more efficient for evaluating purely
computational expressions (i.e. with few dead code).

\begin{Variants}
\item {\tt compute} \tacindex{compute}\\
      {\tt cbv}
  
  These are synonyms for {\tt cbv beta delta iota zeta}.

\item {\tt lazy}
  
  This is a synonym for {\tt lazy beta delta iota zeta}.

\item {\tt compute [\qualid$_1$\ldots\qualid$_k$]}\\
      {\tt cbv [\qualid$_1$\ldots\qualid$_k$]}

  These are synonyms of {\tt cbv beta delta
  [\qualid$_1$\ldots\qualid$_k$] iota zeta}.
  
\item {\tt compute -[\qualid$_1$\ldots\qualid$_k$]}\\
      {\tt cbv -[\qualid$_1$\ldots\qualid$_k$]}

  These are synonyms of {\tt cbv beta delta
  -[\qualid$_1$\ldots\qualid$_k$] iota zeta}.

\item {\tt lazy [\qualid$_1$\ldots\qualid$_k$]}\\
      {\tt lazy -[\qualid$_1$\ldots\qualid$_k$]}

  These are respectively synonyms of {\tt cbv beta delta
  [\qualid$_1$\ldots\qualid$_k$] iota zeta} and {\tt cbv beta delta
  -[\qualid$_1$\ldots\qualid$_k$] iota zeta}.

\item {\tt vm\_compute} \tacindex{vm\_compute}

  This tactic evaluates the goal using the optimized call-by-value
  evaluation bytecode-based virtual machine. This algorithm is
  dramatically more efficient than the algorithm used for the {\tt
  cbv} tactic, but it cannot be fine-tuned. It is specially
  interesting for full evaluation of algebraic objects. This includes
  the case of reflexion-based tactics.

\end{Variants}

% Obsolete? Anyway not very important message
%\begin{ErrMsgs}
%\item \errindex{Delta must be specified before}
%  
%  A list of constants appeared before the {\tt delta} flag.
%\end{ErrMsgs}


\subsection{{\tt red}
\tacindex{red}}

This tactic applies to a goal which has the form {\tt
  forall (x:T1)\dots(xk:Tk), c t1 \dots\ tn} where {\tt c} is a constant.  If
{\tt c} is transparent then it replaces {\tt c} with its definition
(say {\tt t}) and then reduces {\tt (t t1 \dots\ tn)} according to
$\beta\iota\zeta$-reduction rules.

\begin{ErrMsgs}
\item \errindex{Not reducible}
\end{ErrMsgs}

\subsection{{\tt hnf}
\tacindex{hnf}}

This tactic applies to any goal. It replaces the current goal with its
head normal form according to the $\beta\delta\iota\zeta$-reduction
rules, i.e.  it reduces the head of the goal until it becomes a
product or an irreducible term.

\Example
The term \verb+forall n:nat, (plus (S n) (S n))+ is not reduced by {\tt hnf}.

\Rem The $\delta$ rule only applies to transparent constants
(see Section~\ref{Opaque} on transparency and opacity).

\subsection{\tt simpl
\tacindex{simpl}}

This tactic applies to any goal. The tactic {\tt simpl} first applies
$\beta\iota$-reduction rule.  Then it expands transparent constants
and tries to reduce {\tt T'} according, once more, to $\beta\iota$
rules. But when the $\iota$ rule is not applicable then possible
$\delta$-reductions are not applied.  For instance trying to use {\tt
simpl} on {\tt (plus n O)=n} changes nothing.  Notice that only
transparent constants whose name can be reused as such in the
recursive calls are possibly unfolded. For instance a constant defined
by {\tt plus' := plus} is possibly unfolded and reused in the
recursive calls, but a constant such as {\tt succ := plus (S O)} is
never unfolded.

\tacindex{simpl \dots\ in}
\begin{Variants}
\item {\tt simpl {\term}}
  
  This applies {\tt simpl} only to the occurrences of {\term} in the
  current goal.

\item {\tt simpl {\term} at \num$_1$ \dots\ \num$_i$}
  
  This applies {\tt simpl} only to the \num$_1$, \dots, \num$_i$
  occurrences of {\term} in the current goal.

  \ErrMsg {\tt Too few occurrences}

\item {\tt simpl {\ident}}
  
  This applies {\tt simpl} only to the applicative subterms whose head
  occurrence is {\ident}.

\item {\tt simpl {\ident} at \num$_1$ \dots\ \num$_i$}
  
  This applies {\tt simpl} only to the \num$_1$, \dots, \num$_i$
applicative subterms whose head occurrence is {\ident}.

\end{Variants}

\subsection{\tt unfold \qualid
\tacindex{unfold}
\label{unfold}}

This tactic applies to any goal. The argument {\qualid} must denote a
defined transparent constant or local definition (see Sections~\ref{Basic-definitions} and~\ref{Transparent}).  The tactic {\tt
  unfold} applies the $\delta$ rule to each occurrence of the constant
to which {\qualid} refers in the current goal and then replaces it
with its $\beta\iota$-normal form.

\begin{ErrMsgs}
\item {\qualid} \errindex{does not denote an evaluable constant}

\end{ErrMsgs}

\begin{Variants}
\item {\tt unfold {\qualid}$_1$, \dots, \qualid$_n$}
  \tacindex{unfold \dots\ in}
  
  Replaces {\em simultaneously} {\qualid}$_1$, \dots, {\qualid}$_n$
  with their definitions and replaces the current goal with its
  $\beta\iota$ normal form.

\item {\tt unfold {\qualid}$_1$ at \num$_1^1$, \dots, \num$_i^1$,
\dots,\ \qualid$_n$ at \num$_1^n$ \dots\ \num$_j^n$}
  
  The lists \num$_1^1$, \dots, \num$_i^1$ and \num$_1^n$, \dots,
  \num$_j^n$ specify the occurrences of {\qualid}$_1$, \dots,
  \qualid$_n$ to be unfolded. Occurrences are located from left to
  right.

  \ErrMsg {\tt bad occurrence number of {\qualid}$_i$}

  \ErrMsg {\qualid}$_i$ {\tt does not occur}

\item {\tt unfold {\qstring}}

  If {\qstring} denotes the discriminating symbol of a notation (e.g. {\tt
  "+"}) or an expression defining a notation (e.g. \verb!"_ + _"!), and
  this notation refers to an unfoldable constant, then the tactic
  unfolds it.

\item {\tt unfold {\qstring}\%{\delimkey}}

  This is variant of {\tt unfold {\qstring}} where {\qstring} gets its
  interpretation from the scope bound to the delimiting key
  {\delimkey} instead of its default interpretation (see
  Section~\ref{scopechange}).

\item {\tt unfold \qualidorstring$_1$ at \num$_1^1$, \dots, \num$_i^1$,
\dots,\ \qualidorstring$_n$ at \num$_1^n$ \dots\ \num$_j^n$}

  This is the most general form, where {\qualidorstring} is either a
  {\qualid} or a {\qstring} referring to a notation.

\end{Variants}

\subsection{{\tt fold} \term
\tacindex{fold}}

This tactic applies to any goal. The term \term\ is reduced using the {\tt red}
tactic. Every occurrence of the resulting term in the goal is then
replaced by \term.

\begin{Variants}
\item {\tt fold} \term$_1$ \dots\ \term$_n$ 
  
  Equivalent to {\tt fold} \term$_1${\tt;}\ldots{\tt; fold} \term$_n$.
\end{Variants}

\subsection{{\tt pattern {\term}}
\tacindex{pattern}
\label{pattern}}

This command applies to any goal. The argument {\term} must be a free
subterm of the current goal.  The command {\tt pattern} performs
$\beta$-expansion (the inverse of $\bt$-reduction) of the current goal
(say \T) by
\begin{enumerate}
\item replacing all occurrences of {\term} in {\T} with a fresh variable
\item abstracting this variable
\item applying the abstracted goal to {\term}
\end{enumerate}

For instance, if the current goal $T$ is expressible has $\phi(t)$
where the notation captures all the instances of $t$ in $\phi(t)$,
then {\tt pattern $t$} transforms it into {\tt (fun x:$A$ => $\phi(${\tt
x}$)$) $t$}.  This command can be used, for instance, when the tactic
{\tt apply} fails on matching.

\begin{Variants}
\item {\tt pattern {\term} at {\num$_1$} \dots\ {\num$_n$}}
  
  Only the occurrences {\num$_1$} \dots\ {\num$_n$} of {\term} are
  considered for $\beta$-expansion. Occurrences are located from left
  to right.

\item {\tt pattern {\term} at - {\num$_1$} \dots\ {\num$_n$}}
  
  All occurrences except the occurrences of indexes {\num$_1$} \dots\
  {\num$_n$} of {\term} are considered for
  $\beta$-expansion. Occurrences are located from left to right.

\item {\tt pattern {\term$_1$}, \dots, {\term$_m$}}
  
  Starting from a goal $\phi(t_1 \dots\ t_m)$, the tactic
   {\tt pattern $t_1$, \dots,\ $t_m$} generates the equivalent goal {\tt
   (fun (x$_1$:$A_1$) \dots\ (x$_m$:$A_m$) => $\phi(${\tt x$_1$\dots\
   x$_m$}$)$) $t_1$ \dots\ $t_m$}.\\ If $t_i$ occurs in one of the
   generated types $A_j$ these occurrences will also be considered and
   possibly abstracted.

\item {\tt pattern {\term$_1$} at {\num$_1^1$} \dots\ {\num$_{n_1}^1$}, \dots,
    {\term$_m$} at {\num$_1^m$} \dots\ {\num$_{n_m}^m$}}
  
  This behaves as above but processing only the occurrences \num$_1^1$,
  \dots, \num$_i^1$ of \term$_1$, \dots, \num$_1^m$, \dots, \num$_j^m$
  of \term$_m$ starting from \term$_m$.

\item {\tt pattern} {\term$_1$} \zeroone{{\tt at \zeroone{-}} {\num$_1^1$} \dots\ {\num$_{n_1}^1$}} {\tt ,} \dots {\tt ,}
    {\term$_m$} \zeroone{{\tt at \zeroone{-}} {\num$_1^m$} \dots\ {\num$_{n_m}^m$}}
  
  This is the most general syntax that combines the different variants.

\end{Variants}

\subsection{Conversion tactics applied to hypotheses}

{\convtactic} {\tt in} \ident$_1$ \dots\ \ident$_n$ 

Applies the conversion tactic {\convtactic} to the
hypotheses \ident$_1$, \ldots, \ident$_n$. The tactic {\convtactic} is
any of the conversion tactics listed in this section. 

If \ident$_i$ is a local definition, then \ident$_i$ can be replaced
by (Type of \ident$_i$) to address not the body but the type of the
local definition. Example: {\tt unfold not in (Type of H1) (Type of H3).}

\begin{ErrMsgs}
\item \errindex{No such hypothesis} : {\ident}.
\end{ErrMsgs}


\section{Introductions}

Introduction tactics address goals which are inductive constants.
They are used when one guesses that the goal can be obtained with one
of its constructors' type.

\subsection{\tt constructor \num
\label{constructor}
\tacindex{constructor}}

This tactic applies to a goal such that the head of its conclusion is
an inductive constant (say {\tt I}).  The argument {\num} must be less
or equal to the numbers of constructor(s) of {\tt I}. Let {\tt ci} be
the {\tt i}-th constructor of {\tt I}, then {\tt constructor i} is
equivalent to {\tt intros; apply ci}.

\begin{ErrMsgs}
\item \errindex{Not an inductive product}
\item \errindex{Not enough constructors}
\end{ErrMsgs}

\begin{Variants}
\item \texttt{constructor} 
  
  This tries \texttt{constructor 1} then \texttt{constructor 2},
  \dots\ , then \texttt{constructor} \textit{n} where \textit{n} if
  the number of constructors of the head of the goal.

\item {\tt constructor \num~with} {\bindinglist}
  
  Let {\tt ci} be the {\tt i}-th constructor of {\tt I}, then {\tt
    constructor i with \bindinglist} is equivalent to {\tt intros;
    apply ci with \bindinglist}.

  \Warning the terms in the \bindinglist\ are checked
  in the context where {\tt constructor} is executed and not in the
  context where {\tt apply} is executed (the introductions are not
  taken into account).

% To document?
% \item {\tt constructor {\tactic}}

\item {\tt split}\tacindex{split}

  Applies if {\tt I} has only one constructor, typically in the case
  of conjunction $A\land B$. Then, it is equivalent to {\tt constructor 1}.

\item {\tt exists {\bindinglist}}\tacindex{exists} 

  Applies if {\tt I} has only one constructor, for instance in the
  case of existential quantification $\exists x\cdot P(x)$. 
  Then, it is equivalent to {\tt intros; constructor 1 with \bindinglist}.

\item {\tt exists \nelist{\bindinglist}{,}}

  This iteratively applies {\tt exists {\bindinglist}}.

\item {\tt left}\tacindex{left}\\
      {\tt right}\tacindex{right}

  Apply if {\tt I} has two constructors, for instance in the case of
  disjunction $A\lor B$. Then, they are respectively equivalent to {\tt
    constructor 1} and {\tt constructor 2}.
  
\item {\tt left with \bindinglist}\\
      {\tt right with \bindinglist}\\
      {\tt split with \bindinglist}
  
  As soon as the inductive type has the right number of constructors,
    these expressions are equivalent to calling {\tt
    constructor $i$ with \bindinglist} for the appropriate $i$.

\item \texttt{econstructor}\tacindex{econstructor}\\
      \texttt{eexists}\tacindex{eexists}\\
      \texttt{esplit}\tacindex{esplit}\\
      \texttt{eleft}\tacindex{eleft}\\
      \texttt{eright}\tacindex{eright}\\

  These tactics and their variants behave like \texttt{constructor},
  \texttt{exists}, \texttt{split}, \texttt{left}, \texttt{right} and
  their variants but they introduce existential variables instead of
  failing when the instantiation of a variable cannot be found (cf
  \texttt{eapply} and Section~\ref{eapply-example}).

\end{Variants}

\section[Induction and Case Analysis]{Induction and Case Analysis
\label{Tac-induction}}

The tactics presented in this section implement induction or case
analysis on inductive or coinductive objects (see
Section~\ref{Cic-inductive-definitions}).

\subsection{\tt induction \term
\tacindex{induction}}

This tactic applies to any goal. The type of the argument {\term} must
be an inductive constant. Then, the tactic {\tt induction}
generates subgoals, one for each possible form of {\term}, i.e. one
for each constructor of the inductive type.

The tactic {\tt induction} automatically replaces every occurrences
of {\term} in the conclusion and the hypotheses of the goal.  It
automatically adds induction hypotheses (using names of the form {\tt
  IHn1}) to the local context. If some hypothesis must not be taken
into account in the induction hypothesis, then it needs to be removed
first (you can also use the tactics {\tt elim} or {\tt simple induction},
see below).

There are particular cases:

\begin{itemize}

\item If {\term} is an identifier {\ident} denoting a quantified
variable of the conclusion of the goal, then {\tt induction {\ident}}
behaves as {\tt intros until {\ident}; induction {\ident}}.

\item If {\term} is a {\num}, then {\tt induction {\num}} behaves as
{\tt intros until {\num}} followed by {\tt induction} applied to the
last introduced hypothesis.

\Rem For simple induction on a numeral, use syntax {\tt induction
({\num})} (not very interesting anyway).

\end{itemize}

\Example

\begin{coq_example}
Lemma induction_test : forall n:nat, n = n -> n <= n.
intros n H.
induction n.
\end{coq_example}

\begin{ErrMsgs}
\item \errindex{Not an inductive product}
\item \errindex{Unable to find an instance for the variables
{\ident} \ldots {\ident}}
  
  Use in this case 
  the variant {\tt elim \dots\ with \dots} below.
\end{ErrMsgs}

\begin{Variants}
\item{\tt induction {\term} as {\disjconjintropattern}}
  
  This behaves as {\tt induction {\term}} but uses the names in
  {\disjconjintropattern} to name the variables introduced in the context.
  The {\disjconjintropattern} must typically be of the form
  {\tt [} $p_{11}$ \ldots
  $p_{1n_1}$ {\tt |} {\ldots} {\tt |} $p_{m1}$ \ldots $p_{mn_m}$ {\tt
    ]} with $m$ being the number of constructors of the type of
  {\term}. Each variable introduced by {\tt induction} in the context
  of the $i^{th}$ goal gets its name from the list $p_{i1}$ \ldots
  $p_{in_i}$ in order. If there are not enough names, {\tt induction}
  invents names for the remaining variables to introduce. More
  generally, the $p_{ij}$ can be any disjunctive/conjunctive
  introduction pattern (see Section~\ref{intros-pattern}). For instance,
  for an inductive type with one constructor, the pattern notation
  {\tt ($p_{1}$,\ldots,$p_{n}$)} can be used instead of
  {\tt [} $p_{1}$ \ldots $p_{n}$ {\tt ]}.

\item{\tt induction {\term} as {\namingintropattern}}

  This behaves as {\tt induction {\term}} but adds an equation between
  {\term} and the value that {\term} takes in each of the induction
  case.  The name of the equation is built according to
  {\namingintropattern} which can be an identifier, a ``?'', etc, as
  indicated in Section~\ref{intros-pattern}.

\item{\tt induction {\term} as {\namingintropattern} {\disjconjintropattern}}

  This combines the two previous forms.

\item{\tt induction {\term} with \bindinglist}

  This behaves like \texttt{induction {\term}} providing explicit
  instances for the premises of the type of {\term} (see the syntax of
  bindings in Section~\ref{Binding-list}).

\item{\tt einduction {\term}\tacindex{einduction}}

  This tactic behaves like \texttt{induction {\term}} excepts that it
  does not fail if some dependent premise of the type of {\term} is
  not inferable. Instead, the unresolved premises are posed as
  existential variables to be inferred later, in the same way as {\tt
  eapply} does (see Section~\ref{eapply-example}).

\item {\tt induction {\term$_1$} using {\term$_2$}}

  This behaves as {\tt induction {\term$_1$}} but using {\term$_2$} as
  induction scheme. It does not expect the conclusion of the type of
  {\term$_1$} to be inductive.

\item {\tt induction {\term$_1$} using {\term$_2$} with {\bindinglist}}

  This behaves as {\tt induction {\term$_1$} using {\term$_2$}} but
  also providing instances for the premises of the type of {\term$_2$}.

\item \texttt{induction {\term}$_1$ $\ldots$ {\term}$_n$ using {\qualid}}

  This syntax is used for the case {\qualid} denotes an induction principle
  with complex predicates as the induction principles generated by
  {\tt Function} or {\tt Functional Scheme} may be.

\item \texttt{induction {\term} in {\occgoalset}}

  This syntax is used for selecting which occurrences of {\term} the
  induction has to be carried on. The {\tt in {\atoccurrences}} clause is an
  occurrence clause whose syntax and behavior is described in
  Section~\ref{Occurrences clauses}.

  When an occurrence clause is given, an equation between {\term} and
  the value it gets in each case of the induction is added to the
  context of the subgoals corresponding to the induction cases (even
  if no clause {\tt as {\namingintropattern}} is given).

\item {\tt induction {\term$_1$} with {\bindinglist$_1$} as {\namingintropattern} {\disjconjintropattern} using {\term$_2$} with {\bindinglist$_2$} in {\occgoalset}}\\
     {\tt einduction {\term$_1$} with {\bindinglist$_1$} as {\namingintropattern} {\disjconjintropattern} using {\term$_2$} with {\bindinglist$_2$} in {\occgoalset}}

  These are the most general forms of {\tt induction} and {\tt
  einduction}.  It combines the effects of the {\tt with}, {\tt as},
  {\tt using}, and {\tt in} clauses.

\item {\tt elim \term}\label{elim}
  
  This is a more basic induction tactic.  Again, the type of the
  argument {\term} must be an inductive type. Then, according to
  the type of the goal, the tactic {\tt elim} chooses the appropriate
  destructor and applies it as the tactic {\tt apply}
  would do. For instance, if the proof context contains {\tt
  n:nat} and the current goal is {\tt T} of type {\tt
  Prop}, then {\tt elim n} is equivalent to {\tt apply nat\_ind with
  (n:=n)}.  The tactic {\tt elim} does not modify the context of
  the goal, neither introduces the induction loading into the context
  of hypotheses.

  More generally, {\tt elim \term} also works when the type of {\term}
  is a statement with premises and whose conclusion is inductive.  In
  that case the tactic performs induction on the conclusion of the
  type of {\term} and leaves the non-dependent premises of the type as
  subgoals.  In the case of dependent products, the tactic tries to
  find an instance for which the elimination lemma applies and fails
  otherwise.

\item {\tt elim {\term} with {\bindinglist}}
  
  Allows to give explicit instances to the premises of the type
  of {\term} (see Section~\ref{Binding-list}).

\item{\tt eelim {\term}\tacindex{eelim}}

  In case the type of {\term} has dependent premises, this turns them into
  existential variables to be resolved later on.

\item{\tt elim {\term$_1$} using {\term$_2$}}\\
     {\tt elim {\term$_1$} using {\term$_2$} with {\bindinglist}\tacindex{elim \dots\ using}}

Allows the user to give explicitly an elimination predicate
{\term$_2$} which is not the standard one for the underlying inductive
type of {\term$_1$}. The {\bindinglist} clause allows to
instantiate premises of the type of {\term$_2$}.

\item{\tt elim {\term$_1$} with {\bindinglist$_1$} using {\term$_2$} with {\bindinglist$_2$}}\\
     {\tt eelim {\term$_1$} with {\bindinglist$_1$} using {\term$_2$} with {\bindinglist$_2$}}

  These are the most general forms of {\tt elim} and {\tt eelim}.  It
  combines the effects of the {\tt using} clause and of the two uses
  of the {\tt with} clause.

\item {\tt elimtype \form}\tacindex{elimtype}
  
  The argument {\form} must be inductively defined. {\tt elimtype I}
  is equivalent to {\tt cut I. intro H{\rm\sl n}; elim H{\rm\sl n};
    clear H{\rm\sl n}}. Therefore the hypothesis {\tt H{\rm\sl n}} will
  not appear in the context(s) of the subgoal(s).  Conversely, if {\tt
    t} is a term of (inductive) type {\tt I} and which does not occur
  in the goal then {\tt elim t} is equivalent to {\tt elimtype I; 2:
    exact t.}

\item {\tt simple induction \ident}\tacindex{simple induction}
  
  This tactic behaves as {\tt intros until
    {\ident}; elim {\tt {\ident}}} when {\ident} is a quantified
  variable of the goal.

\item {\tt simple induction {\num}}
  
  This tactic behaves as {\tt intros until
    {\num}; elim {\tt {\ident}}} where {\ident} is the name given by
  {\tt intros until {\num}} to the {\num}-th non-dependent premise of
  the goal.

%% \item {\tt simple induction {\term}}\tacindex{simple induction}
  
%%   If {\term} is an {\ident} corresponding to a quantified variable of
%%   the goal then the tactic behaves as {\tt intros until {\ident}; elim
%%   {\tt {\ident}}}.  If {\term} is a {\num} then the tactic behaves as
%%   {\tt intros until {\ident}; elim {\tt {\ident}}}.  Otherwise, it is
%%   a synonym for {\tt elim {\term}}.

%%   \Rem For simple induction on a numeral, use syntax {\tt simple
%%   induction ({\num})}.

\end{Variants}

\subsection{\tt destruct \term
\tacindex{destruct}}
\label{destruct}

The tactic {\tt destruct} is used to perform case analysis without
recursion. Its behavior is similar to {\tt induction} except
that no induction hypothesis is generated.  It applies to any goal and
the type of {\term} must be inductively defined. There are particular cases:

\begin{itemize}

\item If {\term} is an identifier {\ident} denoting a quantified
variable of the conclusion of the goal, then {\tt destruct {\ident}}
behaves as {\tt intros until {\ident}; destruct {\ident}}.

\item If {\term} is a {\num}, then {\tt destruct {\num}} behaves as
{\tt intros until {\num}} followed by {\tt destruct} applied to the
last introduced hypothesis.

\Rem For destruction of a numeral, use syntax {\tt destruct
({\num})} (not very interesting anyway).

\end{itemize}

\begin{Variants}
\item{\tt destruct {\term} as {\disjconjintropattern}}
  
  This behaves as {\tt destruct {\term}} but uses the names in
  {\intropattern} to name the variables introduced in the context.
  The {\intropattern} must have the form {\tt [} $p_{11}$ \ldots
  $p_{1n_1}$ {\tt |} {\ldots} {\tt |} $p_{m1}$ \ldots $p_{mn_m}$ {\tt
    ]} with $m$ being the number of constructors of the type of
  {\term}. Each variable introduced by {\tt destruct} in the context
  of the $i^{th}$ goal gets its name from the list $p_{i1}$ \ldots
  $p_{in_i}$ in order. If there are not enough names, {\tt destruct}
  invents names for the remaining variables to introduce. More
  generally, the $p_{ij}$ can be any disjunctive/conjunctive
  introduction pattern (see Section~\ref{intros-pattern}). This
  provides a concise notation for nested destruction.

%  It is recommended to use this variant of {\tt destruct} for 
%  robust proof scripts.

\item{\tt destruct {\term} as {\disjconjintropattern} \_eqn}

  This behaves as {\tt destruct {\term}} but adds an equation between
  {\term} and the value that {\term} takes in each of the possible
  cases.  The name of the equation is chosen by Coq. If  
  {\disjconjintropattern} is simply {\tt []}, it is automatically considered
  as a disjunctive pattern of the appropriate size.

\item{\tt destruct {\term} as {\disjconjintropattern} \_eqn: {\namingintropattern}}

  This behaves as {\tt destruct {\term} as
    {\disjconjintropattern} \_eqn} but use {\namingintropattern} to
  name the equation (see Section~\ref{intros-pattern}). Note that spaces 
  can generally be removed around {\tt \_eqn}.

\item{\tt destruct {\term} with \bindinglist}

  This behaves like \texttt{destruct {\term}} providing explicit
  instances for the dependent premises of the type of {\term} (see
  syntax of bindings in Section~\ref{Binding-list}).

\item{\tt edestruct {\term}\tacindex{edestruct}}

  This tactic behaves like \texttt{destruct {\term}} excepts that it
  does not fail if the instance of a dependent premises of the type of
  {\term} is not inferable. Instead, the unresolved instances are left
  as existential variables to be inferred later, in the same way as
  {\tt eapply} does (see Section~\ref{eapply-example}).

\item{\tt destruct {\term$_1$} using {\term$_2$}}\\
     {\tt destruct {\term$_1$} using {\term$_2$} with {\bindinglist}}

  These are synonyms of {\tt induction {\term$_1$} using {\term$_2$}} and
  {\tt induction {\term$_1$} using {\term$_2$} with {\bindinglist}}.

\item \texttt{destruct {\term} in {\occgoalset}}

  This syntax is used for selecting which occurrences of {\term} the
  case analysis has to be done on. The {\tt in {\occgoalset}} clause is an
  occurrence clause whose syntax and behavior is described in
  Section~\ref{Occurrences clauses}.

  When an occurrence clause is given, an equation between {\term} and
  the value it gets in each case of the analysis is added to the
  context of the subgoals corresponding to the cases (even
  if no clause {\tt as {\namingintropattern}} is given).

\item{\tt destruct {\term$_1$} with {\bindinglist$_1$} as {\disjconjintropattern} \_eqn: {\namingintropattern}  using {\term$_2$} with {\bindinglist$_2$} in {\occgoalset}}\\
     {\tt edestruct {\term$_1$} with {\bindinglist$_1$} as {\disjconjintropattern} \_eqn: {\namingintropattern} using {\term$_2$} with {\bindinglist$_2$} in {\occgoalset}}

  These are the general forms of {\tt destruct} and {\tt edestruct}.
  They combine the effects of the {\tt with}, {\tt as}, {\tt using},
  and {\tt in} clauses.

\item{\tt case \term}\label{case}\tacindex{case}
  
  The tactic {\tt case} is a more basic tactic to perform case
  analysis without recursion. It behaves as {\tt elim \term} but using
  a case-analysis elimination principle and not a recursive one.

\item{\tt case\_eq \term}\label{case_eq}\tacindex{case\_eq}

 The tactic {\tt case\_eq} is a variant of the {\tt case} tactic that
 allow to perform case analysis on a term without completely
 forgetting its original form. This is done by generating equalities
 between the original form of the term and the outcomes of the case
 analysis. The effect of this tactic is similar to the effect of {\tt
 destruct {\term} in |- *} with the exception that no new hypotheses 
 are introduced in the context.

\item {\tt case {\term} with {\bindinglist}}

  Analogous to {\tt elim {\term} with {\bindinglist}} above.

\item{\tt ecase {\term}\tacindex{ecase}}\\
  {\tt ecase {\term} with {\bindinglist}}
  
  In case the type of {\term} has dependent premises, or dependent
  premises whose values are not inferable from the {\tt with
  {\bindinglist}} clause, {\tt ecase} turns them into existential
  variables to be resolved later on.

\item {\tt simple destruct \ident}\tacindex{simple destruct}
  
  This tactic behaves as {\tt intros until
    {\ident}; case {\tt {\ident}}} when {\ident} is a quantified
  variable of the goal.

\item {\tt simple destruct {\num}}
  
  This tactic behaves as {\tt intros until
    {\num}; case {\tt {\ident}}} where {\ident} is the name given by
  {\tt intros until {\num}} to the {\num}-th non-dependent premise of
  the goal.


\end{Variants}

\subsection{\tt intros {\intropattern} {\ldots} {\intropattern}
\label{intros-pattern}
\tacindex{intros \intropattern}}
\index{Introduction patterns}
\index{Naming introduction patterns}
\index{Disjunctive/conjunctive introduction patterns}

This extension of the tactic {\tt intros} combines introduction of
variables or hypotheses and case analysis. An {\em introduction pattern} is
either:
\begin{itemize}
\item A {\em naming introduction pattern}, i.e. either one of:
  \begin{itemize}
  \item the pattern \texttt{?}
  \item the pattern \texttt{?\ident}
  \item an identifier
  \end{itemize}
\item A {\em disjunctive/conjunctive introduction pattern}, i.e. either one of:
  \begin{itemize}
  \item a disjunction of lists of patterns:
  {\tt [$p_{11}$ {\ldots} $p_{1m_1}$ | {\ldots} | $p_{11}$ {\ldots} $p_{nm_n}$]}
  \item a conjunction of patterns: {\tt (} $p_1$ {\tt ,} {\ldots} {\tt ,} $p_n$ {\tt )}
  \item a list of patterns {\tt (} $p_1$\ {\tt \&}\ {\ldots}\ {\tt \&}\ $p_n$ {\tt )}
   for sequence of right-associative binary constructs
  \end{itemize}
\item the wildcard: {\tt \_}
\item the rewriting orientations: {\tt ->} or {\tt <-}
\end{itemize}

Assuming a goal of type {\tt $Q$ -> $P$} (non dependent product), or
of type {\tt forall $x$:$T$, $P$} (dependent product), the behavior of
{\tt intros $p$} is defined inductively over the structure of the
introduction pattern $p$:
\begin{itemize}
\item introduction on \texttt{?} performs the introduction, and lets {\Coq}
  choose a fresh name for the variable;
\item introduction on \texttt{?\ident} performs the introduction, and
  lets {\Coq} choose a fresh name for the variable based on {\ident};
\item introduction on \texttt{\ident} behaves as described in
  Section~\ref{intro};
\item introduction over a disjunction of list of patterns {\tt
  [$p_{11}$ {\ldots} $p_{1m_1}$ | {\ldots} | $p_{11}$ {\ldots}
    $p_{nm_n}$]} expects the product to be over an inductive type
  whose number of constructors is $n$ (or more generally over a type
  of conclusion an inductive type built from $n$ constructors,
  e.g. {\tt C -> A$\backslash$/B if $n=2$}): it destructs the introduced
  hypothesis as {\tt destruct} (see Section~\ref{destruct}) would and
  applies on each generated subgoal the corresponding tactic;
  \texttt{intros}~$p_{i1}$ {\ldots} $p_{im_i}$; if the disjunctive
  pattern is part of a sequence of patterns and is not the last
  pattern of the sequence, then {\Coq} completes the pattern so as all
  the argument of the constructors of the inductive type are
  introduced (for instance, the list of patterns {\tt [$\;$|$\;$] H}
  applied on goal {\tt forall x:nat, x=0 -> 0=x} behaves the same as
  the list of patterns {\tt [$\,$|$\,$?$\,$] H});
\item introduction over a conjunction of patterns {\tt ($p_1$, \ldots,
  $p_n$)} expects the goal to be a product over an inductive type $I$ with a
  single constructor that itself has at least $n$ arguments: it
  performs a case analysis over the hypothesis, as {\tt destruct}
  would, and applies the patterns $p_1$~\ldots~$p_n$ to the arguments
  of the constructor of $I$ (observe that {\tt ($p_1$, {\ldots},
  $p_n$)} is an alternative notation for {\tt [$p_1$ {\ldots}
  $p_n$]});
\item introduction via {\tt ( $p_1$ \& \ldots \& $p_n$ )}
  is a shortcut for introduction via
  {\tt ($p_1$,(\ldots,(\dots,$p_n$)\ldots))}; it expects the
  hypothesis to be a sequence of right-associative binary inductive 
  constructors such as {\tt conj} or {\tt ex\_intro}; for instance, an
  hypothesis with type {\tt A\verb|/\|exists x, B\verb|/\|C\verb|/\|D} can be
  introduced via pattern {\tt (a \& x \& b \& c \& d)};
\item introduction on the wildcard depends on whether the product is
  dependent or not: in the non dependent case, it erases the
  corresponding hypothesis (i.e. it behaves as an {\tt intro} followed
  by a {\tt clear}, cf Section~\ref{clear}) while in the dependent
  case, it succeeds and erases the variable only if the wildcard is
  part of a more complex list of introduction patterns that also
  erases the hypotheses depending on this variable;
\item introduction over {\tt ->} (respectively {\tt <-}) expects the
  hypothesis to be an equality and the right-hand-side (respectively
  the left-hand-side) is replaced by the left-hand-side (respectively
  the right-hand-side) in both the conclusion and the context of the goal;
  if moreover the term to substitute is a variable, the hypothesis is
  removed.
\end{itemize}

\Rem {\tt intros $p_1~\ldots~p_n$} is not equivalent to \texttt{intros
  $p_1$;\ldots; intros $p_n$} for the following reasons:
\begin{itemize}
\item A wildcard pattern never succeeds when applied isolated on a
  dependent product, while it succeeds as part of a list of
  introduction patterns if the hypotheses that depends on it are
  erased too.
\item A disjunctive or conjunctive pattern followed by an introduction
  pattern forces the introduction in the context of all arguments of
  the constructors before applying the next pattern while a terminal
  disjunctive or conjunctive pattern does not. Here is an example

\begin{coq_example}
Goal forall n:nat, n = 0 -> n = 0.
intros [ | ] H.
Show 2.
Undo.
intros [ | ]; intros H.
Show 2.
\end{coq_example}

\end{itemize}

\begin{coq_example}
Lemma intros_test : forall A B C:Prop, A \/ B /\ C -> (A -> C) -> C.
intros A B C [a| [_ c]] f.
apply (f a).
exact c.
Qed.
\end{coq_example}

%\subsection[\tt FixPoint \dots]{\tt FixPoint \dots\tacindex{Fixpoint}}
%Not yet documented.

\subsection{\tt double induction \ident$_1$ \ident$_2$}
%\tacindex{double induction}}
This tactic is deprecated and should be replaced by {\tt induction \ident$_1$; induction \ident$_2$} (or {\tt induction \ident$_1$; destruct \ident$_2$} depending on the exact needs).

%% This tactic applies to any goal. If the variables {\ident$_1$} and
%% {\ident$_2$} of the goal have an inductive type, then this tactic
%% performs double induction on these variables.  For instance, if the
%% current goal is \verb+forall n m:nat, P n m+ then, {\tt double induction n
%%   m} yields the four cases with their respective inductive hypotheses.

%% In particular, for proving \verb+(P (S n) (S m))+, the generated induction
%% hypotheses are \verb+(P (S n) m)+ and \verb+(m:nat)(P n m)+ (of the latter, 
%% \verb+(P n m)+ and \verb+(P n (S m))+ are derivable).

%% \Rem When the induction hypothesis \verb+(P (S n) m)+ is not
%% needed, {\tt induction \ident$_1$; destruct \ident$_2$} produces
%% more concise subgoals.

\begin{Variant}

\item {\tt double induction \num$_1$ \num$_2$}

This tactic is deprecated and should be replaced by {\tt induction
  \num$_1$; induction \num$_3$} where \num$_3$ is the result of
\num$_2$-\num$_1$.

%% This tactic applies to any goal. If the variables {\ident$_1$} and

%% This applies double induction on the \num$_1^{th}$ and \num$_2^{th}$ {\it
%% non dependent} premises of the goal. More generally, any combination of an
%% {\ident} and a {\num} is valid.

\end{Variant}

\subsection{\tt dependent induction \ident
  \tacindex{dependent induction}
  \label{DepInduction}}

The \emph{experimental} tactic \texttt{dependent induction} performs
induction-inversion on an instantiated inductive predicate.
One needs to first require the {\tt Coq.Program.Equality} module to use
this tactic. The tactic is based on the BasicElim tactic by Conor
McBride \cite{DBLP:conf/types/McBride00} and the work of Cristina Cornes
around inversion \cite{DBLP:conf/types/CornesT95}. From an instantiated
inductive predicate and a goal it generates an equivalent goal where the
hypothesis has been generalized over its indexes which are then
constrained by equalities to be the right instances. This permits to
state lemmas without resorting to manually adding these equalities and
still get enough information in the proofs. 
A simple example is the following:

\begin{coq_eval}
Reset Initial.
\end{coq_eval}
\begin{coq_example}
Lemma le_minus : forall n:nat, n < 1 -> n = 0.
intros n H ; induction H.
\end{coq_example}

Here we didn't get any information on the indexes to help fulfill this
proof. The problem is that when we use the \texttt{induction} tactic
we lose information on the hypothesis instance, notably that the second
argument is \texttt{1} here. Dependent induction solves this problem by
adding the corresponding equality to the context.

\begin{coq_eval}
Reset Initial.
\end{coq_eval}
\begin{coq_example}
Require Import Coq.Program.Equality.
Lemma le_minus : forall n:nat, n < 1 -> n = 0.
intros n H ; dependent induction H.
\end{coq_example}

The subgoal is cleaned up as the tactic tries to automatically
simplify the subgoals with respect to the generated equalities.
In this enriched context it becomes possible to solve this subgoal.
\begin{coq_example}
reflexivity.
\end{coq_example}

Now we are in a contradictory context and the proof can be solved.
\begin{coq_example}
inversion H.
\end{coq_example}

This technique works with any inductive predicate.
In fact, the \texttt{dependent induction} tactic is just a wrapper around
the \texttt{induction} tactic. One can make its own variant by just
writing a new tactic based on the definition found in
\texttt{Coq.Program.Equality}. Common useful variants are the following,
defined in the same file:

\begin{Variants}
\item {\tt dependent induction {\ident} generalizing {\ident$_1$} \dots
    {\ident$_n$}}\tacindex{dependent induction \dots\ generalizing}
  
  Does dependent induction on the hypothesis {\ident} but first
  generalizes the goal by the given variables so that they are
  universally quantified in the goal. This is generally what one wants
  to do with the variables that are inside some constructors in the
  induction hypothesis. The other ones need not be further generalized.

\item {\tt dependent destruction {\ident}}\tacindex{dependent destruction}
  
  Does the generalization of the instance {\ident} but uses {\tt destruct}
  instead of {\tt induction} on the generalized hypothesis. This gives
  results equivalent to {\tt inversion} or {\tt dependent inversion} if
  the hypothesis is dependent.
\end{Variants}

A larger example of dependent induction and an explanation of the
underlying technique are developed in section~\ref{dependent-induction-example}.

\subsection{\tt decompose [ {\qualid$_1$} \dots\ {\qualid$_n$} ] \term
\label{decompose}
\tacindex{decompose}}

This tactic allows to recursively decompose a
complex proposition in order to obtain atomic ones.
Example: 

\begin{coq_eval}
Reset Initial.
\end{coq_eval}
\begin{coq_example}
Lemma ex1 : forall A B C:Prop, A /\ B /\ C \/ B /\ C \/ C /\ A -> C.
intros A B C H; decompose [and or] H; assumption.
\end{coq_example}
\begin{coq_example*}
Qed.
\end{coq_example*}

{\tt decompose} does not work on right-hand sides of implications or products.

\begin{Variants}
  
\item {\tt decompose sum \term}\tacindex{decompose sum}
  This decomposes sum types (like \texttt{or}).
\item {\tt decompose record \term}\tacindex{decompose record}
  This decomposes record types (inductive types with one constructor,
  like \texttt{and} and \texttt{exists} and those defined with the
  \texttt{Record} macro, see Section~\ref{Record}).
\end{Variants}


\subsection{\tt functional induction (\qualid\ \term$_1$ \dots\ \term$_n$).
\tacindex{functional induction}
\label{FunInduction}}

The \emph{experimental} tactic \texttt{functional induction} performs
case analysis and induction following the definition of a function. It
makes use of a principle generated by \texttt{Function}
(see Section~\ref{Function}) or \texttt{Functional Scheme}
(see Section~\ref{FunScheme}).

\begin{coq_eval}
Reset Initial.
\end{coq_eval}
\begin{coq_example}
Functional Scheme minus_ind := Induction for minus Sort Prop.

Lemma le_minus : forall n m:nat, (n - m <= n).
intros n m.
functional induction (minus n m); simpl; auto.
\end{coq_example}
\begin{coq_example*}
Qed.
\end{coq_example*}

\Rem \texttt{(\qualid\ \term$_1$ \dots\ \term$_n$)} must be a correct
full application of \qualid. In particular, the rules for implicit
arguments are the same as usual. For example use \texttt{@\qualid} if
you want to write implicit arguments explicitly.

\Rem Parenthesis over \qualid \dots \term$_n$ are mandatory.

\Rem \texttt{functional induction (f x1 x2 x3)} is actually a wrapper
for \texttt{induction x1 x2 x3 (f x1 x2 x3) using \qualid} followed by
a cleaning phase, where $\qualid$ is the induction principle
registered for $f$ (by the \texttt{Function} (see Section~\ref{Function})
or \texttt{Functional Scheme} (see Section~\ref{FunScheme}) command)
corresponding to the sort of the goal.  Therefore \texttt{functional
  induction} may fail if the induction scheme (\texttt{\qualid}) is
not defined. See also Section~\ref{Function} for the function terms
accepted by \texttt{Function}.

\Rem There is a difference between obtaining an induction scheme for a
function by using \texttt{Function} (see Section~\ref{Function}) and by
using \texttt{Functional Scheme} after a normal definition using
\texttt{Fixpoint} or \texttt{Definition}. See \ref{Function} for
details.

\SeeAlso{\ref{Function},\ref{FunScheme},\ref{FunScheme-examples},
  \ref{sec:functional-inversion}}

\begin{ErrMsgs}
\item \errindex{Cannot find induction information on \qualid}

  ~

\item \errindex{Not the right number of induction arguments}
\end{ErrMsgs}

\begin{Variants}
\item {\tt functional induction (\qualid\ \term$_1$ \dots\ \term$_n$)
   using \term$_{m+1}$ with {\term$_{n+1}$} \dots {\term$_m$}}

 Similar to \texttt{Induction} and \texttt{elim}
 (see Section~\ref{Tac-induction}), allows to give explicitly the
 induction principle and the values of dependent premises of the
 elimination scheme, including \emph{predicates} for mutual induction
 when {\qualid} is part of a mutually recursive definition.

\item {\tt functional induction (\qualid\ \term$_1$ \dots\ \term$_n$)
    using \term$_{m+1}$ with {\vref$_1$} := {\term$_{n+1}$} \dots\
    {\vref$_m$} := {\term$_n$}}

  Similar to \texttt{induction} and \texttt{elim}
  (see Section~\ref{Tac-induction}).

\item All previous variants can be extended by the usual \texttt{as
    \intropattern} construction, similar for example to
  \texttt{induction} and \texttt{elim} (see Section~\ref{Tac-induction}).
    
\end{Variants}



\section{Equality}

These tactics use the equality {\tt eq:forall A:Type, A->A->Prop}
defined in file {\tt Logic.v} (see Section~\ref{Equality}). The
notation for {\tt eq}~$T~t~u$ is simply {\tt $t$=$u$} dropping the
implicit type of $t$ and $u$.

\subsection{\tt rewrite \term
\label{rewrite}
\tacindex{rewrite}}

This tactic applies to any goal. The type of {\term}
must have the form

\texttt{forall (x$_1$:A$_1$) \dots\ (x$_n$:A$_n$)}\texttt{eq} \term$_1$ \term$_2$. 

\noindent where \texttt{eq} is the Leibniz equality or a registered
setoid equality.

\noindent Then {\tt rewrite \term} finds the first subterm matching
\term$_1$ in the goal, resulting in instances \term$_1'$ and \term$_2'$
and then replaces every occurrence of \term$_1'$ by \term$_2'$.
Hence, some of the variables x$_i$ are
solved by unification, and some of the types \texttt{A}$_1$, \dots,
\texttt{A}$_n$ become new subgoals.

% \Rem In case the type of  
% \term$_1$ contains occurrences of variables bound in the
% type of \term, the tactic tries first to find a subterm of the goal
% which matches this term in order to find a closed instance \term$'_1$
% of \term$_1$, and then all instances of \term$'_1$ will be replaced.

\begin{ErrMsgs}
\item \errindex{The term provided does not end with an equation}

\item \errindex{Tactic generated a subgoal identical to the original goal}\\
This happens if \term$_1$ does not occur in the goal.
\end{ErrMsgs}

\begin{Variants}
\item {\tt rewrite -> {\term}}\tacindex{rewrite ->}\\
  Is equivalent to {\tt rewrite \term}

\item {\tt rewrite <- {\term}}\tacindex{rewrite <-}\\
  Uses the equality \term$_1${\tt=}\term$_2$ from right to left

\item {\tt rewrite {\term} in \textit{clause}}
  \tacindex{rewrite \dots\ in}\\
  Analogous to {\tt rewrite {\term}} but rewriting is done following
  \textit{clause} (similarly to \ref{Conversion-tactics}). For
  instance:
  \begin{itemize}
  \item \texttt{rewrite H in H1} will rewrite \texttt{H} in the hypothesis
    \texttt{H1} instead of the current goal.
  \item \texttt{rewrite H in H1 at 1, H2 at - 2 |- *} means \texttt{rewrite H; rewrite H in H1 at 1;
      rewrite H in H2 at - 2}. In particular a failure will happen if any of
    these three simpler tactics fails. 
  \item \texttt{rewrite H in * |- } will do \texttt{rewrite H in
      H$_i$} for all hypothesis \texttt{H$_i$ <> H}. A success will happen
    as soon as at least one of these simpler tactics succeeds.
  \item \texttt{rewrite H in *} is a combination of \texttt{rewrite H} 
    and \texttt{rewrite H in * |-} that succeeds if at
    least one of these two tactics succeeds. 
  \end{itemize}
  Orientation {\tt ->} or {\tt <-} can be
  inserted before the term to rewrite.

\item {\tt rewrite {\term} at {\occlist}}
  \tacindex{rewrite \dots\ at}

  Rewrite only the given occurrences of \term$_1'$. Occurrences are
  specified from left to right as for \texttt{pattern} (\S
  \ref{pattern}). The rewrite is always performed using setoid
  rewriting, even for Leibniz's equality, so one has to 
  \texttt{Import Setoid} to use this variant.

\item {\tt rewrite {\term} by {\tac}}
  \tacindex{rewrite \dots\ by}

  Use {\tac} to completely solve the side-conditions arising from the
  rewrite.

\item {\tt rewrite $\term_1$, \ldots, $\term_n$}\\
  Is equivalent to the $n$ successive tactics {\tt rewrite $\term_1$}
  up to {\tt rewrite $\term_n$}, each one working on the first subgoal
  generated by the previous one.
  Orientation {\tt ->} or {\tt <-} can be
  inserted before each term to rewrite. One unique \textit{clause}
  can be added at the end after the keyword {\tt in}; it will 
  then affect all rewrite operations.

\item In all forms of {\tt rewrite} described above, a term to rewrite
  can be immediately prefixed by one of the following modifiers:
  \begin{itemize}
  \item {\tt ?} : the tactic {\tt rewrite ?$\term$} performs the
    rewrite of $\term$  as many times as possible (perhaps zero time).
    This form never fails. 
  \item {\tt $n$?} : works similarly, except that it will do at most 
   $n$ rewrites. 
  \item {\tt !} : works as {\tt ?}, except that at least one rewrite 
    should succeed, otherwise the tactic fails. 
  \item {\tt $n$!} (or simply {\tt $n$}) : precisely $n$ rewrites 
    of $\term$ will be done, leading to failure if these $n$ rewrites are not possible. 
  \end{itemize}

\item {\tt erewrite {\term}\tacindex{erewrite}}

This tactic works as {\tt rewrite {\term}} but turning unresolved
bindings into existential variables, if any, instead of failing. It has
the same variants as {\tt rewrite} has.

\end{Variants}


\subsection{\tt cutrewrite -> \term$_1$ = \term$_2$
\label{cutrewrite}
\tacindex{cutrewrite}}

This tactic acts like {\tt replace {\term$_1$} with {\term$_2$}}
(see below).

\subsection{\tt replace {\term$_1$} with {\term$_2$}
\label{tactic:replace}
\tacindex{replace \dots\ with}}

This tactic applies to any goal. It replaces all free occurrences of
{\term$_1$} in the current goal with {\term$_2$} and generates the
equality {\term$_2$}{\tt =}{\term$_1$} as a subgoal. This equality is
automatically solved if it occurs amongst the assumption, or if its
symmetric form occurs.  It is equivalent to {\tt cut
\term$_2$=\term$_1$; [intro H{\sl n}; rewrite <- H{\sl n}; clear H{\sl
n}| assumption || symmetry; try assumption]}.

\begin{ErrMsgs}
\item \errindex{terms do not have convertible types}
\end{ErrMsgs}

\begin{Variants}
\item {\tt replace {\term$_1$} with {\term$_2$} by \tac}\\ This acts
  as {\tt replace {\term$_1$} with {\term$_2$}} but applies {\tt \tac}
  to solve the generated subgoal {\tt \term$_2$=\term$_1$}.
\item {\tt replace {\term}}\\ Replace {\term} with {\term'} using the
  first assumption whose type has the form {\tt \term=\term'} or {\tt
    \term'=\term}
\item {\tt replace -> {\term}}\\ Replace {\term} with {\term'} using the
  first assumption whose type has the form {\tt \term=\term'}
\item {\tt replace <- {\term}}\\ Replace {\term} with {\term'} using the
  first assumption whose type has the form {\tt \term'=\term}
\item {\tt replace {\term$_1$} with {\term$_2$} \textit{clause} }\\
    {\tt replace {\term$_1$} with {\term$_2$} \textit{clause} by \tac }\\ 
    {\tt replace {\term} \textit{clause}}\\ 
    {\tt replace -> {\term} \textit{clause}}\\ 
    {\tt replace <- {\term} \textit{clause}}\\ 
    Act as before but the replacements take place in
    \textit{clause}~(see Section~\ref{Conversion-tactics}) and not only
    in the conclusion of the goal.\\
    The  \textit{clause} argument must  not contain  any \texttt{type  of} nor  \texttt{value  of}.
\end{Variants}

\subsection{\tt reflexivity
\label{reflexivity}
\tacindex{reflexivity}}

This tactic applies to a goal which has the form {\tt t=u}. It checks
that {\tt t} and {\tt u} are convertible and then solves the goal.
It is equivalent to {\tt apply refl\_equal}.

\begin{ErrMsgs}
\item \errindex{The conclusion is not a substitutive equation}
\item \errindex{Impossible to unify \dots\ with \dots.}
\end{ErrMsgs}

\subsection{\tt symmetry
\tacindex{symmetry}
\tacindex{symmetry in}}
This tactic applies to a goal which has the form {\tt t=u} and changes it
into {\tt u=t}.

\variant {\tt symmetry in {\ident}}\\
If the statement of the hypothesis {\ident} has the form {\tt t=u},
the tactic changes it to {\tt u=t}.

\subsection{\tt transitivity \term
\tacindex{transitivity}}
This tactic applies to a goal which has the form {\tt t=u}
and transforms it into the two subgoals 
{\tt t={\term}} and {\tt {\term}=u}.

\subsection{\tt subst {\ident}
\tacindex{subst}}

This tactic applies to a goal which has \ident\ in its context and
(at least) one hypothesis, say {\tt H}, of type {\tt
  \ident=t} or {\tt t=\ident}. Then it replaces 
\ident\ by {\tt t} everywhere in the goal (in the hypotheses 
and in the conclusion) and clears \ident\ and {\tt H} from the context.

\Rem 
When several hypotheses have the form {\tt \ident=t} or {\tt
  t=\ident}, the first one is used. 

\begin{Variants}
  \item {\tt subst \ident$_1$ \dots \ident$_n$} \\
    Is equivalent to {\tt subst \ident$_1$; \dots; subst \ident$_n$}.
  \item {\tt subst} \\
    Applies {\tt subst} repeatedly to all identifiers from the context
    for which an equality exists.
\end{Variants}

\subsection[{\tt stepl {\term}}]{{\tt stepl {\term}}\tacindex{stepl}}

This tactic is for chaining rewriting steps. It assumes a goal of the
form ``$R$ {\term}$_1$ {\term}$_2$'' where $R$ is a binary relation
and relies on a database of lemmas of the form {\tt forall} $x$ $y$
$z$, $R$ $x$ $y$ {\tt ->} $eq$ $x$ $z$ {\tt ->} $R$ $z$ $y$ where $eq$
is typically a setoid equality. The application of {\tt stepl {\term}}
then replaces the goal by ``$R$ {\term} {\term}$_2$'' and adds a new
goal stating ``$eq$ {\term} {\term}$_1$''.

Lemmas are added to the database using the command 
\comindex{Declare Left Step}
\begin{quote}
{\tt Declare Left Step {\term}.}
\end{quote}

The tactic is especially useful for parametric setoids which are not
accepted as regular setoids for {\tt rewrite} and {\tt
  setoid\_replace} (see Chapter~\ref{setoid_replace}).

\tacindex{stepr}
\comindex{Declare Right Step}
\begin{Variants}
\item{\tt stepl {\term} by {\tac}}\\
This applies {\tt stepl {\term}} then applies {\tac} to the second goal.

\item{\tt stepr {\term}}\\
     {\tt stepr {\term} by {\tac}}\\
This behaves as {\tt stepl} but on the right-hand-side of the binary relation.
Lemmas are expected to be of the form
``{\tt forall} $x$ $y$
$z$, $R$ $x$ $y$ {\tt ->} $eq$ $y$ $z$ {\tt ->} $R$ $x$ $z$''
and are registered using the command
\begin{quote}
{\tt Declare Right Step {\term}.}
\end{quote}
\end{Variants}


\subsection{\tt f\_equal
\label{f-equal}
\tacindex{f\_equal}}

This tactic applies to a goal of the form $f\ a_1\ \ldots\ a_n = f'\
a'_1\ \ldots\ a'_n$. Using {\tt f\_equal} on such a goal leads to
subgoals $f=f'$ and $a_1=a'_1$ and so on up to $a_n=a'_n$. Amongst 
these subgoals, the simple ones (e.g. provable by
reflexivity or congruence) are automatically solved by {\tt f\_equal}.


\section{Equality and inductive sets}

We describe in this section some special purpose tactics dealing with
equality and inductive sets or types. These tactics use the equality
{\tt eq:forall (A:Type), A->A->Prop}, simply written with the
infix symbol {\tt =}.

\subsection{\tt decide equality
\label{decideequality}
\tacindex{decide equality}}

This tactic solves a goal of the form
{\tt forall $x$ $y$:$R$, \{$x$=$y$\}+\{\verb|~|$x$=$y$\}}, where $R$
is an inductive type such that its constructors do not take proofs or
functions as arguments, nor objects in dependent types.

\begin{Variants}
\item {\tt decide equality {\term}$_1$ {\term}$_2$ }.\\
 Solves a goal of the form {\tt \{}\term$_1${\tt =}\term$_2${\tt
\}+\{\verb|~|}\term$_1${\tt =}\term$_2${\tt \}}.
\end{Variants}

\subsection{\tt compare \term$_1$ \term$_2$
\tacindex{compare}}

This tactic compares two given objects \term$_1$ and \term$_2$ 
of an inductive datatype. If $G$ is the current goal, it leaves the sub-goals
\term$_1${\tt =}\term$_2$ {\tt ->} $G$ and \verb|~|\term$_1${\tt =}\term$_2$
{\tt ->} $G$. The type
of \term$_1$ and \term$_2$ must satisfy the same restrictions as in the tactic
\texttt{decide equality}.

\subsection{\tt discriminate {\term}
\label{discriminate}
\tacindex{discriminate}
\tacindex{ediscriminate}}

This tactic proves any goal from an assumption stating that two
structurally different terms of an inductive set are equal. For
example, from {\tt (S (S O))=(S O)} we can derive by absurdity any
proposition.

The argument {\term} is assumed to be a proof of a statement
of conclusion {\tt{\term$_1$} = {\term$_2$}} with {\term$_1$} and
{\term$_2$} being elements of an inductive set.  To build the proof,
the tactic traverses the normal forms\footnote{Reminder: opaque
  constants will not be expanded by $\delta$ reductions} of
{\term$_1$} and {\term$_2$} looking for a couple of subterms {\tt u}
and {\tt w} ({\tt u} subterm of the normal form of {\term$_1$} and
{\tt w} subterm of the normal form of {\term$_2$}), placed at the same
positions and whose head symbols are two different constructors. If
such a couple of subterms exists, then the proof of the current goal
is completed, otherwise the tactic fails.

\Rem The syntax {\tt discriminate {\ident}} can be used to refer to a
hypothesis quantified in the goal. In this case, the quantified
hypothesis whose name is {\ident} is first introduced in the local
context using \texttt{intros until \ident}.

\begin{ErrMsgs}
\item \errindex{No primitive equality found}
\item \errindex{Not a discriminable equality}
\end{ErrMsgs}  

\begin{Variants}
\item \texttt{discriminate} \num

  This does the same thing as \texttt{intros until \num} followed by
  \texttt{discriminate \ident} where {\ident} is the identifier for
  the last introduced hypothesis.

\item \texttt{discriminate} {\term} {\tt with} {\bindinglist}

  This does the same thing as \texttt{discriminate {\term}} but using
the given bindings to instantiate parameters or hypotheses of {\term}.

\item \texttt{ediscriminate} \num\\
      \texttt{ediscriminate} {\term} \zeroone{{\tt with} {\bindinglist}}

  This works the same as {\tt discriminate} but if the type of {\term},
  or the type of the hypothesis referred to by {\num}, has uninstantiated
  parameters, these parameters are left as existential variables.

\item \texttt{discriminate}

  This behaves like {\tt discriminate {\ident}} if {\ident} is the
  name of an hypothesis to which {\tt discriminate} is applicable; if
  the current goal is of the form {\term$_1$} {\tt <>} {\term$_2$},
  this behaves as {\tt intro {\ident}; injection {\ident}}.

  \begin{ErrMsgs}
  \item \errindex{No discriminable equalities} \\
  occurs when the goal does not verify the expected preconditions.
  \end{ErrMsgs}
\end{Variants}

\subsection{\tt injection {\term}
\label{injection}
\tacindex{injection}
\tacindex{einjection}}

The {\tt injection} tactic is based on the fact that constructors of
inductive sets are injections. That means that if $c$ is a constructor
of an inductive set, and if $(c~\vec{t_1})$ and $(c~\vec{t_2})$ are two
terms that are equal then $~\vec{t_1}$ and $~\vec{t_2}$ are equal
too.

If {\term} is a proof of a statement of conclusion
 {\tt {\term$_1$} = {\term$_2$}},
then {\tt injection} applies injectivity as deep as possible to
derive the equality of all the subterms of {\term$_1$} and {\term$_2$}
placed in the same positions. For example, from {\tt (S
  (S n))=(S (S (S m))} we may derive {\tt n=(S m)}.  To use this
tactic {\term$_1$} and {\term$_2$} should be elements of an inductive
set and they should be neither explicitly equal, nor structurally
different. We mean by this that, if {\tt n$_1$} and {\tt n$_2$} are
their respective normal forms, then:
\begin{itemize}
\item {\tt n$_1$} and {\tt n$_2$} should not be syntactically equal,
\item there must not exist any pair of subterms {\tt u} and {\tt w},
  {\tt u} subterm of {\tt n$_1$} and {\tt w} subterm of {\tt n$_2$} ,
  placed in the same positions and having different constructors as
  head symbols.
\end{itemize}
If these conditions are satisfied, then, the tactic derives the
equality of all the subterms of {\term$_1$} and {\term$_2$} placed in
the same positions and puts them as antecedents of the current goal.

\Example Consider the following goal:

\begin{coq_example*}
Inductive list : Set :=
  | nil : list
  | cons : nat -> list -> list.
Variable P : list -> Prop.
\end{coq_example*}
\begin{coq_eval}
Lemma ex :
 forall (l:list) (n:nat), P nil -> cons n l = cons 0 nil -> P l.
intros l n H H0.
\end{coq_eval}
\begin{coq_example}
Show.
injection H0.
\end{coq_example}
\begin{coq_eval}
Abort.
\end{coq_eval}

Beware that \texttt{injection} yields always an equality in a sigma type
whenever the injected object has a dependent type.

\Rem There is a special case for dependent pairs. If we have a decidable 
equality over the type of the first argument, then it is safe to do 
the projection on the second one, and so {\tt injection} will work fine.
To define such an equality, you have to use the {\tt Scheme} command 
(see \ref{Scheme}).

\Rem If some quantified hypothesis of the goal is named {\ident}, then
{\tt injection {\ident}} first introduces the hypothesis in the local
context using \texttt{intros until \ident}.

\begin{ErrMsgs}
\item \errindex{Not a projectable equality but a discriminable one}
\item \errindex{Nothing to do, it is an equality between convertible terms}
\item \errindex{Not a primitive equality}
\end{ErrMsgs}

\begin{Variants}
\item \texttt{injection} \num{}

  This does the same thing as \texttt{intros until \num} followed by
\texttt{injection \ident} where {\ident} is the identifier for the last
introduced hypothesis.

\item \texttt{injection} \term{} {\tt with} {\bindinglist}

  This does the same as \texttt{injection {\term}} but using
  the given bindings to instantiate parameters or hypotheses of {\term}.

\item \texttt{einjection} \num\\
      \texttt{einjection} \term{} \zeroone{{\tt with} {\bindinglist}}

  This works the same as {\tt injection} but if the type of {\term},
  or the type of the hypothesis referred to by {\num}, has uninstantiated
  parameters, these parameters are left as existential variables.

\item{\tt injection}
  
  If the current goal is of the form {\term$_1$} {\tt <>} {\term$_2$},
  this behaves as {\tt intro {\ident}; injection {\ident}}.
  
  \ErrMsg \errindex{goal does not satisfy the expected preconditions}

\item \texttt{injection} \term{} \zeroone{{\tt with} {\bindinglist}} \texttt{as} \nelist{\intropattern}{}\\
\texttt{injection} \num{} \texttt{as} {\intropattern} {\ldots} {\intropattern}\\
\texttt{injection} \texttt{as} {\intropattern} {\ldots} {\intropattern}\\
\texttt{einjection} \term{} \zeroone{{\tt with} {\bindinglist}} \texttt{as} \nelist{\intropattern}{}\\
\texttt{einjection} \num{} \texttt{as} {\intropattern} {\ldots} {\intropattern}\\
\texttt{einjection} \texttt{as} {\intropattern} {\ldots} {\intropattern}\\
\tacindex{injection \ldots{} as}
 
These variants apply \texttt{intros} \nelist{\intropattern}{} after
the call to \texttt{injection} or \texttt{einjection}.

\end{Variants}

\subsection{\tt simplify\_eq {\term}
\tacindex{simplify\_eq}
\tacindex{esimplify\_eq}
\label{simplify-eq}}

Let {\term} be the proof of a statement of conclusion {\tt
  {\term$_1$}={\term$_2$}}. If {\term$_1$} and
{\term$_2$} are structurally different (in the sense described for the
tactic {\tt discriminate}), then the tactic {\tt simplify\_eq} behaves as {\tt
  discriminate {\term}}, otherwise it behaves as {\tt injection
  {\term}}.

\Rem If some quantified hypothesis of the goal is named {\ident}, then
{\tt simplify\_eq {\ident}} first introduces the hypothesis in the local
context using \texttt{intros until \ident}.

\begin{Variants}
\item \texttt{simplify\_eq} \num

  This does the same thing as \texttt{intros until \num} then
\texttt{simplify\_eq \ident} where {\ident} is the identifier for the last
introduced hypothesis.

\item \texttt{simplify\_eq} \term{} {\tt with} {\bindinglist}

  This does the same as \texttt{simplify\_eq {\term}} but using
  the given bindings to instantiate parameters or hypotheses of {\term}.

\item \texttt{esimplify\_eq} \num\\
      \texttt{esimplify\_eq} \term{} \zeroone{{\tt with} {\bindinglist}}

  This works the same as {\tt simplify\_eq} but if the type of {\term},
  or the type of the hypothesis referred to by {\num}, has uninstantiated
  parameters, these parameters are left as existential variables.

\item{\tt simplify\_eq}

If the current goal has form $t_1\verb=<>=t_2$, it behaves as
\texttt{intro {\ident}; simplify\_eq {\ident}}.
\end{Variants}

\subsection{\tt dependent rewrite -> {\ident}
\tacindex{dependent rewrite ->}
\label{dependent-rewrite}}

This tactic applies to any goal.  If \ident\ has type 
\verb+(existT B a b)=(existT B a' b')+ 
in the local context (i.e. each term of the
equality has a sigma type $\{ a:A~ \&~(B~a)\}$) this tactic rewrites
\verb+a+ into \verb+a'+ and \verb+b+ into \verb+b'+ in the current
goal. This tactic works even if $B$ is also a sigma type.  This kind
of equalities between dependent pairs may be derived by the injection
and inversion tactics.

\begin{Variants}
\item{\tt dependent rewrite <- {\ident}}
\tacindex{dependent rewrite <-} \\
Analogous to {\tt dependent rewrite ->} but uses the equality from
right to left.
\end{Variants}

\section{Inversion
\label{inversion}}

\subsection{\tt inversion {\ident}
\tacindex{inversion}}

Let the type of \ident~ in the local context be $(I~\vec{t})$,
where $I$ is a (co)inductive predicate. Then,
\texttt{inversion} applied to \ident~ derives for each possible
constructor $c_i$ of $(I~\vec{t})$, {\bf all} the necessary
conditions that should hold for the instance $(I~\vec{t})$ to be
proved by $c_i$.

\Rem If {\ident} does not denote a hypothesis in the local context
but refers to a hypothesis quantified in the goal, then the
latter is first introduced in the local context using
\texttt{intros until \ident}.

\begin{Variants}
\item \texttt{inversion} \num
  
  This does the same thing as \texttt{intros until \num} then
  \texttt{inversion \ident} where {\ident} is the identifier for the
  last introduced hypothesis.

\item \tacindex{inversion\_clear} \texttt{inversion\_clear} \ident

  This behaves as \texttt{inversion} and then erases \ident~ from the
  context.

\item \tacindex{inversion \dots\ as} \texttt{inversion} {\ident} \texttt{as} {\intropattern}
  
  This behaves as \texttt{inversion} but using names in
  {\intropattern} for naming hypotheses. The {\intropattern} must have
  the form {\tt [} $p_{11}$ \ldots $p_{1n_1}$ {\tt |} {\ldots} {\tt |}
  $p_{m1}$ \ldots $p_{mn_m}$ {\tt ]} with $m$ being the number of
  constructors of the type of {\ident}. Be careful that the list must
  be of length $m$ even if {\tt inversion} discards some cases (which
  is precisely one of its roles): for the discarded cases, just use an
  empty list (i.e. $n_i=0$).

  The arguments of the $i^{th}$ constructor and the
  equalities that {\tt inversion} introduces in the context of the
  goal corresponding to the $i^{th}$ constructor, if it exists, get
  their names from the list $p_{i1}$ \ldots $p_{in_i}$ in order. If
  there are not enough names, {\tt induction} invents names for the
  remaining variables to introduce. In case an equation splits into
  several equations (because {\tt inversion} applies {\tt injection}
  on the equalities it generates), the corresponding name $p_{ij}$ in
  the list must be replaced by a sublist of the form {\tt [$p_{ij1}$
  \ldots $p_{ijq}$]} (or, equivalently, {\tt ($p_{ij1}$,
  \ldots, $p_{ijq}$)}) where $q$ is the number of subequalities
  obtained from splitting the original equation. Here is an example.

\begin{coq_eval}
Require Import List.
\end{coq_eval}

\begin{coq_example}
Inductive contains0 : list nat -> Prop :=
  | in_hd : forall l, contains0 (0 :: l)
  | in_tl : forall l b, contains0 l -> contains0 (b :: l).
Goal forall l:list nat, contains0 (1 :: l) -> contains0 l.
intros l H; inversion H as [ | l' p Hl' [Heqp Heql'] ].
\end{coq_example}

\begin{coq_eval}
Abort.
\end{coq_eval}

\item \texttt{inversion} {\num} {\tt as} {\intropattern} 

  This allows to name the hypotheses introduced by
  \texttt{inversion} {\num} in the context.

\item \tacindex{inversion\_cleardots\ as} \texttt{inversion\_clear}
  {\ident} {\tt as} {\intropattern}  

  This allows to name the hypotheses introduced by
  \texttt{inversion\_clear} in the context.
  
\item \tacindex{inversion \dots\ in} \texttt{inversion } {\ident}
  \texttt{in} \ident$_1$ \dots\ \ident$_n$

  Let \ident$_1$ \dots\ \ident$_n$, be identifiers in the local context. This
  tactic behaves as generalizing \ident$_1$ \dots\ \ident$_n$, and
  then performing \texttt{inversion}.
  
\item \tacindex{inversion \dots\ as \dots\ in} \texttt{inversion }
  {\ident} {\tt as} {\intropattern} \texttt{in} \ident$_1$ \dots\ 
  \ident$_n$
  
  This allows to name the hypotheses introduced in the context by
  \texttt{inversion} {\ident} \texttt{in} \ident$_1$ \dots\ 
  \ident$_n$.
  
\item \tacindex{inversion\_clear \dots\ in} \texttt{inversion\_clear}
  {\ident} \texttt{in} \ident$_1$ \ldots \ident$_n$
 
  Let \ident$_1$ \dots\ \ident$_n$, be identifiers in the local context. This
  tactic behaves as generalizing \ident$_1$ \dots\ \ident$_n$, and
  then performing {\tt inversion\_clear}.
  
\item \tacindex{inversion\_clear \dots\ as \dots\ in}
  \texttt{inversion\_clear} {\ident} \texttt{as} {\intropattern}
  \texttt{in} \ident$_1$ \ldots \ident$_n$

  This allows to name the hypotheses introduced in the context by
  \texttt{inversion\_clear} {\ident} \texttt{in} \ident$_1$ \ldots
  \ident$_n$.

\item \tacindex{dependent inversion} \texttt{dependent inversion}
  {\ident}  
  
  That must be used when \ident\ appears in the current goal.  It acts
  like \texttt{inversion} and then substitutes \ident\ for the
  corresponding term in the goal.
  
\item \tacindex{dependent inversion \dots\ as } \texttt{dependent
    inversion} {\ident} \texttt{as} {\intropattern} 
  
  This allows to name the hypotheses introduced in the context by
  \texttt{dependent inversion} {\ident}.

\item \tacindex{dependent inversion\_clear} \texttt{dependent
    inversion\_clear} {\ident} 
  
  Like \texttt{dependent inversion}, except that {\ident} is cleared
  from the local context.

\item \tacindex{dependent inversion\_clear \dots\ as}
  \texttt{dependent inversion\_clear} {\ident}\texttt{as} {\intropattern}
  
  This allows to name the hypotheses introduced in the context by
  \texttt{dependent inversion\_clear} {\ident}.

\item \tacindex{dependent inversion \dots\ with} \texttt{dependent
    inversion } {\ident} \texttt{ with } \term  
  
  This variant allows you to specify the generalization of the goal. It
  is useful when the system fails to generalize the goal automatically. If
  {\ident} has type $(I~\vec{t})$ and $I$ has type
  $forall (\vec{x}:\vec{T}), s$,   then \term~  must be of type
  $I:forall (\vec{x}:\vec{T}), I~\vec{x}\to s'$ where $s'$ is the
  type of the goal.

\item \tacindex{dependent inversion \dots\ as \dots\ with}
  \texttt{dependent inversion } {\ident} \texttt{as} {\intropattern}
  \texttt{ with } \term  
  
  This allows to name the hypotheses introduced in the context by
  \texttt{dependent inversion } {\ident} \texttt{ with } \term.

\item \tacindex{dependent inversion\_clear \dots\ with}
  \texttt{dependent inversion\_clear } {\ident} \texttt{ with } \term 
  
  Like \texttt{dependent inversion \dots\ with} but clears {\ident} from
  the local context.

\item \tacindex{dependent inversion\_clear \dots\ as \dots\ with}
  \texttt{dependent inversion\_clear } {\ident} \texttt{as}
  {\intropattern} \texttt{ with } \term 
  
  This allows to name the hypotheses introduced in the context by
  \texttt{dependent inversion\_clear } {\ident} \texttt{ with } \term.

\item \tacindex{simple inversion} \texttt{simple inversion} {\ident}
  
  It is a very primitive inversion tactic that derives all the necessary
  equalities  but it does not simplify the  constraints as
  \texttt{inversion} does.

\item \tacindex{simple inversion \dots\ as} \texttt{simple inversion}
  {\ident} \texttt{as} {\intropattern} 
  
  This allows to name the hypotheses introduced in the context by
  \texttt{simple inversion}.

\item \tacindex{inversion \dots\ using} \texttt{inversion} \ident
  \texttt{ using} \ident$'$  
  
  Let {\ident} have type $(I~\vec{t})$ ($I$ an inductive
  predicate) in the local context, and \ident$'$ be a (dependent) inversion
  lemma. Then, this tactic refines the current goal with the specified
  lemma.

\item \tacindex{inversion \dots\ using \dots\ in} \texttt{inversion}
  {\ident} \texttt{using} \ident$'$ \texttt{in} \ident$_1$\dots\ \ident$_n$
  
  This tactic behaves as generalizing \ident$_1$\dots\ \ident$_n$,
  then doing \texttt{inversion} {\ident} \texttt{using} \ident$'$.

\end{Variants}

\SeeAlso~\ref{inversion-examples} for detailed examples

\subsection{\tt Derive Inversion {\ident} with
  ${\tt forall (}\vec{x}{\tt :}\vec{T}{\tt),} I~\vec{t}$ Sort \sort
\label{Derive-Inversion}
\comindex{Derive Inversion}}

This command generates an inversion principle for the
\texttt{inversion \dots\ using} tactic.
Let $I$ be an inductive predicate and $\vec{x}$ the variables
occurring in $\vec{t}$. This command generates and stocks the
inversion lemma for the sort \sort~ corresponding to the instance
$forall (\vec{x}:\vec{T}), I~\vec{t}$ with the name {\ident} in the {\bf
global} environment. When applied it is equivalent to have inverted
the instance with the tactic {\tt inversion}.

\begin{Variants}
\item \texttt{Derive Inversion\_clear} {\ident} \texttt{with}
  \comindex{Derive Inversion\_clear}
  $forall (\vec{x}:\vec{T}), I~\vec{t}$ \texttt{Sort} \sort~ \\ 
  \index{Derive Inversion\_clear \dots\ with}
  When applied it is equivalent to having
  inverted the instance with the tactic \texttt{inversion}
  replaced by the tactic \texttt{inversion\_clear}.
\item \texttt{Derive Dependent Inversion} {\ident} \texttt{with}
  $forall (\vec{x}:\vec{T}), I~\vec{t}$ \texttt{Sort} \sort~\\
  \comindex{Derive Dependent Inversion}
  When applied it is equivalent to having
  inverted the instance with the tactic \texttt{dependent inversion}.
\item \texttt{Derive Dependent Inversion\_clear} {\ident} \texttt{with}
  $forall (\vec{x}:\vec{T}), I~\vec{t}$ \texttt{Sort} \sort~\\
  \comindex{Derive Dependent Inversion\_clear}
  When applied it is equivalent to having
  inverted the instance with the tactic \texttt{dependent inversion\_clear}.
\end{Variants}

\SeeAlso \ref{inversion-examples} for examples



\subsection[\tt functional inversion \ident]{\tt functional inversion \ident\label{sec:functional-inversion}}

\texttt{functional inversion} is a \emph{highly} experimental tactic
which performs inversion on hypothesis \ident\ of the form
\texttt{\qualid\ \term$_1$\dots\term$_n$\ = \term} or \texttt{\term\ =
  \qualid\ \term$_1$\dots\term$_n$} where \qualid\ must have been
defined using \texttt{Function} (see Section~\ref{Function}).

\begin{ErrMsgs}
\item \errindex{Hypothesis {\ident} must contain at least one Function}
\item \errindex{Cannot find inversion information for hypothesis \ident}
  This error may be raised when  some inversion lemma failed to be
  generated by Function.
\end{ErrMsgs}

\begin{Variants}
\item {\tt functional inversion \num}

  This does the same thing as \texttt{intros until \num} then
  \texttt{functional inversion \ident} where {\ident} is the
  identifier for the last introduced hypothesis.
\item {\tt functional inversion \ident\ \qualid}\\
  {\tt functional inversion \num\ \qualid}

  In case the hypothesis {\ident} (or {\num}) has a type of the form
  \texttt{\qualid$_1$\ \term$_1$\dots\term$_n$\ =\ \qualid$_2$\
    \term$_{n+1}$\dots\term$_{n+m}$} where \qualid$_1$ and \qualid$_2$
  are valid candidates to functional inversion, this variant allows to
  choose which must be inverted.
\end{Variants}



\subsection{\tt quote \ident
\tacindex{quote}
\index{2-level approach}}

This kind of inversion has nothing to do with the tactic
\texttt{inversion} above. This tactic does \texttt{change (\ident\
  t)}, where \texttt{t} is a term built in order to ensure the
convertibility. In other words, it does inversion of the function
\ident. This function must be a fixpoint on a simple recursive
datatype: see~\ref{quote-examples} for the full details.

\begin{ErrMsgs}
\item \errindex{quote: not a simple fixpoint}\\
  Happens when \texttt{quote} is not able to perform inversion properly.
\end{ErrMsgs}

\begin{Variants}
\item \texttt{quote {\ident} [ \ident$_1$ \dots \ident$_n$ ]}\\
  All terms that are built only with \ident$_1$ \dots \ident$_n$ will be
  considered by \texttt{quote} as constants rather than variables.
\end{Variants}

% En attente d'un moyen de valoriser les fichiers de demos
% \SeeAlso file \texttt{theories/DEMOS/DemoQuote.v} in the distribution

\section[Classical tactics]{Classical tactics\label{ClassicalTactics}}

In order to ease the proving process, when the {\tt Classical} module is loaded. A few more tactics are available. Make sure to load the module using the \texttt{Require Import} command.

\subsection{{\tt classical\_left, classical\_right} \tacindex{classical\_left} \tacindex{classical\_right}}

The tactics \texttt{classical\_left} and \texttt{classical\_right} are the analog of the \texttt{left} and \texttt{right} but using classical logic. They can only be used for disjunctions.
Use  \texttt{classical\_left} to prove the left part of the disjunction with the assumption that the negation of right part holds. 
Use \texttt{classical\_right} to prove the right part of the disjunction with the assumption that the negation of left part holds. 

\section{Automatizing
\label{Automatizing}}

\subsection{\tt auto
\label{auto}
\tacindex{auto}}

This tactic implements a Prolog-like resolution procedure to solve the
current goal. It first tries to solve the goal using the {\tt
  assumption} tactic, then it reduces the goal to an atomic one using
{\tt intros} and introducing the newly generated hypotheses as hints.
Then it looks at the list of tactics associated to the head symbol of
the goal and tries to apply one of them (starting from the tactics
with lower cost). This process is recursively applied to the generated
subgoals. 

By default, \texttt{auto} only uses the hypotheses of the current goal and the
hints of the database named {\tt core}. 

\begin{Variants}

\item  {\tt auto \num}

  Forces the search depth to be \num. The maximal search depth is 5 by
  default. 

\item {\tt auto with \ident$_1$ \dots\ \ident$_n$}
  
  Uses the hint databases $\ident_1$ \dots\ $\ident_n$ in addition to
  the database {\tt core}. See Section~\ref{Hints-databases} for the
  list of pre-defined databases and the way to create or extend a
  database.  This option can be combined with the previous one.

\item {\tt auto with *}

  Uses all existing hint databases, minus the special database
  {\tt v62}. See Section~\ref{Hints-databases}

\item \texttt{auto using \nterm{lemma}$_1$ , \ldots , \nterm{lemma}$_n$}

  Uses \nterm{lemma}$_1$, \ldots, \nterm{lemma}$_n$ in addition to
  hints (can be combined with the \texttt{with \ident} option). If
  $lemma_i$ is an inductive type, it is the collection of its
  constructors which is added as hints.

\item \texttt{auto using \nterm{lemma}$_1$ , \ldots , \nterm{lemma}$_n$ with \ident$_1$ \dots\ \ident$_n$}

  This combines the effects of the {\tt using} and {\tt with} options.

\item {\tt trivial}\tacindex{trivial}

  This tactic is a restriction of {\tt auto} that is not recursive and 
  tries only hints which cost 0. Typically it solves trivial
  equalities like $X=X$.

\item \texttt{trivial with \ident$_1$ \dots\ \ident$_n$}

\item \texttt{trivial with *}

\end{Variants}

\Rem {\tt auto} either solves completely the goal or else leaves it
intact. \texttt{auto} and \texttt{trivial} never fail.

\SeeAlso Section~\ref{Hints-databases}

\subsection{\tt eauto
\tacindex{eauto}
\label{eauto}}

This tactic generalizes {\tt auto}. In contrast with 
the latter, {\tt eauto} uses unification of the goal
against the hints rather than pattern-matching
(in other words, it uses {\tt eapply} instead of
{\tt apply}).
As a consequence, {\tt eauto} can solve such a goal:

\begin{coq_example}
Hint Resolve ex_intro.
Goal forall P:nat -> Prop, P 0 ->  exists n, P n.
eauto.
\end{coq_example}
\begin{coq_eval}
Abort.
\end{coq_eval}

Note that {\tt ex\_intro} should be declared as an
hint.

\SeeAlso Section~\ref{Hints-databases}

\subsection{\tt autounfold with  \ident$_1$ \dots\ \ident$_n$
\tacindex{autounfold}
\label{autounfold}}

This tactic unfolds constants that were declared through a {\tt Hint
  Unfold} in the given databases.

\begin{Variants}
\item {\tt autounfold with \ident$_1$ \dots\ \ident$_n$ in \textit{clause}}
  
  Perform the unfolding in the given clause.

\item {\tt autounfold with *}
  
  Uses the unfold hints declared in all the hint databases.
\end{Variants}


% EXISTE ENCORE ?
% 
% \subsection{\tt Prolog [ \term$_1$ \dots\ \term$_n$ ] \num}
% \tacindex{Prolog}\label{Prolog}
% This tactic, implemented by Chet Murthy, is based upon the concept of
% existential variables of Gilles Dowek, stating that resolution is a
% kind of unification. It tries to solve the current goal using the {\tt
%   Assumption} tactic, the {\tt intro} tactic, and applying hypotheses
% of the local context and terms of the given list {\tt [ \term$_1$
%   \dots\ \term$_n$\ ]}.  It is more powerful than {\tt auto} since it
% may apply to any theorem, even those of the form {\tt (x:A)(P x) -> Q}
% where {\tt x} does not appear free in {\tt Q}.  The maximal search
% depth is {\tt \num}.

% \begin{ErrMsgs}
% \item \errindex{Prolog failed}\\
%   The Prolog tactic was not able to prove the subgoal.
% \end{ErrMsgs}

\subsection{\tt tauto
\tacindex{tauto}
\label{tauto}}

This tactic implements a decision procedure for intuitionistic propositional
calculus based on the contraction-free sequent calculi LJT* of Roy Dyckhoff
\cite{Dyc92}. Note that {\tt tauto} succeeds on any instance of an
intuitionistic tautological proposition. {\tt tauto} unfolds negations
and logical equivalence but does not unfold any other definition.

The following goal can be proved by {\tt tauto} whereas {\tt auto}
would fail:

\begin{coq_example}
Goal forall (x:nat) (P:nat -> Prop), x = 0 \/ P x -> x <> 0 -> P x.
  intros.
  tauto.
\end{coq_example}
\begin{coq_eval}
Abort.
\end{coq_eval}

Moreover, if it has nothing else to do, {\tt tauto} performs
introductions. Therefore, the use of {\tt intros} in the previous
proof is unnecessary. {\tt tauto} can for instance prove the
following:
\begin{coq_example}
(* auto would fail *)
Goal forall (A:Prop) (P:nat -> Prop),
    A \/ (forall x:nat, ~ A -> P x) -> forall x:nat, ~ A -> P x.

  tauto.
\end{coq_example}
\begin{coq_eval}
Abort.
\end{coq_eval}

\Rem In contrast, {\tt tauto} cannot solve the following goal

\begin{coq_example*}
Goal forall (A:Prop) (P:nat -> Prop),
    A \/ (forall x:nat, ~ A -> P x) -> forall x:nat, ~ ~ (A \/ P x).
\end{coq_example*}
\begin{coq_eval}
Abort.
\end{coq_eval}

because \verb=(forall x:nat, ~ A -> P x)= cannot be treated as atomic and an
instantiation of \verb=x= is necessary.

\subsection{\tt intuition {\tac}
\tacindex{intuition}
\label{intuition}}

The tactic \texttt{intuition} takes advantage of the search-tree built
by the decision procedure involved in the tactic {\tt tauto}. It uses
this information to generate a set of subgoals equivalent to the
original one (but simpler than it) and applies the tactic 
{\tac} to them \cite{Mun94}. If this tactic fails on some goals then
{\tt intuition} fails. In fact, {\tt tauto} is simply {\tt intuition
  fail}.

For instance, the tactic {\tt intuition auto} applied to the goal
\begin{verbatim}
(forall (x:nat), P x)/\B -> (forall (y:nat),P y)/\ P O \/B/\ P O
\end{verbatim}
internally replaces it by the equivalent one:
\begin{verbatim}
(forall (x:nat), P x), B |- P O
\end{verbatim}
and then uses {\tt auto} which completes the proof.

Originally due to C{\'e}sar~Mu{\~n}oz, these tactics ({\tt tauto} and {\tt intuition})
have been completely re-engineered by David~Delahaye using mainly the tactic
language (see Chapter~\ref{TacticLanguage}). The code is now much shorter and
a significant increase in performance has been noticed. The general behavior
with respect to dependent types, unfolding and introductions has
slightly changed to get clearer semantics. This may lead to some
incompatibilities.

\begin{Variants}
\item {\tt intuition}\\
  Is equivalent to {\tt intuition auto with *}.
\end{Variants}

% En attente d'un moyen de valoriser les fichiers de demos
%\SeeAlso file \texttt{contrib/Rocq/DEMOS/Demo\_tauto.v}


\subsection{\tt rtauto
\tacindex{rtauto}
\label{rtauto}}

The {\tt rtauto} tactic solves propositional tautologies similarly to what {\tt tauto} does. The main difference is that the proof term is built using a reflection scheme applied to a sequent calculus proof of the goal. The search procedure is also implemented using a different technique. 

Users should be aware that this difference may result in faster proof-search but  slower proof-checking, and {\tt rtauto} might not solve goals that {\tt tauto} would be able to solve (e.g. goals involving universal quantifiers). 

\subsection{{\tt firstorder}
\tacindex{firstorder}
\label{firstorder}}

The tactic \texttt{firstorder} is an {\it experimental} extension of
\texttt{tauto} to  
first-order reasoning, written by Pierre Corbineau. 
It is not restricted to usual logical connectives but
instead may reason about any first-order class inductive definition.

\begin{Variants}
 \item {\tt firstorder {\tac}}
   \tacindex{firstorder {\tac}}

   Tries to solve the goal with {\tac} when no logical rule may apply.

 \item {\tt firstorder with \ident$_1$ \dots\ \ident$_n$ }
   \tacindex{firstorder with}

   Adds lemmas \ident$_1$ \dots\ \ident$_n$ to the proof-search
   environment.

 \item {\tt firstorder using {\qualid}$_1$ , \dots\ , {\qualid}$_n$ }
   \tacindex{firstorder using}

   Adds lemmas in {\tt auto} hints bases {\qualid}$_1$ \dots\ {\qualid}$_n$
   to the proof-search environment. If {\qualid}$_i$ refers to an inductive
   type, it is the collection of its constructors which is added as hints.

\item \texttt{firstorder using {\qualid}$_1$ , \dots\ , {\qualid}$_n$ with \ident$_1$ \dots\ \ident$_n$}

  This combines the effects of the {\tt using} and {\tt with} options.

\end{Variants}

Proof-search is bounded by a depth parameter which can be set by typing the
{\nobreak \tt Set Firstorder Depth $n$} \comindex{Set Firstorder Depth} 
vernacular command.

%% \subsection{{\tt jp} {\em (Jprover)}
%% \tacindex{jp}
%% \label{jprover}}

%% The tactic \texttt{jp}, due to Huang Guan-Shieng, is an experimental
%% port of the {\em Jprover}\cite{SLKN01} semi-decision procedure for
%% first-order intuitionistic logic implemented in {\em
%%   NuPRL}\cite{Kre02}.

%% The tactic \texttt{jp}, due to Huang Guan-Shieng, is an {\it
%%   experimental} port of the {\em Jprover}\cite{SLKN01} semi-decision 
%% procedure for first-order intuitionistic logic implemented in {\em
%%   NuPRL}\cite{Kre02}. 

%% Search may optionnaly be bounded by a multiplicity parameter
%% indicating how many (at most) copies of a formula may be used in 
%% the proof process, its absence may lead to non-termination of the tactic.

%% %\begin{coq_eval}
%% %Variable S:Set.
%% %Variables P Q:S->Prop.
%% %Variable f:S->S.
%% %\end{coq_eval}

%% %\begin{coq_example*}
%% %Lemma example: (exists x |P x\/Q x)->(exists x |P x)\/(exists x |Q x).
%% %jp.
%% %Qed.

%% %Lemma example2: (forall x ,P x->P (f x))->forall x,P x->P (f(f x)).
%% %jp.
%% %Qed.
%% %\end{coq_example*}

%% \begin{Variants}
%%  \item {\tt jp $n$}\\
%%    \tacindex{jp $n$} 
%%    Tries the {\em Jprover} procedure with multiplicities up to $n$,
%%    starting from 1.
%%  \item {\tt jp}\\
%%    Tries the {\em Jprover} procedure without multiplicity bound, 
%%    possibly running forever.
%% \end{Variants}

%% \begin{ErrMsgs}
%%  \item \errindex{multiplicity limit reached}\\
%%    The procedure tried all multiplicities below the limit and
%%    failed. Goal might be solved by increasing the multiplicity limit. 
%%  \item \errindex{formula is not provable}\\
%%    The procedure determined that goal was not provable in
%%    intuitionistic first-order logic, no matter how big the
%%    multiplicity is.
%% \end{ErrMsgs}


% \subsection[\tt Linear]{\tt Linear\tacindex{Linear}\label{Linear}}
% The tactic \texttt{Linear}, due to Jean-Christophe Filli{\^a}atre
% \cite{Fil94}, implements a decision procedure for {\em Direct
%   Predicate Calculus}, that is first-order Gentzen's Sequent Calculus
% without contraction rules \cite{KeWe84,BeKe92}.  Intuitively, a
% first-order goal is provable in Direct Predicate Calculus if it can be
% proved using each hypothesis at most once.

% Unlike the previous tactics, the \texttt{Linear} tactic does not belong
% to the initial state of the system, and it must be loaded explicitly
% with the command

% \begin{coq_example*}
% Require Linear.
% \end{coq_example*}

% For instance, assuming that \texttt{even} and \texttt{odd} are two
% predicates on natural numbers, and \texttt{a} of type \texttt{nat}, the
% tactic \texttt{Linear} solves the following goal

% \begin{coq_eval}
% Variables even,odd : nat -> Prop.
% Variable a:nat.
% \end{coq_eval}

% \begin{coq_example*}
% Lemma example : (even a) 
%               -> ((x:nat)((even x)->(odd (S x))))
%               -> (EX y | (odd y)).
% \end{coq_example*}

% You can find examples of the use of \texttt{Linear} in
% \texttt{theories/DEMOS/DemoLinear.v}.
% \begin{coq_eval}
% Abort.
% \end{coq_eval}

% \begin{Variants}
% \item {\tt Linear with \ident$_1$ \dots\ \ident$_n$}\\
%   \tacindex{Linear with} 
%   Is equivalent to apply first {\tt generalize \ident$_1$ \dots
%     \ident$_n$} (see Section~\ref{generalize}) then the \texttt{Linear}
%   tactic.  So one can use axioms, lemmas or hypotheses of the local
%   context with \texttt{Linear} in this way.
% \end{Variants}

% \begin{ErrMsgs}
% \item \errindex{Not provable in Direct Predicate Calculus}
% \item \errindex{Found $n$ classical proof(s) but no intuitionistic one}\\ 
%   The decision procedure looks actually for classical proofs of the
%   goals, and then checks that they are intuitionistic.  In that case,
%   classical proofs have been found, which do not correspond to
%   intuitionistic ones.
% \end{ErrMsgs}

\subsection{\tt congruence
\tacindex{congruence}
\label{congruence}}

The tactic {\tt congruence}, by Pierre Corbineau, implements the standard Nelson and Oppen
congruence closure algorithm, which is a decision procedure for ground
equalities with uninterpreted symbols. It also include the constructor theory
(see \ref{injection} and \ref{discriminate}).
If the goal is a non-quantified equality, {\tt congruence} tries to
prove it with non-quantified equalities in the context. Otherwise it
tries to infer a discriminable equality from those in the context. Alternatively, congruence tries to prove that a hypothesis is equal to the goal or to the negation of another hypothesis.

{\tt congruence} is also able to take advantage of hypotheses stating quantified equalities, you have to provide a bound for the number of extra equalities generated that way. Please note that one of the members of the equality must contain all the quantified variables in order for {\tt congruence} to match against it. 

\begin{coq_eval}
Reset Initial.
Variable A:Set.
Variables a b:A.
Variable f:A->A.
Variable g:A->A->A.
\end{coq_eval}

\begin{coq_example}
Theorem T: 
  a=(f a) -> (g b (f a))=(f (f a)) -> (g a b)=(f (g b a)) -> (g a b)=a.
intros.
congruence.
\end{coq_example}

\begin{coq_eval}
Reset Initial.
Variable A:Set.
Variables a c d:A.
Variable f:A->A*A.
\end{coq_eval}

\begin{coq_example}
Theorem inj : f = pair a -> Some (f c) = Some (f d) -> c=d.
intros.
congruence.
\end{coq_example}

\begin{Variants}
 \item {\tt congruence {\sl n}}\\
  Tries to add at most {\tt \sl n} instances of hypotheses stating quantified equalities to the problem in order to solve it. A bigger value of {\tt \sl n} does not make success slower, only failure. You might consider adding some lemmas as hypotheses using {\tt assert} in order for congruence to use them.

\end{Variants}

\begin{Variants}
\item {\tt congruence with \term$_1$ \dots\ \term$_n$}\\
  Adds {\tt \term$_1$ \dots\ \term$_n$} to the pool of terms used by
  {\tt congruence}. This helps in case you have partially applied
  constructors in your goal.
\end{Variants}

\begin{ErrMsgs}
  \item \errindex{I don't know how to handle dependent equality} \\
    The decision procedure managed to find a proof of the goal or of
    a discriminable equality but this proof couldn't be built in {\Coq}
    because of dependently-typed functions.
  \item \errindex{I couldn't solve goal} \\
    The decision procedure didn't find any way to solve the goal.
  \item \errindex{Goal is solvable by congruence but some arguments are missing. Try "congruence with \dots", replacing metavariables by arbitrary terms.} \\
    The decision procedure could solve the goal with the provision
    that additional arguments are supplied for some partially applied
    constructors. Any term of an appropriate type will allow the
    tactic to successfully solve the goal. Those additional arguments
    can be given to {\tt congruence} by filling in the holes in the
    terms given in the error message, using the {\tt with} variant
    described above.
\end{ErrMsgs}

\subsection{\tt omega
\tacindex{omega}
\label{omega}}

The tactic \texttt{omega}, due to Pierre Cr{\'e}gut,
is an automatic decision procedure for Presburger
arithmetic. It solves quantifier-free 
formulas built with \verb|~|, \verb|\/|, \verb|/\|,
\verb|->| on top of equalities, inequalities and disequalities on
both the type \texttt{nat} of natural numbers and \texttt{Z} of binary
integers. This tactic must be loaded by the command \texttt{Require Import
  Omega}. See the additional documentation about \texttt{omega}
(see Chapter~\ref{OmegaChapter}).

\subsection{{\tt ring} and {\tt ring\_simplify \term$_1$ \dots\ \term$_n$}
\tacindex{ring}
\tacindex{ring\_simplify}
\comindex{Add Ring}}

The {\tt ring} tactic solves equations upon polynomial expressions of
a ring (or semi-ring) structure. It proceeds by normalizing both hand
sides of the equation (w.r.t. associativity, commutativity and
distributivity, constant propagation) and comparing syntactically the
results.

{\tt ring\_simplify} applies the normalization procedure described
above to the terms given. The tactic then replaces all occurrences of
the terms given in the conclusion of the goal by their normal
forms. If no term is given, then the conclusion should be an equation
and both hand sides are normalized.

See Chapter~\ref{ring} for more information on the tactic and how to
declare new ring structures.

\subsection{{\tt field}, {\tt field\_simplify \term$_1$\dots\ \term$_n$}
            and {\tt field\_simplify\_eq}
\tacindex{field}
\tacindex{field\_simplify}
\tacindex{field\_simplify\_eq}
\comindex{Add Field}}

The {\tt field} tactic is built on the same ideas as {\tt ring}: this
is a reflexive tactic that solves or simplifies equations in a field
structure. The main idea is to reduce a field expression (which is an
extension of ring expressions with the inverse and division
operations) to a fraction made of two polynomial expressions.

Tactic {\tt field} is used to solve subgoals, whereas {\tt
  field\_simplify \term$_1$\dots\term$_n$} replaces the provided terms
by their reduced fraction. {\tt field\_simplify\_eq} applies when the
conclusion is an equation: it simplifies both hand sides and multiplies
so as to cancel denominators. So it produces an equation without
division nor inverse.

All of these 3 tactics may generate a subgoal in order to prove that
denominators are different from zero.

See Chapter~\ref{ring} for more information on the tactic and how to
declare new field structures.

\Example
\begin{coq_example*}
Require Import Reals.
Goal forall x y:R,
    (x * y > 0)%R ->
    (x * (1 / x + x / (x + y)))%R =
    ((- 1 / y) * y * (- x * (x / (x + y)) - 1))%R.
\end{coq_example*}

\begin{coq_example}
intros; field.
\end{coq_example}

\begin{coq_eval}
Reset Initial.
\end{coq_eval}

\SeeAlso file {\tt plugins/setoid\_ring/RealField.v} for an example of instantiation,\\
\phantom{\SeeAlso}theory {\tt theories/Reals} for many examples of use of {\tt
field}.

\subsection{\tt fourier
\tacindex{fourier}}

This tactic written by Lo{\"\i}c Pottier solves linear inequalities on
real numbers using Fourier's method~\cite{Fourier}. This tactic must
be loaded by {\tt Require Import Fourier}.

\Example
\begin{coq_example*}
Require Import Reals.
Require Import Fourier.
Goal forall x y:R, (x < y)%R -> (y + 1 >= x - 1)%R.
\end{coq_example*}

\begin{coq_example}
intros; fourier.
\end{coq_example}

\begin{coq_eval}
Reset Initial.
\end{coq_eval}

\subsection{\tt autorewrite with \ident$_1$ \dots \ident$_n$.
\label{tactic:autorewrite}
\tacindex{autorewrite}}

This tactic \footnote{The behavior of this tactic has much changed compared to
the versions available in the previous distributions (V6). This may cause
significant changes in your theories to obtain the same result. As a drawback
of the re-engineering of the code, this tactic has also been completely revised
to get a very compact and readable version.} carries out rewritings according
the rewriting rule bases {\tt \ident$_1$ \dots \ident$_n$}.

Each rewriting rule of a base \ident$_i$ is applied to the main subgoal until
it fails. Once all the rules have been processed, if the main subgoal has
progressed (e.g., if it is distinct from the initial main goal) then the rules
of this base are processed again. If the main subgoal has not progressed then
the next base is processed. For the bases, the behavior is exactly similar to
the processing of the rewriting rules.

The rewriting rule bases are built with the {\tt Hint~Rewrite} vernacular
command.

\Warning{} This tactic may loop if you build non terminating rewriting systems.

\begin{Variant}
\item {\tt autorewrite with \ident$_1$ \dots \ident$_n$ using \tac}\\
Performs, in the same way, all the rewritings of the bases {\tt \ident$_1$ $...$
\ident$_n$} applying {\tt \tac} to the main subgoal after each rewriting step.

\item \texttt{autorewrite with {\ident$_1$} \dots \ident$_n$ in {\qualid}}

  Performs all the rewritings in hypothesis {\qualid}.
\item \texttt{autorewrite with {\ident$_1$} \dots \ident$_n$ in {\qualid} using \tac}

  Performs all  the rewritings  in hypothesis {\qualid}  applying {\tt
    \tac} to the main subgoal after each rewriting step.

\item \texttt{autorewrite with {\ident$_1$} \dots \ident$_n$ in \textit{clause}}
  Performs all  the rewritings  in the clause \textit{clause}. \\
  The  \textit{clause} argument must  not contain  any \texttt{type  of} nor  \texttt{value  of}.

\end{Variant}

\SeeAlso Section~\ref{HintRewrite} for feeding the database of lemmas used by {\tt autorewrite}.

\SeeAlso Section~\ref{autorewrite-example} for examples showing the use of
this tactic. 

% En attente d'un moyen de valoriser les fichiers de demos
%\SeeAlso file \texttt{contrib/Rocq/DEMOS/Demo\_AutoRewrite.v}

\section{Controlling automation}

\subsection{The hints databases for {\tt auto} and {\tt eauto}
\index{Hints databases}
\label{Hints-databases}
\comindex{Hint}}

The hints for \texttt{auto} and \texttt{eauto} are stored in
databases.  Each database maps head symbols to a list of hints. One can
use the command \texttt{Print Hint \ident} to display the hints
associated to the head symbol \ident{} (see \ref{PrintHint}). Each
hint has a cost that is an nonnegative integer, and an optional pattern. 
The hints with lower cost are tried first. A hint is tried by 
\texttt{auto} when the conclusion of the current goal
matches its pattern or when it has no pattern. 

\subsubsection*{Creating Hint databases
  \label{CreateHintDb}\comindex{CreateHintDb}}

One can optionally declare a hint database using the command
\texttt{Create HintDb}. If a hint is added to an unknown database, it
will be automatically created. 

\medskip
\texttt{Create HintDb} {\ident} [\texttt{discriminated}]
\medskip

This command creates a new database named \ident.
The database is implemented by a Discrimination Tree (DT) that serves as
an index of all the lemmas. The DT can use transparency information to decide
if a constant should be indexed or not (c.f. \ref{HintTransparency}),
making the retrieval more efficient.
The legacy implementation (the default one for new databases) uses the
DT only on goals without existentials (i.e., auto goals), for non-Immediate
hints and do not make use of transparency hints, putting more work on the
unification that is run after retrieval (it keeps a list of the lemmas
in case the DT is not used). The new implementation enabled by
the {\tt discriminated} option makes use of DTs in all cases and takes
transparency information into account. However, the order in which hints
are retrieved from the DT may differ from the order in which they were
inserted, making this implementation observationaly different from the
legacy one. 

\begin{Variants}
\item\texttt{Local Hint} \textsl{hint\_definition} \texttt{:}
  \ident$_1$ \ldots\ \ident$_n$
  
  This is used to declare a hint database that must not be exported to the other
  modules that require and import the current module. Inside a
  section, the option {\tt Local} is useless since hints do not
  survive anyway to the closure of sections.

\end{Variants}

The general
command to add a hint to some database \ident$_1$, \dots, \ident$_n$ is:
\begin{tabbing}
  \texttt{Hint} \textsl{hint\_definition} \texttt{:} \ident$_1$ \ldots\ \ident$_n$
\end{tabbing}
where {\sl hint\_definition} is one of the following expressions:

\begin{itemize}
\item \texttt{Resolve} {\term} 
  \comindex{Hint Resolve}
  
  This command adds {\tt apply {\term}} to the hint list
  with the head symbol of the type of \term. The cost of that hint is
  the number of subgoals generated by {\tt apply {\term}}.
  
  In case the inferred type of \term\ does not start with a product the
  tactic added in the hint list is {\tt exact {\term}}. In case this
  type can be reduced to a type starting with a product, the tactic {\tt
    apply {\term}} is also stored in the hints list.
  
  If the inferred type of \term\ contains a dependent
  quantification on a predicate, it is added to the hint list of {\tt
    eapply} instead of the hint list of {\tt apply}. In this case, a
  warning is printed since the hint is only used by the tactic {\tt
    eauto} (see \ref{eauto}). A typical example of a hint that is used
  only by \texttt{eauto} is a transitivity lemma.

  \begin{ErrMsgs}
  \item \errindex{Bound head variable}

    The head symbol of the type of {\term} is a bound variable such
    that this tactic cannot be associated to a constant.

  \item \term\ \errindex{cannot be used as a hint}

    The type of \term\ contains products over variables which do not
    appear in the conclusion. A typical example is a transitivity axiom.
    In that case the {\tt apply} tactic fails, and thus is useless.

  \end{ErrMsgs}

  \begin{Variants}

  \item \texttt{Resolve} {\term$_1$} \dots {\term$_m$}

    Adds each \texttt{Resolve} {\term$_i$}.

  \end{Variants}

\item \texttt{Immediate {\term}} 
\comindex{Hint Immediate}
  
  This command adds {\tt apply {\term}; trivial} to the hint list
  associated with the head symbol of the type of {\ident} in the given
  database. This tactic will fail if all the subgoals generated by
  {\tt apply {\term}} are not solved immediately by the {\tt trivial}
  tactic (which only tries tactics with cost $0$).
  
  This command is useful for theorems such as the symmetry of equality
  or $n+1=m+1 \to n=m$ that we may like to introduce with a
  limited use in order to avoid useless proof-search.
  
  The cost of this tactic (which never generates subgoals) is always 1,
  so that it is not used by {\tt trivial} itself.

  \begin{ErrMsgs}

  \item \errindex{Bound head variable}

  \item \term\ \errindex{cannot be used as a hint} 

  \end{ErrMsgs}

  \begin{Variants}

    \item \texttt{Immediate} {\term$_1$} \dots {\term$_m$} 

      Adds each \texttt{Immediate} {\term$_i$}.

  \end{Variants}

\item \texttt{Constructors} {\ident}
\comindex{Hint Constructors}
  
  If {\ident} is an inductive type, this command adds all its
  constructors as hints of type \texttt{Resolve}. Then, when the
  conclusion of current goal has the form \texttt{({\ident} \dots)},
  \texttt{auto} will try to apply each constructor.

  \begin{ErrMsgs}

    \item {\ident} \errindex{is not an inductive type}

    \item {\ident} \errindex{not declared}

  \end{ErrMsgs}

  \begin{Variants}

    \item \texttt{Constructors} {\ident$_1$} \dots {\ident$_m$} 

      Adds each \texttt{Constructors} {\ident$_i$}.

  \end{Variants}

\item \texttt{Unfold} {\qualid}
\comindex{Hint Unfold}
  
  This adds the tactic {\tt unfold {\qualid}} to the hint list that
  will only be used when the head constant of the goal is \ident.  Its
  cost is 4.

  \begin{Variants}

    \item \texttt{Unfold} {\ident$_1$} \dots {\ident$_m$} 

      Adds each \texttt{Unfold} {\ident$_i$}.

  \end{Variants}

\item \texttt{Transparent}, \texttt{Opaque} {\qualid}
\label{HintTransparency}
\comindex{Hint Transparent}
\comindex{Hint Opaque}
  
  This adds a transparency hint to the database, making {\tt {\qualid}}
  a transparent or opaque constant during resolution. This information 
  is used during unification of the goal with any lemma in the database
  and inside the discrimination network to relax or constrain it in the
  case of \texttt{discriminated} databases.
  
  \begin{Variants}

    \item \texttt{Transparent}, \texttt{Opaque} {\ident$_1$} \dots {\ident$_m$} 

      Declares each {\ident$_i$} as a transparent or opaque constant.
      
  \end{Variants}

\item \texttt{Extern \num\ [\pattern]\ => }\textsl{tactic}
\comindex{Hint Extern}

  This hint type is to extend \texttt{auto} with tactics other than
  \texttt{apply} and \texttt{unfold}. For that, we must specify a
  cost, an optional pattern and a tactic to execute. Here is an example:

\begin{quotation}
\begin{verbatim}
Hint Extern 4 (~(_ = _)) => discriminate.
\end{verbatim}
\end{quotation}

  Now, when the head of the goal is a disequality, \texttt{auto} will
  try \texttt{discriminate} if it does not manage to solve the goal
  with hints with a cost less than 4.
  
  One can even use some sub-patterns of the pattern in the tactic
  script. A sub-pattern is a question mark followed by an ident, like
  \texttt{?X1} or \texttt{?X2}. Here is an example:

% Require EqDecide.
\begin{coq_example*}
Require Import List.
\end{coq_example*}
\begin{coq_example}
Hint Extern 5   ({?X1 = ?X2} + {?X1 <> ?X2}) =>
 generalize X1, X2; decide equality : eqdec.
Goal 
forall a b:list (nat * nat), {a = b} + {a <> b}.
info auto with eqdec.
\end{coq_example}
\begin{coq_eval}
Abort.
\end{coq_eval}

\end{itemize}

\Rem One can use an \texttt{Extern} hint with no pattern to do
pattern-matching on hypotheses using \texttt{match goal with} inside 
the tactic.

\begin{Variants}
\item \texttt{Hint} \textsl{hint\_definition} 
  
  No database name is given: the hint is registered in the {\tt core} 
    database. 

\item\texttt{Hint Local} \textsl{hint\_definition} \texttt{:}
   \ident$_1$ \ldots\ \ident$_n$

  This is used to declare hints that must not be exported to the other
  modules that require and import the current module. Inside a
  section, the option {\tt Local} is useless since hints do not
  survive anyway to the closure of sections.

\item\texttt{Hint Local} \textsl{hint\_definition} 

  Idem for the {\tt core} database.
    
\end{Variants}

% There are shortcuts that allow to define several goal at once:

% \begin{itemize}
% \item \comindex{Hints Resolve}\texttt{Hints Resolve \ident$_1$ \dots\ \ident$_n$ : \ident.}\\
%   This command is a shortcut for the following ones:
%   \begin{quotation}
%    \noindent\texttt{Hint \ident$_1$ : \ident\ := Resolve \ident$_1$}\\
%    \dots\\
%    \texttt{Hint \ident$_1$ : \ident := Resolve \ident$_1$}
%   \end{quotation}
%   Notice that the hint name is the same that the theorem given as
%   hint.
% \item \comindex{Hints Immediate}\texttt{Hints Immediate \ident$_1$ \dots\ \ident$_n$ : \ident.}\\
% \item \comindex{Hints Unfold}\texttt{Hints Unfold \qualid$_1$ \dots\ \qualid$_n$ : \ident.}\\
% \end{itemize}

%\begin{Warnings}
%  \item \texttt{Overriding hint named \dots\ in database \dots}
%\end{Warnings}



\subsection{Hint databases defined in the \Coq\ standard library}

Several hint databases are defined in the \Coq\ standard library.  The
actual content of a database is the collection of the hints declared
to belong to this database in each of the various modules currently
loaded.  Especially, requiring new modules potentially extend a
database. At {\Coq} startup, only the {\tt core} and {\tt v62}
databases are non empty and can be used.

\begin{description}

\item[\tt core] This special database is automatically used by
  \texttt{auto}. It contains only basic lemmas about negation,
  conjunction, and so on from. Most of the hints in this database come 
  from the \texttt{Init} and \texttt{Logic} directories.

\item[\tt arith] This database contains all lemmas about Peano's
  arithmetic proved in the directories \texttt{Init} and
  \texttt{Arith}

\item[\tt zarith] contains lemmas about binary signed integers from
  the directories \texttt{theories/ZArith}. When required, the module
  {\tt Omega} also extends the database {\tt zarith} with a high-cost
  hint that calls {\tt omega} on equations and inequalities in {\tt
  nat} or {\tt Z}.

\item[\tt bool] contains lemmas about booleans, mostly from directory
  \texttt{theories/Bool}.

\item[\tt datatypes] is for lemmas about lists, streams and so on that 
  are mainly proved in the \texttt{Lists} subdirectory.

\item[\tt sets] contains lemmas about sets and relations from the 
  directories \texttt{Sets} and \texttt{Relations}.

\item[\tt typeclass\_instances] contains all the type class instances
  declared in the environment, including those used for \texttt{setoid\_rewrite},
  from the \texttt{Classes} directory.
\end{description}

There is also a special database called {\tt v62}. It collects all
hints that were declared in the versions of {\Coq} prior to version
6.2.4 when the databases {\tt core}, {\tt arith}, and so on were
introduced.  The purpose of the database {\tt v62} is to ensure
compatibility with further versions of {\Coq} for developments done in
versions prior to 6.2.4 ({\tt auto} being replaced by {\tt auto with v62}).
The database {\tt v62} is intended not to be extended (!). It is not
included in the hint databases list used in the {\tt auto with *} tactic.

Furthermore, you are advised not to put your own hints in the
{\tt core} database, but use one or several databases specific to your
development.

\subsection{\tt Print Hint
\label{PrintHint}
\comindex{Print Hint}}

This command displays all hints that apply to the current goal. It
fails if no proof is being edited, while the two variants can be used at
every moment.

\begin{Variants}

\item {\tt  Print Hint {\ident} }

 This command displays only tactics associated with \ident\ in the
 hints list. This is independent of the goal being edited, so this
 command will not fail if no goal is being edited.

\item {\tt Print Hint *}

  This command displays all declared hints. 

\item {\tt  Print HintDb {\ident} }
\label{PrintHintDb}
\comindex{Print HintDb}

 This command displays all hints from database \ident.

\end{Variants}

\subsection{\tt Hint Rewrite \term$_1$ \dots \term$_n$ : \ident
\label{HintRewrite}
\comindex{Hint Rewrite}}

This vernacular command adds the terms {\tt \term$_1$ \dots \term$_n$}
(their types must be equalities) in the rewriting base {\tt \ident}
with the default orientation (left to right). Notice that the
rewriting bases are distinct from the {\tt auto} hint bases and that
{\tt auto} does not take them into account.

This command is synchronous with the section mechanism (see \ref{Section}):
when closing a section, all aliases created by \texttt{Hint Rewrite} in that
section are lost. Conversely, when loading a module, all \texttt{Hint Rewrite}
declarations at the global level of that module are loaded.

\begin{Variants}
\item {\tt Hint Rewrite -> \term$_1$ \dots \term$_n$ : \ident}\\
This is strictly equivalent to the command above (we only make explicit the
orientation which otherwise defaults to {\tt ->}).

\item {\tt Hint Rewrite <- \term$_1$ \dots \term$_n$ : \ident}\\
Adds the rewriting rules {\tt \term$_1$ \dots \term$_n$} with a right-to-left
orientation in the base {\tt \ident}.

\item {\tt Hint Rewrite \term$_1$ \dots \term$_n$ using {\tac} : {\ident}}\\
When the rewriting rules {\tt \term$_1$ \dots \term$_n$} in {\tt \ident} will
be used, the tactic {\tt \tac} will be applied to the generated subgoals, the
main subgoal excluded.

%% \item 
%% {\tt Hint Rewrite [ \term$_1$ \dots \term$_n$ ] in \ident}\\
%% {\tt Hint Rewrite [ \term$_1$ \dots \term$_n$ ] in {\ident} using {\tac}}\\
%% These are deprecated syntactic variants for
%% {\tt Hint Rewrite \term$_1$ \dots \term$_n$ : \ident} and
%% {\tt Hint Rewrite \term$_1$ \dots \term$_n$ using {\tac} : {\ident}}.

\item \texttt{Print Rewrite HintDb {\ident}}

  This command displays all rewrite hints contained in {\ident}.

\end{Variants}

\subsection{Hints and sections
\label{Hint-and-Section}}

Hints provided by the \texttt{Hint} commands are erased when closing a
section. Conversely, all hints of a module \texttt{A} that are not
defined inside a section (and not defined with option {\tt Local}) become
available when the module {\tt A} is imported (using
e.g. \texttt{Require Import A.}).

\subsection{Setting implicit automation tactics}

\subsubsection[\tt Proof with {\tac}.]{\tt Proof with {\tac}.\label{ProofWith}
\comindex{Proof with}}

  This command may be used to start a proof. It defines a default
  tactic to be used each time a tactic command {\tac$_1$} is ended by
  ``\verb#...#''. In this case the tactic command typed by the user is
  equivalent to \tac$_1$;{\tac}.

\SeeAlso {\tt Proof.} in Section~\ref{BeginProof}.

\subsubsection[\tt Declare Implicit Tactic {\tac}.]{\tt Declare Implicit Tactic {\tac}.\comindex{Declare Implicit Tactic}}

This command declares a tactic to be used to solve implicit arguments
that {\Coq} does not know how to solve by unification. It is used
every time the term argument of a tactic has one of its holes not
fully resolved.

Here is an example:

\begin{coq_example}
Parameter quo : nat -> forall n:nat, n<>0 -> nat.
Notation "x // y" := (quo x y _) (at level 40).

Declare Implicit Tactic assumption.
Goal forall n m, m<>0 -> { q:nat & { r | q * m + r = n } }.
intros.
exists (n // m).
\end{coq_example}

The tactic {\tt exists (n // m)} did not fail. The hole was solved by
{\tt assumption} so that it behaved as {\tt exists (quo n m H)}.

\section{Generation of induction principles with {\tt Scheme}
\label{Scheme}
\index{Schemes}
\comindex{Scheme}}

The {\tt Scheme} command is a high-level tool for generating
automatically (possibly mutual) induction principles for given types
and sorts.  Its syntax follows the schema:
\begin{quote}
{\tt Scheme {\ident$_1$} := Induction for \ident'$_1$ Sort {\sort$_1$} \\
  with\\
  \mbox{}\hspace{0.1cm} \dots\\
        with {\ident$_m$} := Induction for {\ident'$_m$} Sort
        {\sort$_m$}}
\end{quote}
where \ident'$_1$ \dots\ \ident'$_m$ are different inductive type
identifiers belonging to the same package of mutual inductive
definitions. This command generates {\ident$_1$}\dots{} {\ident$_m$}
to be mutually recursive definitions. Each term {\ident$_i$} proves a
general principle of mutual induction for objects in type {\term$_i$}.

\begin{Variants}
\item {\tt Scheme {\ident$_1$} := Minimality for \ident'$_1$ Sort {\sort$_1$} \\
    with\\
    \mbox{}\hspace{0.1cm} \dots\ \\
    with {\ident$_m$} := Minimality for {\ident'$_m$} Sort
    {\sort$_m$}}

  Same as before but defines a non-dependent elimination principle more
  natural in case of inductively defined relations. 

\item {\tt Scheme Equality for \ident$_1$\comindex{Scheme Equality}}

  Tries to generate a boolean equality and a proof of the
  decidability of the usual equality.

\item {\tt Scheme Induction for \ident$_1$ Sort {\sort$_1$} \\
  with\\
  \mbox{}\hspace{0.1cm} \dots\\
        with Induction for {\ident$_m$} Sort
        {\sort$_m$}}

  If you do not provide the name of the schemes, they will be automatically 
  computed from the sorts involved (works also with Minimality).

\end{Variants}

\SeeAlso Section~\ref{Scheme-examples}

\subsection{Automatic declaration of schemes}
\comindex{Set Equality Schemes}
\comindex{Set Elimination Schemes}
It is possible to deactivate the automatic declaration of the induction
 principles when defining a new inductive type  with the
 {\tt Unset Elimination Schemes} command. It may be
reactivated at any time with {\tt Set Elimination Schemes}. 
\\

You can also activate the automatic declaration of those boolean equalities 
(see the second variant of {\tt Scheme})  with the {\tt Set Equality Schemes}
 command. However you have to be careful with this option since
\Coq~ may now reject well-defined inductive types because it cannot compute
a boolean equality for them.

\subsection{\tt Combined Scheme\label{CombinedScheme}
\comindex{Combined Scheme}}
The {\tt Combined Scheme} command is a tool for combining 
induction principles generated by the {\tt Scheme} command.
Its syntax follows the schema :

\noindent
{\tt Combined Scheme {\ident$_0$} from {\ident$_1$}, .., {\ident$_n$}}\\
\ident$_1$ \ldots \ident$_n$ are different inductive principles that must belong to
the same package of mutual inductive principle definitions. This command
generates {\ident$_0$} to be the conjunction of the principles: it is
built from the common premises of the principles and concluded by the
conjunction of their conclusions.

\SeeAlso Section~\ref{CombinedScheme-examples}

\section{Generation of induction principles with {\tt Functional Scheme}
\label{FunScheme}
\comindex{Functional Scheme}}

The {\tt Functional Scheme} command is a high-level experimental
tool for generating automatically induction principles
corresponding to (possibly mutually recursive) functions.  Its
syntax follows the schema:
\begin{quote}
{\tt Functional Scheme {\ident$_1$} := Induction for \ident'$_1$ Sort {\sort$_1$} \\
  with\\
  \mbox{}\hspace{0.1cm} \dots\ \\
        with {\ident$_m$} := Induction for {\ident'$_m$} Sort
        {\sort$_m$}}
\end{quote}  
where \ident'$_1$ \dots\ \ident'$_m$ are different mutually defined function
names (they must be in the same order as when they were defined).
This command generates the induction principles
\ident$_1$\dots\ident$_m$, following the recursive structure and case
analyses of the functions \ident'$_1$ \dots\ \ident'$_m$.


\paragraph{\texttt{Functional Scheme}} 
There is a difference between obtaining an induction scheme by using
\texttt{Functional Scheme} on a function defined by \texttt{Function}
or not. Indeed \texttt{Function} generally produces smaller
principles, closer to the definition written by the user.


\SeeAlso Section~\ref{FunScheme-examples}


\section{Simple tactic macros
\index{Tactic macros}
\comindex{Tactic Definition}
\label{TacticDefinition}}

A simple example has more value than a long explanation: 

\begin{coq_example}
Ltac Solve := simpl; intros; auto.
Ltac ElimBoolRewrite b H1 H2 :=
  elim b; [ intros; rewrite H1; eauto | intros; rewrite H2; eauto ].
\end{coq_example}

The tactics macros are synchronous with the \Coq\ section mechanism:
a tactic definition is deleted from the current environment
when you close the section (see also \ref{Section}) 
where it was defined. If you want that a
tactic macro defined in a module is usable in the modules that
require it, you should put it outside of any section.

Chapter~\ref{TacticLanguage} gives examples of more complex
user-defined tactics.


% $Id$ 

%%% Local Variables: 
%%% mode: latex
%%% TeX-master: "Reference-Manual"
%%% TeX-master: "Reference-Manual"
%%% End: 
% Tactics and tacticals
\chapter[The tactic language]{The tactic language\label{TacticLanguage}}

%\geometry{a4paper,body={5in,8in}}

This chapter gives a compact documentation of Ltac, the tactic
language available in {\Coq}. We start by giving the syntax, and next,
we present the informal semantics. If you want to know more regarding
this language and especially about its foundations, you can refer
to~\cite{Del00}. Chapter~\ref{Tactics-examples} is devoted to giving
examples of use of this language on small but also with non-trivial
problems.


\section{Syntax}

\def\tacexpr{\textrm{\textsl{expr}}}
\def\tacexprlow{\textrm{\textsl{tacexpr$_1$}}}
\def\tacexprinf{\textrm{\textsl{tacexpr$_2$}}}
\def\tacexprpref{\textrm{\textsl{tacexpr$_3$}}}
\def\atom{\textrm{\textsl{atom}}}
%%\def\recclause{\textrm{\textsl{rec\_clause}}}
\def\letclause{\textrm{\textsl{let\_clause}}}
\def\matchrule{\textrm{\textsl{match\_rule}}}
\def\contextrule{\textrm{\textsl{context\_rule}}}
\def\contexthyp{\textrm{\textsl{context\_hyp}}}
\def\tacarg{\nterm{tacarg}}
\def\cpattern{\nterm{cpattern}}

The syntax of the tactic language is given Figures~\ref{ltac}
and~\ref{ltac_aux}. See Chapter~\ref{BNF-syntax} for a description of
the BNF metasyntax used in these grammar rules. Various already
defined entries will be used in this chapter: entries
{\naturalnumber}, {\integer}, {\ident}, {\qualid}, {\term},
{\cpattern} and {\atomictac} represent respectively the natural and
integer numbers, the authorized identificators and qualified names,
{\Coq}'s terms and patterns and all the atomic tactics described in
Chapter~\ref{Tactics}. The syntax of {\cpattern} is the same as that
of terms, but it is extended with pattern matching metavariables. In
{\cpattern}, a pattern-matching metavariable is represented with the
syntax {\tt ?id} where {\tt id} is an {\ident}. The notation {\tt \_}
can also be used to denote metavariable whose instance is
irrelevant. In the notation {\tt ?id}, the identifier allows us to
keep instantiations and to make constraints whereas {\tt \_} shows
that we are not interested in what will be matched. On the right hand
side of pattern-matching clauses, the named metavariable are used
without the question mark prefix. There is also a special notation for
second-order pattern-matching problems: in an applicative pattern of
the form {\tt @?id id$_1$ \ldots id$_n$}, the variable {\tt id}
matches any complex expression with (possible) dependencies in the
variables {\tt id$_1$ \ldots id$_n$} and returns a functional term of
the form {\tt fun id$_1$ \ldots id$_n$ => {\term}}.


The main entry of the grammar is {\tacexpr}. This language is used in
proof mode but it can also be used in toplevel definitions as shown in
Figure~\ref{ltactop}.

\begin{Remarks}
\item The infix tacticals ``\dots\ {\tt ||} \dots'' and ``\dots\ {\tt
    ;} \dots'' are associative. 

\item In {\tacarg}, there is an overlap between {\qualid} as a
direct tactic argument and {\qualid} as a particular case of
{\term}. The resolution is done by first looking for a reference of
the tactic language and if it fails, for a reference to a term. To
force the resolution as a reference of the tactic language, use the
form {\tt ltac :} {\qualid}. To force the resolution as a reference to
a term, use the syntax {\tt ({\qualid})}.

\item As shown by the figure, tactical {\tt ||} binds more than the
prefix tacticals {\tt try}, {\tt repeat}, {\tt do}, {\tt info} and
{\tt abstract} which themselves bind more than the postfix tactical
``{\tt \dots\ ;[ \dots\ ]}'' which binds more than ``\dots\ {\tt ;}
\dots''.

For instance
\begin{quote}
{\tt try repeat \tac$_1$ ||
  \tac$_2$;\tac$_3$;[\tac$_{31}$|\dots|\tac$_{3n}$];\tac$_4$.}
\end{quote}
is understood as 
\begin{quote}
{\tt (try (repeat (\tac$_1$ || \tac$_2$)));} \\
{\tt ((\tac$_3$;[\tac$_{31}$|\dots|\tac$_{3n}$]);\tac$_4$).}
\end{quote}
\end{Remarks}


\begin{figure}[htbp]
\begin{centerframe}
\begin{tabular}{lcl}
{\tacexpr} & ::= &
           {\tacexpr} {\tt ;} {\tacexpr}\\
& | & {\tacexpr} {\tt ; [} \nelist{\tacexpr}{|} {\tt ]}\\
& | & {\tacexprpref}\\
\\
{\tacexprpref} & ::= &
           {\tt do} {\it (}{\naturalnumber} {\it |} {\ident}{\it )} {\tacexprpref}\\
& | & {\tt info} {\tacexprpref}\\
& | & {\tt progress} {\tacexprpref}\\
& | & {\tt repeat} {\tacexprpref}\\
& | & {\tt try} {\tacexprpref}\\
& | & {\tacexprinf} \\
\\
{\tacexprinf} & ::= &
           {\tacexprlow} {\tt ||} {\tacexprpref}\\
& | & {\tacexprlow}\\
\\
{\tacexprlow} & ::= &
{\tt fun} \nelist{\name}{} {\tt =>} {\atom}\\
& | &
{\tt let} \zeroone{\tt rec} \nelist{\letclause}{\tt with} {\tt in}
{\atom}\\
& | &
{\tt match goal with} \nelist{\contextrule}{\tt |} {\tt end}\\
& | &
{\tt match reverse goal with} \nelist{\contextrule}{\tt |} {\tt end}\\
& | &
{\tt match} {\tacexpr} {\tt with} \nelist{\matchrule}{\tt |} {\tt end}\\
& | &
{\tt lazymatch goal with} \nelist{\contextrule}{\tt |} {\tt end}\\
& | &
{\tt lazymatch reverse goal with} \nelist{\contextrule}{\tt |} {\tt end}\\
& | &
{\tt lazymatch} {\tacexpr} {\tt with} \nelist{\matchrule}{\tt |} {\tt end}\\
& | & {\tt abstract} {\atom}\\
& | & {\tt abstract} {\atom} {\tt using} {\ident} \\
& | & {\tt first [} \nelist{\tacexpr}{\tt |} {\tt ]}\\
& | & {\tt solve [} \nelist{\tacexpr}{\tt |} {\tt ]}\\
& | & {\tt idtac} \sequence{\messagetoken}{}\\
& | & {\tt fail} \zeroone{\naturalnumber} \sequence{\messagetoken}{}\\
& | & {\tt fresh} ~|~ {\tt fresh} {\qstring}\\
& | & {\tt context} {\ident} {\tt [} {\term} {\tt ]}\\
& | & {\tt eval} {\nterm{redexpr}} {\tt in} {\term}\\
& | & {\tt type of} {\term}\\
& | & {\tt external} {\qstring} {\qstring} \nelist{\tacarg}{}\\
& | & {\tt constr :} {\term}\\
& | & \atomictac\\
& | & {\qualid} \nelist{\tacarg}{}\\
& | & {\atom}\\
\\
{\atom} & ::= &
           {\qualid} \\
& | & ()\\
& | & {\integer}\\
& | & {\tt (} {\tacexpr} {\tt )}\\
\\
{\messagetoken}\!\!\!\!\!\! & ::= & {\qstring} ~|~ {\ident} ~|~ {\integer} \\
\end{tabular}
\end{centerframe}
\caption{Syntax of the tactic language}
\label{ltac}
\end{figure}



\begin{figure}[htbp]
\begin{centerframe}
\begin{tabular}{lcl}
\tacarg & ::= & 
        {\qualid}\\
& $|$ & {\tt ()} \\
& $|$ & {\tt ltac :} {\atom}\\
& $|$ & {\term}\\
\\
\letclause & ::= & {\ident} \sequence{\name}{} {\tt :=} {\tacexpr}\\
\\
\contextrule & ::= &
  \nelist{\contexthyp}{\tt ,} {\tt |-}{\cpattern} {\tt =>} {\tacexpr}\\
& $|$ & {\tt |-} {\cpattern} {\tt =>} {\tacexpr}\\
& $|$ & {\tt \_ =>} {\tacexpr}\\
\\
\contexthyp & ::= & {\name} {\tt :} {\cpattern}\\
             & $|$ & {\name} {\tt :=} {\cpattern} \zeroone{{\tt :} {\cpattern}}\\
\\
\matchrule & ::= &
           {\cpattern} {\tt =>} {\tacexpr}\\
& $|$ & {\tt context} {\zeroone{\ident}} {\tt [} {\cpattern} {\tt ]}
           {\tt =>} {\tacexpr}\\
& $|$ & {\tt appcontext} {\zeroone{\ident}} {\tt [} {\cpattern} {\tt ]}
           {\tt =>} {\tacexpr}\\
& $|$ & {\tt \_ =>} {\tacexpr}\\
\end{tabular}
\end{centerframe}
\caption{Syntax of the tactic language (continued)}
\label{ltac_aux}
\end{figure}

\begin{figure}[ht]
\begin{centerframe}
\begin{tabular}{lcl}
\nterm{top} & ::= & \zeroone{\tt Local} {\tt Ltac} \nelist{\nterm{ltac\_def}} {\tt with} \\
\\
\nterm{ltac\_def} & ::= & {\ident} \sequence{\ident}{} {\tt :=}
{\tacexpr}\\
& $|$ &{\qualid} \sequence{\ident}{} {\tt ::=}{\tacexpr}
\end{tabular}
\end{centerframe}
\caption{Tactic toplevel definitions}
\label{ltactop}
\end{figure}


%%
%% Semantics
%%
\section{Semantics}
%\index[tactic]{Tacticals}
\index{Tacticals}
%\label{Tacticals}

Tactic expressions can only be applied in the context of a goal.  The
evaluation yields either a term, an integer or a tactic. Intermediary
results can be terms or integers but the final result must be a tactic
which is then applied to the current goal.

There is a special case for {\tt match goal} expressions of which
the clauses evaluate to tactics. Such expressions can only be used as
end result of a tactic expression (never as argument of a non recursive local
definition or of an application).

The rest of this section explains the semantics of every construction
of Ltac.


%% \subsection{Values}

%% Values are given by Figure~\ref{ltacval}. All these values are tactic values,
%% i.e. to be applied to a goal, except {\tt Fun}, {\tt Rec} and $arg$ values.

%% \begin{figure}[ht]
%% \noindent{}\framebox[6in][l]
%% {\parbox{6in}
%% {\begin{center}
%% \begin{tabular}{lp{0.1in}l}
%% $vexpr$ & ::= & $vexpr$ {\tt ;} $vexpr$\\
%% & | & $vexpr$ {\tt ; [} {\it (}$vexpr$ {\tt |}{\it )}$^*$ $vexpr$ {\tt
%% ]}\\
%% & | & $vatom$\\
%% \\
%% $vatom$ & ::= & {\tt Fun} \nelist{\inputfun}{}  {\tt ->} {\tacexpr}\\
%% %& | & {\tt Rec} \recclause\\
%% & | &
%% {\tt Rec} \nelist{\recclause}{\tt And} {\tt In}
%% {\tacexpr}\\
%% & | &
%% {\tt Match Context With} {\it (}$context\_rule$ {\tt |}{\it )}$^*$
%% $context\_rule$\\
%% & | & {\tt (} $vexpr$ {\tt )}\\
%% & | & $vatom$ {\tt Orelse} $vatom$\\
%% & | & {\tt Do} {\it (}{\naturalnumber} {\it |} {\ident}{\it )} $vatom$\\
%% & | & {\tt Repeat} $vatom$\\
%% & | & {\tt Try} $vatom$\\
%% & | & {\tt First [} {\it (}$vexpr$ {\tt |}{\it )}$^*$ $vexpr$ {\tt ]}\\
%% & | & {\tt Solve [} {\it (}$vexpr$ {\tt |}{\it )}$^*$ $vexpr$ {\tt ]}\\
%% & | & {\tt Idtac}\\
%% & | & {\tt Fail}\\
%% & | & {\primitivetactic}\\
%% & | & $arg$
%% \end{tabular}
%% \end{center}}}
%% \caption{Values of ${\cal L}_{tac}$}
%% \label{ltacval}
%% \end{figure}

%% \subsection{Evaluation}

\subsubsection[Sequence]{Sequence\tacindex{;}
\index{Tacticals!;@{\tt {\tac$_1$};\tac$_2$}}}

A sequence is an expression of the following form:
\begin{quote}
{\tacexpr}$_1$ {\tt ;} {\tacexpr}$_2$
\end{quote}
The expressions {\tacexpr}$_1$ and {\tacexpr}$_2$ are evaluated
to $v_1$ and $v_2$ which have to be tactic values. The tactic $v_1$ is
then applied and $v_2$ is applied to every subgoal generated by the
application of $v_1$. Sequence is left-associative.

\subsubsection[General sequence]{General sequence\tacindex{;[\ldots$\mid$\ldots$\mid$\ldots]}}
%\tacindex{; [ | ]}
%\index{; [ | ]@{\tt ;[\ldots$\mid$\ldots$\mid$\ldots]}}
\index{Tacticals!; [ \mid ]@{\tt {\tac$_0$};[{\tac$_1$}$\mid$\ldots$\mid$\tac$_n$]}}

A general sequence has the following form:
\begin{quote}
{\tacexpr}$_0$ {\tt ; [} {\tacexpr}$_1$ {\tt |} $...$ {\tt |}
{\tacexpr}$_n$ {\tt ]}
\end{quote}
The expressions {\tacexpr}$_i$ are evaluated to $v_i$, for $i=0,...,n$
and all have to be tactics. The tactic $v_0$ is applied and $v_i$ is
applied to the $i$-th generated subgoal by the application of $v_0$,
for $=1,...,n$. It fails if the application of $v_0$ does not generate
exactly $n$ subgoals.

\begin{Variants}
  \item If no tactic is given for the $i$-th generated subgoal, it
behaves as if the tactic {\tt idtac} were given. For instance, {\tt
split ; [ | auto ]} is a shortcut for
{\tt split ; [ idtac | auto ]}.

  \item {\tacexpr}$_0$ {\tt ; [} {\tacexpr}$_1$ {\tt |} $...$ {\tt |}
    {\tacexpr}$_i$ {\tt |} {\tt ..} {\tt |} {\tacexpr}$_{i+1+j}$ {\tt |}
    $...$ {\tt |} {\tacexpr}$_n$ {\tt ]}

  In this variant, {\tt idtac} is used for the subgoals numbered from
  $i+1$ to $i+j$ (assuming $n$ is the number of subgoals).

  Note that {\tt ..} is part of the syntax, while $...$ is the meta-symbol used
  to describe a list of {\tacexpr} of arbitrary length.

  \item {\tacexpr}$_0$ {\tt ; [} {\tacexpr}$_1$ {\tt |} $...$ {\tt |}
    {\tacexpr}$_i$ {\tt |} {\tacexpr} {\tt ..} {\tt |}
    {\tacexpr}$_{i+1+j}$ {\tt |} $...$ {\tt |} {\tacexpr}$_n$ {\tt ]}

  In this variant, {\tacexpr} is used instead of {\tt idtac} for the
  subgoals numbered from $i+1$ to $i+j$.

\end{Variants}



\subsubsection[For loop]{For loop\tacindex{do}
\index{Tacticals!do@{\tt do}}}

There is a for loop that repeats a tactic {\num} times:
\begin{quote}
{\tt do} {\num} {\tacexpr}
\end{quote}
{\tacexpr} is evaluated to $v$. $v$ must be a tactic value. $v$ is
applied {\num} times. Supposing ${\num}>1$, after the first
application of $v$, $v$ is applied, at least once, to the generated
subgoals and so on. It fails if the application of $v$ fails before
the {\num} applications have been completed.

\subsubsection[Repeat loop]{Repeat loop\tacindex{repeat}
\index{Tacticals!repeat@{\tt repeat}}}

We have a repeat loop with:
\begin{quote}
{\tt repeat} {\tacexpr}
\end{quote}
{\tacexpr} is evaluated to $v$. If $v$ denotes a tactic, this tactic
is applied to the goal. If the application fails, the tactic is
applied recursively to all the generated subgoals until it eventually
fails.  The recursion stops in a subgoal when the tactic has failed.
The tactic {\tt repeat {\tacexpr}} itself never fails.

\subsubsection[Error catching]{Error catching\tacindex{try}
\index{Tacticals!try@{\tt try}}}

We can catch the tactic errors with:
\begin{quote}
{\tt try} {\tacexpr}
\end{quote}
{\tacexpr} is evaluated to $v$. $v$ must be a tactic value. $v$ is
applied. If the application of $v$ fails, it catches the error and
leaves the goal unchanged. If the level of the exception is positive,
then the exception is re-raised with its level decremented.

\subsubsection[Detecting progress]{Detecting progress\tacindex{progress}}

We can check if a tactic made progress with:
\begin{quote}
{\tt progress} {\tacexpr}
\end{quote}
{\tacexpr} is evaluated to $v$. $v$ must be a tactic value. $v$ is
applied. If the application of $v$ produced one subgoal equal to the
initial goal (up to syntactical equality), then an error of level 0 is
raised. 

\ErrMsg \errindex{Failed to progress}

\subsubsection[Branching]{Branching\tacindex{$\mid\mid$}
\index{Tacticals!orelse@{\tt $\mid\mid$}}}

We can easily branch with the following structure:
\begin{quote}
{\tacexpr}$_1$ {\tt ||} {\tacexpr}$_2$
\end{quote}
{\tacexpr}$_1$ and {\tacexpr}$_2$ are evaluated to $v_1$ and
$v_2$. $v_1$ and $v_2$ must be tactic values. $v_1$ is applied and if
it fails to progress then $v_2$ is applied. Branching is left-associative.

\subsubsection[First tactic to work]{First tactic to work\tacindex{first}
\index{Tacticals!first@{\tt first}}}

We may consider the first tactic to work (i.e. which does not fail) among a
panel of tactics:
\begin{quote}
{\tt first [} {\tacexpr}$_1$ {\tt |} $...$ {\tt |} {\tacexpr}$_n$ {\tt ]}
\end{quote}
{\tacexpr}$_i$ are evaluated to $v_i$ and $v_i$ must be tactic values, for 
$i=1,...,n$. Supposing $n>1$, it applies $v_1$, if it works, it stops else it
tries to apply $v_2$ and so on. It fails when there is no applicable tactic.

\ErrMsg \errindex{No applicable tactic}

\subsubsection[Solving]{Solving\tacindex{solve}
\index{Tacticals!solve@{\tt solve}}}

We may consider the first to solve (i.e. which generates no subgoal) among a
panel of tactics:
\begin{quote}
{\tt solve [} {\tacexpr}$_1$ {\tt |} $...$ {\tt |} {\tacexpr}$_n$ {\tt ]}
\end{quote}
{\tacexpr}$_i$ are evaluated to $v_i$ and $v_i$ must be tactic values, for 
$i=1,...,n$. Supposing $n>1$, it applies $v_1$, if it solves, it stops else it
tries to apply $v_2$ and so on. It fails if there is no solving tactic.

\ErrMsg \errindex{Cannot solve the goal}

\subsubsection[Identity]{Identity\tacindex{idtac}
\index{Tacticals!idtac@{\tt idtac}}}

The constant {\tt idtac} is the identity tactic: it leaves any goal
unchanged but it appears in the proof script.

\variant {\tt idtac \nelist{\messagetoken}{}}

This prints the given tokens. Strings and integers are printed
literally. If a (term) variable is given, its contents are printed.


\subsubsection[Failing]{Failing\tacindex{fail}
\index{Tacticals!fail@{\tt fail}}}

The tactic {\tt fail} is the always-failing tactic: it does not solve
any goal. It is useful for defining other tacticals since it can be
catched by {\tt try} or {\tt match goal}. 

\begin{Variants}
\item {\tt fail $n$}\\
The number $n$ is the failure level. If no level is specified, it
defaults to $0$.  The level is used by {\tt try} and {\tt match goal}.
If $0$, it makes {\tt match goal} considering the next clause
(backtracking). If non zero, the current {\tt match goal} block or
{\tt try} command is aborted and the level is decremented.

\item {\tt fail \nelist{\messagetoken}{}}\\
The given tokens are used for printing the failure message.

\item {\tt fail $n$ \nelist{\messagetoken}{}}\\
This is a combination of the previous variants.
\end{Variants}

\ErrMsg \errindex{Tactic Failure {\it message} (level $n$)}.

\subsubsection[Local definitions]{Local definitions\index{Ltac!let}
\index{Ltac!let rec}
\index{let!in Ltac}
\index{let rec!in Ltac}}

Local definitions can be done as follows:
\begin{quote}
{\tt let} {\ident}$_1$ {\tt :=} {\tacexpr}$_1$\\
{\tt with} {\ident}$_2$ {\tt :=} {\tacexpr}$_2$\\
...\\
{\tt with} {\ident}$_n$ {\tt :=} {\tacexpr}$_n$ {\tt in}\\
{\tacexpr}
\end{quote}
each {\tacexpr}$_i$ is evaluated to $v_i$, then, {\tacexpr} is
evaluated by substituting $v_i$ to each occurrence of {\ident}$_i$,
for $i=1,...,n$. There is no dependencies between the {\tacexpr}$_i$
and the {\ident}$_i$.

Local definitions can be recursive by using {\tt let rec} instead of
{\tt let}. In this latter case, the definitions are evaluated lazily
so that the {\tt rec} keyword can be used also in non recursive cases
so as to avoid the eager evaluation of local definitions.

\subsubsection{Application}

An application is an expression of the following form:
\begin{quote}
{\qualid} {\tacarg}$_1$ ... {\tacarg}$_n$
\end{quote}
The reference {\qualid} must be bound to some defined tactic
definition expecting at least $n$ arguments.  The expressions
{\tacexpr}$_i$ are evaluated to $v_i$, for $i=1,...,n$.
%If {\tacexpr} is a {\tt Fun} or {\tt Rec} value then the body is evaluated by
%substituting $v_i$ to the formal parameters, for $i=1,...,n$. For recursive
%clauses, the bodies are lazily substituted (when an identifier to be evaluated
%is the name of a recursive clause).

%\subsection{Application of tactic values}

\subsubsection[Function construction]{Function construction\index{fun!in Ltac}
\index{Ltac!fun}}

A parameterized tactic can be built anonymously (without resorting to
local definitions) with:
\begin{quote}
{\tt fun} {\ident${}_1$} ... {\ident${}_n$} {\tt =>} {\tacexpr}
\end{quote}
Indeed, local definitions of functions are a syntactic sugar for
binding a {\tt fun} tactic to an identifier.

\subsubsection[Pattern matching on terms]{Pattern matching on terms\index{Ltac!match}
\index{match!in Ltac}}

We can carry out pattern matching on terms with:
\begin{quote}
{\tt match} {\tacexpr} {\tt with}\\
~~~{\cpattern}$_1$ {\tt =>} {\tacexpr}$_1$\\
~{\tt |} {\cpattern}$_2$ {\tt =>} {\tacexpr}$_2$\\
~...\\
~{\tt |} {\cpattern}$_n$ {\tt =>} {\tacexpr}$_n$\\
~{\tt |} {\tt \_} {\tt =>} {\tacexpr}$_{n+1}$\\
{\tt end}
\end{quote}
The expression {\tacexpr} is evaluated and should yield a term which
is matched against {\cpattern}$_1$. The matching is non-linear: if a
metavariable occurs more than once, it should match the same
expression every time. It is first-order except on the
variables of the form {\tt @?id} that occur in head position of an
application. For these variables, the matching is second-order and
returns a functional term.

If the matching with {\cpattern}$_1$ succeeds, then {\tacexpr}$_1$ is
evaluated into some value by substituting the pattern matching
instantiations to the metavariables. If {\tacexpr}$_1$ evaluates to a
tactic and the {\tt match} expression is in position to be applied to
a goal (e.g. it is not bound to a variable by a {\tt let in}), then
this tactic is applied. If the tactic succeeds, the list of resulting
subgoals is the result of the {\tt match} expression. If
{\tacexpr}$_1$ does not evaluate to a tactic or if the {\tt match}
expression is not in position to be applied to a goal, then the result
of the evaluation of {\tacexpr}$_1$ is the result of the {\tt match}
expression.

If the matching with {\cpattern}$_1$ fails, or if it succeeds but the
evaluation of {\tacexpr}$_1$ fails, or if the evaluation of
{\tacexpr}$_1$ succeeds but returns a tactic in execution position
whose execution fails, then {\cpattern}$_2$ is used and so on.  The
pattern {\_} matches any term and shunts all remaining patterns if
any. If all clauses fail (in particular, there is no pattern {\_})
then a no-matching-clause error is raised.

\begin{ErrMsgs}

\item \errindex{No matching clauses for match}

  No pattern can be used and, in particular, there is no {\tt \_} pattern.

\item \errindex{Argument of match does not evaluate to a term}

  This happens when {\tacexpr} does not denote a term.

\end{ErrMsgs}

\begin{Variants}

\item \index{lazymatch!in Ltac}
\index{Ltac!lazymatch}
Using {\tt lazymatch} instead of {\tt match} has an effect if the
right-hand-side of a clause returns a tactic. With {\tt match}, the
tactic is applied to the current goal (and the next clause is tried if
it fails). With {\tt lazymatch}, the tactic is directly returned as
the result of the whole {\tt lazymatch} block without being first
tried to be applied to the goal. Typically, if the {\tt lazymatch}
block is bound to some variable $x$ in a {\tt let in}, then tactic
expression gets bound to the variable $x$.

\item \index{context!in pattern}
There is a special form of patterns to match a subterm against the
pattern:
\begin{quote}
{\tt context} {\ident} {\tt [} {\cpattern} {\tt ]}
\end{quote}
It matches any term which one subterm matches {\cpattern}. If there is
a match, the optional {\ident} is assign the ``matched context'', that
is the initial term where the matched subterm is replaced by a
hole. The definition of {\tt context} in expressions below will show
how to use such term contexts.

If the evaluation of the right-hand-side of a valid match fails, the
next matching subterm is tried. If no further subterm matches, the
next clause is tried. Matching subterms are considered top-bottom and
from left to right (with respect to the raw printing obtained by
setting option {\tt Printing All}, see Section~\ref{SetPrintingAll}).

\begin{coq_example}
Ltac f x :=
  match x with
    context f [S ?X] => 
    idtac X;                    (* To display the evaluation order *)
    assert (p := refl_equal 1 : X=1);    (* To filter the case X=1 *)
    let x:= context f[O] in assert (x=O) (* To observe the context *)
  end.
Goal True.
f (3+4).
\end{coq_example}

\item \index{appcontext!in pattern}
For historical reasons, {\tt context} considers $n$-ary applications
such as {\tt (f 1 2)} as a whole, and not as a sequence of unary
applications {\tt ((f 1) 2)}. Hence {\tt context [f ?x]} will fail
to find a matching subterm in {\tt (f 1 2)}: if the pattern is a partial
application, the matched subterms will be necessarily be
applications with exactly the same number of arguments.
Alternatively, one may now use the following variant of {\tt context}:
\begin{quote}
{\tt appcontext} {\ident} {\tt [} {\cpattern} {\tt ]}
\end{quote}
The behavior of {\tt appcontext} is the same as the one of {\tt
  context}, except that a matching subterm could be a partial
part of a longer application. For instance, in {\tt (f 1 2)},
an {\tt appcontext [f ?x]} will find the matching subterm {\tt (f 1)}.

\end{Variants}

\subsubsection[Pattern matching on goals]{Pattern matching on goals\index{Ltac!match goal}
\index{Ltac!match reverse goal}
\index{match goal!in Ltac}
\index{match reverse goal!in Ltac}}

We can make pattern matching on goals using the following expression:
\begin{quote}
\begin{tabbing}
{\tt match goal with}\\
~~\={\tt |} $hyp_{1,1}${\tt ,}...{\tt ,}$hyp_{1,m_1}$
   ~~{\tt |-}{\cpattern}$_1${\tt =>} {\tacexpr}$_1$\\
  \>{\tt |} $hyp_{2,1}${\tt ,}...{\tt ,}$hyp_{2,m_2}$
   ~~{\tt |-}{\cpattern}$_2${\tt =>} {\tacexpr}$_2$\\
~~...\\
  \>{\tt |} $hyp_{n,1}${\tt ,}...{\tt ,}$hyp_{n,m_n}$
   ~~{\tt |-}{\cpattern}$_n${\tt =>} {\tacexpr}$_n$\\
  \>{\tt |\_}~~~~{\tt =>} {\tacexpr}$_{n+1}$\\
{\tt end}
\end{tabbing}
\end{quote}

If each hypothesis pattern $hyp_{1,i}$, with $i=1,...,m_1$
is matched (non-linear first-order unification) by an hypothesis of
the goal and if {\cpattern}$_1$ is matched by the conclusion of the
goal, then {\tacexpr}$_1$ is evaluated to $v_1$ by substituting the
pattern matching to the metavariables and the real hypothesis names
bound to the possible hypothesis names occurring in the hypothesis
patterns. If $v_1$ is a tactic value, then it is applied to the
goal. If this application fails, then another combination of
hypotheses is tried with the same proof context pattern. If there is
no other combination of hypotheses then the second proof context
pattern is tried and so on. If the next to last proof context pattern
fails then {\tacexpr}$_{n+1}$ is evaluated to $v_{n+1}$ and $v_{n+1}$
is applied. Note also that matching against subterms (using the {\tt
context} {\ident} {\tt [} {\cpattern} {\tt ]}) is available and may
itself induce extra backtrackings.

\ErrMsg \errindex{No matching clauses for match goal}

No clause succeeds, i.e. all matching patterns, if any,
fail at the application of the right-hand-side.

\medskip

It is important to know that each hypothesis of the goal can be
matched by at most one hypothesis pattern. The order of matching is
the following: hypothesis patterns are examined from the right to the
left (i.e. $hyp_{i,m_i}$ before $hyp_{i,1}$). For each hypothesis
pattern, the goal hypothesis are matched in order (fresher hypothesis
first), but it possible to reverse this order (older first) with
the {\tt match reverse goal with} variant.

\variant
\index{lazymatch goal!in Ltac}
\index{Ltac!lazymatch goal}
\index{lazymatch reverse goal!in Ltac}
\index{Ltac!lazymatch reverse goal}
Using {\tt lazymatch} instead of {\tt match} has an effect if the
right-hand-side of a clause returns a tactic. With {\tt match}, the
tactic is applied to the current goal (and the next clause is tried if
it fails). With {\tt lazymatch}, the tactic is directly returned as
the result of the whole {\tt lazymatch} block without being first
tried to be applied to the goal. Typically, if the {\tt lazymatch}
block is bound to some variable $x$ in a {\tt let in}, then tactic
expression gets bound to the variable $x$.

\begin{coq_example}
Ltac test_lazy :=
  lazymatch goal with
  | _ => idtac "here"; fail 
  | _ => idtac "wasn't lazy"; trivial
  end.
Ltac test_eager :=
  match goal with
  | _ => idtac "here"; fail 
  | _ => idtac "wasn't lazy"; trivial
  end.
Goal True.
test_lazy || idtac "was lazy".
test_eager || idtac "was lazy".
\end{coq_example}

\subsubsection[Filling a term context]{Filling a term context\index{context!in expression}}

The following expression is not a tactic in the sense that it does not
produce subgoals but generates a term to be used in tactic
expressions:
\begin{quote}
{\tt context} {\ident} {\tt [} {\tacexpr} {\tt ]}
\end{quote}
{\ident} must denote a context variable bound by a {\tt context}
pattern of a {\tt match} expression. This expression evaluates
replaces the hole of the value of {\ident} by the value of
{\tacexpr}.

\ErrMsg \errindex{not a context variable}


\subsubsection[Generating fresh hypothesis names]{Generating fresh hypothesis names\index{Ltac!fresh}
\index{fresh!in Ltac}}

Tactics sometimes have to generate new names for hypothesis. Letting
the system decide a name with the {\tt intro} tactic is not so good
since it is very awkward to retrieve the name the system gave.
The following expression returns an identifier:
\begin{quote}
{\tt fresh} \nelist{\textrm{\textsl{component}}}{}
\end{quote}
It evaluates to an identifier unbound in the goal. This fresh
identifier is obtained by concatenating the value of the
\textrm{\textsl{component}}'s (each of them is, either an {\ident} which
has to refer to a name, or directly a name denoted by a
{\qstring}). If the resulting name is already used, it is padded
with a number so that it becomes fresh. If no component is
given, the name is a fresh derivative of the name {\tt H}.

\subsubsection[Computing in a constr]{Computing in a constr\index{Ltac!eval}
\index{eval!in Ltac}}

Evaluation of a term can be performed with:
\begin{quote}
{\tt eval} {\nterm{redexpr}} {\tt in} {\term}
\end{quote}
where \nterm{redexpr} is a reduction tactic among {\tt red}, {\tt
hnf}, {\tt compute}, {\tt simpl}, {\tt cbv}, {\tt lazy}, {\tt unfold},
{\tt fold}, {\tt pattern}.

\subsubsection{Type-checking a term}
%\tacindex{type of}
\index{Ltac!type of}
\index{type of!in Ltac}

The following returns the type of {\term}:

\begin{quote}
{\tt type of} {\term}
\end{quote}

\subsubsection[Accessing tactic decomposition]{Accessing tactic decomposition\tacindex{info}
\index{Tacticals!info@{\tt info}}}

Tactical ``{\tt info} {\tacexpr}'' is not really a tactical. For
elementary tactics, this is equivalent to \tacexpr. For complex tactic
like \texttt{auto}, it displays the operations performed by the
tactic.

\subsubsection[Proving a subgoal as a separate lemma]{Proving a subgoal as a separate lemma\tacindex{abstract}
\index{Tacticals!abstract@{\tt abstract}}}

From the outside ``\texttt{abstract \tacexpr}'' is the same as
{\tt solve \tacexpr}. Internally it saves an auxiliary lemma called 
{\ident}\texttt{\_subproof}\textit{n} where {\ident} is the name of the
current goal and \textit{n} is chosen so that this is a fresh name.

This tactical is useful with tactics such as \texttt{omega} or
\texttt{discriminate} that generate huge proof terms. With that tool
the user can avoid the explosion at time of the \texttt{Save} command
without having to cut manually the proof in smaller lemmas.

\begin{Variants}
\item \texttt{abstract {\tacexpr} using {\ident}}.\\
  Give explicitly the name of the auxiliary lemma.
\end{Variants}

\ErrMsg \errindex{Proof is not complete}

\subsubsection[Calling an external tactic]{Calling an external tactic\index{Ltac!external}}

The tactic {\tt external} allows to run an executable outside the
{\Coq} executable. The communication is done via an XML encoding of
constructions. The syntax of the command is

\begin{quote}
{\tt external} "\textsl{command}" "\textsl{request}" \nelist{\tacarg}{}
\end{quote}

The string \textsl{command}, to be interpreted in the default
execution path of the operating system, is the name of the external
command. The string \textsl{request} is the name of a request to be
sent to the external command. Finally the list of tactic arguments
have to evaluate to terms. An XML tree of the following form is sent
to the standard input of the external command.
\medskip

\begin{tabular}{l}
\texttt{<REQUEST req="}\textsl{request}\texttt{">}\\
the XML tree of the first argument\\
{\ldots}\\
the XML tree of the last argument\\
\texttt{</REQUEST>}\\
\end{tabular}
\medskip

Conversely, the external command must send on its standard output an
XML tree of the following forms:

\medskip
\begin{tabular}{l}
\texttt{<TERM>}\\
the XML tree of a term\\
\texttt{</TERM>}\\
\end{tabular}
\medskip

\noindent or 

\medskip
\begin{tabular}{l}
\texttt{<CALL uri="}\textsl{ltac\_qualified\_ident}\texttt{">}\\
the XML tree of a first argument\\
{\ldots}\\
the XML tree of a last argument\\
\texttt{</CALL>}\\
\end{tabular}

\medskip
\noindent where \textsl{ltac\_qualified\_ident} is the name of a
defined {\ltac} function and each subsequent XML tree is recursively a
\texttt{CALL} or a \texttt{TERM} node.

The Document Type Definition (DTD) for terms of the Calculus of
Inductive Constructions is the one developed as part of the MoWGLI
European project. It can be found in the file {\tt dev/doc/cic.dtd} of
the {\Coq} source archive.

An example of parser for this DTD, written in the Objective Caml -
Camlp4 language, can be found in the file {\tt parsing/g\_xml.ml4} of
the {\Coq} source archive.

\section[Tactic toplevel definitions]{Tactic toplevel definitions\comindex{Ltac}}

\subsection{Defining {\ltac} functions}

Basically, {\ltac} toplevel definitions are made as follows:
%{\tt Tactic Definition} {\ident} {\tt :=} {\tacexpr}\\
%
%{\tacexpr} is evaluated to $v$ and $v$ is associated to {\ident}. Next, every
%script is evaluated by substituting $v$ to {\ident}.
%
%We can define functional definitions by:\\
\begin{quote}
{\tt Ltac} {\ident} {\ident}$_1$ ... {\ident}$_n$ {\tt :=}
{\tacexpr}
\end{quote}
This defines a new {\ltac} function that can be used in any tactic
script or new {\ltac} toplevel definition.

\Rem The preceding definition can equivalently be written:
\begin{quote}
{\tt Ltac} {\ident} {\tt := fun} {\ident}$_1$ ... {\ident}$_n$
{\tt =>} {\tacexpr}
\end{quote}
Recursive and mutual recursive function definitions are also
possible with the syntax:
\begin{quote}
{\tt Ltac} {\ident}$_1$ {\ident}$_{1,1}$ ...
{\ident}$_{1,m_1}$~~{\tt :=} {\tacexpr}$_1$\\
{\tt with} {\ident}$_2$ {\ident}$_{2,1}$ ... {\ident}$_{2,m_2}$~~{\tt :=}
{\tacexpr}$_2$\\
...\\
{\tt with} {\ident}$_n$ {\ident}$_{n,1}$ ... {\ident}$_{n,m_n}$~~{\tt :=}
{\tacexpr}$_n$
\end{quote}
\medskip
It is also possible to \emph{redefine} an existing user-defined tactic
using the syntax:
\begin{quote}
{\tt Ltac} {\qualid} {\ident}$_1$ ... {\ident}$_n$ {\tt ::=}
{\tacexpr}
\end{quote}
A previous definition of \qualid must exist in the environment.
The new definition will always be used instead of the old one and
it goes accross module boundaries.

If preceded by the keyword {\tt Local} the tactic definition will not
be exported outside the current module.

\subsection[Printing {\ltac} tactics]{Printing {\ltac} tactics\comindex{Print Ltac}}

Defined {\ltac} functions can be displayed using the command

\begin{quote}
{\tt Print Ltac {\qualid}.}
\end{quote}

\section[Debugging {\ltac} tactics]{Debugging {\ltac} tactics\comindex{Set Ltac Debug}
\comindex{Unset Ltac Debug}
\comindex{Test Ltac Debug}}

The {\ltac} interpreter comes with a step-by-step debugger. The
debugger can be activated using the command

\begin{quote}
{\tt Set Ltac Debug.}
\end{quote}

\noindent and deactivated using the command

\begin{quote}
{\tt Unset Ltac Debug.}
\end{quote}

To know if the debugger is on, use the command \texttt{Test Ltac Debug}.
When the debugger is activated, it stops at every step of the
evaluation of the current {\ltac} expression and it prints information
on what it is doing. The debugger stops, prompting for a command which
can be one of the following:

\medskip
\begin{tabular}{ll}
simple newline: & go to the next step\\
h: & get help\\
x: & exit current evaluation\\
s: & continue current evaluation without stopping\\
r$n$: & advance $n$ steps further\\
\end{tabular}
\endinput

\subsection{Permutation on closed lists}

\begin{figure}[b]
\begin{center}
\fbox{\begin{minipage}{0.95\textwidth}
\begin{coq_example*}
Require Import List.
Section Sort.
Variable A : Set.
Inductive permut : list A -> list A -> Prop :=
  | permut_refl   : forall l, permut l l
  | permut_cons   :
      forall a l0 l1, permut l0 l1 -> permut (a :: l0) (a :: l1)
  | permut_append : forall a l, permut (a :: l) (l ++ a :: nil)
  | permut_trans  :
      forall l0 l1 l2, permut l0 l1 -> permut l1 l2 -> permut l0 l2.
End Sort.
\end{coq_example*}
\end{center}
\caption{Definition of the permutation predicate}
\label{permutpred}
\end{figure}


Another more complex example is the problem of permutation on closed
lists. The aim is to show that a closed list is a permutation of
another one.  First, we define the permutation predicate as shown on
Figure~\ref{permutpred}.
 
\begin{figure}[p]
\begin{center}
\fbox{\begin{minipage}{0.95\textwidth}
\begin{coq_example}
Ltac Permut n :=
  match goal with
  | |- (permut _ ?l ?l) => apply permut_refl
  | |- (permut _ (?a :: ?l1) (?a :: ?l2)) =>
      let newn := eval compute in (length l1) in
      (apply permut_cons; Permut newn)
  | |- (permut ?A (?a :: ?l1) ?l2) =>
      match eval compute in n with
      | 1 => fail
      | _ =>
          let l1' := constr:(l1 ++ a :: nil) in
          (apply (permut_trans A (a :: l1) l1' l2);
            [ apply permut_append | compute; Permut (pred n) ])
      end
  end.
Ltac PermutProve :=
  match goal with
  | |- (permut _ ?l1 ?l2) =>
      match eval compute in (length l1 = length l2) with
      | (?n = ?n) => Permut n
      end
  end.
\end{coq_example}
\end{minipage}}
\end{center}
\caption{Permutation tactic}
\label{permutltac}
\end{figure}

\begin{figure}[p]
\begin{center}
\fbox{\begin{minipage}{0.95\textwidth}
\begin{coq_example*}
Lemma permut_ex1 :
  permut nat (1 :: 2 :: 3 :: nil) (3 :: 2 :: 1 :: nil).
Proof.
PermutProve.
Qed.

Lemma permut_ex2 :
  permut nat
    (0 :: 1 :: 2 :: 3 :: 4 :: 5 :: 6 :: 7 :: 8 :: 9 :: nil)
    (0 :: 2 :: 4 :: 6 :: 8 :: 9 :: 7 :: 5 :: 3 :: 1 :: nil).
Proof.
PermutProve.
Qed.
\end{coq_example*}
\end{minipage}}
\end{center}
\caption{Examples of {\tt PermutProve} use}
\label{permutlem}
\end{figure}

Next, we can write naturally the tactic and the result can be seen on
Figure~\ref{permutltac}. We can notice that we use two toplevel
definitions {\tt PermutProve} and {\tt Permut}. The function to be
called is {\tt PermutProve} which computes the lengths of the two
lists and calls {\tt Permut} with the length if the two lists have the
same length. {\tt Permut} works as expected.  If the two lists are
equal, it concludes. Otherwise, if the lists have identical first
elements, it applies {\tt Permut} on the tail of the lists.  Finally,
if the lists have different first elements, it puts the first element
of one of the lists (here the second one which appears in the {\tt
  permut} predicate) at the end if that is possible, i.e., if the new
first element has been at this place previously. To verify that all
rotations have been done for a list, we use the length of the list as
an argument for {\tt Permut} and this length is decremented for each
rotation down to, but not including, 1 because for a list of length
$n$, we can make exactly $n-1$ rotations to generate at most $n$
distinct lists. Here, it must be noticed that we use the natural
numbers of {\Coq} for the rotation counter. On Figure~\ref{ltac}, we
can see that it is possible to use usual natural numbers but they are
only used as arguments for primitive tactics and they cannot be
handled, in particular, we cannot make computations with them. So, a
natural choice is to use {\Coq} data structures so that {\Coq} makes
the computations (reductions) by {\tt eval compute in} and we can get
the terms back by {\tt match}.

With {\tt PermutProve}, we can now prove lemmas such those shown on
Figure~\ref{permutlem}.


\subsection{Deciding intuitionistic propositional logic}

\begin{figure}[tbp]
\begin{center}
\fbox{\begin{minipage}{0.95\textwidth}
\begin{coq_example}
Ltac Axioms :=
  match goal with
  | |- True => trivial
  | _:False |- _  => elimtype False; assumption
  | _:?A |- ?A  => auto
  end.
Ltac DSimplif :=
  repeat
   (intros;
    match goal with
     | id:(~ _) |- _ => red in id
     | id:(_ /\ _) |- _ =>
         elim id; do 2 intro; clear id
     | id:(_ \/ _) |- _ =>
         elim id; intro; clear id
     | id:(?A /\ ?B -> ?C) |- _ =>
         cut (A -> B -> C);
          [ intro | intros; apply id; split; assumption ]
     | id:(?A \/ ?B -> ?C) |- _ =>
         cut (B -> C);
          [ cut (A -> C);
             [ intros; clear id
             | intro; apply id; left; assumption ]
          | intro; apply id; right; assumption ]
     | id0:(?A -> ?B),id1:?A |- _ =>
         cut B; [ intro; clear id0 | apply id0; assumption ]
     | |- (_ /\ _) => split
     | |- (~ _) => red
     end).
\end{coq_example}
\end{minipage}}
\end{center}
\caption{Deciding intuitionistic propositions (1)}
\label{tautoltaca}
\end{figure}

\begin{figure}
\begin{center}
\fbox{\begin{minipage}{0.95\textwidth}
\begin{coq_example}
Ltac TautoProp :=
  DSimplif;
   Axioms ||
     match goal with
     | id:((?A -> ?B) -> ?C) |- _ =>
          cut (B -> C);
          [ intro; cut (A -> B);
             [ intro; cut C;
                [ intro; clear id | apply id; assumption ]
             | clear id ]
          | intro; apply id; intro; assumption ]; TautoProp
     | id:(~ ?A -> ?B) |- _ =>
         cut (False -> B);
          [ intro; cut (A -> False);
             [ intro; cut B;
                [ intro; clear id | apply id; assumption ]
             | clear id ]
          | intro; apply id; red; intro; assumption ]; TautoProp
     | |- (_ \/ _) => (left; TautoProp) || (right; TautoProp)
     end.
\end{coq_example}
\end{minipage}}
\end{center}
\caption{Deciding intuitionistic propositions (2)}
\label{tautoltacb}
\end{figure}

The pattern matching on goals allows a complete and so a powerful
backtracking when returning tactic values. An interesting application
is the problem of deciding intuitionistic propositional logic.
Considering the contraction-free sequent calculi {\tt LJT*} of
Roy~Dyckhoff (\cite{Dyc92}), it is quite natural to code such a tactic
using the tactic language. On Figure~\ref{tautoltaca}, the tactic {\tt
  Axioms} tries to conclude using usual axioms. The {\tt DSimplif}
tactic applies all the reversible rules of Dyckhoff's system.
Finally, on Figure~\ref{tautoltacb}, the {\tt TautoProp} tactic (the
main tactic to be called) simplifies with {\tt DSimplif}, tries to
conclude with {\tt Axioms} and tries several paths using the
backtracking rules (one of the four Dyckhoff's rules for the left
implication to get rid of the contraction and the right or).

\begin{figure}[tb]
\begin{center}
\fbox{\begin{minipage}{0.95\textwidth}
\begin{coq_example*}
Lemma tauto_ex1 : forall A B:Prop, A /\ B -> A \/ B.
Proof.
TautoProp.
Qed.

Lemma tauto_ex2 :
   forall A B:Prop, (~ ~ B -> B) -> (A -> B) -> ~ ~ A -> B.
Proof.
TautoProp.
Qed.
\end{coq_example*}
\end{minipage}}
\end{center}
\caption{Proofs of tautologies with {\tt TautoProp}}
\label{tautolem}
\end{figure}

For example, with {\tt TautoProp}, we can prove tautologies like those of
Figure~\ref{tautolem}.


\subsection{Deciding type isomorphisms}

A more tricky problem is to decide equalities between types and modulo
isomorphisms. Here, we choose to use the isomorphisms of the simply typed
$\lb{}$-calculus with Cartesian product and $unit$ type (see, for example,
\cite{RC95}). The axioms of this $\lb{}$-calculus are given by
Figure~\ref{isosax}.

\begin{figure}
\begin{center}
\fbox{\begin{minipage}{0.95\textwidth}
\begin{coq_eval}
Reset Initial.
\end{coq_eval}
\begin{coq_example*}
Open Scope type_scope.
Section Iso_axioms.
Variables A B C : Set.
Axiom Com : A * B = B * A.
Axiom Ass : A * (B * C) = A * B * C.
Axiom Cur : (A * B -> C) = (A -> B -> C).
Axiom Dis : (A -> B * C) = (A -> B) * (A -> C).
Axiom P_unit : A * unit = A.
Axiom AR_unit : (A -> unit) = unit.
Axiom AL_unit : (unit -> A) = A.
Lemma Cons : B = C -> A * B = A * C.
Proof.
intro Heq; rewrite Heq; apply refl_equal.
Qed.
End Iso_axioms.
\end{coq_example*}
\end{minipage}}
\end{center}
\caption{Type isomorphism axioms}
\label{isosax}
\end{figure}

The tactic to judge equalities modulo this axiomatization can be written as
shown on Figures~\ref{isosltac1} and~\ref{isosltac2}. The algorithm is quite
simple. Types are reduced using axioms that can be oriented (this done by {\tt
MainSimplif}). The normal forms are sequences of Cartesian
products without Cartesian product in the left component. These normal forms
are then compared modulo permutation of the components (this is done by {\tt
CompareStruct}). The main tactic to be called and realizing this algorithm is
{\tt IsoProve}.

\begin{figure}
\begin{center}
\fbox{\begin{minipage}{0.95\textwidth}
\begin{coq_example}
Ltac DSimplif trm :=
  match trm with
  | (?A * ?B * ?C) =>
      rewrite <- (Ass A B C); try MainSimplif
  | (?A * ?B -> ?C) =>
      rewrite (Cur A B C); try MainSimplif
  | (?A -> ?B * ?C) =>
      rewrite (Dis A B C); try MainSimplif
  | (?A * unit) =>
      rewrite (P_unit A); try MainSimplif
  | (unit * ?B) =>
      rewrite (Com unit B); try MainSimplif
  | (?A -> unit) =>
      rewrite (AR_unit A); try MainSimplif
  | (unit -> ?B) =>
      rewrite (AL_unit B); try MainSimplif
  | (?A * ?B) =>
      (DSimplif A; try MainSimplif) || (DSimplif B; try MainSimplif)
  | (?A -> ?B) =>
      (DSimplif A; try MainSimplif) || (DSimplif B; try MainSimplif)
  end
 with MainSimplif :=
  match goal with
  | |- (?A = ?B) => try DSimplif A; try DSimplif B
  end.
Ltac Length trm :=
  match trm with
  | (_ * ?B) => let succ := Length B in constr:(S succ)
  | _ => constr:1
  end.
Ltac assoc := repeat rewrite <- Ass.
\end{coq_example}
\end{minipage}}
\end{center}
\caption{Type isomorphism tactic (1)}
\label{isosltac1}
\end{figure}

\begin{figure}
\begin{center}
\fbox{\begin{minipage}{0.95\textwidth}
\begin{coq_example}
Ltac DoCompare n :=
  match goal with
  | [ |- (?A = ?A) ] => apply refl_equal
  | [ |- (?A * ?B = ?A * ?C) ] =>
    apply Cons; let newn := Length B in DoCompare newn
  | [ |- (?A * ?B = ?C) ] =>
    match eval compute in n with
    | 1 => fail
    | _ =>
      pattern (A * B) at 1; rewrite Com; assoc; DoCompare (pred n)
    end
  end.
Ltac CompareStruct :=
  match goal with
  | [ |- (?A = ?B) ] =>
      let l1 := Length A
      with l2 := Length B in
      match eval compute in (l1 = l2) with
      | (?n = ?n) => DoCompare n
      end
  end.
Ltac IsoProve := MainSimplif; CompareStruct.
\end{coq_example}
\end{minipage}}
\end{center}
\caption{Type isomorphism tactic (2)}
\label{isosltac2}
\end{figure}

Figure~\ref{isoslem} gives examples of what can be solved by {\tt IsoProve}.

\begin{figure}
\begin{center}
\fbox{\begin{minipage}{0.95\textwidth}
\begin{coq_example*}
Lemma isos_ex1 : 
  forall A B:Set, A * unit * B = B * (unit * A).
Proof.
intros; IsoProve.
Qed.

Lemma isos_ex2 :
  forall A B C:Set,
    (A * unit -> B * (C * unit)) =
    (A * unit -> (C -> unit) * C) * (unit -> A -> B).
Proof.
intros; IsoProve.
Qed.
\end{coq_example*}
\end{minipage}}
\end{center}
\caption{Type equalities solved by {\tt IsoProve}}
\label{isoslem}
\end{figure}

%%% Local Variables: 
%%% mode: latex
%%% TeX-master: "Reference-Manual"
%%% End: 
% Writing tactics
\chapter[Detailed examples of tactics]{Detailed examples of tactics\label{Tactics-examples}}

This chapter presents detailed examples of certain tactics, to
illustrate their behavior.

\section[\tt dependent induction]{\tt dependent induction\label{dependent-induction-example}}
\def\depind{{\tt dependent induction}~}
\def\depdestr{{\tt dependent destruction}~}

The tactics \depind and \depdestr are another solution for inverting
inductive predicate instances and potentially doing induction at the
same time. It is based on the \texttt{BasicElim} tactic of Conor McBride which
works by abstracting each argument of an inductive instance by a variable
and constraining it by equalities afterwards. This way, the usual 
{\tt induction} and {\tt destruct} tactics can be applied to the
abstracted instance and after simplification of the equalities we get
the expected goals.

The abstracting tactic is called {\tt generalize\_eqs} and it takes as
argument an hypothesis to generalize. It uses the {\tt JMeq} datatype
defined in {\tt Coq.Logic.JMeq}, hence we need to require it before.
For example, revisiting the first example of the inversion documentation above:

\begin{coq_example*}
Require Import Coq.Logic.JMeq.
\end{coq_example*}
\begin{coq_eval}
Require Import Coq.Program.Equality.
\end{coq_eval}

\begin{coq_eval}
Inductive Le : nat -> nat -> Set :=
  | LeO : forall n:nat, Le 0 n
  | LeS : forall n m:nat, Le n m -> Le (S n) (S m).
Variable P : nat -> nat -> Prop.
Variable Q : forall n m:nat, Le n m -> Prop.
\end{coq_eval}

\begin{coq_example*}
Goal forall n m:nat, Le (S n) m -> P n m.
intros n m H.
\end{coq_example*}
\begin{coq_example}
generalize_eqs H.
\end{coq_example}

The index {\tt S n} gets abstracted by a variable here, but a
corresponding equality is added under the abstract instance so that no
information is actually lost. The goal is now almost amenable to do induction
or case analysis. One should indeed first move {\tt n} into the goal to
strengthen it before doing induction, or {\tt n} will be fixed in
the inductive hypotheses (this does not matter for case analysis). 
As a rule of thumb, all the variables that appear inside constructors in
the indices of the hypothesis should be generalized. This is exactly
what the \texttt{generalize\_eqs\_vars} variant does:

\begin{coq_eval} 
Undo 1.
\end{coq_eval}
\begin{coq_example}
generalize_eqs_vars H.
induction H.
\end{coq_example}

As the hypothesis itself did not appear in the goal, we did not need to
use an heterogeneous equality to relate the new hypothesis to the old
one (which just disappeared here). However, the tactic works just as well
in this case, e.g.:

\begin{coq_eval}
Admitted.
\end{coq_eval}

\begin{coq_example}
Goal forall n m (p : Le (S n) m), Q (S n) m p.
intros n m p ; generalize_eqs_vars p.
\end{coq_example}

One drawback of this approach is that in the branches one will have to
substitute the equalities back into the instance to get the right
assumptions. Sometimes injection of constructors will also be needed to
recover the needed equalities. Also, some subgoals should be directly
solved because of inconsistent contexts arising from the constraints on 
indexes. The nice thing is that we can make a tactic based on
discriminate, injection and variants of substitution to automatically 
do such simplifications (which may involve the K axiom). 
This is what the {\tt simplify\_dep\_elim} tactic from
{\tt Coq.Program.Equality} does. For example, we might simplify the
previous goals considerably:
% \begin{coq_eval} 
% Abort.
% Goal forall n m:nat, Le (S n) m -> P n m.
% intros n m H ; generalize_eqs_vars H.
% \end{coq_eval}

\begin{coq_example}
induction p ; simplify_dep_elim.
\end{coq_example}

The higher-order tactic {\tt do\_depind} defined in {\tt
  Coq.Program.Equality} takes a tactic and combines the
building blocks we have seen with it: generalizing by equalities
calling the given tactic with the
generalized induction hypothesis as argument and cleaning the subgoals
with respect to equalities. Its most important instantiations are
\depind and \depdestr that do induction or simply case analysis on the
generalized hypothesis. For example we can redo what we've done manually
with \depdestr:

\begin{coq_eval}
Abort.
\end{coq_eval}
\begin{coq_example*}
Require Import Coq.Program.Equality.
Lemma ex : forall n m:nat, Le (S n) m -> P n m.
intros n m H.
\end{coq_example*}
\begin{coq_example}
dependent destruction H.
\end{coq_example}
\begin{coq_eval}
Abort.
\end{coq_eval}

This gives essentially the same result as inversion. Now if the
destructed hypothesis actually appeared in the goal, the tactic would
still be able to invert it, contrary to {\tt dependent
 inversion}. Consider the following example on vectors:

\begin{coq_example*}
Require Import Coq.Program.Equality.
Set Implicit Arguments.
Variable A : Set.
Inductive vector : nat -> Type := 
| vnil : vector 0 
| vcons : A -> forall n, vector n -> vector (S n).
Goal forall n, forall v : vector (S n), 
  exists v' : vector n, exists a : A, v = vcons a v'.
  intros n v.
\end{coq_example*}
\begin{coq_example}
  dependent destruction v.
\end{coq_example}
\begin{coq_eval}
Abort.
\end{coq_eval}

In this case, the {\tt v} variable can be replaced in the goal by the
generalized hypothesis only when it has a type of the form {\tt vector
 (S n)}, that is only in the second case of the {\tt destruct}. The
first one is dismissed because {\tt S n <> 0}.

\subsection{A larger example}

Let's see how the technique works with {\tt induction} on inductive
predicates on a real example. We will develop an example application to the
theory of simply-typed lambda-calculus formalized in a dependently-typed style:

\begin{coq_example*}
Inductive type : Type :=
| base : type
| arrow : type -> type -> type.
Notation " t --> t' " := (arrow t t') (at level 20, t' at next level).
Inductive ctx : Type :=
| empty : ctx
| snoc : ctx -> type -> ctx.
Notation " G , tau " := (snoc G tau) (at level 20, t at next level).
Fixpoint conc (G D : ctx) : ctx :=
  match D with
    | empty => G
    | snoc D' x => snoc (conc G D') x
  end.
Notation " G ; D " := (conc G D) (at level 20).
Inductive term : ctx -> type -> Type :=
| ax : forall G tau, term (G, tau) tau
| weak : forall G tau, 
  term G tau -> forall tau', term (G, tau') tau
| abs : forall G tau tau', 
  term (G , tau) tau' -> term G (tau --> tau')
| app : forall G tau tau', 
  term G (tau --> tau') -> term G tau -> term G tau'.
\end{coq_example*}

We have defined types and contexts which are snoc-lists of types. We
also have a {\tt conc} operation that concatenates two contexts.
The {\tt term} datatype represents in fact the possible typing
derivations of the calculus, which are isomorphic to the well-typed
terms, hence the name. A term is either an application of:
\begin{itemize}
\item the axiom rule to type a reference to the first variable in a context,
\item the weakening rule to type an object in a larger context
\item the abstraction or lambda rule to type a function
\item the application to type an application of a function to an argument
\end{itemize}

Once we have this datatype we want to do proofs on it, like weakening:

\begin{coq_example*}
Lemma weakening : forall G D tau, term (G ; D) tau -> 
  forall tau', term (G , tau' ; D) tau.
\end{coq_example*}
\begin{coq_eval}
  Abort.
\end{coq_eval}

The problem here is that we can't just use {\tt induction} on the typing
derivation because it will forget about the {\tt G ; D} constraint
appearing in the instance. A solution would be to rewrite the goal as:
\begin{coq_example*}
Lemma weakening' : forall G' tau, term G' tau -> 
  forall G D, (G ; D) = G' ->
  forall tau', term (G, tau' ; D) tau.
\end{coq_example*}
\begin{coq_eval}
  Abort.
\end{coq_eval}

With this proper separation of the index from the instance and the right
induction loading (putting {\tt G} and {\tt D} after the inducted-on
hypothesis), the proof will go through, but it is a very tedious
process. One is also forced to make a wrapper lemma to get back the
more natural statement. The \depind tactic alleviates this trouble by
doing all of this plumbing of generalizing and substituting back automatically.
Indeed we can simply write:

\begin{coq_example*}
Require Import Coq.Program.Tactics.
Lemma weakening : forall G D tau, term (G ; D) tau -> 
  forall tau', term (G , tau' ; D) tau.
Proof with simpl in * ; simpl_depind ; auto.
  intros G D tau H. dependent induction H generalizing G D ; intros.
\end{coq_example*}

This call to \depind has an additional arguments which is a list of
variables appearing in the instance that should be generalized in the
goal, so that they can vary in the induction hypotheses. By default, all
variables appearing inside constructors (except in a parameter position)
of the instantiated hypothesis will be generalized automatically but
one can always give the list explicitly.

\begin{coq_example}
  Show.
\end{coq_example}

The {\tt simpl\_depind} tactic includes an automatic tactic that tries
to simplify equalities appearing at the beginning of induction
hypotheses, generally using trivial applications of
reflexivity. In cases where the equality is not between constructor
forms though, one must help the automation by giving
some arguments, using the {\tt specialize} tactic.

\begin{coq_example*}
destruct D... apply weak ; apply ax. apply ax.
destruct D...
\end{coq_example*}
\begin{coq_example}
Show.
\end{coq_example}
\begin{coq_example}
  specialize (IHterm G empty).
\end{coq_example}

Then the automation can find the needed equality {\tt G = G} to narrow
the induction hypothesis further. This concludes our example.

\begin{coq_example}
  simpl_depind.
\end{coq_example}

\SeeAlso The induction \ref{elim}, case \ref{case} and inversion \ref{inversion} tactics.

\section[\tt autorewrite]{\tt autorewrite\label{autorewrite-example}}

Here are two examples of {\tt autorewrite} use. The first one ({\em Ackermann
function}) shows actually a quite basic use where there is no conditional
rewriting. The second one ({\em Mac Carthy function}) involves conditional
rewritings and shows how to deal with them using the optional tactic of the
{\tt Hint~Rewrite} command.

\firstexample
\example{Ackermann function}
%Here is a basic use of {\tt AutoRewrite} with the Ackermann function:

\begin{coq_example*}
Reset Initial.
Require Import Arith.
Variable Ack : 
           nat -> nat -> nat.
Axiom Ack0 : 
        forall m:nat, Ack 0 m = S m.
Axiom Ack1 : forall n:nat, Ack (S n) 0 = Ack n 1.
Axiom Ack2 : forall n m:nat, Ack (S n) (S m) = Ack n (Ack (S n) m).
\end{coq_example*}

\begin{coq_example}
Hint Rewrite Ack0 Ack1 Ack2 : base0.
Lemma ResAck0 : 
 Ack 3 2 = 29.
autorewrite with base0 using try reflexivity.
\end{coq_example}

\begin{coq_eval}
Reset Initial.
\end{coq_eval}

\example{Mac Carthy function}
%The Mac Carthy function shows a more complex case:

\begin{coq_example*}
Require Import Omega.
Variable g :   
           nat -> nat -> nat.
Axiom g0 : 
        forall m:nat, g 0 m = m.
Axiom
  g1 :
    forall n m:nat,
      (n > 0) -> (m > 100) -> g n m = g (pred n) (m - 10).
Axiom
  g2 :
    forall n m:nat,
      (n > 0) -> (m <= 100) -> g n m = g (S n) (m + 11).
\end{coq_example*}

\begin{coq_example}
Hint Rewrite g0 g1 g2 using omega : base1.
Lemma Resg0 : 
 g 1 110 = 100.
autorewrite with base1 using reflexivity || simpl.
\end{coq_example}

\begin{coq_eval}
Abort.
\end{coq_eval}

\begin{coq_example}
Lemma Resg1 : g 1 95 = 91.
autorewrite with base1 using reflexivity || simpl.
\end{coq_example}

\begin{coq_eval}
Reset Initial.
\end{coq_eval}

\section[\tt quote]{\tt quote\tacindex{quote}
\label{quote-examples}}

The tactic \texttt{quote} allows to use Barendregt's so-called
2-level approach without writing any ML code. Suppose you have a
language \texttt{L} of 
'abstract terms' and a type \texttt{A} of 'concrete terms' 
and a function \texttt{f : L -> A}. If \texttt{L} is a simple
inductive datatype and \texttt{f} a simple fixpoint, \texttt{quote f}
will replace the head of current goal by a convertible term of the form 
\texttt{(f t)}. \texttt{L} must have a constructor of type: \texttt{A
  -> L}. 

Here is an example:

\begin{coq_example}
Require Import Quote.
Parameters A B C : Prop.
Inductive formula : Type :=
  | f_and : formula -> formula -> formula (* binary constructor *)
  | f_or : formula -> formula -> formula
  | f_not : formula -> formula (* unary constructor *)
  | f_true : formula (* 0-ary constructor *)
  | f_const : Prop -> formula (* constructor for constants *).
Fixpoint interp_f (f:
                   formula) : Prop :=
  match f with
  | f_and f1 f2 => interp_f f1 /\ interp_f f2
  | f_or f1 f2 => interp_f f1 \/ interp_f f2
  | f_not f1 => ~ interp_f f1
  | f_true => True
  | f_const c => c
  end.
Goal A /\ (A \/ True) /\ ~ B /\ (A <-> A).
quote interp_f.
\end{coq_example}

The algorithm to perform this inversion is: try to match the
term with right-hand sides expression of \texttt{f}. If there is a
match, apply the corresponding left-hand side and call yourself
recursively on sub-terms. If there is no match, we are at a leaf:
return the corresponding constructor (here \texttt{f\_const}) applied
to the term. 

\begin{ErrMsgs}
\item \errindex{quote: not a simple fixpoint} \\
  Happens when \texttt{quote} is not able to perform inversion properly.
\end{ErrMsgs}

\subsection{Introducing variables map}

The normal use of \texttt{quote} is to make proofs by reflection: one
defines a function \texttt{simplify : formula -> formula} and proves a 
theorem \texttt{simplify\_ok: (f:formula)(interp\_f (simplify f)) ->
  (interp\_f f)}. Then, one can simplify formulas by doing:
\begin{verbatim}
   quote interp_f.
   apply simplify_ok.
   compute.
\end{verbatim}
But there is a problem with leafs: in the example above one cannot
write a function that implements, for example, the logical simplifications 
$A \land A \ra A$ or $A \land \lnot A \ra \texttt{False}$. This is
because the \Prop{} is impredicative.

It is better to use that type of formulas:

\begin{coq_eval}
Reset formula.
\end{coq_eval}
\begin{coq_example}
Inductive formula : Set :=
  | f_and : formula -> formula -> formula
  | f_or : formula -> formula -> formula
  | f_not : formula -> formula
  | f_true : formula
  | f_atom : index -> formula.
\end{coq_example*}

\texttt{index} is defined in module \texttt{quote}. Equality on that
type is decidable so we are able to simplify $A \land A$ into $A$ at
the abstract level. 

When there are variables, there are bindings, and \texttt{quote}
provides also a type \texttt{(varmap A)} of bindings from
\texttt{index} to any set \texttt{A}, and a function
\texttt{varmap\_find} to search in such maps. The interpretation
function has now another argument, a variables map:

\begin{coq_example}
Fixpoint interp_f (vm:
                    varmap Prop) (f:formula) {struct f} : Prop :=
  match f with
  | f_and f1 f2 => interp_f vm f1 /\ interp_f vm f2
  | f_or f1 f2 => interp_f vm f1 \/ interp_f vm f2
  | f_not f1 => ~ interp_f vm f1
  | f_true => True
  | f_atom i => varmap_find True i vm
  end.
\end{coq_example}

\noindent\texttt{quote} handles this second case properly:

\begin{coq_example}
Goal A /\ (B \/ A) /\ (A \/ ~ B).
quote interp_f.
\end{coq_example}

It builds \texttt{vm} and \texttt{t} such that \texttt{(f vm t)} is
convertible with the conclusion of current goal.

\subsection{Combining variables and constants}

One can have both variables and constants in abstracts terms; that is
the case, for example, for the \texttt{ring} tactic (chapter
\ref{ring}). Then one must provide to \texttt{quote} a list of
\emph{constructors of constants}. For example, if the list is
\texttt{[O S]} then closed natural numbers will be considered as
constants and other terms as variables. 

Example: 

\begin{coq_eval}
Reset formula.
\end{coq_eval}
\begin{coq_example*}
Inductive formula : Type :=
  | f_and : formula -> formula -> formula
  | f_or : formula -> formula -> formula
  | f_not : formula -> formula
  | f_true : formula
  | f_const : Prop -> formula (* constructor for constants *)
  | f_atom : index -> formula.
Fixpoint interp_f
 (vm:            (* constructor for variables *)
  varmap Prop) (f:formula) {struct f} : Prop :=
  match f with
  | f_and f1 f2 => interp_f vm f1 /\ interp_f vm f2
  | f_or f1 f2 => interp_f vm f1 \/ interp_f vm f2
  | f_not f1 => ~ interp_f vm f1
  | f_true => True
  | f_const c => c
  | f_atom i => varmap_find True i vm
  end.
Goal 
A /\ (A \/ True) /\ ~ B /\ (C <-> C).
\end{coq_example*}

\begin{coq_example}
quote interp_f [ A B ].
Undo.
  quote interp_f [ B C iff ].
\end{coq_example}

\Warning Since function inversion
is undecidable in general case, don't expect miracles from it!

\begin{Variants}

\item {\tt quote {\ident} in {\term} using {\tac}}

  \tac\ must be a functional tactic (starting with {\tt fun x =>})
  and will be called with the quoted version of \term\ according to
  \ident.

\item {\tt quote {\ident} [ \ident$_1$ \dots\ \ident$_n$ ] in {\term} using {\tac}}

  Same as above, but will use \ident$_1$, \dots, \ident$_n$ to
  chose which subterms are constants (see above).

\end{Variants}

% \SeeAlso file \texttt{theories/DEMOS/DemoQuote.v}

\SeeAlso comments of source file \texttt{plugins/quote/quote.ml}

\SeeAlso the \texttt{ring} tactic (Chapter~\ref{ring})



\section{Using the tactical language}

\subsection{About the cardinality of the set of natural numbers}

A first example which shows how to use the pattern matching over the proof
contexts is the proof that natural numbers have more than two elements. The
proof of such a lemma can be done as %shown on Figure~\ref{cnatltac}.
follows:
%\begin{figure}
%\begin{centerframe}
\begin{coq_eval}
Reset Initial.
Require Import Arith.
Require Import List.
\end{coq_eval}
\begin{coq_example*}
Lemma card_nat :
 ~ (exists x : nat, exists y : nat, forall z:nat, x = z \/ y = z).
Proof.
red; intros (x, (y, Hy)).
elim (Hy 0); elim (Hy 1); elim (Hy 2); intros;
 match goal with
 | [_:(?a = ?b),_:(?a = ?c) |- _ ] =>
     cut (b = c); [ discriminate | transitivity a; auto ]
 end.
Qed.
\end{coq_example*}
%\end{centerframe}
%\caption{A proof on cardinality of natural numbers}
%\label{cnatltac}
%\end{figure}

We can notice that all the (very similar) cases coming from the three
eliminations (with three distinct natural numbers) are successfully solved by
a {\tt match goal} structure and, in particular, with only one pattern (use
of non-linear matching).

\subsection{Permutation on closed lists}

Another more complex example is the problem of permutation on closed lists. The
aim is to show that a closed list is a permutation of another one.

First, we define the permutation predicate as shown in table~\ref{permutpred}.

\begin{figure}
\begin{centerframe}
\begin{coq_example*}
Section Sort.
Variable A : Set.
Inductive permut : list A -> list A -> Prop :=
  | permut_refl   : forall l, permut l l
  | permut_cons   :
      forall a l0 l1, permut l0 l1 -> permut (a :: l0) (a :: l1)
  | permut_append : forall a l, permut (a :: l) (l ++ a :: nil)
  | permut_trans  :
      forall l0 l1 l2, permut l0 l1 -> permut l1 l2 -> permut l0 l2.
End Sort.
\end{coq_example*}
\end{centerframe}
\caption{Definition of the permutation predicate}
\label{permutpred}
\end{figure}

A more complex example is the problem of permutation on closed lists.
The aim is to show that a closed list is a permutation of another one.
First, we define the permutation predicate as shown on
Figure~\ref{permutpred}.

\begin{figure}
\begin{centerframe}
\begin{coq_example}
Ltac Permut n :=
  match goal with
  | |- (permut _ ?l ?l) => apply permut_refl
  | |- (permut _ (?a :: ?l1) (?a :: ?l2)) =>
      let newn := eval compute in (length l1) in
      (apply permut_cons; Permut newn)
  | |- (permut ?A (?a :: ?l1) ?l2) =>
      match eval compute in n with
      | 1 => fail
      | _ =>
          let l1' := constr:(l1 ++ a :: nil) in
          (apply (permut_trans A (a :: l1) l1' l2);
            [ apply permut_append | compute; Permut (pred n) ])
      end
  end.
Ltac PermutProve :=
  match goal with
  | |- (permut _ ?l1 ?l2) =>
      match eval compute in (length l1 = length l2) with
      | (?n = ?n) => Permut n
      end
  end.
\end{coq_example}
\end{centerframe}
\caption{Permutation tactic}
\label{permutltac}
\end{figure}

Next, we can write naturally the tactic and the result can be seen on
Figure~\ref{permutltac}. We can notice that we use two toplevel
definitions {\tt PermutProve} and {\tt Permut}. The function to be
called is {\tt PermutProve} which computes the lengths of the two
lists and calls {\tt Permut} with the length if the two lists have the
same length. {\tt Permut} works as expected.  If the two lists are
equal, it concludes. Otherwise, if the lists have identical first
elements, it applies {\tt Permut} on the tail of the lists.  Finally,
if the lists have different first elements, it puts the first element
of one of the lists (here the second one which appears in the {\tt
  permut} predicate) at the end if that is possible, i.e., if the new
first element has been at this place previously. To verify that all
rotations have been done for a list, we use the length of the list as
an argument for {\tt Permut} and this length is decremented for each
rotation down to, but not including, 1 because for a list of length
$n$, we can make exactly $n-1$ rotations to generate at most $n$
distinct lists. Here, it must be noticed that we use the natural
numbers of {\Coq} for the rotation counter. On Figure~\ref{ltac}, we
can see that it is possible to use usual natural numbers but they are
only used as arguments for primitive tactics and they cannot be
handled, in particular, we cannot make computations with them. So, a
natural choice is to use {\Coq} data structures so that {\Coq} makes
the computations (reductions) by {\tt eval compute in} and we can get
the terms back by {\tt match}.
 
With {\tt PermutProve}, we can now prove lemmas as 
% shown on Figure~\ref{permutlem}.
follows:
%\begin{figure}
%\begin{centerframe}

\begin{coq_example*}
Lemma permut_ex1 :
  permut nat (1 :: 2 :: 3 :: nil) (3 :: 2 :: 1 :: nil).
Proof. PermutProve. Qed.
Lemma permut_ex2 :
  permut nat
    (0 :: 1 :: 2 :: 3 :: 4 :: 5 :: 6 :: 7 :: 8 :: 9 :: nil)
    (0 :: 2 :: 4 :: 6 :: 8 :: 9 :: 7 :: 5 :: 3 :: 1 :: nil).
Proof. PermutProve. Qed.
\end{coq_example*}
%\end{centerframe}
%\caption{Examples of {\tt PermutProve} use}
%\label{permutlem}
%\end{figure}


\subsection{Deciding intuitionistic propositional logic}

\begin{figure}[b]
\begin{centerframe}
\begin{coq_example}
Ltac Axioms :=
  match goal with
  | |- True => trivial
  | _:False |- _  => elimtype False; assumption
  | _:?A |- ?A  => auto
  end.
\end{coq_example}
\end{centerframe}
\caption{Deciding intuitionistic propositions (1)}
\label{tautoltaca}
\end{figure}


\begin{figure}
\begin{centerframe}
\begin{coq_example}
Ltac DSimplif :=
  repeat
   (intros;
    match goal with
     | id:(~ _) |- _ => red in id
     | id:(_ /\ _) |- _ =>
         elim id; do 2 intro; clear id
     | id:(_ \/ _) |- _ =>
         elim id; intro; clear id
     | id:(?A /\ ?B -> ?C) |- _ =>
         cut (A -> B -> C);
          [ intro | intros; apply id; split; assumption ]
     | id:(?A \/ ?B -> ?C) |- _ =>
         cut (B -> C);
          [ cut (A -> C);
             [ intros; clear id
             | intro; apply id; left; assumption ]
          | intro; apply id; right; assumption ]
     | id0:(?A -> ?B),id1:?A |- _ =>
         cut B; [ intro; clear id0 | apply id0; assumption ]
     | |- (_ /\ _) => split
     | |- (~ _) => red
     end).
Ltac TautoProp :=
  DSimplif;
   Axioms ||
     match goal with
     | id:((?A -> ?B) -> ?C) |- _ =>
          cut (B -> C);
          [ intro; cut (A -> B);
             [ intro; cut C;
                [ intro; clear id | apply id; assumption ]
             | clear id ]
          | intro; apply id; intro; assumption ]; TautoProp
     | id:(~ ?A -> ?B) |- _ =>
         cut (False -> B);
          [ intro; cut (A -> False);
             [ intro; cut B;
                [ intro; clear id | apply id; assumption ]
             | clear id ]
          | intro; apply id; red; intro; assumption ]; TautoProp
     | |- (_ \/ _) => (left; TautoProp) || (right; TautoProp)
     end.
\end{coq_example}
\end{centerframe}
\caption{Deciding intuitionistic propositions (2)}
\label{tautoltacb}
\end{figure}

The pattern matching on goals allows a complete and so a powerful
backtracking when returning tactic values. An interesting application
is the problem of deciding intuitionistic propositional logic.
Considering the contraction-free sequent calculi {\tt LJT*} of
Roy~Dyckhoff (\cite{Dyc92}), it is quite natural to code such a tactic
using the tactic language as shown on Figures~\ref{tautoltaca}
and~\ref{tautoltacb}. The tactic {\tt Axioms} tries to conclude using
usual axioms. The tactic {\tt DSimplif} applies all the reversible
rules of Dyckhoff's system. Finally, the tactic {\tt TautoProp} (the
main tactic to be called) simplifies with {\tt DSimplif}, tries to
conclude with {\tt Axioms} and tries several paths using the
backtracking rules (one of the four Dyckhoff's rules for the left
implication to get rid of the contraction and the right or).

For example, with {\tt TautoProp}, we can prove tautologies like
 those:
% on Figure~\ref{tautolem}.
%\begin{figure}[tbp]
%\begin{centerframe}
\begin{coq_example*}
Lemma tauto_ex1 : forall A B:Prop, A /\ B -> A \/ B.
Proof. TautoProp. Qed.
Lemma tauto_ex2 :
   forall A B:Prop, (~ ~ B -> B) -> (A -> B) -> ~ ~ A -> B.
Proof. TautoProp. Qed.
\end{coq_example*}
%\end{centerframe}
%\caption{Proofs of tautologies with {\tt TautoProp}}
%\label{tautolem}
%\end{figure}

\subsection{Deciding type isomorphisms}

A more tricky problem is to decide equalities between types and modulo
isomorphisms. Here, we choose to use the isomorphisms of the simply typed
$\lb{}$-calculus with Cartesian product and $unit$ type (see, for example,
\cite{RC95}). The axioms of this $\lb{}$-calculus are given by
table~\ref{isosax}.

\begin{figure}
\begin{centerframe}
\begin{coq_eval}
Reset Initial.
\end{coq_eval}
\begin{coq_example*}
Open Scope type_scope.
Section Iso_axioms.
Variables A B C : Set.
Axiom Com : A * B = B * A.
Axiom Ass : A * (B * C) = A * B * C.
Axiom Cur : (A * B -> C) = (A -> B -> C).
Axiom Dis : (A -> B * C) = (A -> B) * (A -> C).
Axiom P_unit : A * unit = A.
Axiom AR_unit : (A -> unit) = unit.
Axiom AL_unit : (unit -> A) = A.
Lemma Cons : B = C -> A * B = A * C.
Proof.
intro Heq; rewrite Heq; reflexivity.
Qed.
End Iso_axioms.
\end{coq_example*}
\end{centerframe}
\caption{Type isomorphism axioms}
\label{isosax}
\end{figure}

A more tricky problem is to decide equalities between types and modulo
isomorphisms. Here, we choose to use the isomorphisms of the simply typed
$\lb{}$-calculus with Cartesian product and $unit$ type (see, for example,
\cite{RC95}). The axioms of this $\lb{}$-calculus are given on
Figure~\ref{isosax}.

\begin{figure}[ht]
\begin{centerframe}
\begin{coq_example}
Ltac DSimplif trm :=
  match trm with
  | (?A * ?B * ?C) =>
      rewrite <- (Ass A B C); try MainSimplif
  | (?A * ?B -> ?C) =>
      rewrite (Cur A B C); try MainSimplif
  | (?A -> ?B * ?C) =>
      rewrite (Dis A B C); try MainSimplif
  | (?A * unit) =>
      rewrite (P_unit A); try MainSimplif
  | (unit * ?B) =>
      rewrite (Com unit B); try MainSimplif
  | (?A -> unit) =>
      rewrite (AR_unit A); try MainSimplif
  | (unit -> ?B) =>
      rewrite (AL_unit B); try MainSimplif
  | (?A * ?B) =>
      (DSimplif A; try MainSimplif) || (DSimplif B; try MainSimplif)
  | (?A -> ?B) =>
      (DSimplif A; try MainSimplif) || (DSimplif B; try MainSimplif)
  end
 with MainSimplif :=
  match goal with
  | |- (?A = ?B) => try DSimplif A; try DSimplif B
  end.
Ltac Length trm :=
  match trm with
  | (_ * ?B) => let succ := Length B in constr:(S succ)
  | _ => constr:1
  end.
Ltac assoc := repeat rewrite <- Ass.
\end{coq_example}
\end{centerframe}
\caption{Type isomorphism tactic (1)}
\label{isosltac1}
\end{figure}

\begin{figure}[ht]
\begin{centerframe}
\begin{coq_example}
Ltac DoCompare n :=
  match goal with
  | [ |- (?A = ?A) ] => reflexivity
  | [ |- (?A * ?B = ?A * ?C) ] =>
      apply Cons; let newn := Length B in
                  DoCompare newn
  | [ |- (?A * ?B = ?C) ] =>
      match eval compute in n with
      | 1 => fail
      | _ =>
          pattern (A * B) at 1; rewrite Com; assoc; DoCompare (pred n)
      end
  end.
Ltac CompareStruct :=
  match goal with
  | [ |- (?A = ?B) ] =>
      let l1 := Length A
      with l2 := Length B in
      match eval compute in (l1 = l2) with
      | (?n = ?n) => DoCompare n
      end
  end.
Ltac IsoProve := MainSimplif; CompareStruct.
\end{coq_example}
\end{centerframe}
\caption{Type isomorphism tactic (2)}
\label{isosltac2}
\end{figure}

The tactic to judge equalities modulo this axiomatization can be written as
shown on Figures~\ref{isosltac1} and~\ref{isosltac2}. The algorithm is quite
simple. Types are reduced using axioms that can be oriented (this done by {\tt
MainSimplif}). The normal forms are sequences of Cartesian
products without Cartesian product in the left component. These normal forms
are then compared modulo permutation of the components (this is done by {\tt
CompareStruct}). The main tactic to be called and realizing this algorithm is
{\tt IsoProve}.

% Figure~\ref{isoslem} gives 
Here are examples of what can be solved by {\tt IsoProve}.
%\begin{figure}[ht]
%\begin{centerframe}
\begin{coq_example*}
Lemma isos_ex1 : 
  forall A B:Set, A * unit * B = B * (unit * A).
Proof.
intros; IsoProve.
Qed.

Lemma isos_ex2 :
  forall A B C:Set,
    (A * unit -> B * (C * unit)) =
    (A * unit -> (C -> unit) * C) * (unit -> A -> B).
Proof.
intros; IsoProve.
Qed.
\end{coq_example*}
%\end{centerframe}
%\caption{Type equalities solved by {\tt IsoProve}}
%\label{isoslem}
%\end{figure}

%%% Local Variables: 
%%% mode: latex
%%% TeX-master: "Reference-Manual"
%%% End: 
% Detailed Examples of tactics
\include{RefMan-decl.v}% The mathematical proof language

\part{User extensions}
\include{RefMan-syn.v}% The Syntax and the Grammar commands
%%SUPPRIME \include{RefMan-tus.v}% Writing tactics
\include{RefMan-sch.v}% The Scheme commands

\part{Practical tools}
\include{RefMan-com}% The coq commands (coqc coqtop)
\include{RefMan-uti}% utilities (gallina, do_Makefile, etc)
\include{RefMan-ide}% Coq IDE

%BEGIN LATEX
\RefManCutCommand{BEGINADDENDUM=\thepage}
%END LATEX
\part{Addendum to the Reference Manual}
\include{AddRefMan-pre}%
\include{Cases.v}%
\include{Coercion.v}%
\include{CanonicalStructures.v}%
\include{Classes.v}%
%%SUPPRIME \include{Natural.v}%
\include{Omega.v}%
\achapter{Micromega : tactics for solving arithmetic goals over ordered rings}
\aauthor{Fr�d�ric Besson and Evgeny Makarov}
\newtheorem{theorem}{Theorem}

 
\asection{Short description of the tactics}
\tacindex{psatz}  \tacindex{lra} 
\label{sec:psatz-hurry}
The {\tt Psatz} module ({\tt Require Psatz.}) gives access to several tactics for solving arithmetic goals over
 {\tt Z}\footnote{Support for {\tt nat} and {\tt N} is obtained by pre-processing the goal with the {\tt zify} tactic.}, {\tt Q} and {\tt R}:
\begin{itemize}
\item {\tt lia} is a decision procedure for linear integer arithmetic (see Section~\ref{sec:lia});
\item {\tt nia} is an incomplete proof procedure for integer non-linear arithmetic (see Section~\ref{sec:nia});
\item {\tt lra} is a decision procedure for linear (real or rational) arithmetic goals (see Section~\ref{sec:lra});
\item {\tt psatz D n} where {\tt D} is {\tt Z}, {\tt Q} or {\tt R} and {\tt n} is an optional integer limiting the proof search depth is
is an incomplete proof procedure for non-linear arithmetic. It is based on John Harrison's Hol light driver to the external prover {\tt cspd}\footnote{Sources and binaries can be found at \url{https://projects.coin-or.org/Csdp}}. 
   Note that the {\tt csdp} driver is generating 
   a \emph{proof cache} thus allowing to rerun scripts even without {\tt csdp} (see Section~\ref{sec:psatz}). 
\end{itemize}

The tactics solve propositional formulas parameterised by atomic arithmetics expressions
interpreted over a domain $D \in \{\mathbb{Z}, \mathbb{Q}, \mathbb{R} \}$.
The syntax of the formulas is the following:
\[
\begin{array}{lcl}
 F &::=&  A \mid P \mid \mathit{True} \mid \mathit{False} \mid F_1 \land F_2 \mid F_1 \lor F_2 \mid F_1 \leftrightarrow F_2 \mid F_1 \to F_2 \mid \sim F\\
 A &::=& p_1 = p_2 \mid  p_1 > p_2 \mid p_1 < p_2 \mid p_1 \ge p_2 \mid p_1 \le p_2 \\
 p &::=& c \mid x \mid {-}p \mid p_1 - p_2 \mid p_1 + p_2 \mid p_1 \times p_2 \mid p \verb!^! n
 \end{array}
 \]
 where $c$ is a numeric constant, $x\in D$ is a numeric variable and the operators $-$, $+$, $\times$, are
 respectively subtraction, addition, product, $p \verb!^!n $ is exponentiation by a constant $n$, $P$ is an
 arbitrary proposition.
 %
 For {\tt Q}, equality is not leibnitz equality {\tt =} but the equality of rationals {\tt ==}.

For {\tt Z} (resp. {\tt Q} ), $c$ ranges over integer constants (resp. rational constants).
%% The following table details for each domain $D \in \{\mathbb{Z},\mathbb{Q},\mathbb{R}\}$ the range of constants $c$ and exponent $n$.
%% \[
%% \begin{array}{|c|c|c|c|}
%%   \hline
%%   &\mathbb{Z} & \mathbb{Q} & \mathbb{R} \\
%%   \hline
%%   c &\mathtt{Z} & \mathtt{Q} & (see below) \\
%%   \hline
%%   n &\mathtt{Z} & \mathtt{Z} & \mathtt{nat}\\
%%   \hline
%% \end{array}
%% \]
For {\tt R}, the tactic recognises as real constants the following expressions:
\begin{verbatim}
c ::= R0 | R1 | Rmul(c,c) | Rplus(c,c) | Rminus(c,c) | IZR z | IQR q | Rdiv(c,c) | Rinv c
\end{verbatim}
where ${\tt z}$ is a constant in {\tt Z} and {\tt q} is a constant in {\tt Q}.
This includes integer constants written using the decimal notation \emph{i.e.,} {\tt c\%R}.

\asection{\emph{Positivstellensatz} refutations}
\label{sec:psatz-back}

The name {\tt psatz} is an abbreviation for \emph{positivstellensatz} -- literally positivity theorem -- which
generalises Hilbert's \emph{nullstellensatz}.
%
It relies on the notion of $\mathit{Cone}$. Given  a (finite) set of polynomials $S$, $Cone(S)$ is
inductively defined as the smallest set of polynomials closed under the following rules:
\[
\begin{array}{l}
\dfrac{p \in S}{p \in Cone(S)} \quad 
\dfrac{}{p^2 \in Cone(S)} \quad
\dfrac{p_1 \in Cone(S) \quad p_2 \in Cone(S) \quad \Join \in \{+,*\}} {p_1 \Join p_2 \in Cone(S)}\\
\end{array}
\]
The following theorem provides a proof principle for checking that a set of polynomial inequalities do not have solutions\footnote{Variants deal with equalities and strict inequalities.}:
\begin{theorem}
  \label{thm:psatz}
  Let $S$ be a set of polynomials.\\
  If ${-}1$ belongs to $Cone(S)$ then the conjunction $\bigwedge_{p \in S} p\ge 0$ is unsatisfiable.
\end{theorem}
A proof based on this theorem is called a \emph{positivstellensatz} refutation.
%
The tactics work as follows. Formulas are normalised into conjonctive normal form $\bigwedge_i C_i$ where
$C_i$ has the general form $(\bigwedge_{j\in S_i} p_j \Join 0) \to \mathit{False})$ and $\Join \in \{>,\ge,=\}$ for $D\in
\{\mathbb{Q},\mathbb{R}\}$ and $\Join \in \{\ge, =\}$ for $\mathbb{Z}$.
%
For each conjunct $C_i$, the tactic calls a oracle which searches for $-1$ within the cone.
%
Upon success, the oracle returns a \emph{cone expression} that is normalised by the {\tt ring} tactic (see chapter~\ref{ring}) and checked to be
$-1$.


\asection{{\tt lra} : a decision procedure for linear real and rational arithmetic}
\label{sec:lra}
The {\tt lra} tactic is searching for \emph{linear} refutations using
Fourier elimination\footnote{More efficient linear programming techniques could equally be employed}.  As a
result, this tactic explores a subset of the $Cone$ defined as:
\[
LinCone(S) =\left\{ \left. \sum_{p \in S} \alpha_p \times p\ \right|\ \alpha_p \mbox{ are positive constants} \right\}
\]
The deductive power of {\tt lra} is the combined deductive power of {\tt ring\_simplify} and {\tt fourier}.
%
There is also an overlap with the {\tt field} tactic {\emph e.g.}, {\tt x = 10 * x / 10} is solved by {\tt lra}.

\asection{ {\tt psatz} : a proof procedure for non-linear arithmetic}
\label{sec:psatz}
The {\tt psatz} tactic explores the $Cone$ by increasing degrees -- hence the depth parameter $n$.
In theory, such a proof search is complete -- if the goal is provable the search eventually stops.
Unfortunately, the external oracle is using numeric (approximate) optimisation techniques that might miss a
refutation. 

To illustrate the working of the tactic, consider we wish to prove the following Coq goal.\\
\begin{coq_eval}
  Require Import ZArith Psatz.
  Open Scope Z_scope.
\end{coq_eval}
\begin{coq_example*}
  Goal forall x, -x^2 >= 0 -> x - 1 >= 0 -> False.
\end{coq_example*}
\begin{coq_eval}
intro x; psatz Z 2.
\end{coq_eval}
Such a goal is solved by {\tt intro x; psatz Z 2}. The oracle returns the cone expression $2 \times
(\mathbf{x-1}) + \mathbf{x-1}\times\mathbf{x-1} + \mathbf{-x^2}$ (polynomial hypotheses are printed in bold). By construction, this
expression belongs to $Cone(\{-x^2, x -1\})$.  Moreover, by running {\tt ring} we obtain $-1$. By
Theorem~\ref{thm:psatz}, the goal is valid.
%

%% \paragraph{The {\tt sos} tactic} -- where {\tt sos} stands for \emph{sum of squares} -- tries to prove that a
%% single polynomial $p$ is positive by expressing it as a sum of squares \emph{i.e.,} $\sum_{i\in S} p_i^2$.
%% This amounts to searching for $p$ in the cone without generators \emph{i.e.}, $Cone(\{\})$.
%

\asection{ {\tt lia} : a tactic for linear integer arithmetic }
\tacindex{lia}
\label{sec:lia}

The tactic {\tt lia} ({\tt Require Lia.}) offers an alternative to the {\tt omega} and {\tt romega} tactic (see
Chapter~\ref{OmegaChapter}). 
%
Rougthly speaking, the deductive power of {\tt lia} is the combined deductive power of {\tt ring\_simplify} and {\tt omega}.
%
However, it solves linear goals that {\tt omega} and {\tt romega} do not solve, such as the
following so-called \emph{omega nightmare}~\cite{TheOmegaPaper}.
\begin{coq_example*}
  Goal forall x y, 
       27 <= 11 * x + 13 * y <= 45 -> 
       -10 <= 7 * x - 9 * y <= 4 ->   False.
\end{coq_example*}
\begin{coq_eval}
intro x; lia.
\end{coq_eval}
The estimation of the relative efficiency of lia \emph{vs} {\tt omega}
and {\tt romega} is under evaluation.

\paragraph{High level view of {\tt lia}.}
Over $\mathbb{R}$,  \emph{positivstellensatz} refutations are a complete proof principle\footnote{In practice, the oracle might fail to produce such a refutation.}.
%
However, this is not the case over $\mathbb{Z}$.
%
Actually, \emph{positivstellensatz} refutations are not even sufficient to decide linear \emph{integer} 
arithmetics.
%
The canonical exemple is {\tt 2 * x = 1 -> False} which is a theorem of $\mathbb{Z}$ but not a theorem of $\mathbb{R}$.
%
To remedy this weakness, the {\tt lia} tactic is using recursively a combination of:
%
\begin{itemize}
\item linear \emph{positivstellensatz} refutations;
\item cutting plane proofs;
\item case split.
\end{itemize}

\paragraph{Cutting plane proofs} are a way to take into account the discreetness of $\mathbb{Z}$ by rounding up
(rational) constants up-to the closest integer. 
%
\begin{theorem}
  Let $p$ be an integer and $c$ a rational constant.
  \[
  p \ge c \Rightarrow p \ge \lceil c \rceil
  \]
\end{theorem}
For instance, from $2 * x = 1$ we can deduce 
\begin{itemize}
\item $x \ge 1/2$ which cut plane is $ x \ge \lceil 1/2 \rceil = 1$;
\item $ x \le 1/2$ which cut plane is $ x \le \lfloor 1/2 \rfloor = 0$.
\end{itemize}
By combining these two facts (in normal form) $x - 1 \ge 0$ and $-x \ge 0$, we conclude by exhibiting a
\emph{positivstellensatz} refutation ($-1 \equiv \mathbf{x-1} + \mathbf{-x}  \in Cone(\{x-1,x\})$).

Cutting plane proofs and linear \emph{positivstellensatz} refutations are a complete proof principle for integer linear arithmetic.

\paragraph{Case split} allow to enumerate over the possible values of an expression. 
\begin{theorem}
  Let $p$ be an integer and $c_1$ and $c_2$  integer constants.
  \[
  c_1 \le p \le c_2 \Rightarrow \bigvee_{x \in [c_1,c_2]} p = x
  \]
\end{theorem}
Our current oracle tries to find an expression $e$ with a small range $[c_1,c_2]$.
%
We generate $c_2 - c_1$ subgoals which contexts are enriched with an equation $e = i$ for $i \in [c_1,c_2]$ and
recursively search for a proof.

\asection{ {\tt nia} : a proof procedure for non-linear integer arithmetic}
\tacindex{nia}
\label{sec:nia}
The {\tt nia} tactic is an {\emph experimental} proof procedure for non-linear  integer arithmetic.
%
The tactic performs a limited amount of non-linear reasoning before running the
linear prover of {\tt lia}.
This pre-processing does the following:
\begin{itemize}
\item If the context contains an arithmetic expression of the form $e[x^2]$ where $x$ is a
  monomial, the context is enriched with $x^2\ge 0$;
\item For all pairs of hypotheses $e_1\ge 0$, $e_2 \ge 0$, the context is enriched with $e_1 \times e_2 \ge 0$.
\end{itemize}
After pre-processing, the linear prover of {\tt lia} is searching for a proof by abstracting monomials by variables.



%%% Local Variables: 
%%% mode: latex
%%% TeX-master: "Reference-Manual"
%%% End: 

%%SUPPRIME \include{Correctness.v}% = preuve de pgms imperatifs
\achapter{Extraction of programs in Objective Caml and Haskell}
\label{Extraction}
\aauthor{Jean-Christophe Filli�tre and Pierre Letouzey}
\index{Extraction}

We present here the \Coq\ extraction commands, used to build certified
and relatively efficient functional programs, extracting them from
either \Coq\ functions or \Coq\ proofs of specifications. The
functional languages available as output are currently \ocaml{},
\textsc{Haskell} and \textsc{Scheme}.  In the following, ``ML'' will
be used (abusively) to refer to any of the three.

\paragraph{Differences with old versions.}
The current extraction mechanism is new for version 7.0 of {\Coq}.
In particular, the \FW\ toplevel used as an intermediate step between 
\Coq\ and ML has been withdrawn.  It is also not possible 
any more to import ML objects in this \FW\ toplevel.
The current mechanism also differs from
the one in previous versions of \Coq: there is no more
an explicit toplevel for the language (formerly called \textsc{Fml}). 

\asection{Generating ML code}
\comindex{Extraction}
\comindex{Recursive Extraction}
\comindex{Extraction Module}
\comindex{Recursive Extraction Module}

The next two commands are meant to be used for rapid preview of
extraction. They both display extracted term(s) inside \Coq.

\begin{description}
\item {\tt Extraction \qualid.} ~\par
  Extracts one constant or module in the \Coq\ toplevel.

\item {\tt Recursive Extraction  \qualid$_1$ \dots\ \qualid$_n$.} ~\par
  Recursive extraction of all the globals (or modules) \qualid$_1$ \dots\
  \qualid$_n$ and all their dependencies in the \Coq\ toplevel.
\end{description}

%% TODO error messages

All the following commands produce real ML files. User can choose to produce
one monolithic file or one file per \Coq\ library. 

\begin{description}
\item {\tt Extraction "{\em file}"}  
      \qualid$_1$ \dots\ \qualid$_n$. ~\par
  Recursive extraction of all the globals (or modules) \qualid$_1$ \dots\
  \qualid$_n$ and all their dependencies in one monolithic file {\em file}.
  Global and local identifiers are renamed according to the chosen ML
  language to fulfill its syntactic conventions, keeping original
  names as much as possible.
  
\item {\tt Extraction Library} \ident. ~\par 
  Extraction of the whole \Coq\ library {\tt\ident.v} to an ML module
  {\tt\ident.ml}.  In case of name clash, identifiers are here renamed
  using prefixes \verb!coq_!  or \verb!Coq_! to ensure a
  session-independent renaming.

\item {\tt Recursive Extraction Library} \ident. ~\par
  Extraction of the \Coq\ library {\tt\ident.v} and all other modules 
  {\tt\ident.v} depends on. 
\end{description}

The list of globals \qualid$_i$ does not need to be
exhaustive: it is automatically completed into a complete and minimal
environment. 

\asection{Extraction options}

\asubsection{Setting the target language}
\comindex{Extraction Language}

The ability to fix target language is the first and more important
of the extraction options. Default is Ocaml.
\begin{description}
\item {\tt Extraction Language Ocaml}.
\item {\tt Extraction Language Haskell}.
\item {\tt Extraction Language Scheme}.
\end{description}

\asubsection{Inlining and optimizations}

Since Objective Caml is a strict language, the extracted
code has to be optimized in order to be efficient (for instance, when
using induction principles we do not want to compute all the recursive
calls but only the needed ones). So the extraction mechanism provides
an automatic optimization routine that will be
called each time the user want to generate Ocaml programs. Essentially,
it performs constants inlining and reductions.  Therefore some
constants may not appear in resulting monolithic Ocaml program.
In the case of modular extraction, even if some inlining is done, the
inlined constant are nevertheless printed, to ensure
session-independent programs.

Concerning Haskell, such optimizations are less useful because of
lazyness. We still make some optimizations, for example in order to
produce more readable code. 

All these optimizations are controled by the following \Coq\ options: 

\begin{description}

\item \comindex{Set Extraction Optimize}
{\tt Set Extraction Optimize.}

\item \comindex{Unset Extraction Optimize}
{\tt Unset Extraction Optimize.}

Default is Set. This control all optimizations made on the ML terms 
(mostly reduction of dummy beta/iota redexes, but also simplifications on
Cases, etc). Put this option to Unset if you want a ML term as close as 
possible to the Coq term.

\item \comindex{Set Extraction AutoInline}
{\tt Set Extraction AutoInline.} 

\item \comindex{Unset Extraction AutoInline}
{\tt Unset Extraction AutoInline.} 

Default is Set, so by default, the extraction mechanism feels free to 
inline the bodies of some defined constants, according to some heuristics 
like size of bodies, useness of some arguments, etc. Those heuristics are 
not always perfect, you may want to disable this feature, do it by Unset. 

\item \comindex{Extraction Inline}
{\tt Extraction Inline} \qualid$_1$ \dots\ \qualid$_n$. 

\item \comindex{Extraction NoInline}
{\tt Extraction NoInline} \qualid$_1$ \dots\ \qualid$_n$. 

In addition to the automatic inline feature, you can now tell precisely to 
inline some more constants by the {\tt Extraction Inline} command. Conversely, 
you can forbid the automatic inlining of some specific constants by
the {\tt Extraction NoInline} command.
Those two commands enable a precise control of what is inlined and what is not. 

\item \comindex{Print Extraction Inline}
{\tt Print Extraction Inline}. 

Prints the current state of the table recording the custom inlinings 
declared by the two previous commands. 

\item \comindex{Reset Extraction Inline}
{\tt Reset Extraction Inline}. 

Puts the table recording the custom inlinings back to empty. 

\end{description}


\paragraph{Inlining and printing of a constant declaration.}

A user can explicitly ask for a constant to be extracted by two means:
\begin{itemize}
\item by mentioning it on the extraction command line
\item by extracting the whole \Coq\ module of this constant.
\end{itemize}
In both cases, the declaration of this constant will be present in the
produced file. 
But this same constant may or may not be inlined in the following
terms, depending on the automatic/custom inlining mechanism.  


For the constants non-explicitly required but needed for dependency
reasons, there are two cases: 
\begin{itemize}
\item If an inlining decision is taken, whether automatically or not,
all occurrences of this constant are replaced by its extracted body, and
this constant is not declared in the generated file.
\item If no inlining decision is taken, the constant is normally
  declared in the produced file. 
\end{itemize}

\asubsection{Extra elimination of useless arguments}

\begin{description}
\item \comindex{Extraction Implicit}
 {\tt Extraction Implicit} \qualid\ [ \ident$_1$ \dots\ \ident$_n$ ].

This experimental command allows to declare some arguments of
\qualid\ as implicit, i.e. useless in extracted code and hence to
be removed by extraction. Here \qualid\ can be any function or
inductive constructor, and \ident$_i$ are the names of the concerned
arguments. In fact, an argument can also be referred by a number
indicating its position, starting from 1. When an actual extraction
takes place, an error is raised if the {\tt Extraction Implicit}
declarations cannot be honored, that is if any of the implicited
variables still occurs in the final code. This declaration of useless
arguments is independent but complementary to the main elimination
principles of extraction (logical parts and types).
\end{description}

\asubsection{Realizing axioms}\label{extraction:axioms}

Extraction will fail if it encounters an informative
axiom not realized (see Section~\ref{extraction:axioms}). 
A warning will be issued if it encounters an logical axiom, to remind 
user that inconsistent logical axioms may lead to incorrect or
non-terminating extracted terms. 

It is possible to assume some axioms while developing a proof. Since
these axioms can be any kind of proposition or object or type, they may
perfectly well have some computational content. But a program must be
a closed term, and of course the system cannot guess the program which
realizes an axiom.  Therefore, it is possible to tell the system
what ML term corresponds to a given axiom. 

\comindex{Extract Constant}
\begin{description}
\item{\tt Extract Constant \qualid\ => \str.} ~\par
  Give an ML extraction for the given constant.
  The \str\ may be an identifier or a quoted string.
\item{\tt Extract Inlined Constant \qualid\ => \str.} ~\par
  Same as the previous one, except that the given ML terms will
  be inlined everywhere instead of being declared via a let.
\end{description}

Note that the {\tt Extract Inlined Constant} command is sugar
for an {\tt Extract Constant} followed by a {\tt Extraction Inline}. 
Hence a {\tt Reset Extraction Inline} will have an effect on the
realized and inlined axiom.

Of course, it is the responsibility of the user to ensure that the ML
terms given to realize the axioms do have the expected types.  In
fact, the strings containing realizing code are just copied in the
extracted files. The extraction recognizes whether the realized axiom
should become a ML type constant or a ML object declaration.

\Example
\begin{coq_example}
Axiom X:Set.
Axiom x:X.
Extract Constant X => "int".
Extract Constant x => "0".
\end{coq_example}   

Notice that in the case of type scheme axiom (i.e. whose type is an
arity, that is a sequence of product finished by a sort), then some type
variables has to be given. The syntax is then: 

\begin{description}
\item{\tt Extract Constant \qualid\ \str$_1$ \ldots \str$_n$ => \str.} ~\par
\end{description}

The number of type variables is checked by the system. 

\Example
\begin{coq_example}
Axiom Y : Set -> Set -> Set.
Extract Constant Y "'a" "'b" => " 'a*'b ".
\end{coq_example}

Realizing an axiom via {\tt Extract Constant} is only useful in the
case of an informative axiom (of sort Type or Set). A logical axiom
have no computational content and hence will not appears in extracted
terms. But a warning is nonetheless issued if extraction encounters a
logical axiom. This warning reminds user that inconsistent logical
axioms may lead to incorrect or non-terminating extracted terms.

If an informative axiom has not been realized before an extraction, a
warning is also issued and the definition of the axiom is filled with
an exception labeled {\tt AXIOM TO BE REALIZED}. The user must then
search these exceptions inside the extracted file and replace them by
real code.

\comindex{Extract Inductive} 

The system also provides a mechanism to specify ML terms for inductive
types and constructors.  For instance, the user may want to use the ML
native boolean type instead of \Coq\ one.  The syntax is the following:

\begin{description}
\item{\tt Extract Inductive \qualid\ => \str\ [ \str\ \dots \str\ ]\
{\it optstring}.} ~\par
  Give an ML extraction for the given inductive type. You must specify
  extractions for the type itself (first \str) and all its
  constructors (between square brackets). If given, the final optional
  string should contain a function emulating pattern-matching over this
  inductive type. If this optional string is not given, the ML
  extraction must be an ML inductive datatype, and the native
  pattern-matching of the language will be used.
\end{description}

For an inductive type with $k$ constructor, the function used to
emulate the match should expect $(k+1)$ arguments, first the $k$
branches in functional form, and then the inductive element to
destruct. For instance, the match branch \verb$| S n => foo$ gives the
functional form \verb$(fun n -> foo)$. Note that a constructor with no
argument is considered to have one unit argument, in order to block
early evaluation of the branch: \verb$| O => bar$ leads to the functional
form \verb$(fun () -> bar)$. For instance, when extracting {\tt nat}
into {\tt int}, the code to provide has type:
{\tt (unit->'a)->(int->'a)->int->'a}.
    
As for {\tt Extract Inductive}, this command should be used with care:
\begin{itemize}
\item The ML code provided by the user is currently \emph{not} checked at all by
  extraction, even for syntax errors.

\item Extracting an inductive type to a pre-existing ML inductive type
is quite sound. But extracting to a general type (by providing an
ad-hoc pattern-matching) will often \emph{not} be fully rigorously
correct.  For instance, when extracting {\tt nat} to Ocaml's {\tt
int}, it is theoretically possible to build {\tt nat} values that are
larger than Ocaml's {\tt max\_int}. It is the user's responsability to
be sure that no overflow or other bad events occur in practice.

\item Translating an inductive type to an ML type does \emph{not}
magically improve the asymptotic complexity of functions, even if the
ML type is an efficient representation. For instance, when extracting
{\tt nat} to Ocaml's {\tt int}, the function {\tt mult} stays
quadratic. It might be interesting to associate this translation with
some specific {\tt Extract Constant} when primitive counterparts exist.
\end{itemize}

\Example
Typical examples are the following:
\begin{coq_example}
Extract Inductive unit => "unit" [ "()" ].
Extract Inductive bool => "bool" [ "true" "false" ].
Extract Inductive sumbool => "bool" [ "true" "false" ].
\end{coq_example}

If an inductive constructor or type has arity 2 and the corresponding 
string is enclosed by parenthesis, then the rest of the string is used
as infix constructor or type. 
\begin{coq_example}
Extract Inductive list => "list" [ "[]" "(::)" ].
Extract Inductive prod => "(*)"  [ "(,)" ].
\end{coq_example}

As an example of translation to a non-inductive datatype, let's turn
{\tt nat} into Ocaml's {\tt int} (see caveat above):
\begin{coq_example}
Extract Inductive nat => int [ "0" "succ" ]
 "(fun fO fS n -> if n=0 then fO () else fS (n-1))".
\end{coq_example}

\asubsection{Avoiding conflicts with existing filenames}

\comindex{Extraction Blacklist}

When using {\tt Extraction Library}, the names of the extracted files
directly depends from the names of the \Coq\ files. It may happen that
these filenames are in conflict with already existing files, 
either in the standard library of the target language or in other
code that is meant to be linked with the extracted code. 
For instance the module {\tt List} exists both in \Coq\ and in Ocaml.
It is possible to instruct the extraction not to use particular filenames.

\begin{description}
\item{\tt Extraction Blacklist \ident \ldots \ident.} ~\par
  Instruct the extraction to avoid using these names as filenames
  for extracted code. 
\item{\tt Print Extraction Blacklist.} ~\par
  Show the current list of filenames the extraction should avoid.
\item{\tt Reset Extraction Blacklist.} ~\par
  Allow the extraction to use any filename.
\end{description}

For Ocaml, a typical use of these commands is
{\tt Extraction Blacklist String List}.

\asection{Differences between \Coq\ and ML type systems}


Due to differences between \Coq\ and ML type systems, 
some extracted programs are not directly typable in ML. 
We now solve this problem (at least in Ocaml) by adding 
when needed some unsafe casting {\tt Obj.magic}, which give
a generic type {\tt 'a} to any term.

For example, here are two kinds of problem that can occur:

\begin{itemize}
  \item If some part of the program is {\em very} polymorphic, there
    may be no ML type for it. In that case the extraction to ML works
    all right but the generated code may be refused by the ML
    type-checker. A very well known example is the {\em distr-pair}
    function:
\begin{verbatim}
Definition dp := 
 fun (A B:Set)(x:A)(y:B)(f:forall C:Set, C->C) => (f A x, f B y).
\end{verbatim}

In Ocaml, for instance, the direct extracted term would be:

\begin{verbatim}
let dp x y f = Pair((f () x),(f () y))
\end{verbatim}

and would have type:
\begin{verbatim}
dp : 'a -> 'a -> (unit -> 'a -> 'b) -> ('b,'b) prod
\end{verbatim}

which is not its original type, but a restriction.

We now produce the following correct version:
\begin{verbatim}
let dp x y f = Pair ((Obj.magic f () x), (Obj.magic f () y))
\end{verbatim}

  \item Some definitions of \Coq\ may have no counterpart in ML. This
    happens when there is a quantification over types inside the type
    of a constructor; for example:
\begin{verbatim}
Inductive anything : Set := dummy : forall A:Set, A -> anything.
\end{verbatim}

which corresponds to the definition of an ML dynamic type.
In Ocaml, we must cast any argument of the constructor dummy.

\end{itemize}

Even with those unsafe castings, you should never get error like
``segmentation fault''. In fact even if your program may seem
ill-typed to the Ocaml type-checker, it can't go wrong: it comes 
from a Coq well-typed terms, so for example inductives will always 
have the correct number of arguments, etc. 

More details about the correctness of the extracted programs can be 
found in \cite{Let02}.

We have to say, though, that in most ``realistic'' programs, these
problems do not occur. For example all the programs of Coq library are
accepted by Caml type-checker without any {\tt Obj.magic} (see examples below).



\asection{Some examples}

We present here two examples of extractions, taken from the 
\Coq\ Standard Library. We choose \ocaml\ as target language, 
but all can be done in the other dialects with slight modifications.
We then indicate where to find other examples and tests of Extraction.

\asubsection{A detailed example: Euclidean division}

The file {\tt Euclid} contains the proof of Euclidean division
(theorem {\tt eucl\_dev}). The natural numbers defined in the example
files are unary integers defined by two constructors $O$ and $S$:
\begin{coq_example*}
Inductive nat : Set :=
  | O : nat
  | S : nat -> nat.
\end{coq_example*}

This module contains a theorem {\tt eucl\_dev}, whose type is:
\begin{verbatim}
forall b:nat, b > 0 -> forall a:nat, diveucl a b
\end{verbatim}
where {\tt diveucl} is a type for the pair of the quotient and the
modulo, plus some logical assertions that disappear during extraction.
We can now extract this program to \ocaml:

\begin{coq_eval}
Reset Initial.
\end{coq_eval}
\begin{coq_example}
Require Import Euclid Wf_nat.
Extraction Inline gt_wf_rec lt_wf_rec induction_ltof2.
Recursive Extraction eucl_dev.
\end{coq_example}

The inlining of {\tt gt\_wf\_rec} and others is not
mandatory. It only enhances readability of extracted code.
You can then copy-paste the output to a file {\tt euclid.ml} or let 
\Coq\ do it for you with the following command: 

\begin{coq_example}
Extraction "euclid" eucl_dev.
\end{coq_example}

Let us play the resulting program:

\begin{verbatim}
# #use "euclid.ml";;
type nat = O | S of nat
type sumbool = Left | Right
val minus : nat -> nat -> nat = <fun>
val le_lt_dec : nat -> nat -> sumbool = <fun>
val le_gt_dec : nat -> nat -> sumbool = <fun>
type diveucl = Divex of nat * nat
val eucl_dev : nat -> nat -> diveucl = <fun>
# eucl_dev (S (S O)) (S (S (S (S (S O)))));;
- : diveucl = Divex (S (S O), S O)
\end{verbatim}
It is easier to test on \ocaml\ integers:
\begin{verbatim}
# let rec nat_of_int = function 0 -> O | n -> S (nat_of_int (n-1));;
val i2n : int -> nat = <fun>
# let rec int_of_nat = function O -> 0 | S p -> 1+(int_of_nat p);;
val n2i : nat -> int = <fun>
# let div a b = 
     let Divex (q,r) = eucl_dev (nat_of_int b) (nat_of_int a)
     in (int_of_nat q, int_of_nat r);;
val div : int -> int -> int * int = <fun>
# div 173 15;;
- : int * int = (11, 8)
\end{verbatim}

Note that these {\tt nat\_of\_int} and {\tt int\_of\_nat} are now
available via a mere {\tt Require Import ExtrOcamlIntConv} and then
adding these functions to the list of functions to extract. This file
{\tt ExtrOcamlIntConv.v} and some others in {\tt plugins/extraction/}
are meant to help building concrete program via extraction.

\asubsection{Extraction's horror museum}

Some pathological examples of extraction are grouped in the file
{\tt test-suite/success/extraction.v} of the sources of \Coq.

\asubsection{Users' Contributions}

 Several of the \Coq\ Users' Contributions use extraction to produce 
 certified programs. In particular the following ones have an automatic 
 extraction test (just run {\tt make} in those directories): 

 \begin{itemize}
 \item Bordeaux/Additions
 \item Bordeaux/EXCEPTIONS
 \item Bordeaux/SearchTrees
 \item Dyade/BDDS
 \item Lannion
 \item Lyon/CIRCUITS
 \item Lyon/FIRING-SQUAD
 \item Marseille/CIRCUITS
 \item Muenchen/Higman
 \item Nancy/FOUnify
 \item Rocq/ARITH/Chinese
 \item Rocq/COC
 \item Rocq/GRAPHS
 \item Rocq/HIGMAN
 \item Sophia-Antipolis/Stalmarck
 \item Suresnes/BDD
 \end{itemize}

 Lannion, Rocq/HIGMAN and Lyon/CIRCUITS are a bit particular. They are 
 examples of developments where {\tt Obj.magic} are needed.
 This is probably due to an heavy use of impredicativity.
 After compilation those two examples run nonetheless,
 thanks to the correction of the extraction~\cite{Let02}. 

% $Id$ 

%%% Local Variables: 
%%% mode: latex
%%% TeX-master: "Reference-Manual"
%%% End: 
%
\include{Program.v}%
\include{Polynom.v}%  = Ring
\include{Nsatz.v}%
\include{Setoid.v}% Tactique pour les setoides
\achapter{Asynchronous Proof Processing}
\aauthor{Enrico Tassi}

\label{pralitp}
\index{Asynchronous Proof Processing!presentation}

This chapter explains how proofs can be asynchronously processed by Coq.
This feature improves the reactiveness of the system when used in interactive
mode via CoqIDE.  In addition to that it allows Coq to take advantage of
parallel hardware when used as a batch compiler by decoupling the checking
of statements and definitions from the contruction and checking of proofs
objects.

This feature is desingned to help dealing with huge libraries of theorems
characterized by long proofs.  At the current state it may not be beneficial
on small set of short files.

This feature has some technical limitations that may make it unsuitable for
some use cases.

For example in interactive mode errors coming from the kernel of Coq are
signalled late.  The most notable type of errors belonging to this category are
universes inconsistency or unsatisfied guardedness conditions for fixpoints
built using tactics.

Last, at the time of writing, only opaque proofs (ending with $Qed$) can be
processed asynchronously.

\subsection{Proof annotations}

To process a proof asynchronously Coq needs to know the precise statement
of the theorem without looking at the proof.  This requires some annotations
if the theorem is proved inside a $Section$ (see Section~\ref{Section}).

When a section is ended Coq looks at the proof object to decide which
section variables are actually used and hence have to be quantified in the
statement of the theorem.  To make the construction of the proofs not
mandatory for ending a section one can start each proof with the
$Proof using$ command~\ref{ProofUsing} that declares the subset of section
variables the theorem uses.

The presence of $Proof using$ is mandatory to process proofs asynchronously
in interactive mode.

It is not strictly mandatory in batch mode if it is not the first time the
file is compiled and if the file itself did not change.  In case the
proof does not begin with $Proof using$ the system records in an auxiliary
file, produced along with the $.vo$ file, the list of section variable used.

\subsubsection{Automatic suggestion of proof annotations}

The command $Set Suggest Proof Using$ makes Coq suggest, when a $Qed$
command is processed, a correct proof annotation.  It is up to the user
to modify the proof script adding the proof annotation.

\subsection{Interactive mode}

At the time of writing the only user interface supporting asynchronous proof
processing is CoqIDE.  

When CoqIDE is started two Coq processes are created.  The master one follows
the user, giving feedback as soon as possible by skipping proofs, that are
delegated to the worker process.  The worker process, whose state can be seen
by clicking on the button in the lower right corner of the main CoqIDE window,
asynchronously processes the proofs.  If a proof contains an error, it is
reported in red in the label of the very same button, that can also be used to
see the list of errors and jump to the corresponding line.

If a proof is processed asynchronously the corresponding $Qed$ command is
is coloured using a color that is lighter than usual.  This signals that
the proof has been delegated to a worker process (or will be processed
lazily if the $-async-proofs lazy$ option is used).  Once finished the
worker process will provide the proof object, but this will not be
automatically checked by the kernel of the main process.  To force
the kernel to check all proof objects one has to click the button
with the gears.

% THIS PARAGRAPH MAY CHANGE.
% IF QED IS FULLY PURELY FUNCTIONAL (not yet because of vm compute) THE CHECKING
% COULD BE MADE BY THE THREAD THAT MANAGES THE WORKER.

Only when the gears button is pressed all universe constraints are checked.

\subsubsection{Caveats}
The number of worker processes can be increased by passing CoqIDE the
$-async-proofs-j 2$ flag.  Note that the memory consumption increases
too, since each worker has to be an exact copy of the master process
and requires the same amount of memory.  Also note that the master process
has to both respond to the user and manage the workers, hence increasing
their number may slow down the master process.

To disable this feature it is enough to pass the $-async-proofs off$ flag to
CoqIDE.

\subsection{Batch mode}

When Coq is used as a batch compiler by running $coqc$ or $coqtop -compile$
it produces a $.vo$ file for each $.v$ file.  A $.vo$ file contains, among
other things, theorems statements and proofs.  Hence to produce a $.vo$
Coq need to process all the proofs of the $.v$ file.

The asynchronous processing of proofs can decouple the generation of a
compiled file (like the $.vo$ one) that can be $Required$ from the generation
and checking of the proof objects.  The $-quick$ flag can be passed to
$coqc$ or $coqtop$ to produce, quickly, $.vi$ files.  Alternatively, if
the $Makefile$ one uses was produced by $coq\_makefile$ the $quick$
target can be used to compile all files using the $-quick$ flag.

A $.vi$ file can be $Required$ exactly as a $.vo$ file but: 1) proofs are
not available (the $Print$ command produces an error); 2) some universe
constraints are missing, hence universes inconsistencies may not be
discovered.  A $.vi$ file does not contain proof objects, but proof tasks,
i.e. what a worker process can transform into a proof object.

Compiling a set of files with the $-quick$ flag allows one to work,
interactively, on any file without waiting for all proofs to be checked.

While one works interactively, he can fully check all $.v$ files by
running $coqc$ as usual.

Alternatively one can check the proof tasks store in $.vi$ files.  This may be
faster, since all proof tasks are independent and can be checked in parallel.
The $coqtop -schedule-vi-checking 3 a b c$ command can be used to obtain
a good scheduling for 3 workers to check all proof tasks of $a.vi$, $b.vi$ and
$c.vi$.  Auxiliary files are used to predict how long a proof task will take,
assuming it will take the same amount of time it took last time.
The output of $coqtop -schedule-vi-checking$ is a list of commands one has
to execute in order to check all proof tasks.

\subsubsection{Caveats}

At the time of writing producing a $.vo$ file from a $.vi$ file is not
possible.  When a proof task is run, he proof object is generated, checked but
then discarded.  This may change in the future, since generating $.vo$ files
from $.vi$ is, in theory, possible.  

Checking all proof tasks does not guarantee the same degree of safety
that producing a $.vo$ file gives.  In particular universe constraints
are checked to be consistent for every single proof, but not globally.

%%% Local Variables: 
%%% mode: latex
%%% TeX-master: "Reference-Manual"
%%% End: 
% Paral-ITP
\include{Misc.v}
%BEGIN LATEX
\RefManCutCommand{ENDADDENDUM=\thepage}
%END LATEX
\nocite{*}
\bibliographystyle{plain}
\bibliography{biblio}
\cutname{biblio.html}

\printindex
\cutname{general-index.html}

\printindex[tactic]
\cutname{tactic-index.html}

\printindex[command]
\cutname{command-index.html}

\printindex[error]
\cutname{error-index.html}

%BEGIN LATEX
\listoffigures
\addcontentsline{toc}{chapter}{\listfigurename}
%END LATEX

\end{document}


